\chapter{Articulation Work} % (fold)
\label{cha:articulation_work}

In diesem Kapitel wird das Konzept „Articulation Work“ dargestellt und in den Kontext von menschlicher Arbeit gestellt. Im ersten Teil des Kapitels wird auf die historische Entwicklung des Begriffs „Articulation Work“ und die unterschiedlichen Herangehensweise zu dessen Verständnis eingegangen. Der zweite Teil des Kapitels widmet sich den Aktivitäten, die „Articulation Work“ ausmachen, den Merkmalen, an denen sich gute „Articulation Work“ zeigt, sowie den Möglichkeiten der Unterstützung von „Articulation Work“ durch organisationale und technische Maßnahmen.

\section{Begriffsbestimmung} % (fold)
\label{sec:aw_begriffsbestimmung}

Das Konzept der "Articulation Work" wurde als Erklärungsmodell für eine bestimmte Art von menschlicher Arbeit Mitte der 1980er Jahre von \citet{Strauss85} eingeführt. Neben \citet{Strauss85} tragen auch die Arbeiten von \citet{Gerson86} und \citet{Fujimura87} wesentlich zur Begriffsbestimmung und Konzeptbildung bei. Die vorhandene Literatur, die Bezug auf „Articulation Work“ nimmt, referenziert im Wesentlichen auf diese drei Arbeiten bzw. eine dieser drei Arbeiten. Der Kontext, in dem die Entwicklung der im folgenden vorgestellten Konzepte erfolgte, war die komplexe, von viel Interaktion an zahlreichen Schnittstellen geprägte Arbeit in Krankenhäusern \citep{Strauss85}, in der Wissenschaft \citep{Fujimura87} und in Versicherungsunternehmen \citep{Gerson86}, die die jeweiligen Autoren in mehreren Fallstudien untersuchten. 

Um in der Folge einen einheitlichen Begriffsraum aufspannen zu können, ist vorab der Begriff „Arbeit“ zu klären. Die eben genannten Autoren führen keine explizite Definition an, weshalb hier auf eine Definition zurückgegriffen wird, die im Kontext der folgenden Ausführungen zur „inneren“ Struktur von Arbeit nach „außen“ hinreichend umfassend ist\footnote{Auf eine umfassende Literaturstudie und die Entwicklung eines darauf aufbauenden „Arbeits“-Begriffs wurde hier verzichtet, da dies über den Betrachtungsbereich und Anspruch dieser Arbeit hinausgeht}. \citep{Semmer04} definieren vor dem Hintergrund der Organisationspsychologie „Arbeit“ wie folgt: \emph{„Arbeit ist zielgerichtete menschliche Tätigkeit zum Zwecke der Transformation und Aneignung der Umwelt aufgrund selbst- oder fremddefinierter Aufgaben, mit gesellschaftlicher, materieller oder ideeller Bewertung, zur Realisierung oder Weiterentwicklung individueller oder kollektiver Bedürfnisse, Ansprüche und Kompetenzen.“}}. Arbeit ist also ein menschliches Phänomen, Träger von Arbeit sind immer Menschen. Arbeit definiert sich außerdem durch ihre Zielgerichtetheit und findet immer in Interaktion mit der Umwelt statt. Die Ziele, auf die Arbeit ausgerichtet ist, leiten sich aus Aufgaben ab, die sich Menschen selbst setzten können oder die ihnen vorgegeben werden. Diese Aufgaben dienen der Erreichung von individuellen oder kollektiven Bedürfnissen und Ansprüchen bzw. der (Weiter-)Entwicklung von Kompetenzen. Die Bewertung der Zielerreichung muss nicht unbedingt aus materieller Perspektive erfolgen sondern kann auch ideell oder gesellschaftlich begründet sein.

„Articulation Work“ ist jener Anteil der gesamten durchgeführten Arbeit, der der Abstimmung mit anderen Individuen dient. Diese Abstimmung ist notwendig, um das eigentliche Arbeitsziel erreichen zu können. Arbeit wird von den oben angeführten Autoren als inhärent kooperativer Prozess gesehen, der immer auf Interaktion mit anderen Menschen basiert bzw. diese bedingt (Strauss formuliert diese Annahme in Bezugnahme auf \citet{Hughes71} prägnant mit der Aussage \emph{„work rests ultimately on interaction“}). Diese Annahme erscheint insofern als zulässig, als dass selbst Arbeitsabläufe, die selbst keine Kooperation mit anderen Menschen mit sich bringen, zumindest auf den Ergebnissen anderer Arbeitsabläufe aufbauen oder als Grundlage weiterer Arbeitsabläufe dienen. Interaktion tritt also in jedem Arbeitsprozess zumindest zu Beginn und am Ende in unmittelbarer oder mittelbarer\footnote{Unter "mittelbar" ist hier Interaktion zu verstehen, die nicht im direkten Kontakt zwischen Individuen abläuft sondern lediglich indirekt durch die Ergebnisse eines Arbeitsprozesses (Materialien, Dokumente, \ldots) vermittelt wird.} Form auf.

\textbf{Abbildung, in der kooperative Arbeitsprozesse und solche mit mittelbarer und unmittelbarer Interaktion zu Beginn oder am Ende dargestellt werden}

Jener Teil von Arbeit, der der eigentlichen Zielerreichung dient, wird im hier vorgestellten Erklärungsmodell als „Production Work“ bezeichnet \citep{Fujimura87}. „Production Work“ ist komplementär zu „Articulation Work“ zu sehen und umfasst alle Aktivitäten, die der „Wertschöpfung“ im wörtlichen Sinn dienen. „Production Work“ sind also alle Tätigkeiten, die mit der Schaffung jener Werte (oder Ergebnisse) befasst sind, die durch den Arbeitsablauf erreicht werden sollen.  

\begin{figure}[htbp]
	\centering
		\includegraphics[height=3in]{img/ArticulationWork/ArbeitInteraktion.png}
	\caption{Struktur von Arbeitsabläufen}
	\label{fig:img_ArticulationWork_ArbeitInteraktion}
\end{figure}

Teile eines Arbeitsablaufs dienen also der Zielerreichung an sich („Production Work“). Andere Teile dienen der Abstimmung zwischen den involvierten Akteuren, um ein gemeinsames Verständnis über die jeweiligen Schnittstellen – also die Berührungspunkte zwischen den Tätigkeiten – zu entwickeln (siehe Abbildung \ref{fig:img_ArticulationWork_ArbeitInteraktion}). Diese Entwicklung eines gemeinsamen Verständnisses oder „Koordination“ ist kritisch für den Erfolg von kooperativer Arbeit \citep{Strauss93} und wird als „Articulation Work“ bezeichnet.\footnote{\emph{„Without an understanding of articulation, the gap between requirements and the actual work process in the office will remain inaccessible to analysis. That is, it will be possible to describe tasks in an idealized form but not to describe actual situations.“}\citep{Gerson86}} 

„Articulation Work“ ist also ein Enabler für funktionierende Kommunikation und Koordination im eigentlichen Arbeitsprozess. Sie umfasst dabei konkret die gegenseitigen Offenlegung der Annahmen aller beteiligten Personen, die den individuellen Arbeitsbeiträgen zugrunde liegen, und dem jeweiligen Vorgehen\footnote{\emph{"Reconciling incommensurate assumptions and procedures in the absence of enforceable standards is the essence of articulation.}\citep{Gerson86}} „Articulation Work“ ist keine Tätigkeit, die zu einem bestimmten Zeitpunkt im Arbeitsprozess durchgeführt wird und dann als abgeschlossen betrachtet werden kann. Vielmehr wird „Articulation Work“ immer auch begleitend zur eigentlichen produktiven Arbeit durchgeführt und umfasst neben planenden und koordinierenden Tätigkeiten auch das Erkennen von Fehlentwicklungen bzw. von Situationen, in denen eine erneute  Koordination notwendig ist\footnote{\emph{Articulation consists of all the tasks involved in assembling, scheduling, monitoring, and coordinating all of the steps necessary to complete a production task."}\citep{Gerson86}}. 

\begin{figure}[htbp]
	\centering
		\includegraphics[height=3in]{img/ArticulationWork/ArtikulationProduktion.png}
	\caption{Konzeptualisierung von „Arbeit“ nach \citep{Strauss85} und \citep{Fujimura87}}
	\label{fig:img_ArticulationWork_ArtikulationProduktion}
\end{figure}

Der Begriff „Articulation Work“ ist im Englischen zweideutig und von \citeauthor{Strauss85} auch bewusst so gewählt. Einerseits wird damit ausgedrückt, dass \emph{Arbeit} ("Work") artikuliert wird, andererseits zeigt der Begriff, das die \emph{Artikulation} selbst ebenfalls Arbeit ist (also Zeit und Ressourcen in Anspruch nimmt) und auch also solche wertgeschätzt werden muss \citep{Fujimura87}. „Articulation Work“ ist kein klar abgegrenztes und strukturiertes Konzept – sie tritt je nach Arbeitssituation in unterschiedlichen Spielarten auf. Die Unterscheidung dieser Arten von „Articulation Work“ ist für die Unterstützung derselben relevant und wird daher im folgenden Abschnitt genauer betrachtet.
% section begriffsbestimmung (end)

\section{Kontext} % (fold)
\label{sec:kontext}
Arc of Work, Due Process, ...
% section kontext (end)

\section{Arten von Articulation Work} % (fold)
\label{sec:arten_von_articulation_work}

Wie bereits von \citet{Gerson86} angeführt (siehe oben), argumentiert auch \citeauthor{Strauss88}, dass Artikulation immer passieren muss (und passiert), wo Menschen zusammenarbeiten, um zu vermeiden, dass unbekannte Aspekte Probleme bei der Durchführung der Arbeit verursachen \citep{Strauss88}. „Articulation Work“ ist kein revolutionäres Konzept, sondern fasst Tätigkeiten unter einem Begriff zusammen, die seit jeher Teil jeder Zusammenarbeit zwischen Menschen sind. Grundsätzlich geht Strauss davon aus, dass Artikulation immer abläuft, egal wie einfach oder kompliziert, wie eingespielt oder neuartig eine (Zusammen-)Arbeit ist \citep{Strauss88}. Sehr wohl existieren jedoch Unterschiede in der Qualität der Arbeit, die sich auf die Form der Artikulation auswirken, die zu deren Abstimmung notwendig ist: \emph{„A useful fundamental distinction between classes of interaction is between the routine and the problematic. Problematic interactions involve 'thought', or when more than one interactant is involved then also 'discussion'.“} \citep{Strauss93}. Dieses Zitat zeigt im Übrigen auch, dass „Interaction“ im Sinne von Strauss nicht unbedingt ein kollektives Phänomen ist, sondern auch individuell (im Bezug auf die (unbelebte) Umgebung) auftreten kann.

Je komplexer („problematic“) eine Interaktion ist, desto notwendiger wird laut Strauss eine explizite Beschäftigung mit dem Vorgang der Artikulation. Bei einfachen, eingespielten („routine“) Interaktionen bleibt die Artikulation zumeist implizit, verborgen und informell \citep{Hampson05} (entsprechend der „Sozialisation“ im aus der Domäne der Wissensgenerierung und -teilung stammenden SECI-Zyklus \citep{Nonaka95}). Ein grundlegendes Problem, dass Artikulation für jeden noch so einfach erscheinend Arbeitsvorgang potentiell relevant macht, spricht Strauss mit den Worten von Hughes unmittelbar nach der Definition von „problematic interaction“ an: \emph{„[O]ne man's routine of work is made up of the emergencies of other people“} \citep{Hughes71} zitiert nach \citep{Strauss93}.

„Articulation Work“ tritt also in zwei Qualitäten auf. Ist der Bedarf zur Abstimmung bekannt und werden Tätigkeiten zur Abdeckung dieses Bedarf bewusst durchgeführt, so spricht man von \emph{expliziter} „Articulation Work“ \citep{Strauss88} \citep{Fjuk97}. Die Abstimmung von Tätigkeiten, die ständig während der Zusammenarbeit unbewusst ausgeführt wird, bezeichnet man als implizite „Articulation Work“\footnote{\emph{The explicit articulation is thus connected to the planning and decisions regarding the salient dimensions of work -- who, what, when, how -- while implicit articulation is invaluable when carrying out activities in situated  circumstances, in order to handle contingencies.}\citep[][S.5]{Fjuk97}}. Letztgenannte Art ist es auch, die von den Arbeitenden „automatisch“ zur Anwendung gebracht wird, sobald Änderungen in der Arbeitsumgebung oder Probleme auftreten \citep{Strauss88}. Implizite „Articulation Work“ stößt aber an ihre Grenzen, wenn die Arbeitssituation als „problematisch“ \citep{Strauss88} oder „komplex“ \citep[][S. 23f]{Schmidt90} wahrgenommen wird. Es wird dann notwendig, dezidierte Abstimmungs-Aktivitäten anzustoßen, also explizite „Articulation Work“ durchzuführen.

Neben der Unterscheidung zwischen impliziter und expliziter Articulation Work anhand der Komplexität der zugrunde liegenden Interaktion führt Strauss keine systematische Betrachtung von Articulation Work hinsichtlich deren Ausprägungen durch. Offensichtlich wird in seinen Texten jedoch, dass es Articulation Work als beobachtbares und eindeutig also solche identifizierbares Phänomen nicht gibt. Abhängig vom betrachteten Arbeitsablauf, der Arbeitsumgebung und den beteiligten Personen zeigt sich Articulation Work in unterschiedlichen Formen.

In der Literatur existieren zwei Ansätze zur Differenzierung zwischen unterschiedlichen Arten von Articulation Work. \citet{Fjuk97} stellen Articulation Work der Activity Theory \citep{Leontev78} gegenüber und unterscheiden so verschiedene Ebenen. \citet{Hampson05} führen ein Raster ein, das Articulation Work hinsichtlich der Art des Arbeitsprozesses unterschiedet, in dem sie zur Anwendung kommt. Beide Ansätze werden in der Folge im Detail beschrieben und bezüglich ihrer Implikationen für diese Arbeit betrachtet.

\subsection{Unterscheidung nach Fjuk, Smørdal und Nurminen}
\label{sub:arten_fjuk}

\citet{Fjuk97} betrachten Articulation Work im Kontext von \gls{CSCW} und versuchen ein konzeptionelles Framework zu entwickeln, das die Rolle von Computersystemen im Kontext indvidueller und kollektiver Tätigkeiten erklärt -- sie entwickeln also ein Erklärungsmodell für die Funktionsweise sozio-technischer Systeme \citep{Emery60}. Während die Implikationen von „Articulation Work“ für \gls{CSCW} an dieser Stelle nicht näher von Belang sind ist aber das theoretische Framework, das die Autoren ihren Ausführungen zu Grunde legen von Interesse. 

\citet{Fjuk97} beziehen sich bei ihren Überlegungen auf die „Activity Theorie“ (Tätigkeits-Theorie), die maßgeblich von \citep{Leontev72} geprägt wurde. Die Autoren argumentieren, dass diese einen Ansatzpunkt bietet, die von Strauss als relevant erkannten aber nicht näher behandelten „externen Faktoren“, die Arbeit beeinflussen, zu berücksichtigen. Der Begriff der „externen Faktoren“ wird mit allen Einflussfaktoren beschrieben, die nicht unmittelbar Teil des Arbeitsablaufs sind sondern technologischer, organisationaler, kultureller, wirtschaftlischer oder physiologischer Natur sind. 

Ohne an dieser Stelle näher auf die „Activity Theory“ \footnote{für eine allgemein verständliche Einführung unter Berücksichtigung der praktischen Implikationen siehe \citep{Dahme97} oder \citep{Nardi06}} einzugehen, seien hier die drei Kernkonzepte der Theorie erwähnt:
\begin{itemize}
	\item Activity
	\item Action 
	\item Operation  
\end{itemize}

Diese drei Konzepte bilden eine Hierarchie, in denen eine „Activity“ an oberster Stelle steht. Eine „Activity“ ist eine menschliche Tätigkeit, die durch ein Motiv getrieben ist und der (vorerst) individuellen Bedürfnisbefriedigung dient. Eine „Activity“ setzt sich aus mehreren „Actions“ zusammen, die jede für sich ein aus dem Motiv heraus begründbares Ziel haben und zur Bedürfnisbefriedigung direkt oder indirekt betragen. „Actions“ setzen sich wiederum aus „Operations“ zusammen, also einzelnen, nicht mehr bewusst ausgeführten Handlungen, die durch die Bedingungen des jeweiligen Umgebungskontexts bestimmt werden. Während sie lernen, transformieren Individuen laufend „Actions“ zu „Operations“, automatisieren also deren Ausführung, sodass sich die kognitive Belastung verringert (als klassisches Beispiel kann hier das Erlernen des Autofahrens dienen).

Die „Activity Theory“ beschreibt als psychologisches Modell vorerst das Individuum und dessen Verhalten. In sozialen Systemen, die auf Interaktion basieren, stößt das Modell jedoch an die Grenzen der erklärbaren Phänomene. \citet{Engestrom87} baut auf der klassischen „Activity Theory“ auf und erweitert diese um den Aspekt der Gemeinschaft sowie der Interaktion in dieser bzw. der Rolle von Artefakten („Objects“) in derartigen Settings. \citet{Fjuk97} bemängeln aber in ihren Ausführungen, dass \citeauthor{Engestrom87} in seinen Ausführungen abstrakt bleibt und nicht den Konkretisierungsgrad der originären „Activity Theory“ erreicht, was das Zusammenspiel der unterschiedlichen Ebenen („Activity“, „Action“ und „Operation“) betrifft.

Basierend auf der Unterscheidung zwischen „Activity“, „Action“ und „Operation“ führen \citet{Fjuk97} zwei unterschiedliche Arten von „Articulation Work“ ein, die sich in ihren Bezugspunkten unterschieden und jeweils für den individuellen und kollektiven Fall betrachtet werden.  

\begin{description}
	\item[Individual articulation of action within activity] 
	\item[Individual articulation of operation within action] 
	\item[Collective articulation of action within activity] 
	\item[Collective articulation of operation within action] 
\end{description}

\subsection{Unterscheidung nach Hampson und Junor}
\label{sub:arten_hampson}

\citep{Hampson05} verwenden „Articulation Work“ als Framework zur Erklärung von „interactive customer service“, also dem jenen Kundenbeziehungen, bei denen die Interaktion zwischen Anbieter und Kunden im Vordergrund steht. Im Rahmen dieser Arbeit zeigen die Autoren auch die historische Entwicklung des Begriffs „Articulation Work“ auf und entwickeln einen Raster zur Einordnung unterschiedlicher Ausprägungen von Arbeit, die wiederum unterschiedliche Arten von „Articulation Work“ bedingen. Dieses Raster ist hier von Interesse.

Bezugnehmend auf \citep{Strauss93} unterschieden die Autoren einerseits zwischen Arbeitsabläufen, die \emph{routine} sind, und solchen, die \emph{non-routine} sind. Außerdem kann zwischen Arbeitsabläufen unterschieden werden, die \emph{visible} oder \emph{invisible} sind \citep{Star99}. Während \emph{visible work} all jene Arbeitsabläufe umfasst, die als solche wahrgenommen werden, bezieht sich \emph{invisible work} auf alle Arbeitsabläufe, die stattfinden aber nicht „offiziell“ wahrgenommen werden (also etwa nicht in einem Prozessmodell aufscheinen).  Daraus ergeben sich vier zu unterscheidende Settings, in denen „Articulation Work“ stattfindet und die sich sowohl in der konkret als „Articulation Work“ ausgeführten Tätigkeit als auch in der möglichen methodischen und/oder technischen Unterstützung unterscheiden.

\begin{description}
	\item[Visible routine work] beschreibt jene Arbeitsabläufe, die von klassischen Management-Ansätzen erfasst werden, formalisiert werden können und in Unternehmen oft normiert vorgegeben sind (etwa in Form von Prozessmodellen oder durch die Vorgaben eines Workflow-Management-Systems). „Articulation Work“ findet hier zu definierten Zeitpunkten und explizit ausgelöst statt, um die normierten Abläufe zu definieren bzw. diese an veränderte Rahmenbedingungen anzupassen. 
	\item[Visible non-routine work] beschreibt Arbeitsabläufe in Umgebungen, die so dynamisch sind, dass normierte Abläufe aufgrund der raschen, nicht absehbaren Veränderungen der Anforderungen nicht sinnvoll einsetzbar sind. „Articulation Work“ tritt hier regelmäßig implizit und explizit auf, da jede Veränderung eine -- je nach Ausmaß der Veränderung implizite oder explizite -- Neuabstimmung der Zusammenarbeit nach innen und außen benötigt.
	\item[Invisible routine work] umfasst all jene Arbeitsabläufe in Unternehmen, die zwar etabliert sind, von den traditionellen Steuer- und Kontroll-Werkzeugen im Unternehmen jedoch nicht erfasst werden. Sie sind formal nicht normiert, treten jedoch so regelmäßig auf, das sich eine routinemäßige Herangehensweise herausbildet. Articulation Work läuft hier bei Veränderungen der Rahmenbedingungen zumeist implizit ab und sorgt dafür, dass die Interaktion zwischen den Beteiligten weiter funktioniert. Explizite „Articulation Work“ unter Einbeziehung der betroffenen Personen kann hier dafür sorgen, Arbeitsabläufe dieser Kategorie in den Bereich der „visible routine work“ überzuführen.
	\item[Invisible non-routine work] umfasst jene Arbeitsabläufe, die zur Behandlung von unvorhergesehenen Anforderungen durchgeführt werden und die nach außen hin nicht sichtbar wird. Typisch treten derartige Situationen bei Ausnahmefällen in etablierten Arbeitsabläufen auf, bei denen die Tätigkeiten zu Wiederherstellung einer „regelkonformen“ Situation oft nicht durch Steuer- und Kontrollelemente erfasst werden und durch die Einzigartigkeit der Ausnahme oder des Kontexts, in dem diese auftritt, keine etablierten Handlungsmuster existieren. „Articulation Work“ ist hier ad-hoc notwendig, um adäquat auf die Anforderungen der Umwelt reagieren zu können. Sowohl explizite und implizite „Articulation Work“ kann hier zu Anwendung kommen, wobei als Entscheidungskriterien zwischen diesen beiden Ausprägungen die wahrgenommene Komplexität der Situation sowie die zur Lösung zur Verfügung stehende Zeit zu berücksichtigen sind.
\end{description}

In unterschiedlichen Arbeitssituationen können diese vier Kategorien auch kombiniert auftreten. Manche Arbeitsabläufe können durch erfolgreich durchgeführte „Articulation Work“ in eine andere Kategorie verschoben werden, wo der Bedarf an laufender ad-hoc Abstimmung geringer oder nicht vorhanden ist. Andere Arbeitsabläufe sind ihrer Natur nach nicht strukturierbar und formalisierbar, so dass „Articulation Work“ ein inhärenter Bestandteil des Ablaufs ist und trotz wiederholter Durchführung auch bleibt.

% section arten_von_articulation_work (end)

\section{Unterstützung von Articulation Work} % (fold)
\label{sec:unterstützung_von_articulation_work}

Nach den ersten Arbeiten von Strauss zum Thema „Articulation Work“ wurde das Konzept rasch als Erklärungsmodell für die Vorgänge im Zuge kooperativer Arbeit aufgenommen und darauf basierend Maßnahmen zur Unterstützung derselben abgeleitet. Anhand der historischen Entwicklung von Mitte der 1980er-Jahre bis Ende des ersten Jahrzehntes des neuen Jahrtausends werden im Folgenden diese Maßnahmen beschrieben, in den jeweiligen Anwendungskontext gesetzt und auch hinsichtlich des zu unterstützenden Verständnisses von „Articulation Work“ betrachtet. Hierbei werden alle Arbeiten berücksichtigt, die sich direkt auf den von Strauss geprägten „Articulation Work“-Begriff beziehen. In der Literatursuche wurden dazu Datenbanken aus den Bereichen Informatik, Psychologie, Soziologie, den Wirtschaftswissenschaften sowie der Organisationslehre durchsucht. Nach der initialen Suche wurde jeweils auch die in den gefundenen Arbeiten referenzierte Sekundärliteratur aufgearbeitet. Des weiteren wurden mithilfe von rückwärts verlinkenden Datenbanken (wo vorhanden) Publikationen erfasst, die die bislang gefundenen Arbeiten referenzieren und diese hinsichtlich ihrer Relevanz überprüft. Insgesamt ergab sich so eine Sammlung von 47 Publikationen (inklusive der hier nicht nochmals behandelten grundlegenden Arbeiten, die bereits oben beschrieben wurden). Von diesen 47 Publikationen trafen XY eine Aussage zu Aspketen, die auf die Unterstützung von „Articulation Work“ abzielen. Die übrigen Arbeiten verwenden „Articulation Work“ als Erklärungs-Framework für Fallstudien und werden weiter unten zusammenfassend angeführt ohne näher auf sie einzugehen.

Zur strukturierten Umsetzung der Betrachtung der Unterstützung von „Articulation Work“ wird ein einheitlicher Raster angewandt, anhand dessen die aus unterschiedlichen Forschungsgebieten stammenden und in unterschiedlichen Anwendungsdomänen angewandten Arbeiten einander gegenüber gestellt werden können. Neben den eigentlichen Unterstützungsmaßnahmen ist zur Bewertung derselben auch Kontextinformation notwendig, die die unterschiedlichen Ansätze offenlegt. Folgende Merkmale einer Arbeit werden dazu betrachtet:
\begin{description}
 \item[Forschungsdomäne] Forschungsgebiet aus dem das Konstrukt "Articulation Work" betrachtet wird bzw. in dessen Kontext es zur Anwendung gebracht wird.
 \item[Untersuchungsdomäne] Abstraktes oder konkretes Problemfeld, in dem "Articulation Work" als Anaylsedimension oder zur Ableitung von Maßnahmen angewandt wird.
 \item[Verständnis von Articulation Work] Bei der Betrachtung der Untersuchungsdomäne zur Anwendung gebrachte Definition von "Articulation Work" (ggf. auch in Abgrenzung zu anderen Arten von Arbeit).
 \item[Auftretende Phänomene] Tätigkeiten und/oder Handlungsweisen, die im Zuge von "Articulation Work" auftreten und ggf. unterstützt werden können
 \item[Unterstützung] Konkrete oder abstrakte Maßnahmen oder Werkzeuge, die zur Unterstützung von "Articulation Work" vorgeschlagen und/oder umgesetzt werden.
 \item[Auftretende Effekte] Tatsächliche oder vermutete Auswirkungen der Unterstützung auf die durchgeführte "Articulation Work"
\end{description}

Die als relevant betrachteten Publikationen sind methodisch höchst unterschiedlich ausgerichtet. Ein großer Anteil beschreibt rein empirisch-deskriptiv ein beobachtetes Phänomen und zieht Schlüsse hinsichtlich möglicher bzw. notwendiger Ausprägungen von "Articulation Work" in bestimmten Anwendungsdomänen. Ein anderer Teil fokussiert auf die organisationale und/oder technische Unterstützung von "Articulation Work", zum Teil ohne auf eigene empirische Ergebnisse aufzubauen oder diese zu erheben. Aus diesem Grund kann das oben angegebene Raster nicht immer vollständig befüllt werden. Wo hinsichtlich einer bestimmten Dimension keine Information vorhanden ist, wird explizit im Text darauf hingewiesen. Wo mehrere Publikation eines Autors oder einer Gruppe zum gleichen Forschungsgegenstand existieren, wurden diese in einem Abschnitt zusammengefasst und in der jeweiligen Einleitung auf die der Beschreibung zugrundeliegenden Publikationen verwiesen.

%TEMPLATE
\subsection{Papertitel / Titel des Forschungsprojekts}

Angabe der zugrundeliegenden Publikation sowie einer kurzen Zusammenfassung des Inhalts

\paragraph{Forschungsdomäne}

\paragraph{Untersuchungsdomäne}

\paragraph{Verständnis von Articulation Work}

\paragraph{Auftretende Phänomene}

\paragraph{Unterstützung}

\paragraph{Auftretende Effekte}
%TEMPLATE

\subsection{Work and the Division of Labor}

In dieser Publikation \citep{Strauss85} beschreibt Strauss zum ersten Mal sein Konzept „Articulation Work“. Er motiviert seine Forschung mit Erklärungs-Lücken, die in der Konzeptualisierung von kooperativer Arbeit bzw. Arbeitsteilung bestehen. Ziel ist es, mit den entwickelten Ansätzen nicht nur Arbeit(steilung) erklären zu können sondern auch weitergehende Forschung zu ermöglichen bzw. dieser als Leitprinzipen zugrunde zu liegen. 

\paragraph{Forschungsdomäne}
Strauss' Hintergrund, auf Basis dessen die hier betrachtete Publikation geschrieben wurde, ist die Soziologie (im konkreten Artikel die Medizinsoziologie \citep[vgl.][]{Siegrist05}). Methodisch wendet er den von ihm mitgeprägten „Grounded Theory“-Ansatz \citep{Glaser77} an, mithilfe dessen aus Feldbeobachtungen und deren systematischer, qualitativer Auswertung Theorien abzuleiten, die das Verhalten und die Interaktion der beobachteten Akteure erklärt.

\paragraph{Untersuchungsdomäne}
Neben einer konzeptionellen Erörterung der kooperativen Arbeit zugrunde liegenden Denkmodelle und Betrachtungsmuster (siehe Abschnitt \ref{sec:kontext}) werden  

\paragraph{Verständnis von Articulation Work}

\paragraph{Auftretende Phänomene}

\paragraph{Unterstützung}

\paragraph{Auftretende Effekte}
%TEMPLATE

\subsection{Gegenüberstellung und Zusammenfassung} % (fold)
\label{sub:gegenüberstellung_und_zusammenfassung}

% subsection gegenüberstellung_und_zusammenfassung

% section unterstützung_von_articulation_work (end)

\section{Fazit} % (fold)
\label{sec:fazit}

\textbf{hier muss eine zusammenfassende Tabelle der in der Literatur verfügbaren Information rein}

Die Zielsetzung von „Articulation Work“ formulieren die Proponenten des Ansatzes - allen voran Strauss - klar aus. Offen bleiben jedoch bei allen Autoren direkten Aussagen zum eigentlichen Gegenstand von „Articulation Work“ – also Allem was von den beteiligten Individuen zu artikulieren ist – und den notwendigen Leistungen der Individuen im Prozess der Artikulation. Aussagen zu diesen Aspekten sind aber für die Entwicklung von Ansätzen zur Unterstützung von expliziter „Articulation Work“ notwendig. 

Strauss ist sich dieser Auslassung bewusst\footnote{\emph{„[\ldots] many social scientist pay almost no attention to interior activity: ignoring it, taking it for granted, but leaving it unexamined, or giving it the kind of abstract but not very detailed analysis [\ldots]“}\citep[][S. 131]{Strauss93}}, und beschäftigt sich in späteren Arbeiten \citep{Strauss93} auch mit jenen kognitiven Vorgängen, die von ihm als „thought processes“ oder „mental activities“ bezeichnet werden und die untrennbar mit jeder Art von Tätigkeit und Interaktion verbunden sind\footnote{\emph{„These [thought processes] accompany visible action, as well as precede and follow in conditional and consequential modes“}\citep[][S. 146]{Strauss93}} und diese beeinflussen\footnote{\emph{„Even well-grooved, routine action and interaction may be accompanied by thought [\ldots] directly relevant to the work at hand. As I vacuum the house, barely noticing my movements, still I give myself commands [\ldots]“}\citep[][S. 132]{Strauss93}}. 

Im Kontext der Abstimmung von Tätigkeiten kommt den „thought processes“ der Individuen große Bedeutung zu, da sie den sichtbaren individuellen Handlungen zugrunde liegen bzw. diese beeinflussen. „Articulation Work“ wirkt sich also auf die „thought processes“ der beteiligten Individuen aus. „Thought processes“ umfassen \emph{„images, imaginations, projections of scenes, [...] flashes of insight, rehearsals of action, construction and reconstruction of scenarios, the spurting up of metaphors or comparisons, the reworking and reevaluating of past scenes and one's actions within them, and so on and on“} \citep[][S. 130]{Strauss93} - also im Wesentlichen alle kognitiven Vorgänge, die unmittelbar oder mittelbar im Zusammenhang mit den sichtbaren Arbeitsaspekten, insbesondere den Tätigkeiten zur Zielerreichung und der wahrgenommenen Arbeitsumgebung, stehen. Strauss interessiert sich allerdings ausschließlich für die dynamischen Aspekte der Interaktion zwischen Individuen, nicht aber für die Ausgangspunkte und Ergebnisse der zugrunde liegenden „thought processes“.\footnote{\emph{„I use the gerund 'ing' after 'symbol' [bei der Beschreibung von 'symbolizing', Anm.] to signify that my principal interest is, again, in interaction rather than its products, for symbols are precipitates of interaction“}\citep[][S. 149]{Strauss93}}  Wie bereits oben erwähnt sind aber die Repräsentationen, auf den „thought processes“ beruhen und operieren, für die Unterstützung von „Articulation Work“ von Interesse. Die kognitions-wissenschaftlichen Ansätze zu Schemata (\citep{Rumelhart78} \citep[vgl. nach ][]{Hanke06}) und mentalen Modellen (\citep[vgl. ][]{Seel91}) sind ein Erklärungsansatz für diese Lücke.

% section fazit (end)
% chapter articulation_work (end)

