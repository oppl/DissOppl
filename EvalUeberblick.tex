\chapter{Überblick über die empirische Untersuchung}
\label{cha:eval_ueberblick}

In diesem Kapitel wird ein Überblick über die in dieser Arbeit durchgeführte empirische Untersuchung gegeben. Dabei wird auf die einzelnen zu untersuchenden Aspekte, deren theoretische Grundlagen und die Durchführung der Untersuchung gegeben.

Die im Rahmen der empirischen Evaluierung zu untersuchenden Aspekte sind Gegenstand des ersten Abschnitts. Neben einer wiederholenden grundlegenden Betrachtung werden hier die jeweiligen Untersuchungsfragen festgelegt. Eine nähere Betrachtung der einzelnen Aspekte, die Festlegung der Methodik und deren Operationalisierung im Rahmen des konkreten Untersuchungsdesigns erfolgt im Rahmen der übrigen Kapitel in diesem Teil der Arbeit.

Im zweiten Abschnitt wird ein Überblick über das globale Untersuchungsdesign gegeben. Auf Basis der zu evaluierenden Aspekte werden die konkret durchgeführten Teile der Evaluation (im Folgenden: "Evaluierungsblöcke") beschrieben und den Aspekten zugeordnet. Diese Evaluierungsblöcke werden überblicksweise hinsichtlich der intendierten Ziele, der Aufgabenstellung und der jeweiligen Anzahl der Teilnehmer beschrieben. Die Beschreibung bildet die Grundlage für die Beschreibung der Evaluierung der zu prüfenden Aspekte in den folgenden Kapiteln.

\section{Untersuchungsaspekte} % (fold)
\label{sec:untersuchungsaspekte}

Ziel dieser Arbeit ist die Unterstüztung von expliziter Articulation Work. Eine Möglichkeit, explizite Articulation Work zu unterstützen, ist die Externalisierung und Abstimmung der mentalen Modelle über den betreffenden Arbeitsvorgang, die den Handlungen der beteiligten Personen zugrunde liegen (siehe Kapitel XY). Die Externalisierung mentaler Modelle ist mittels unterschiedlicher Methoden möglich, wobei sich Ansätze, die auf der Abbildung mentaler Modelle in diagrammatischen Strukturen basieren, als gut geeignet erwiesen haben (siehe Kapitel XY). Zwei derartige Methoden sind Concept Mapping und Strukturlegetechniken, die beide Vor- und Nachteil hinsichtlich des Einsatzes in kollaborativen Szenarien zeigen (siehe Kapitel XY). In dieser Arbeit wird deshalb versucht, die Vorteile der beiden Ansätze methodisch zu vereinigen und zur Vermeidung der Nachteile durch ein Tabletop Interface zu unterstützen (siehe Kapitel XY).

Anhand dieser Argumentationskette zeigt sich, dass zwischen der Zielformulierung und dem konkreten Werkzeug zur Zielerreichung einige argumentative Schritte liegen, die vorerst lediglich (aus der Literatur begründete) Annahmen darstellen. Im Zuge der Evaluation der Ergebnisse dieser Arbeit müssen nun diese Schritte einzeln betrachtet werden und hinsichtlich der jeweiligen Zielerreichung überprüft werden. Untersuchungsgegenstand ist dabei jeweils das erstellte Werkzeug, die betrachteten Aspekte unterscheiden sich je nach Argumentationsschritt. Die Untersuchungsfragen, die die Argumentationsschritte abdecken sind:
\begin{itemize}
 \item Unterstützen Werkzeug und Methode Articulation Work?
 \item Erlauben Werkzeug und Methode die Abbildung semantisch offener diagrammatischer Modelle?
 \item Sind das Werkzeug und dessen Komponenten verständlich und wie intendiert einsetzbar?
\end{itemize}

Diese Fragen decken die Aspekte der oben beschriebene Argumentationskette ab, die Detaillierung der Fragestellungen ist in den folgenden Abschnitten beschrieben. Die Beschreibung der zu prüfenden Hypothesen sowie die Operationalisierung der Untersuchungsfragen erfolgt in den Kapitel XY bis XY.

\subsection{Evaluierung des Werkzeugs}
\label{sub:eval_werkzeug}

Die Evaluierung des Werkzeugs an sich beschäftigt sich mit der Beantwortung der dritten Untersuchungsfrage. Diese zielt auf die Verständlichkeit des Werkzeugs im weiteren Sinn ab. Unter Verständlichkeit im weiteren Sinn ist hier zu verstehen, dass einerseits geprüft werden muss, ob die Bedeutung und grundlegende Verwendung der Komponenten des Werkzeugs von Benutzern erfasst und verstanden werden und ob andererseits die Interaktionsabläufe, die zur Auslösung bzw. Abwicklung einer Funktion des Werkzeugs führen, für Benutzer verständlich und nachvollziehbar sind. Mögliche Metriken sind hier die Faktoren zur Evaluierung interaktiver Systeme nach Shneiderman (REF).

Neben der quantitativen Bewertung anhand dieser Metriken ist bei der Untersuchung dieses Aspektes vor allem auch das qualitative Feedback der Benutzer notwendig, um Ansatzpunkte zur Verbesserung der Verwendbarkeit des Werkzeugs zu erhalten. Diese Anregungen können im Sinne eines iterativen Designprozesses umgesetzt und deren Auswirkungen erneut einer Evaluierung unterzogen werden. Neben der Erhebung dieser zusätzlich funktionalen Anforderungen für einen iterativen Designprozess sind in diesem Zusammenhang auch Hinweise hinsichtlich nicht-funktionaler Aspekte des Systems zu berücksichtigen, die der Verwendbarkeit negativ beeinflussen bzw. auch unkritisch sein können.

Die Verwendbarkeit des Werkzeugs kann nicht entkoppelt von der Anwendungsdomäne betrachtet werden, muss also im Kontext der Aufgabe, für die es eingesetzt wird, gesehen werden. Das Werkzeug ist zwar grundsätzlich für die Repräsentation beliebiger diagrammatischer Modelle ausgelegt, eignet sich aufgrund der unterschiedlichen Anforderungen jedoch nicht gleich gut für alle möglichen Anwendungsfälle (so sind z.B. ausschließlich Verbindungen mit zwei Endpunkten erstellbar, Verbindung mit mehr Endpunkten werden nicht unterstützt). Die Prüfung der Verwendbarkeit des Werkzeugs kann hier fokussiert auf die in dieser Arbeit verfolgten Anwendungsfälle durchgeführt werden, die im Bereich der konzeptuellen Netze (im Wesentlichen Varianten von Concept Maps) und im Bereich der Abbildung von Arbeitsvorgängen (im Wesentlichen kausale Zusammenhänge mit Kontextinformation) zu finden sind. Die Unterstützung anderer Anwendungsfälle ist möglich und unter Umständen erstrebenswert, stellt jedoch kein Beurteilungskriterium dar.

\subsection{Evaluierung der Modellrepräsentationen}
\label{sub:eval_modell}

Der zweite zu evaluierende Aspekt sind die mit dem Werkzeug erstellten Modelle, die als Mittel zur Durchführung expliziter Articulation Work dienen. Eine wesentliche Eigenschaft, die Modelle dabei aufweisen müssen, ist die Adäquatheit der Modellierungssprache hinsichtlich der durch die Benutzer zu repräsentierenden Information. Diese Eigenschaft wird in der vorliegenden Arbeit durch den in Kapitel XY beschriebenen Ansatz der semantischen Offenheit abgedeckt, der jedoch vor allem hinsichtlich der intersubjektiven Verständlichkeit der Modelle und deren Eindeutigkeit nicht nur Vorteile bringt. Grundlegende ist in dieser Phase zu evaluieren, ob die erstellten Modelle den im Rahmen des Einsatzes zur Unterstützung von Articulation Work intendierten Zweck erfüllen. Dabei sind sowohl das Modell als auch das (hier von den Benutzern festgelegte) Metamodell zu betrachten. Anhaltspunkte zur Identifikation der zu evaluierenden Objekte sowie zum Vorgehen bieten hier der Ansatz der "Interactive Process Models" \citep{Jorgensen04} und die "Grundsätze der ordnungsgemäßen Modellierung" \citep{Becker00} sowie von diesen Arbeiten abgeleitete Ansätze.

Die eben beschriebenen Ansatzpunkte erlauben eine Evaluierung der erstellten Modelle hinsichtlich der Abbildbarkeit der Kernaspekte von "Articulation Work" im engeren Sinne (Strauss' "salient dimensions": \emph{"who, where, when, what and how"} \citep{Fjuk97}), decken also im Wesentlichen eine an organisationalen Abläufen orientierten Sicht auf Modelle ab. Im Sinne der Offenheit der Abbildung müssen aber auch Modelle berücksichtigt werden, die nicht diese "salient dimensions" zur Grundlage haben, also "Concept Maps" \citep{Novak06} im allgemeinen Sinn sind und damit die Abbildung mentaler Modelle nicht nur über unmittelbare Arbeitsaspekte sondern über beliebige Sachverhalte erlauben \citep{Ifenthaler06}. Dabei sind Metriken notwendig, die die erstellten Modelle selbst betrachten und deren Eigenschaften und Verwendung beim Concept Mapping bzw. im Rahmen von Strukturlegetechniken berücksichtigen.

Wie bereits im letzten Abschnitt angeführt, ist auch bei diesem Aspekt der Evaluierung der in dieser Arbeit verfolgte Anwendungszweck des Werkzeugs (bzw. hier: der Modelle) zu berücksichtigen. Dies ist insofern ein einschränkender Faktor, als dass hier Modelle lediglich im Kontext der Externalisierung mentaler Modelle und zur Unterstützung von Articulation Work berücksichtigt werden. Das Werkzeug selbst erlaubt auch die Erstellung von Modellen zu anderen Anwendungszwecken, die jedoch hier nicht weiter berücksichtigt werden.  

\subsection{Evaluierung der Articulation Work}
\label{sub:eval_articulation_work}

Letztendlich muss auch die durchgeführte Articulation Work selbst beurteilt werden. In der Literatur zum Thema "Articulation Work" werden zumeist lediglich das Phänomen "Articulation Work" und dessen konkrete Ausprägungen beschrieben (siehe Kapitel XY), Ansätze zur Bewertung des Erfolgs von Articulation Work sind jedoch selten zu finden. Aus der Verschränkung zwischen Articulation Work und Production Work, also jenem Anteil der Arbeit, der unmittelbar der Zielerreichung dient, die von mehreren Autoren, unter anderem \citep{Fujimura87} und \citep{Strauss93}, erwähnt wird, lassen sich jedoch Ansatzpunkte ableiten.

Articulation Work tritt immer dann auf, wenn eine Zielerreichung in der Production Work aufgrund von Unklarheiten oder Problemen zwischen den beteiligen Individuen nicht möglich ist. Ein erfolgreicher Abschuss der Production Work bei am Beginn oder während der Arbeit bestehenden Unklarheiten weißt also unter Umständen auf erfolgreich durchgeführte Articulation Work hin. Articulation Work manifestiert sich im Arbeitsprozess auf unterschiedliche Arten, so dass bei der Evaluierung hinsichtlich der Auswirkungen des Werkzeugs diese von den übrigen Einflussfaktoren (also auf anderen Wegen durchgeführte Articulation Work) getrennt werden muss. Dazu ist eine Betrachtung des gesamten Arbeitsablaufs unter Berücksichtigung von Production und Articulation Work notwendig. Metriken, die bei der Bewertung des Erfolgs von Articulation Work zu berücksichtigen sind, sind also einerseits im Ergebnis des Arbeitsprozesses, andererseits auch im Arbeitsprozess selbst zu finden.

Ein zweiter Ansatzpunkt zur Bewertung des Erfolgs von Articulation Work liegt in den Aussagen von \citet{Strauss93} hinsichtlich der wahrgenommenen "Problematik" einer Arbeitssituation, die Articulation Work notwendig macht. Diese Wahrnehmung ist individueller Natur, d.h. Articulation Work ist dann notwendig, wenn zumindest einer am Arbeitsablauf beteiligten Person Aspekte der Arbeit unklar sind oder problematisch erscheinen. Im Gegenzug ist keine Articulation Work notwendig bzw. diese abgeschlossen, wenn alle beteiligten Personen die Situation als unproblematisch empfinden bzw. mit den im Rahmen der (expliziten) Articulation Work erzielten Ergebnissen zufrieden sind. Hier liegt der Ansatzpunkt für eine Evaluierung des Erfolgs der durchgeführten Articulation Work, der auf diese auf Basis der individuellen Wahrnehmungen der beteiligten Personen beurteilt.
% section untersuchungsaspekte (end)

\section{Globales Untersuchungsdesign}
\label{sec:globales_untersuchungsdesign}

Die oben beschriebenen Aspekte müssen nun im Rahmen einer empirischen Untersuchung getestet werden. Während die detaillierten Untersuchungsdesigns in den folgenden Kapiteln, die sich jeweils einem der drei zu evaluierenden Aspekte widmen, beschrieben werden, wird an dieser Stelle ein Überblick über das globale Untersuchungsdesign und die im Rahmen der Evaluierung durchgeführten Anwendungen des Werkzeugs gegeben.

Im ursprünglichen globalen Untersuchungsdesign war vorgesehen, jedem der zu untersuchenden Aspekte einen Block an Anwendungen des Werkzeugs mit einer auf den jeweiligen Aspekt abgestimmten Aufgabenstellung zuzuordnen. Nach Durchführung der ersten beiden Blöcke wurde offensichtlich, dass sich während der Evaluierung zusätzlich Hypothesen zu einem Aspekt bildeten, die -- um sie in die Evaluierung einfließen zu lassen -- in einem späteren Block getestet werden mussten. Außerdem wurde offensichtlich, dass vor allem zur Evaluierung des Werkzeugs in allen Blöcken Verbesserungspotential identifiziert werden konnte bzw. Anregungen der Anwender rückgemeldet wurden, die zum Teil im Rahmen des iterativen Entwicklungsprozesses in das Werkzeug einflossen und entsprechend in einem späteren Block erneut getestet werden musste. 

Letztendlich wurden die Blöcke, sofern die jeweilige Aufgabenstellung geeignet war, für die Evaluierung mehrerer bzw. aller Aspekte herangezogen. Bei der nun folgenden Beschreibung der Anwendungs-Blöcke wird deshalb jeweils angegeben und begründet, inwieweit diese in die Evaluierung welcher Blöcke einfließen. Ein Überblick über das globale Untersuchungsdesign mit einer erneuten, überblicksweisen Zuordnung zwischen den zu evaluierenden Aspekten und den Anwendungsblöcken wird in Abschnitt \ref{sec:eval_ueberblick_zusammenfassung} gegeben.

\subsection{Block 1: Technische Evaluierung}
\label{sub:eval_1}

Die Intention von Block 1 war die grundlegende Verständlichkeit und Verwendbarkeit des Werkzeugs zu klären. Fokus dieses Blocks an Anwendungen des Werkzeugs war also die Untersuchung der Eigenschaften des Werkzeugs selbst. Zusätzlich sollte explorativ Hypothesen für die übrigen zu evaluierenden Aspekte gebildet werden.

\subsubsection{Kontext} % (fold)
\label{ssub:1_kontext}

Die Untersuchung wurde im Rahmen einer Diplomarbeit durchgeführt (REF Bohninger), wobei die Untersuchungen in keinen einheitlichen realen Arbeitskontext eingebettet waren. Allerdings war die Aufgabenstellung so formuliert, dass die erstellten Modelle aus den Arbeitskontexten der jeweiligen Teilnehmer stammten.

% subsubsection kontext (end)

\subsubsection{Aufgabenstellung und Ablauf} % (fold)
\label{ssub:1_aufgabenstellung}

Den modellierenden Teilnehmern wurde mitgeteilt, dass sie einen Aspekt aus ihrem täglichen Arbeits- oder Privatleben abbilden sollten, der regelmäßig auftritt oder bereits mehrmals für Probleme sorgte. Die bewusste Offenheit der Aufgabenstellung sollte dabei bewirken, dass sich die Teilnehmer nicht zu sehr auf den abzubildenden Sachverhalt, sondern eher auf den Abbildungsprozess selbst fokussierten.

Nur die Hälfte der Teilnehmer erstellte Modelle. Die zweite Hälfte wurde zur Überprüfung der Verständlichkeit der Modelle sowie der Verwendung des Werkzeugs zur kollaborativen Modellierung herangezogen. Dazu wurde nach Abschluss einer Modellbildung jeweils ein nicht modellierender Teilnehmer an die Modellierungsoberfläche gebeten und gebeten, die Abbildung zu interpretieren. Die Beurteilung der Adäquatheit dieser Interpretation erfolgte durch den modellierenden Teilnehmer.

In einer dritten Phase wurden beide Teilnehmer aufgefordert, dass Modell gemeinsam zu reflektieren und gegebenenfalls zu verändern, um es den Ergebnissen der Reflexion anzupassen. In dieser Phase war das vorrangige Ziel, die Verwendung des Werkzeugs bei der Veränderung von Modellen und dessen kollaborativer Anwendung zu testen. 

% subsubsection aufgabenstellung (end)

\subsubsection{Anwendungen und Teilnehmer} % (fold)
\label{ssub:1_teilnehmer}

Insgesamt wurden neun Anwendungen des Werkzeug wie oben beschrieben durchgeführt. Zusätzlich wurde das Untersuchungsdesign im Rahmen von drei Anwendungen getestet (Pretest), woraus hinsichtlich der technischen Eigenschaften des Werkzeugs ebenfalls bereits Erkenntnisse gewonnen werden konnten. Insgesamt nahmen also 24 Personen an diesem Block von Anwendungen teil, 6 davon in der Pretest-Phase.

Die Teilnehmer stammten aus unterschiedlichen beruflichen Hintergründen (DETAILS) und unterschieden sich auch in Art der höchsten abgeschlossenen Ausbildung (DETAILS). Die Altersspanne lag zwischen XY und XY Jahren, XY Teilnehmer waren weiblich, XY männlich.

Die Modellierungsphasen dauerten im Schnitt XY Minuten, die kürzeste Modellbildung dauerte XY Minuten, die längste XY Minuten. Die Interpretations- und Reflexionsphasen (nicht separat aufschlüsselbar, da zum Großteil ineinander übergehend)  dauerten im Schnitt XY Minuten. 

% subsubsection teilnehmer (end)

\subsubsection{Verwendung der Ergebnisse} % (fold)
\label{ssub:1_verwendung_der_ergebnisse}

Die Ergebnisse dieses Blocks flossen in die Evaluierung des Werkzeugs und in die Hypothesenbildung hinsichtlich der erstellten Modelle ein. Für die Evaluierung der Modelle konnten erste Erkenntnisse hinsichtlich der Verständlichkeit der mittels offener Semantik gewonnen werden. Keine Ergebnisse brachte dieser Block für die Evaluierung der durchgeführten Articulation Work.

% subsubsection verwendung_der_ergebnisse (end)

\subsection{Block 2: Aushandlung von Zusammenarbeit 1}
\label{sub:eval_2}

\subsection{Block 3: Concept Mapping 1}
\label{sub:eval_3}

\subsection{Block 4: Aushandlung von Zusammenarbeit 2}
\label{sub:eval_4}

\subsection{Block 5: Concept Mapping 2}
\label{sub:eval_5}

\section{Zusammenfassung}
\label{sec:eval_ueberblick_zusammenfassung}

% section globales_untersuchungsdesign (end)
% chapter eval_ueberblick (end)