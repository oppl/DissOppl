\chapter{Überblick über die empirische Untersuchung}
\label{cha:eval_ueberblick}

In diesem Kapitel wird ein Überblick über die in dieser Arbeit durchgeführte empirische Untersuchung gegeben. Dabei wird auf die einzelnen zu untersuchenden Aspekte, deren theoretische Grundlagen und die Durchführung der Untersuchung gegeben.

Die im Rahmen der empirischen Evaluierung zu untersuchenden Aspekte sind Gegenstand des ersten Abschnitts. Neben einer wiederholenden grundlegenden Betrachtung werden hier die jeweiligen Untersuchungsfragen festgelegt. Eine nähere Betrachtung der einzelnen Aspekte, die Festlegung der Methodik und deren Operationalisierung im Rahmen des konkreten Untersuchungsdesigns erfolgt im Rahmen der übrigen Kapitel in diesem Teil der Arbeit.

Im zweiten Abschnitt wird ein Überblick über das globale Untersuchungsdesign gegeben. Auf Basis der zu evaluierenden Aspekte werden die konkret durchgeführten Teile der Evaluation (im Folgenden: "Evaluierungsblöcke") beschrieben und den Aspekten zugeordnet. Diese Evaluierungsblöcke werden überblicksweise hinsichtlich der intendierten Ziele, der Aufgabenstellung und der jeweiligen Anzahl der Teilnehmer beschrieben. Die Beschreibung bildet die Grundlage für die Beschreibung der Evaluierung der zu prüfenden Aspekte in den folgenden Kapiteln.

\section{Untersuchungsaspekte} % (fold)
\label{sec:untersuchungsaspekte}

Ziel dieser Arbeit ist die Unterstüztung von expliziter Articulation Work. Eine Möglichkeit, explizite Articulation Work zu unterstützen, ist die Externalisierung und Abstimmung der mentalen Modelle über den betreffenden Arbeitsvorgang, die den Handlungen der beteiligten Personen zugrunde liegen (siehe Kapitel XY). Die Externalisierung mentaler Modelle ist mittels unterschiedlicher Methoden möglich, wobei sich Ansätze, die auf der Abbildung mentaler Modelle in diagrammatischen Strukturen basieren, als gut geeignet erwiesen haben (siehe Kapitel XY). Zwei derartige Methoden sind Concept Mapping und Strukturlegetechniken, die beide Vor- und Nachteil hinsichtlich des Einsatzes in kollaborativen Szenarien zeigen (siehe Kapitel XY). In dieser Arbeit wird deshalb versucht, die Vorteile der beiden Ansätze methodisch zu vereinigen und zur Vermeidung der Nachteile durch ein Tabletop Interface zu unterstützen (siehe Kapitel XY).

Anhand dieser Argumentationskette zeigt sich, dass zwischen der Zielformulierung und dem konkreten Werkzeug zur Zielerreichung einige argumentative Schritte liegen, die vorerst lediglich (aus der Literatur begründete) Annahmen darstellen. Im Zuge der Evaluation der Ergebnisse dieser Arbeit müssen nun diese Schritte einzeln betrachtet werden und hinsichtlich der jeweiligen Zielerreichung überprüft werden. Untersuchungsgegenstand ist dabei jeweils das erstellte Werkzeug, die betrachteten Aspekte unterscheiden sich je nach Argumentationsschritt. Die Untersuchungsfragen, die die Argumentationsschritte abdecken sind:
\begin{itemize}
 \item Unterstützen Werkzeug und Methode Articulation Work? (Aspekt: Werkzeug)
 \item Erlauben Werkzeug und Methode die Abbildung semantisch offener diagrammatischer Modelle? (Aspekt: Modell)
 \item Sind das Werkzeug und dessen Komponenten verständlich und wie intendiert einsetzbar? (Aspekt: Articulation Work)
\end{itemize}

Diese Fragen decken die Aspekte der oben beschriebene Argumentationskette ab, die Detaillierung der Fragestellungen ist in den folgenden Abschnitten beschrieben. Die Beschreibung der zu prüfenden Hypothesen sowie die Operationalisierung der Untersuchungsfragen erfolgt in den Kapitel XY bis XY.

\subsection{Evaluierung des Werkzeugs}
\label{sub:eval_werkzeug}

Die Evaluierung des Werkzeugs an sich beschäftigt sich mit der Beantwortung der dritten Untersuchungsfrage. Diese zielt auf die Verständlichkeit des Werkzeugs im weiteren Sinn ab. Unter Verständlichkeit im weiteren Sinn ist hier zu verstehen, dass einerseits geprüft werden muss, ob die Bedeutung und grundlegende Verwendung der Komponenten des Werkzeugs von Benutzern erfasst und verstanden werden und ob andererseits die Interaktionsabläufe, die zur Auslösung bzw. Abwicklung einer Funktion des Werkzeugs führen, für Benutzer verständlich und nachvollziehbar sind. Mögliche Metriken sind hier die Faktoren zur Evaluierung interaktiver Systeme nach Shneiderman (REF).

Neben der quantitativen Bewertung anhand dieser Metriken ist bei der Untersuchung dieses Aspektes vor allem auch das qualitative Feedback der Benutzer notwendig, um Ansatzpunkte zur Verbesserung der Verwendbarkeit des Werkzeugs zu erhalten. Diese Anregungen können im Sinne eines iterativen Designprozesses umgesetzt und deren Auswirkungen erneut einer Evaluierung unterzogen werden. Neben der Erhebung dieser zusätzlich funktionalen Anforderungen für einen iterativen Designprozess sind in diesem Zusammenhang auch Hinweise hinsichtlich nicht-funktionaler Aspekte des Systems zu berücksichtigen, die der Verwendbarkeit negativ beeinflussen bzw. auch unkritisch sein können.

Die Verwendbarkeit des Werkzeugs kann nicht entkoppelt von der Anwendungsdomäne betrachtet werden, muss also im Kontext der Aufgabe, für die es eingesetzt wird, gesehen werden. Das Werkzeug ist zwar grundsätzlich für die Repräsentation beliebiger diagrammatischer Modelle ausgelegt, eignet sich aufgrund der unterschiedlichen Anforderungen jedoch nicht gleich gut für alle möglichen Anwendungsfälle (so sind z.B. ausschließlich Verbindungen mit zwei Endpunkten erstellbar, Verbindung mit mehr Endpunkten werden nicht unterstützt). Die Prüfung der Verwendbarkeit des Werkzeugs kann hier fokussiert auf die in dieser Arbeit verfolgten Anwendungsfälle durchgeführt werden, die im Bereich der konzeptuellen Netze (im Wesentlichen Varianten von Concept Maps) und im Bereich der Abbildung von Arbeitsvorgängen (im Wesentlichen kausale Zusammenhänge mit Kontextinformation) zu finden sind. Die Unterstützung anderer Anwendungsfälle ist möglich und unter Umständen erstrebenswert, stellt jedoch kein Beurteilungskriterium dar.

\subsection{Evaluierung der Modellrepräsentationen}
\label{sub:eval_modell}

Der zweite zu evaluierende Aspekt sind die mit dem Werkzeug erstellten Modelle, die als Mittel zur Durchführung expliziter Articulation Work dienen. Eine wesentliche Eigenschaft, die Modelle dabei aufweisen müssen, ist die Adäquatheit der Modellierungssprache hinsichtlich der durch die Benutzer zu repräsentierenden Information. Diese Eigenschaft wird in der vorliegenden Arbeit durch den in Kapitel XY beschriebenen Ansatz der semantischen Offenheit abgedeckt, der jedoch vor allem hinsichtlich der intersubjektiven Verständlichkeit der Modelle und deren Eindeutigkeit nicht nur Vorteile bringt. Grundlegende ist in dieser Phase zu evaluieren, ob die erstellten Modelle den im Rahmen des Einsatzes zur Unterstützung von Articulation Work intendierten Zweck erfüllen. Dabei sind sowohl das Modell als auch das (hier von den Benutzern festgelegte) Metamodell zu betrachten. Anhaltspunkte zur Identifikation der zu evaluierenden Objekte sowie zum Vorgehen bieten hier der Ansatz der "Interactive Process Models" \citep{Jorgensen04} und die "Grundsätze der ordnungsgemäßen Modellierung" \citep{Becker00} sowie von diesen Arbeiten abgeleitete Ansätze.

Die eben beschriebenen Ansatzpunkte erlauben eine Evaluierung der erstellten Modelle hinsichtlich der Abbildbarkeit der Kernaspekte von "Articulation Work" im engeren Sinne (Strauss' "salient dimensions": \emph{"who, where, when, what and how"} \citep{Fjuk97}), decken also im Wesentlichen eine an organisationalen Abläufen orientierten Sicht auf Modelle ab. Im Sinne der Offenheit der Abbildung müssen aber auch Modelle berücksichtigt werden, die nicht diese "salient dimensions" zur Grundlage haben, also "Concept Maps" \citep{Novak06} im allgemeinen Sinn sind und damit die Abbildung mentaler Modelle nicht nur über unmittelbare Arbeitsaspekte sondern über beliebige Sachverhalte erlauben \citep{Ifenthaler06}. Dabei sind Metriken notwendig, die die erstellten Modelle selbst betrachten und deren Eigenschaften und Verwendung beim Concept Mapping bzw. im Rahmen von Strukturlegetechniken berücksichtigen.

Wie bereits im letzten Abschnitt angeführt, ist auch bei diesem Aspekt der Evaluierung der in dieser Arbeit verfolgte Anwendungszweck des Werkzeugs (bzw. hier: der Modelle) zu berücksichtigen. Dies ist insofern ein einschränkender Faktor, als dass hier Modelle lediglich im Kontext der Externalisierung mentaler Modelle und zur Unterstützung von Articulation Work berücksichtigt werden. Das Werkzeug selbst erlaubt auch die Erstellung von Modellen zu anderen Anwendungszwecken, die jedoch hier nicht weiter berücksichtigt werden.  

\subsection{Evaluierung der Articulation Work}
\label{sub:eval_articulation_work}

Letztendlich muss auch die durchgeführte Articulation Work selbst beurteilt werden. In der Literatur zum Thema "Articulation Work" werden zumeist lediglich das Phänomen "Articulation Work" und dessen konkrete Ausprägungen beschrieben (siehe Kapitel XY), Ansätze zur Bewertung des Erfolgs von Articulation Work sind jedoch selten zu finden. Aus der Verschränkung zwischen Articulation Work und Production Work, also jenem Anteil der Arbeit, der unmittelbar der Zielerreichung dient, die von mehreren Autoren, unter anderem \citep{Fujimura87} und \citep{Strauss93}, erwähnt wird, lassen sich jedoch Ansatzpunkte ableiten.

Articulation Work tritt immer dann auf, wenn eine Zielerreichung in der Production Work aufgrund von Unklarheiten oder Problemen zwischen den beteiligen Individuen nicht möglich ist. Ein erfolgreicher Abschuss der Production Work bei am Beginn oder während der Arbeit bestehenden Unklarheiten weißt also unter Umständen auf erfolgreich durchgeführte Articulation Work hin. Articulation Work manifestiert sich im Arbeitsprozess auf unterschiedliche Arten, so dass bei der Evaluierung hinsichtlich der Auswirkungen des Werkzeugs diese von den übrigen Einflussfaktoren (also auf anderen Wegen durchgeführte Articulation Work) getrennt werden muss. Dazu ist eine Betrachtung des gesamten Arbeitsablaufs unter Berücksichtigung von Production und Articulation Work notwendig. Metriken, die bei der Bewertung des Erfolgs von Articulation Work zu berücksichtigen sind, sind also einerseits im Ergebnis des Arbeitsprozesses, andererseits auch im Arbeitsprozess selbst zu finden.

Ein zweiter Ansatzpunkt zur Bewertung des Erfolgs von Articulation Work liegt in den Aussagen von \citet{Strauss93} hinsichtlich der wahrgenommenen "Problematik" einer Arbeitssituation, die Articulation Work notwendig macht. Diese Wahrnehmung ist individueller Natur, d.h. Articulation Work ist dann notwendig, wenn zumindest einer am Arbeitsablauf beteiligten Person Aspekte der Arbeit unklar sind oder problematisch erscheinen. Im Gegenzug ist keine Articulation Work notwendig bzw. diese abgeschlossen, wenn alle beteiligten Personen die Situation als unproblematisch empfinden bzw. mit den im Rahmen der (expliziten) Articulation Work erzielten Ergebnissen zufrieden sind. Hier liegt der Ansatzpunkt für eine Evaluierung des Erfolgs der durchgeführten Articulation Work, der auf diese auf Basis der individuellen Wahrnehmungen der beteiligten Personen beurteilt.
% section untersuchungsaspekte (end)

\section{Globales Untersuchungsdesign}
\label{sec:globales_untersuchungsdesign}

Die oben beschriebenen Aspekte müssen nun im Rahmen einer empirischen Untersuchung getestet werden. Während die detaillierten Untersuchungsdesigns in den folgenden Kapiteln, die sich jeweils einem der drei zu evaluierenden Aspekte widmen, beschrieben werden, wird an dieser Stelle ein Überblick über das globale Untersuchungsdesign und die im Rahmen der Evaluierung durchgeführten Anwendungen des Werkzeugs gegeben.

Im ursprünglichen globalen Untersuchungsdesign war vorgesehen, jedem der zu untersuchenden Aspekte einen Block an Anwendungen des Werkzeugs mit einer auf den jeweiligen Aspekt abgestimmten Aufgabenstellung zuzuordnen. Nach Durchführung der ersten beiden Blöcke wurde offensichtlich, dass sich während der Evaluierung zusätzlich Hypothesen zu einem Aspekt bildeten, die -- um sie in die Evaluierung einfließen zu lassen -- in einem späteren Block getestet werden mussten. Außerdem wurde offensichtlich, dass vor allem zur Evaluierung des Werkzeugs in allen Blöcken Verbesserungspotential identifiziert werden konnte bzw. Anregungen der Anwender rückgemeldet wurden, die zum Teil im Rahmen des iterativen Entwicklungsprozesses in das Werkzeug einflossen und entsprechend in einem späteren Block erneut getestet werden musste. 

Letztendlich wurden die Blöcke, sofern die jeweilige Aufgabenstellung geeignet war, für die Evaluierung mehrerer bzw. aller Aspekte herangezogen. Bei der nun folgenden Beschreibung der Anwendungs-Blöcke wird deshalb jeweils angegeben und begründet, inwieweit diese in die Evaluierung welcher Blöcke einfließen. Ein Überblick über das globale Untersuchungsdesign mit einer erneuten, überblicksweisen Zuordnung zwischen den zu evaluierenden Aspekten und den Anwendungsblöcken wird in Abschnitt \ref{sec:eval_ueberblick_zusammenfassung} gegeben.

\subsection{Block 1: Technische Evaluierung}
\label{sub:eval_1}

Die Intention von Block 1 war die grundlegende Verständlichkeit und Verwendbarkeit des Werkzeugs zu klären. Fokus dieses Blocks an Anwendungen des Werkzeugs war also die Untersuchung der Eigenschaften des Werkzeugs selbst. Zusätzlich sollte explorativ Hypothesen für die übrigen zu evaluierenden Aspekte gebildet werden.

\subsubsection{Kontext} % (fold)
\label{ssub:1_kontext}

Die Untersuchung wurde im Rahmen einer Diplomarbeit durchgeführt (REF Bohninger), wobei die Untersuchungen in keinen einheitlichen realen Arbeitskontext eingebettet waren. Allerdings war die Aufgabenstellung so formuliert, dass die erstellten Modelle aus den Arbeitskontexten der jeweiligen Teilnehmer stammten.

% subsubsection kontext (end)

\subsubsection{Aufgabenstellung und Ablauf} % (fold)
\label{ssub:1_aufgabenstellung}

Den modellierenden Teilnehmern wurde mitgeteilt, dass sie einen Aspekt aus ihrem täglichen Arbeits- oder Privatleben abbilden sollten, der regelmäßig auftritt oder bereits mehrmals für Probleme sorgte. Die bewusste Offenheit der Aufgabenstellung sollte dabei bewirken, dass sich die Teilnehmer nicht zu sehr auf den abzubildenden Sachverhalt, sondern eher auf den Abbildungsprozess selbst fokussierten.

Nur die Hälfte der Teilnehmer erstellte Modelle. Die zweite Hälfte wurde zur Überprüfung der Verständlichkeit der Modelle sowie der Verwendung des Werkzeugs zur kollaborativen Modellierung herangezogen. Dazu wurde nach Abschluss einer Modellbildung jeweils ein nicht modellierender Teilnehmer an die Modellierungsoberfläche gebeten und gebeten, die Abbildung zu interpretieren. Die Beurteilung der Adäquatheit dieser Interpretation erfolgte durch den modellierenden Teilnehmer.

In einer dritten Phase wurden beide Teilnehmer aufgefordert, dass Modell gemeinsam zu reflektieren und gegebenenfalls zu verändern, um es den Ergebnissen der Reflexion anzupassen. In dieser Phase war das vorrangige Ziel, die Verwendung des Werkzeugs bei der Veränderung von Modellen und dessen kollaborativer Anwendung zu testen. 

% subsubsection aufgabenstellung (end)

\subsubsection{Anwendungen und Teilnehmer} % (fold)
\label{ssub:1_teilnehmer}

Insgesamt wurden neun Anwendungen des Werkzeug wie oben beschrieben durchgeführt. Zusätzlich wurde das Untersuchungsdesign im Rahmen von drei Anwendungen getestet (Pretest), woraus hinsichtlich der technischen Eigenschaften des Werkzeugs ebenfalls bereits Erkenntnisse gewonnen werden konnten. Insgesamt nahmen also 24 Personen an diesem Block von Anwendungen teil, 6 davon in der Pretest-Phase.

Die Teilnehmer stammten aus unterschiedlichen beruflichen Hintergründen (DETAILS) und unterschieden sich auch in Art der höchsten abgeschlossenen Ausbildung (DETAILS). Die Altersspanne lag zwischen XY und XY Jahren, XY Teilnehmer waren weiblich, XY männlich.

Die Modellierungsphasen dauerten im Schnitt XY Minuten, die kürzeste Modellbildung dauerte XY Minuten, die längste XY Minuten. Die Interpretations- und Reflexionsphasen (nicht separat aufschlüsselbar, da zum Großteil ineinander übergehend) dauerten im Schnitt XY Minuten. 

% subsubsection teilnehmer (end)

\subsubsection{Verwendung der Ergebnisse} % (fold)
\label{ssub:1_verwendung_der_ergebnisse}

Die Ergebnisse dieses Blocks flossen in die Evaluierung des Werkzeugs und in die Hypothesenbildung hinsichtlich der erstellten Modelle ein. Für die Evaluierung der Modelle konnten erste Erkenntnisse hinsichtlich der Verständlichkeit der mittels offener Semantik gewonnen werden. Keine Ergebnisse brachte dieser Block für die Evaluierung der durchgeführten Articulation Work.

% subsubsection verwendung_der_ergebnisse (end)

\subsection{Block 2: Aushandlung von Zusammenarbeit 1}
\label{sub:eval_2}

In Block 2 lag der Fokus der Evaluation erstmals auf der Unterstützung von Articulation Work. In diesem Rahmnen wurden auch die Verwendbarkeit des Werkzeugs im praktischen Anwendungskontext und die Eigenschaften der erstellten Modelle sowie deren Rolle im Prozess der expliziten Articulation Work untersucht.

\subsubsection{Kontext} % (fold)
\label{ssub:2_kontext}

Block 2 wurde im Rahmen eines Seminars aus Wirtschaftsinformatik mit Studierenden derselben Studienrichtung durchgeführt. Die im Seminar zu erstellenden wissenschaftlichen Arbeiten wurden von den Studierenden in Gruppen zu 2-3 Personen ausgearbeitet. Die Gruppen wurden so gebildet, dass sich die Teilnehmer nicht persönlich kannten oder zumindest nicht bereits in anderen Kontexten zusammengearbeitet hatten. Ziel dieser Maßnahme war die Vermeidung der Verfälschung der Untersuchungsergebnissse durch bereits eingespielte Gruppen (Erfahrungen in Seminaren der Vorjahre zeigen tendentiell schlechtere Ergebnisse bei der Zusammenarbeit von einander nicht persönlich bekannten bzw. nicht eingespielten Teilnehmern).

Im Rahmen des Seminars wurden sechs Forschungsgebiete ausgewählt, die in Zusammenhang mit der Erstellung und Verwendung sozio-technischer Systeme stehen (konkret: Organisationales Lernen, eLearning, \gls{CSCW}, Mentale Modelle, Articulation Work und semantische Contentanreicherung). Den Gruppen wurden jeweils zufällig zwei dieser Themen zugewiesen, die Aufgabe für die wissenschaftliche Arbeit war das Finden und Beschreiben einer möglichen Verknüpfung oder eines möglichen Zusammenhanges zwischen diesen Themen. Dieser Zusammenhang sollte im Zentrum der Seminararbeit stehen und aus beiden Grundlagen-Themen argumentiert sein. Ziel dieser Maßnahme war es, die Seminararbeit so offen wie möglich zu gestalten und eine Themenfindungs bzw. -konkretisierungsprozess in den Ablauf zu integrieren. Außerdem wurde so ein Setting geschaffen, in dem eine strikte Arbeitteilung der Gruppenteilnehmer ohne weitere Zusammenarbeit sich tendentiell stark auf das Ergebnis auswirkt und sich konkret  der fehlenden oder schwachen Verknüpfung der Grundlagen-Themen zeigt.

% subsubsection kontext (end)

\subsubsection{Aufgabenstellung und Ablauf} % (fold)
\label{ssub:2_aufgabenstellung}

Das Werkzeug wurde im Rahmen des Seminars für jede Gruppe zweimal eingesetzt. Die erste Anwendung fand zu Beginn des Seminars nach der Themenzuteilung statt. Die Aufgabe war die Aushandlung der Modalitäten der Zusammenarbeit mit der Zielsetzung, das an der resultierenden wissenschaftlichen Arbeit die Ko-Autorenschaft nicht mehr zu erkennen sein sollte (etwa durch plötzlich wechselnde Schreibstile oder Brüche in der Argumentationskette). Den Teilnehmern wurde das Werkzeug und dessen Funktionen vorgestellt und ohne weitere Vorgaben zur Verfügung gestellt (insbesondere wurden weder Vorgaben hinsichlich der Topologie des zu erstellenden Modells oder der Bedeutung der Modellierungselemente gemacht).

In der zweiten Anwendung wurde der Zusammenarbeitsprozess reflektiert und gegebenenfalls eine Adaption vereinbart. Die zweite Anwendung fand in der Mitte des Semesters nach Abschluss der Literaturrecherche und der Grobkonzeption, aber vor der Erstellung der eigentlichen wissenschaftlichen Arbeit statt. Konkrete Zielsetzung für die Teilnehmer war hier, auf Basis der bisherigen Erfahrungen die weitere Zusammenarbeit zu vereinbaren. Das Werkzeug wurde ohne neuerliche Vorstellung und ohne Vorgaben zur Verfügung gestellt.
% subsubsection aufgabenstellung (end)

\subsubsection{Anwendungen und Teilnehmer} % (fold)
\label{ssub:2_teilnehmer}

Insgesamt nahmen an diesem Block 19 Personen in 9 Gruppen zu 2 bzw. einmalig 3 Personen teil. Jede der Gruppen setzte das Werkzeug zweimal ein, wodurch insgesamt 18 Anwendungen die Grundlage für die Auswertung der Ergebnisse bilden.

Die Teilnehmer waren allesamt Studierende der Wirtschaftsinformatik im zweiten Studienabschnitt. 18 Personen waren männlich, eine weiblich. Vier Personen hatten insofern Erfahrung mit wissenschaftlichen Arbeiten bzw. den konkreten Anforderungen in der betreffenden Lehrveranstaltungen, als dass sie bereits zuvor eine Lehrveranstaltung gleichen Typs besucht hatten.

In der ersten Runde dauerten die Anwendungen durchschnittlich XY Minuten, in der zweiten Runde lediglich XY Minuten.

% subsubsection teilnehmer (end)

\subsubsection{Verwendung der Ergebnisse} % (fold)
\label{ssub:2_verwendung_der_ergebnisse}

Die in diesem Block erhobenen Daten fließen in die Auswertung alle drei zu evaluierenden Aspekte ein. Zur Auswertung hinsichtlich des Erfolgs von Articulation Work liegen neben den Aufnahmen der Modellierungsvorgänge und den erstellten Modellen selbst auch Prozessreflexionen der Teilnehmer über den Erstellungsprozess der Seminararbeiten sowie die Seminararbeit ansich vor. Die Auswirkungen von Articlation Work können also am Ergebnis (im Vergleich zu Ergebnissen auf Lehrveranstaltungen mit identischem Konzept) und am subjektiv wahrgenommenen Verlauf des Erstellungsprozesses der Arbeit bewertet werden.

Hinsichtlich des Auswertung des Modell-Aspektes wird durch diesen Block die Betrachtung von Modellen ermöglicht, die im Kontext der Arbeitsabstimmung erstellt wurden, also im Wesentlichen der Definition von Vorgehen und Schnittstellen dienen. Untersucht werden hier Aufbau und Inhalt der Modelle, wobei besonderes Augenmerk auf der Prozess und Ergebniss der Bedeutungszuweisung zu den Modellelementen liegt.

Im Rahmen der Werkzeug-Evaluation bringt dieser Block die ersten Hinweise auf die Anforderungen an das Werkzeug bei der Verwendung desselben im Rahmen einer realen Aufgabenstellung. Außerdem wurde in diesem Block erstmals ein durchgängig kollaboratives Szenario eingesetzt, bei dem immer mindestens zwei Personen gleichzeitig das Werkzeug verwenden.

% subsubsection verwendung_der_ergebnisse (end)

\subsection{Block 3: Concept Mapping 1}
\label{sub:eval_3}

Der Fokus von Block 3 lag auf der Erstellung von semantisch vernetzten Strukturen im Allgemeinen, wobei das Konzept der Concept Maps als ein etabliertes Werkzeug zur Externalisierung mentaler Modelle eingesetzt wurde. Inhaltlich fokussierte dieser Block nicht auf die Unterstützung von Articulation Work im engeren Sinne, wohl aber auf die Externalisierung und Abstimmung mentaler Modelle, was wiederum wie in Kapitel XY beschrieben ein Mittel zur Unterstützung expliziter Articulation Work ist. Im Zentrum der Aufmerksamkeit steht in diesem Block also die Evaluierung der erstellten Modelle und der Nutzen des Werkzeugs zur Aushandlung einer einheitlichen auf einen gegebenen Sachverhalt.

\subsubsection{Kontext} % (fold)
\label{ssub:3_kontext}

Der dritte Block wurde im Rahmen einer Lehrveranstaltung zur Schulung von Methoden der Prozess- und Kommunikationsmodellierung durchgeführt. Diese Lehrveranstaltung ist Teil der im Curriculum definierten Basiskompetenz Wirtschaftsinformatik und wird von Studierenden im zweiten bis dritten Studiensemester besucht.

Im Rahmen der Lehrveranstaltung wurden drei unterschiedliche Prozessmodellierungssprachen (SeeMe \citep{Herrmann04a}, Subjekt-orientierte Modellierung mittels JPass REF und \gls{EPK}s aus dem ARIS-Konzept \citep{Scheer00}) eingeführt und praktisch an einem durchgängigen Beispiel angewandt. Diese Sprachen unterscheiden sich sowohl im Anwendungsgebiet, in den abgebildeten Aspekten des realen Prozesses sowie in der Darstellungsform des Modells. Ziel der letzten Teilaufgabe, die unter Einsatz des hier vorgestellten Werkzeugs durchgeführt wurde, war bei den Studierenden ein Verständnis für die Unterschiede und Gemeinsamkeiten zwischen diesen Sprachen zu erzeugen und sie in die Lage zu versetzen, für einen gegebenen Anwendungsfall eine adäquate Sprache auszuwählen.  

% subsubsection kontext (end)

\subsubsection{Aufgabenstellung und Ablauf} % (fold)
\label{ssub:3_aufgabenstellung}

Die Aufgabe zur Erstellung der Concept Map umfasste zwei Teile, wobei im zweiten Teil das Tabletop Interface eingesetzt wurde. Die Aufgabenstellung lautete in beiden Teilen, eine Concept Map zu erstellen, die die wesentlich erscheinenden Eigenschaften der vorgestellten Sprachen sowie deren Gemeinsamkeiten und Unterschiede darstellt. In der ersten Phase war diese Aufgabe von den Studierenden individuell zu lösen, wobei die Concept Map auf Papier oder mit Hilfe des Werkzeugs CMapTools\footnote{http://cmap.ihmc.us} \citep{Canas04} am Rechner erstellt werden konnte. 

In der zweiten Phase wurden Gruppen zu je drei Teilnehmern gebildet, die nun ihre individuellen Sichten konsolidieren und jeweils eine gemeinsame Concept Map zur gleichen Aufgabenstellung unter Einsatz des hier vorgestellten Werkzeugs erstellen sollten. Die Gruppen wurden zufällig zusammengesetzt, den Teilnehmern war während der individuellen Phase die Zuteilung nicht bekannt, so dass eine Abstimmung vor Anwendung des Werkzeugs weitgehend ausgeschlossen werden kann.

% subsubsection aufgabenstellung (end)

\subsubsection{Anwendungen und Teilnehmer} % (fold)
\label{ssub:3_teilnehmer}

An den Anwendungen, die in diesem Block durchgeführt wurden, nahmen insgesamt 54 Personen teil, die in 18 Gruppen einmalig mit dem Werkzeug arbeiteten. Alle Teilnehmer waren Studierende der Wirtschaftsinformatik im ersten Studienabschnitt (1-4 Semester), 8 waren weiblich, 46 männlich. Keinem der Teilnehmer war der Ansatz des Concept Mapping vor Beginn der betreffenden Aufgabe bekannt, Erfahrungen mit Prozessmodellierungssprachen (also dem Gegenstand der Concept Map) sammelten alle Teilnehmer erstmals im Rahmen der Lehrveranstaltung, in der dieser Evaluierungs-Block durchgeführt wurde.

Den Teilnehmern wurde das Werkzeug vor Beginn der Anwendung demonstriert und in sämtlichen Anwendungsaspekten erklärt. Die Anwendungen selbst dauerten durchschnittlich XY Minuten, wobei die kürzeste Anwendung XY Minuten, die längste XY Minuten dauerte.
% subsubsection teilnehmer (end)

\subsubsection{Verwendung der Ergebnisse} % (fold)
\label{ssub:3_verwendung_der_ergebnisse}

Die Daten, die aus diesem Block gewonnen werden konnten, gehen in die Evaluierung des Modell-Aspekts ein. Hier können einerseits wiederum die erstellten Modelle hinsichtlich Struktur, Inhalt und semantischen Zuweisungen untersucht werden. Der Modellierungsgegenstand ist in diesem Fall jedoch anders gelagert als im vorhergehenden Fall, anstelle eines Arbeitsabstimmung ist hier ein Vergleich von Konzepten durchzuführen. Andererseits können hier die Abstimmungsprozesse der individuellen mentalen Modelle insofern betrachtet werden, als dass für jede Gruppe neben dem kollaborativ erstellten Ergebnis auch noch die individuellen Concept Maps vorliegen und ausgewertet werden können.

Wie bereits in den zuvor beschriebenen Blöcken können auch hier wieder Erkenntnisse hinsichtlich der Verwendung des Werkzeugs gewonnen werden. Aufgrund der der Aufgabe innewohnenden Wichtigkeit der Verbindungen zwischen Konzepten wird vorallem deren Verwendung bzw. der Vorgang deren Erstellung zu betrachten sein.

Der Aspekt Articulation Work bleibt in diesem Block im engeren Sinne außen vor, als dass kein Arbeitskontext vorliegt, keine aufzulösende Problematik vorliegt und keine Zusammenarbeit auszuhandeln ist. Insofern wird dieser Aspekt in diesem Block nicht explizit behandelt. Aufgrund der Durchführung sämtlicher Schritte, die zur Unterstützung expliziter Articulation Work notwendig sind (Externalisierung, Abstimmung) können aber die einzelnen Anwendungen zur Hypothesenbildung für den Evaluierungs-Aspekt Articulation Work herangezogen werden.

% subsubsection verwendung_der_ergebnisse (end)

\subsection{Block 4: Aushandlung von Zusammenarbeit 2}
\label{sub:eval_4}

Block 4 deckt die erste Anwendung des Werkzeugs im realen Unternehmenskontext ab. Im Rahmen einer Diplomarbeit REF wurde das Werkzeug im Rahmen eines Workshops einer Abteilung des betreffenden Unternehmens angewandt.

\subsubsection{Kontext} % (fold)
\label{ssub:4_kontext}

Das Unternehmen, in dem das Werkzeug angewandt wird, ist im Bereich XY tätig und hat XY Mitarbeiter. Die Abteilung, im Rahmen deren Workshop die Anwendung durchgeführt wurde, ist im Unternehmen für XY zuständig und hat XY Mitarbeiter. Der Workshop hatte das Ziel ... und wurde von der Abteilungsleitung in Hinblick auf diese Aufgabenstellung konzpiert. Das Problem dieser Abteilung ist ... Um diese Problematik aufzulösen, wurde das Werkzeug zur Unterstützung eingesetzt.

% subsubsection kontext (end)

\subsubsection{Aufgabenstellung und Ablauf} % (fold)
\label{ssub:4_aufgabenstellung}

exakter Ablauf bzw. Aufgabenstellung noch offen

% subsubsection aufgabenstellung (end)

\subsubsection{Anwendungen und Teilnehmer} % (fold)
\label{ssub:4_teilnehmer}

einmalige Anwendung, Teilnehmer offen 

% subsubsection teilnehmer (end)

\subsubsection{Verwendung der Ergebnisse} % (fold)
\label{ssub:4_verwendung_der_ergebnisse}

Die Daten, die das Ergebnis dieses Blocks bilden, werden zur Evaluierung des Aspekts Artiuclation Work eingesetzt. Betrachtet werden dabei die wahrgenommenen und beobachtbaren Veränderungen am Arbeitsprozess, der unter Einsatz des Werkzeugs reflektiert wurde.

Neben diesem Aspekt wird auch das erstellte Modell, das in diesem Fall wieder aus der Domäne der Arbeitsabstimmung stammt, betrachtet und hinsichtlich seiner Struktur und Semantik ausgewertet. 

Der Werkzeug-Aspekt wird in diesem Teil der Untersuchung nicht gesondert betrachtet, Verbesserungs- und Erweiterungspotential wird nur bei Erwähung oder offensichtlichen Bedienungsfehlern bzw. Verständnisschwierigkeiten explizit identifiziert.

% subsubsection verwendung_der_ergebnisse (end)

\subsection{Block 5: Concept Mapping 2}
\label{sub:eval_5}

In Block 5 wird im Wesentlich der Evaluierungs-Blocks 3 (siehe Abschnitt \ref{sub:eval_3})   inhaltlich erneut durchgeführt (die Modellierungsaufgabe ist identisch). Im Gegensatz zu Block 3, wo die grundlegende Eignung des Werkzeugs zum Concept Mapping im Mittelpunkt stand, wird in Block 5 eine vergleichende Studie durchgeführt, die die Eignung des Tabletop Interfaces zum kollaborativen Concept Mapping mit jener der rechner-basierten CMapTools \citep{Canas04} vergleicht.

\subsubsection{Kontext} % (fold)
\label{ssub:5_kontext}

Die Anwendungssituation ist in diesem Block identisch mit dem in Abschnitt \ref{ssub:3_kontext} beschriebenen Kontext (Lehrveranstaltung im Curriculum Wirtschaftsinformatik zur Schulung von Ansätzen in der Prozess- und Kommunikationsmodellierung).

Der Ablauf der Lehrveranstaltung unterschied sich nur insofern von jenem in Block 3, als dass für jene Modellierungssprache separat eine Reflexion in Gruppen zu zwei Studierenden durchgeführt wurde. In diesen Reflexion wurden die eigenen Anwendungen der jeweiligen Sprache mit einer Musterlösung gegenübergestellt und hinsichtlich ihrer Korrektheit und dem Vorgehen bei der Modellierung betrachtet.

% subsubsection kontext (end)

\subsubsection{Aufgabenstellung und Ablauf} % (fold)
\label{ssub:5_aufgabenstellung}

Die Aufgabenstellung ist identisch mit jener in Block 3. Ziel ist es, die drei vorgestellten Prozessmodellierungssprachen hinsichtlich ihrer als wesentlich empfundenen Eigenschaften und deren Gemeinsamkeiten und Unterschiede zu betrachten und in einer Concept Map abzubilden. Das Vorgehen unterscheiden sich jedoch wegen der unterschiedlichen Zielsetzung der Untersuchung von jenem in Block 3.

Nach Abschluss der letzten Reflexionsphase (also nach drei Modellierungsphasen und drei Reflexionsphasen) wurde eine Gruppeneinteilung für die kollaborative Erstellung der Concept Map vorgenommen. Die Gruppen wurden aus jeweils zwei zufällig ausgewählten Studierenden gebildet. In der Untersuchung erhielt nun eine Hälfte der Gruppen den Auftrag, die Aufgabenstellung unter Verwendung des Tabletop Interfaces durchzuführen, die andere Hälfte verwendete das rechner-basierte Werkzeug CMapTools \citep{Canas04}, um die Concept Map zu erstellen. Die Gruppen wurden zufällig einem Werkzeug zugeordnet und führten die Aufgabenstellung in beiden Fällen kollaborativ in einer kontrollierten Umgebung durch. Im Gegensatz zu Block 3 entfiel hier die explizit geforderte individuelle Vorbereitungsphase, um eine stärkere inhaltliche Auseinandersetzung mit den Inhalten während der Modellierung zu fördern. 

% subsubsection aufgabenstellung (end)

\subsubsection{Anwendungen und Teilnehmer} % (fold)
\label{ssub:5_teilnehmer}

An der Untersuchung nahmen XY Studierende in XY Gruppen teil, wobei XY Gruppen die Aufgabenstellung unter Verwendung des hier vorgestellen Werkzeugs und XY Gruppen unter Verwendung der CMapTools durchführten. Die Teilnehmer waren allesamt Studierende der Wirtschaftsinformatik in der ersten Phase des Bakkelauratsstudiums (erstes bis drittes Semester), XY Teilnehmer waren männlich, XY weiblich. Keiner der Teilnehmer hatte Vorkenntnisse in der Prozessmodellierung oder im Concept Mapping.

Den Teilnehmern wurde das Werkzeug vor Beginn der Anwendung demonstriert und in sämtlichen Anwendungsaspekten erklärt. Die Anwendungen selbst dauerten durchschnittlich XY Minuten, wobei die kürzeste Anwendung XY Minuten, die längste XY Minuten dauerte.

% subsubsection teilnehmer (end)

\subsubsection{Verwendung der Ergebnisse} % (fold)
\label{ssub:5_verwendung_der_ergebnisse}

Die in diesem Block erhobenen Daten fließen in vorrangig in den Modell-Aspekt der Evaluierung ein. Hier wird eine vergleichende Studie durchgeführt, die das Ziel hat, die Eignung der beiden verwendeten Ansätze für die Externalisierung von mentalen Modellen gegenüberzustellen. Grundlage dieser Beurteilung ist das erstellte Modell, außerdem wird der auch Modellierungsprozess in der Auswertung berücksichtigt.

Hinsichtlich des Werkzeug-Aspekts wird in diesem Block neben der Identifikation von Verbesserungspotential und Verständnisschwierigkeiten auch die Zufriedenheit mit dem Werkzeug bzw. dessen Akzeptanz bei den Benutzern explizit erhoben. 

Der Aspekt Articulation Work wird hier wie schon in Block 3 und aus den dort angeführten Gründen (siehe Abschnitt \ref{ssub:3_verwendung_der_ergebnisse}) nicht weiter berücksichtigt.

% subsubsection verwendung_der_ergebnisse (end)

\section{Zusammenfassung}
\label{sec:eval_ueberblick_zusammenfassung}

In diesem Kapitel wurde das globale Untersuchungsdesign zur Evaluierung der hier vorgestellten Arbeit beschrieben. In den ersten Abschnitten wurden die zu evaluierenden Aspekte identifziert und beschrieben. Im Rahmen dieser Beschreibungen wurden auch mögliche Ansatzpunkte für die konkrete Untersuchung angeführt, die die Basis für die detaillierte Konzeption der Evaluierung dieser Aspekte in den Kapiteln XY bis XY bildet. 

Im folgenden Abschnitt wurden die einzelnen im Rahmen der Evaluierung durchgeführten Untersuchungen angeführt. Diese Untersuchungen fokussieren jeweils auf einen zu evaluierenden Aspekt. Ihnen liegt jeweils ein konkretes Szenario zu Grunde, das in einer Reihe von Anwendungen des Werkzeugs in ein Modell umgesetzt wird. Je nach Fokus der Untersuchung werden vor- und nachgelagerte bzw. paralell ablaufende Aktivitäten in die Untersuchung mit einbezogen.

Die ursprüngliche Zuordnung zwischen den zu evaluierenden Aspekten und den einzelnen Evaluierungs-Blöcken ist in Tabelle \ref{tab:evaluierungsMatrixOriginal} nochmals überblicksweise angeführt. Die Zuordnung hatte jeweils Einfluss auf das Szenario, in dem das Werkzeug angewandt wurde sowie auf das Untersuchungsdesign.

\begin{table}[htbp]
	\centering
	\caption{Ursprüngliches globales Untersuchungsdesign}
	\begin{tabular}{| p{3cm} || p{2cm} | p{2cm} | p{2cm} |} \hline
		 & Werkzeug & Modell & Articulation Work \\ \hline \hline
		 Block 1 & x &  &   \\ \hline
		 Block 2 &  &  & x  \\ \hline
		 Block 3 &  & x &   \\ \hline
		 Block 4 &  &  & x  \\ \hline
		 Block 5 &  & x &   \\ \hline
	\end{tabular}
	\label{tab:evaluierungsMatrixOriginal}
\end{table}

Im Zuge der Durchführung der Evaluierung erwies sich die strikte Zuordnung eines Blocks zu genau einem zu evaluierenden Aspekt als nicht durchführbar. Tatsächlich liefern Untersuchungen zu einem (im Sinne der Zielhierarchie) übergeordneten Aspekte (von "unten" nach "oben": Werkzeug -- Modell -- Articulation Work) immer auch Erkenntnisse zu den untergeordneten zu evaluierenden Aspekten. Die Zuordnung der Evaluierungs-Blöcke zu den Aspekten verändert sich also wie in Tabelle \ref{tab:evaluierungsMatrix} angegeben. Diese Zuordnung liegt auch den oben angeführten Beschreibungen der Blöcke zugrunde, in denen jeweils die Beiträge eines Blocks zu den zu evaluierenden Aspekten angegeben wurden.

\begin{table}[htbp]
	\centering
	\caption{Einfluss der Untersuchungen auf die zu evaluierenden Aspekte}
	\begin{tabular}{| p{3cm} || p{2cm} | p{2cm} | p{2cm} |} \hline
		 & Werkzeug & Modell & Articulation Work \\ \hline \hline
		 Block 1 & \textbf{x} &  &   \\ \hline
		 Block 2 & x & x & \textbf{x}  \\ \hline
		 Block 3 & x & \textbf{x} &   \\ \hline
		 Block 4 & x & x & \textbf{x}  \\ \hline
		 Block 5 & x & \textbf{x} &   \\ \hline
	\end{tabular}
	\label{tab:evaluierungsMatrix}
\end{table}

In den folgenden Kapiteln wird nun die Evaluierung der einzelnen Aspekte über die Evaluierungs-Blöcke hinweg im Detail beschrieben. Dabei werden die Hyothesenbildung bzw. die Entwicklung der Hypothesen über die Zeit, die möglichen Ansätze zur Evaluierung der jeweiligen Hypothesen sowie das Untersuchungsdesign das zur Prüfung des Hypothesen führt, beschrieben. Die Kapitel schließen jeweils mit einer Zusammenfassung der Ergebnisse der Hypothesenprüfung und einer Bewertung dieser Ergebnisse im Kontext der globalen Zielsetzung, also der Unterstüztung von expliziter Articulation Work.

% section globales_untersuchungsdesign (end)
% chapter eval_ueberblick (end)