\chapter{Mentale Modelle} % (fold)
\label{cha:mentale_modelle}

In diesem Kapitel wird das Konzept der mentalen Modelle eingeführt, das in dieser Arbeit als Erklärungsansatz für jene Aspekte von "Articulation Work" verwendet wird, die die nicht sichtbaren, kognitiven Beiträge eines beteiligten Individuums betreffen. Nach einer Einführung in die Begriffswelt der mentalen Modelle wird die Argumentation aus dem letzten Kapitel nochmals aufgegriffen und die mögliche Rolle mentaler Modelle für "Articulation Work" erörtert. In der Folge werden Methoden eingeführt mit denen mentale Modelle externalisiert und kommuniziert werden können. Basierend auf diesen Beschreibungen wird im letzten Teil des Kapitels untersucht, welche Herausforderungen sich bei der Anwendung dieser Methoden im Kontext von "Articulation Work" ergeben können.

\section{Articulation Work und mentale Modelle} % (fold)
\label{sec:articulation_work_und_mentale_modelle}

Wie bereits im vorgehenden Kapitel beschrieben, wird in vorhandenen Arbeiten zu Articulation Work deren Auftreten, Kontext und Wirkung beschrieben, nicht aber ihre Durchführung und der eigentliche Gegenstand der Abstimmung. Die Vermeidung von konkreten Aussagen zur Durchführung liegt in der Vielfalt möglicher Ausprägungen begründet. Schon \citet{Strauss88} unterscheidet grob zwischen expliziter und impliziter Articulation Work, begründet die Unterscheidung aus dem Kontext des Auftretens (siehe Abschnitt \ref{sec:arten_von_articulation_work}), lässt aber offen, wie sich implizite und explizite Articulation Work unterscheiden bzw. was explizite Articulation Work im Gegensatz zur ständig im Arbeitsverlauf auftretenden impliziten Articulation Work ausmacht. Der eigentliche Gegenstand der Abstimmung, die im Rahmen der Articulation Work erfolgen soll, wird ebenfalls nicht konkret festgelegt. Strauss spricht von \emph{„putting together tasks, task sequences, task clusters - even aligning larger units such as lines of work and subprojects - in the service of work flow“} \citep[][S. 2]{Strauss88}, und konkretisiert \emph{„the specific questions about tasks of course include: what, where, when, how, for how long, how complex, how weIl defined are their boundaries, how attainable are they under current working conditions, how precisely are they defined in their operational details, and what is the expected level of performance. (Which of those are the most salient dimensions depends on the organizational work context under study, and we cannot emphasize too much that \textbf{it is the researcher who must discover these saliences}.)“} \citep[][S. 6]{Strauss85}. Strauss lässt also offen, was es exakt ist, dass abgestimmt werden muss bzw. verlagert diese Frage in den konkreten Einzelfall. 

Strauss spricht diese Auslassung in einer späteren Arbeit explizit an \citep[][S. 131]{Strauss93} und beschäftigt sich in dieser auch mit jenen kognitiven Vorgängen, die von ihm als „thought processes“ oder „mental activities“ bezeichnet werden und die untrennbar mit jeder Art von Tätigkeit und Interaktion verbunden sind \citep[][S. 146]{Strauss93} und diese beeinflussen \citep[][S. 132]{Strauss93}.  

Im Kontext der Abstimmung von Tätigkeiten kommt den „thought processes“ der Individuen große Bedeutung zu, da sie den sichtbaren individuellen Handlungen zugrunde liegen bzw. diese beeinflussen. „Articulation Work“ wirkt sich also auf die „thought processes“ der beteiligten Individuen aus. „Thought processes“ umfassen \emph{„images, imaginations, projections of scenes, [...] flashes of insight, rehearsals of action, construction and reconstruction of scenarios,  the spurting up of metaphors or comparisons, the reworking and reevaluating of past scenes and one's actions within them, and so on and on“} \citep[][S. 130]{Strauss93} - also im Wesentlichen alle kognitiven Vorgänge, die unmittelbar oder mittelbar im Zusammenhang mit den sichtbaren Arbeitsaspekten, insbesondere den Tätigkeiten zur Zielerreichung und der wahrgenommenen Arbeitsumgebung, stehen. Strauss interessiert sich allerdings ausschließlich für die dynamischen Aspekte der Interaktion zwischen Individuen, nicht aber für die Ausgangspunkte und Ergebnisse der zugrunde liegenden „thought processes“ \citep[][S. 149]{Strauss93}

% section articulation_work_und_mentale_modelle (end)

\section*{Mentale Modelle -- Begriffsbestimmung}

Nach der Identifikation der begrenzten Erklärungsmächtigkeit des Konzepts "Articulation Work" wird nun die Theorie der "mentalen Modelle" herangezogen, um die bei Articulation Work offen bleibenden Aspekte hinsichtlich des Gegenstandes der Abstimmung und deren Untersützung näher zu betrachten. Dazu wird im ersten Schritt der Begriff der "mentalen Modelle" in den historischen Kontext gestellt und dessen Bedeutung dargelegt.

Das Konzept der "mentalen Modelle" wird grundsätzlich verwendet, um zu erklären \emph{"wie Menschen die Welt verstehen -- genauer: wie sie ihr Wissen benutzen, um sich bestimmte Phänomene der Welt subjektiv plausibel zu machen"} \citep[][S. VII]{Seel91}. Mentale Modelle sind dabei Erklärungsmodelle der Welt, die Menschen auf Basis von Alltagserfahrungen, bisherigem Wissen und darauf basierenden Schlussfolgerungen bilden. Ein gebildetes mentales Modell wird dann als Basis verwendet, um die Welt zu verstehen und ggf. Vorhersagen über deren Verhalten zu bilden. \citep[][S. VII]{Seel91}

Im Wesentlichen wurde das Forschungsfeld der mentalen Modelle durch zwei Arbeiten maßgeblich beeinflusst. \citet{Johnson-Laird81} und \citet{de-Kleer81} führen den Begriff als eigenständigen Forschungsgegenstand ein und legen damit die Grundlage für einen Großteil der nachfolgenden Arbeiten in dem Gebiet. Im Kontext dieser Arbeit werden dabei zwei dieser nachfolgenden Arbeiten näher betrachtet. Zum einen stellt \citet{Norman83} den Begriff erstmals im den Kontext der Mensch-Maschine-Interaktion dar. Zum anderen versucht \citet{Seel91} die unterschiedlichen Richtungen der Forschung im Bereich der mentalen Modelle zusammenzuführen und daraus die Bedeutung von Mentalen Modellen für Lernvorgänge (unter die -- im breiten Verständnis von Seel -- auch die hier relevanten Abstimmungsvorgänge fallen) und Möglichkeiten zu deren Unterstütztung abzuleiten.

\subsection{Mentale Modelle nach Johnson-Laird}

\section{Mentale Modelle im Gesamtzusammenhang}

\subsection{Realität}

\subsection{Theorien}

\subsection{Experten- vs. Alltags-Modelle}

\subsection{Schemata}


\section{Bildung mentaler Modelle} % (fold)
\label{sec:bildung_mentaler_modelle}

Nach \citep{Seel91} umfasst die Bildung mentaler Modelle zwei Komponenten: Eine \emph{deklarative Komponente}, in der bereichs- bzw. domänen-spezifisches Wissen in der Form von hier nicht näher spezifizierten, strukurierten Wissensbasen abgelegt wird und eine \emph{operative Komponente}, in der auf Grundlage dieser Wissensbasen Schlüsse gezogen und neues Wissen abgeleitet wird, die über das ursprüngliche domänenspezifische Wissen hinausgeht. 

Das in den Wissensbasen repräsentierte Wissen kann auf Alltagserfahrung begründet sein oder durch Vermittlung oder Instruktion begründet werden. Im ersteren Fall ist das Wissen dann als konkret und handlungsbezogen angesehen werden, im zweiten Fall ist das Wissen eher auf abstrakter, formaler Ebene anzusiedeln. Analog dazu kann auch in der operativen Komponente die Schlussfolgerung induktiv auf Basis eines "intuitionsbegründeten" Regelsystems gezogen werden oder durch Deduktion mittels einem formal begründbaren Regelsystem gebildet werden. 

Die Modifikation und Erweiterung der eigenen Wissensbasen und die (Weiter-)Entwicklung der kognitiven Fähigkeiten, die für die Ableitung von Schlussfolgerungen notwendig sind, bezeichnet \citet{Seel91} als "Lernen". Lernen ist \emph{"mit der Verarbeitung individueller Erfahrungen mit sowie vermittelter Information über die Welt, ihre Struktur und Evidenz verbuneen und kann als ein Prozess permanenter konzeptueller Veränderungen verstanden werden."} \citep[][S. 23]{Seel91}. Lernen setzt damit die Fähigkeit und Bereitschaft voraus, \emph{"vermittelte Weltauffassungen zu verstehen, zu akzeptieren und sodann den eigenen gedanklichen Konstruktionen zugrunde zu legen"} \citep[][S. 23]{Seel91}.

% section bildung_mentaler_modelle (end)

\section{Externalisierung mentaler Modelle} % (fold)
\label{sec:externalisierung_mentaler_modelle}

% section externalisierung_mentaler_modelle (end)

% chapter mentale_modelle (end)

% \chapter{Mentale Modelle}
% \label{cha:mentale_modelle}
% 
% In diesem Kapitel wird das Konzept der mentalen Modelle eingeführt, das in dieser Arbeit als Erklärungsansatz für jene Aspekte von "Articulation Work" verwendet wird, die die nicht sichtbaren, kognitiven Beiträge eines beteiligten Individuums betreffen. Nach einer Einführung in die Begriffswelt der mentalen Modelle wird die Argumentation aus dem letzten Kapitel nochmals aufgegriffen und die mögliche Rolle mentaler Modelle für "Articulation Work" erörtert. In der Folge werden Methoden eingeführt mit denen mentale Modelle externalisiert und kommuniziert werden können. Basierend auf diesen Beschreibungen wird im letzten Teil des Kapitels untersucht, welche Herausforderungen sich bei der Anwendung dieser Methoden im Kontext von "Articulation Work" ergeben können.
% 
% \section{Begriffsbestimmung}e
% \label{sec:mentalemodelle_begriffsbestimmung}
% 
% Der Begriff der Mentalen Modelle wurde von \citet{Johnson-Laird81} geprägt. Ein mentales Modell ist nach 
% 
% \subsection{Mentale Modelle nach Johnson-Laird} % (fold)
% \label{sub:mentale_modelle_nach_johnson_laird}
% 
% % subsection mentale_modelle_nach_johnson_laird (end)
% 
% \subsection{Mentale Modelle nach Norman} % (fold)
% \label{sub:mentale_modelle_nach_norman}
% 
% \citet{Norman83a} formuliert ein Verständnis von mentalen Modellen aus Interaktionssicht. Sein Kontext ist die Untersuchung von Mensch-Maschine-Interaktion und den dort auftretenden Interaktionsabläufen. Mentale Modelle sind in diesem Verständnis individuelle Konstrukte, die von Menschen bei der Interaktion mit der Umwelt, mit anderen Menschen oder mit Technologie gebildet werden, um das Verhalten der Gegenseite erklären und vorhersagen zu können\footnote{\emph{„In interaction with the environment, with others, an with the artifacts of technology, people form internal, mental models of themselves and of the things with which they are interacting. These models provide predictive and explanatory power for understanding the interaction“} \citep{Norman83a}}. Um den Begriff abzugrenzen, führt \citeauthor{Norman83a} ein aus vier Elementen bestehendes Begriffssystem ein, das den Diskussionsbereich abgrenzt und definiert:
% \begin{description}
% 	\item[target system] Das Zielsystem ist jenes System, das von einer Person benutzt wird oder dessen Benutzung von dieser Person erlernt wird.
% 	\item[conceptual model of target system] Ein konzeptionelles Modell ist ein Modell, dass das Zielsystem vollständig, konsistent und exakt beschreibt. Konzeptionelle Modelle werden von Entwicklern, Designern, Wissenschaftern oder Lehrern (im Allgemeinen: Experten in der Domäne des Zielsystems) definiert.
% 	\item[mental model of target system] Mentale Modelle werden von Personen bei der Interaktion mit dem Zielsystem entwickelt, um dessen Verhalten zu erklären. Diese Modelle müssen nicht vollständig und exakt sein, müssen aber für die jeweilige Person funktional sein, d.h. für deren Zwecke ausreichendes Erklärungspotential besitzen. Mentale Modelle haben evolutionären Charakter und entwickeln sich während der Interaktion mit dem System weiter. Die Inhalte eines mentalen Modells werden durch das Vorwissen und die Erfahrung der jeweiligen Person beeinflusst.
% 	\item[scientist's conceptualization of mental model] Die Konzeptualisierung eines mentalen Modells ist der Versuch ein mentales Modell mit wissenschaftlichen Mitteln zu erheben und abzubilden. Sie soll die Inhalte des mentalen Modells möglichst vollständig und genau abbilden. Die Konzeptualisierung ist also ein Modell eines Modells.
% \end{description}
% 
% Im Weiteren nennt \citeauthor{Norman83a} sechs generelle Eigenschaften von mentalen Modellen, die er aus eine Vielzahl von Beobachtungen in unterschiedlichen Kontexten ableitet:
% \begin{enumerate}
% 	\item Mentale Modelle sind unvollständig
% 	\item Mentale Modelle können von ihren Trägern nur sehr einschränkt wiedergegeben werden.
% 	\item Mentale Modell sind instabil und werden vor allem in Bereich ungenau, die Teile des Zielsystems abbilden die lange nicht benötigt wurden.
% 	\item Mentale Modelle sind nicht klar voneinander abgrenzbar -- ähnliche Gegenstände oder Situationen werden oft hinsichtlich der angewandten Interaktionsmuster verwechselt.
% 	\item Mentale Modelle sind unwissenschaftlich -- auch mentale Modelle, die inhaltlich (technisch) überflüssiges Verhalten verursachen, werden beibehalten, wenn der Aufwand der physischen Ausführung gering ist.
% 	\item Mentale Modelle sind simpel -- auch wenn eine effizientere Interaktion möglich wäre, wenn mehr Aufwand in die Planung investiert würde bzw. ein komplexeres mentales Modell zum Einsatz käme, präferieren Benutzer einfache Modelle, deren Anwendung höheren „physischen Aufwand“ mit sich bringen.
% \end{enumerate}
% 
% % subsection mentale_modelle_nach_norman (end)
% 
% \subsection{Mentale Modelle nach Senge} % (fold)
% \label{sub:mentale_modelle_nach_senge}
% 
% % subsection mentale_modelle_nach_senge (end)
% 
% \section{Veränderung mentaler Modelle}
% \label{sub:veränderung_mentaler_modelle}
% Assimilation vs. Akkommodation
% 
% \section{Mentale Modelle und Articulation Work}
% \label{sec:mentale_modelle_und_articulation_work}
% 
% Argumentation mit Wissensspirale (Nonaka \& Takeuchi)
