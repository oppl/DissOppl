\chapter{Mentale Modelle}
\label{cha:mentale_modelle}

In diesem Kapitel wird das Konzept der mentalen Modelle eingeführt, das in dieser Arbeit als Erklärungsansatz für jene Aspekte von "Articulation Work" verwendet wird, die die nicht sichtbaren, kognitiven Beiträge eines beteiligten Individuums betreffen. Nach einer Einführung in die Begriffswelt der mentalen Modelle wird die Argumentation aus dem letzten Kapitel nochmals aufgegriffen und die mögliche Rolle mentaler Modelle für "Articulation Work" erörtert. In der Folge werden Methoden eingeführt mit denen mentale Modelle externalisiert und kommuniziert werden können. Basierend auf diesen Beschreibungen wird im letzten Teil des Kapitels untersucht, welche Herausforderungen sich bei der Anwendung dieser Methoden im Kontext von "Articulation Work" ergeben können.

\section{Begriffsbestimmung}
\label{sec:mentalemodelle_begriffsbestimmung}

Der Begriff der Mentalen Modelle wurde von \citet{Johnson-Laird81} geprägt. Ein mentales Modell ist 

\subsection{Entwicklung des Begriffs}
\label{sub:entwicklung_des_begriffs}

\subsection{Veränderung mentaler Modelle}
\label{sub:veränderung_mentaler_modelle}
Assimilation vs. Akkommodation

\section{Mentale Modelle und Articulation Work}
\label{sec:mentale_modelle_und_articulation_work}

Argumentation mit Wissensspirale (Nonaka \& Takeuchi)

\section{Externalisierung mentaler Modelle}
\label{sec:methoden_zur_externalisierung_mentaler_modelle}

% section methoden_zur_externalisierung_mentaler_modelle (end)

\section{Concept Mapping} % (fold)
\label{sec:concept_mapping}

% section concept_mapping (end)

\section{Strukturlegetechniken} % (fold)
\label{sec:strukturlegetechniken}

Strukturlegetechniken sind Ansätze, in denen gelegte Strukturen zur Repräsentation von "Wissen" eingesetzt werden. Die gelegten Strukturen (die im Wesentlichen aus Knoten und Kanten unterschiedlicher Bedeutung bestehen) bilden dabei die Zusammenhänge einzelner Konstrukte ab, wie sie die legende Person wahrnimmt. Der Prozess des Legens ist eine \emph{"Rekonstruktion subjektiver Theorien"} \citep{Dann92} und stellt eine \emph{"[\ldots] verstehende Beschreibung von Handlungen nicht aus der Perspektive eines außenstehenden Beobachters, sondern aus Sicht der handelnden Person, des Akteurs selber"} \citep{Dann92} dar. 

% section strukturlegetechniken (end)

\section{Herausforderungen bei der Anwendung}
\label{sec:herausforderungen_bei_der_anwendung}

\subsection{Kollaborative Anwendbarkeit}

pro Strukturlegetechniken 

\subsection{Nachhaltige Verwendung der Information}

pro Concept Mapping

\subsection{Zusammenführung}

Offenheit des Conceptmapping

Strukturlegetechniken mit IT-Unterstützung

Nicht SLT in den Computer sondern Computer zur Unterstützung von SLT