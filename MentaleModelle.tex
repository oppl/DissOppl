\chapter{Mentale Modelle}
\label{cha:mentale_modelle}

In diesem Kapitel wird das Konzept der mentalen Modelle eingeführt, das in dieser Arbeit als Erklärungsansatz für jene Aspekte von "Articulation Work" verwendet wird, die die nicht sichtbaren, kognitiven Beiträge eines beteiligten Individuums betreffen. Nach einer Einführung in die Begriffswelt der mentalen Modelle wird die Argumentation aus dem letzten Kapitel nochmals aufgegriffen und die mögliche Rolle mentaler Modelle für "Articulation Work" erörtert. In der Folge werden Methoden eingeführt mit denen mentale Modelle externalisiert und kommuniziert werden können. Basierend auf diesen Beschreibungen wird im letzten Teil des Kapitels untersucht, welche Herausforderungen sich bei der Anwendung dieser Methoden im Kontext von "Articulation Work" ergeben können.

\section{Begriffsbestimmung}
\label{sec:mentalemodelle_begriffsbestimmung}

Der Begriff der Mentalen Modelle wurde von \citet{Johnson-Laird81} geprägt. Ein mentales Modell ist nach 

\subsection{Mentale Modelle nach Johnson-Laird} % (fold)
\label{sub:mentale_modelle_nach_johnson_laird}

% subsection mentale_modelle_nach_johnson_laird (end)

\subsection{Mentale Modelle nach Norman} % (fold)
\label{sub:mentale_modelle_nach_norman}

\citet{Norman83a} formuliert ein Verständnis von mentalen Modellen aus Interaktionssicht. Sein Kontext ist die Untersuchung von Mensch-Maschine-Interaktion und den dort auftretenden Interaktionsabläufen. Mentale Modelle sind in diesem Verständnis individuelle Konstrukte, die von Menschen bei der Interaktion mit der Umwelt, mit anderen Menschen oder mit Technologie gebildet werden, um das Verhalten der Gegenseite erklären und vorhersagen zu können\footnote{\emph{„In interaction with the environment, with others, an with the artifacts of technology, people form internal, mental models of themselves and of the things with which they are interacting. These models provide predictive and explanatory power for understanding the interaction“} \citep{Norman83a}}. Um den Begriff abzugrenzen, führt \citeauthor{Norman83a} ein aus vier Elementen bestehendes Begriffssystem ein, das den Diskussionsbereich abgrenzt und definiert:
\begin{description}
	\item[target system] Das Zielsystem ist jenes System, das von einer Person benutzt wird oder dessen Benutzung von dieser Person erlernt wird.
	\item[conceptual model of target system] Ein konzeptionelles Modell ist ein Modell, dass das Zielsystem vollständig, konsistent und exakt beschreibt. Konzeptionelle Modelle werden von Entwicklern, Designern, Wissenschaftern oder Lehrern (im Allgemeinen: Experten in der Domäne des Zielsystems) definiert.
	\item[mental model of target system] Mentale Modelle werden von Personen bei der Interaktion mit dem Zielsystem entwickelt, um dessen Verhalten zu erklären. Diese Modelle müssen nicht vollständig und exakt sein, müssen aber für die jeweilige Person funktional sein, d.h. für deren Zwecke ausreichendes Erklärungspotential besitzen. Mentale Modelle haben evolutionären Charakter und entwickeln sich während der Interaktion mit dem System weiter. Die Inhalte eines mentalen Modells werden durch das Vorwissen und die Erfahrung der jeweiligen Person beeinflusst.
	\item[scientist's conceptualization of mental model] Die Konzeptualisierung eines mentalen Modells ist der Versuch ein mentales Modell mit wissenschaftlichen Mitteln zu erheben und abzubilden. Sie soll die Inhalte des mentalen Modells möglichst vollständig und genau abbilden. Die Konzeptualisierung ist also ein Modell eines Modells.
\end{description}

Im Weiteren nennt \citeauthor{Norman83a} sechs generelle Eigenschaften von mentalen Modellen, die er aus eine Vielzahl von Beobachtungen in unterschiedlichen Kontexten ableitet:
\begin{enumerate}
	\item Mentale Modelle sind unvollständig
	\item Mentale Modelle können von ihren Trägern nur sehr einschränkt wiedergegeben werden.
	\item Mentale Modell sind instabil und werden vor allem in Bereich ungenau, die Teile des Zielsystems abbilden die lange nicht benötigt wurden.
	\item Mentale Modelle sind nicht klar voneinander abgrenzbar -- ähnliche Gegenstände oder Situationen werden oft hinsichtlich der angewandten Interatkionsmuster verwechselt.
	\item Mentale Modelle sind unwissenschaftlich -- auch mentale Modelle, die inhaltlich (technisch) überflüssiges Verhalten verursachen, werden beibehalten, wenn der Aufwand der physischen Ausführung gering ist.
	\item Mentale Modelle sind simpel -- auch wenn eine effizientere Interaktion möglich wäre, wenn mehr Aufwand in die Planung investiert würde bzw. ein komplexeres mentales Modell zum Einsatz käme, präferieren Benutzer einfache Modelle, deren Anwendung höheren „physischen Aufwand“ mit sich bringen.
\end{enumerate}

% subsection mentale_modelle_nach_norman (end)

\subsection{Mentale Modelle nach Senge} % (fold)
\label{sub:mentale_modelle_nach_senge}

% subsection mentale_modelle_nach_senge (end)

\section{Veränderung mentaler Modelle}
\label{sub:veränderung_mentaler_modelle}
Assimilation vs. Akkommodation

\section{Mentale Modelle und Articulation Work}
\label{sec:mentale_modelle_und_articulation_work}

Argumentation mit Wissensspirale (Nonaka \& Takeuchi)
