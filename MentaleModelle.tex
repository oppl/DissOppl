\chapter{Mentale Modelle}
\label{cha:mentale_modelle}

In diesem Kapitel wird das Konzept der mentalen Modelle eingeführt, das in dieser Arbeit als Erklärungsansatz für jene Aspekte von "Articulation Work" verwendet wird, die die nicht sichtbaren, kognitiven Beiträge eines beteiligten Individuums betreffen. Nach einer Einführung in die Begriffswelt der mentalen Modelle wird die Argumentation aus dem letzten Kapitel nochmals aufgegriffen und die mögliche Rolle mentaler Modelle für "Articulation Work" erörtert. In der Folge werden Methoden eingeführt mit denen mentale Modelle externalisiert und kommuniziert werden können. Basierend auf diesen Beschreibungen wird im letzten Teil des Kapitels untersucht, welche Herausforderungen sich bei der Anwendung dieser Methoden im Kontext von "Articulation Work" ergeben können.

\section{Begriffsbestimmung}
\label{sec:mentalemodelle_begriffsbestimmung}

Der Begriff der Mentalen Modelle 

\subsection{Entwicklung des Begriffs}
\label{sub:entwicklung_des_begriffs}

\subsection{Veränderung mentaler Modelle}
\label{sub:veränderung_mentaler_modelle}
Assimilation vs. Akkommodation

\section{Mentale Modelle und Articulation Work}
\label{sec:mentale_modelle_und_articulation_work}

Argumentation mit Wissensspirale (Nonaka \& Takeuchi)

\section{Methoden zur Externalisierung mentaler Modelle}
\label{sec:methoden_zur_externalisierung_mentaler_modelle}

\subsection{Überblick}

\subsection{Concept Mapping}

\subsection{Strukturlegetechniken}

\section{Herausforderungen bei der Anwendung}
\label{sec:herausforderungen_bei_der_anwendung}

\subsection{Kollaborative Anwendbarkeit}

pro Strukturlegetechniken 

\subsection{Nachhaltige Verwendung der Information}

pro Concept Mapping

\subsection{Zusammenführung}

Offenheit des Conceptmapping

Strukturlegetechniken mit IT-Unterstützung

Nicht SLT in den Computer sondern Computer zur Unterstützung von SLT