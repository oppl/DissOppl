\chapter{Implementierung – Überblick} % (fold)
\label{cha:implementierung_Überblick}

Wie im Kapitel "Design" gefordert, wurde zur Umsetzung des Werkzeugs ein "Tangible Tabletop Interface" verwendet. Tabletop Interface zeichnen sich im Generellen dadurch aus, dass im Gegensatz zu handelsüblichen Rechnern nicht nur die Software sondern auch die Hardware applikationsspezifisch ist und nicht generisch eingesetzt werden kann. Die Hardware bildet dabei einen Teil oder die gesamte Benutzungsschnittstelle ab. Im speziellen Fall eines "Tangible Tabletop Interfaces" basiert der Benutzerinteraktion auf der Verwendung physischer Bausteine ("Tokens"), die auf der physischen Oberfläche des Interfaces manipuliert werden. Dieses Paradigma wird ergänzt von Tabletop Interfaces, die die Benutzerinteraktion ausschließlich auf Gesten bzw. Berührungen der Oberfläche abbilden (horizontal verbaute "Touch-" bzw. "Multi-Touch-Displays").

\section{Grundlegende \& verwandte Arbeiten} % (fold)
\label{sec:grundlegende_&_verwandte_arbeiten}

REFs!!! Die Entwicklung von Tabletop Interfaces begann Mitte der 1990er-Jahren mit den Arbeiten von Ishii \& Ullmer. Auch die erste Anwendung, die sich mit Modellierungs-Ansätzen mit Hilfe von Tabletop Interfaces konzentriert, stammt aus dieser Zeit. Mit dem fortschreiten der technologischen Entwicklung ist heute ein Status erreicht, in dem mit Hilfe generischer Identifikations-Frameworks schnell und ohne großen Aufwand Applikationen mit "tangiblen" Inputkanälen erstellt werden können. Zur Zeit noch im Prototypenstatus befinden sich Ansätze, die sich mit generischen Möglichkeiten des tangiblen Informationsoutputs beschäftigt. Der Rückkanal vom Rechner zum Benutzer wird heute zumeist mit der Projektion von Inhalten auf die Arbeitsoberfläche umgesetzt.

In den folgenden Abschnitten wird die historische Entwicklung von Tabletop Interfaces sowie der aktuelle Stand der Entwicklung im Anwendungsbereich dieser Arbeit betrachtet. Es werden dabei die grundlegenden Konzepte und Eigenschaften der jeweiligen Arbeiten betrachtet und das Potential hinsichtlich der Umsetzung von in Kapitel XY identifizierten Anforderungen an das hier entwickelte Werkzeug betrachtet. 

\subsection{Tangible Interfaces} % (fold)
\label{sub:tangible_interfaces}

Der Begriff der Tangible bzw. Graspable Interfaces – also der "berührbaren" oder "begreifbaren" Benutzungsschnittstellen — stammt aus der Mitte der neunziger Jahre des zwanzigsten Jahrhunderts. \citet{Fitzmaurice95} werden im Allgemeinen als die ersten betrachtet, die den Begriff des "Graspable User Interfaces" prägen und damit die Manipulierbarkeit digitaler Information durch physische Mittel beschreiben. \citet{Fitzmaurice96} präzisiert später den Begriff durch die Abgrenzung zwischen (herkömmlichen, maus-, tastatur- und bildschirmbasierenden) zeitlich gemultiplexten Schnittstellen, bei denen der Informationsaustausch zwischen Benutzer und System über einen Kanal zeitlich hintereinander erfolgt und den (neuartigen, berührbaren) räumlich gemultiplexten Schnittstellen, bei denen mehrere Kanäle gleichzeitig zur Interaktion zwischen Benutzer und System verwendet werden können. 

Der Begriff des "Tangible User Interfaces" wurde kurz danach bzw. parallel dazu von \citet{Ishii97} eingeführt. \citeauthor{Ishii97} verfolgen dabei bei der Definition den umgekehrten Weg und sprechen von einer "Augmentation der realen Welt durch eine Kopplung von digitaler Information and physische Objekte"\footnote{\emph{“augment the real physical world by coupling digital information to everyday physical objects and environments”}\citep{Ishii97}}. 

% subsection tangible_interfaces (end)

\subsection{Historische Entwicklung von Tabletop Interfaces} % (fold)
\label{sub:historische_entwicklung_von_tabletop_interfaces}

\paragraph{Sensetable} % (fold)
\label{par:sensetable}
Der Sensetable \citep{Patten01}
% paragraph sensetable (end)

\paragraph{BUILD-IT} % (fold)
\label{par:build_it}
\citep{Fjeld01}
% paragraph build_it (end)
% subsection historische_entwicklung_von_tabletop_interfaces (end)

\subsection{Tangible Interfaces zur Modellbildung} % (fold)
\label{sub:tangible_interfaces_zur_modellbildung}

% subsection tangible_interfaces_zur_modellbildung (end)

\subsection{Aktuelle verwandte Ansätze} % (fold)
\label{sub:aktuelle_verwandte_ansätze}

% subsection aktuelle_verwandte_ansätze (end)
\begin{itemize}
	\item Historische Entwicklung von Tabletop Interfaces
	\begin{itemize}
		\item Sensetable
		\item Morten Fjeld
		\item ReacTable
		\item Eva Hornecker
	\end{itemize}
	\item Historische Entwicklung von Tangible Interfaces zur Modellbildung
	\begin{itemize}
		\item Sensetable Modeling Application
		\item Designer's Outpost (Klemmer)
	\end{itemize}
	\item Aktuelle verwandte Ansätze
	\begin{itemize}
		\item Antle (TEI Mail-Pointer)
		\item Sun (TEI Demo)
	\end{itemize}
\end{itemize}


% section grundlegende_&_verwandte_arbeiten (end)

% chapter implementierung_Überblick (end)