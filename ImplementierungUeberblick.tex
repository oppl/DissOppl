\chapter{Grundlagen der Implementierung} % (fold)
\label{cha:implementierung_Überblick}

Wie im Kapitel "Design" gefordert, wurde zur Umsetzung des Werkzeugs ein "Tangible Tabletop Interface" verwendet. Tabletop Interface zeichnen sich im Generellen dadurch aus, dass im Gegensatz zu handelsüblichen Rechnern nicht nur die Software sondern auch die Hardware applikationsspezifisch ist und nicht generisch eingesetzt werden kann. Die Hardware bildet dabei einen Teil oder die gesamte Benutzungsschnittstelle ab. Im speziellen Fall eines "Tangible Tabletop Interfaces" basiert der Benutzerinteraktion auf der Verwendung physischer Bausteine ("Tokens"), die auf der physischen Oberfläche des Interfaces manipuliert werden. Dieses Paradigma wird ergänzt von Tabletop Interfaces, die die Benutzerinteraktion ausschließlich auf Gesten bzw. Berührungen der Oberfläche abbilden (horizontal verbaute "Touch-" bzw. "Multi-Touch-Displays").

REFs!!! Die Entwicklung von Tabletop Interfaces begann Mitte der 1990er-Jahren mit den Arbeiten von Ishii \& Ullmer. Auch die erste Anwendung, die sich mit Modellierungs-Ansätzen mit Hilfe von Tabletop Interfaces konzentriert, stammt aus dieser Zeit. Mit dem fortschreiten der technologischen Entwicklung ist heute ein Status erreicht, in dem mit Hilfe generischer Identifikations-Frameworks schnell und ohne großen Aufwand Applikationen mit "tangiblen" Inputkanälen erstellt werden können. Zur Zeit noch im Prototypenstatus befinden sich Ansätze, die sich mit generischen Möglichkeiten des tangiblen Informationsoutputs beschäftigt. Der Rückkanal vom Rechner zum Benutzer wird heute zumeist mit der Projektion von Inhalten auf die Arbeitsoberfläche umgesetzt.

In den folgenden Abschnitten wird die historische Entwicklung von Tabletop Interfaces sowie der aktuelle Stand der Entwicklung im Anwendungsbereich dieser Arbeit betrachtet. Es werden dabei die grundlegenden Konzepte und Eigenschaften der jeweiligen Arbeiten betrachtet und das Potential hinsichtlich der Umsetzung von in Kapitel XY identifizierten Anforderungen an das hier entwickelte Werkzeug betrachtet. 

\section{Entwicklung von Tangible Interfaces} % (fold)
\label{sec:tangible_interfaces}

Der Begriff der Tangible bzw. Graspable Interfaces – also der "berührbaren" oder "begreifbaren" Benutzungsschnittstellen — stammt aus der Mitte der neunziger Jahre des zwanzigsten Jahrhunderts. \citet{Fitzmaurice95} werden im Allgemeinen als die ersten betrachtet, die den Begriff des "Graspable User Interfaces" prägen und damit die Manipulierbarkeit digitaler Information durch physische Mittel beschreiben. \citet{Fitzmaurice96} präzisiert später den Begriff durch die Abgrenzung zwischen (herkömmlichen, maus-, tastatur- und bildschirmbasierenden) zeitlich gemultiplexten Schnittstellen, bei denen der Informationsaustausch zwischen Benutzer und System über einen Kanal zeitlich hintereinander erfolgt und den (neuartigen, berührbaren) räumlich gemultiplexten Schnittstellen, bei denen mehrere Kanäle gleichzeitig zur Interaktion zwischen Benutzer und System verwendet werden können. 

Der Begriff des "Tangible User Interfaces" wurde kurz danach bzw. parallel dazu von \citet{Ishii97} eingeführt. \citeauthor{Ishii97} verfolgen dabei bei der Definition den umgekehrten Weg und sprechen von einer "Augmentation der realen Welt durch eine Kopplung von digitaler Information and physische Objekte"\footnote{\emph{“augment the real physical world by coupling digital information to everyday physical objects and environments”}\citep{Ishii97}}. 

\subsection{Ubiquitous Computing}

\subsection{Augmented Reality} 

% subsection tangibles_historischer_hintergrund (end)

\section{Konzeptualisierung und Einteilung von Tangible Interfaces} % (fold)
\label{sec:konzeptualisierungen_von_tangible_interfaces}

Die Entwicklung des Forschungsgebiets der "Tangible Interfaces" wurde von mehreren konzeptuellen Arbeiten maßgeblich beeinflusst. Die dort vorschlagenen Erklärungsmodelle definieren das Gebiet und grenzen es gegenüber anderen Forschungsbereichen ab. Sie dienen außerdem als Grundlage für Erklärung und Konzeption konkreter Tangible Interfaces. Im Folgenden wird die historische Entwicklung dieser konzeptuellen Modelle beschrieben und auf deren Spezifika eingegangen.

Zur struktrierten Betrachtung von Tangible Interfaces ist es außerdem notwendig, jene Dimensionen zu identifizieren, an denen sich einzelne Tangible Interfaces einordnen und unterscheiden lassen. Die Ausprägungen dieser Dimensionen liefern kombiniert ein Begriffssystem, dass bei der Aufbereitung von unterschiedlichen Ansätzen im Bereich der Tangible Interface sowie deren Vergleich helfen kann. Die hier vorgestellten Ansätze tragen unterschiedlich detailliert und aus unterschiedlichen Gesichtspunkten zu dieser Thematik bei. Die einzelnen Ansätze werden hier dargestellt und in Kapitel XY auf das in dieser Arbeit entwickelte System angewandt um so das System-Design aus konzeptueller Sicht zu reflektieren und potentielle Verbesserungs- und Erweiterungsmöglichkeiten zu identifizieren.

Allgemein ist anzumerken, dass eine Vielzahl von Ausdrücken im sich entwickelnden Forschungsgebiet mehrfach belegt wurden und/oder nicht eindeutig definiert sind. Im Folgenden werden die Ausdrücke der jeweiligen Autoren übernommen, eine Interpretation bzw. Abbildung auf die Terminologie anderer Autoren wird nur vorgenommen, wo sie im jeweiligen Artikel explizit angeführt wurde. In der Zusammenfassung dieses Abschnitts wird versucht, die unterschiedlichen Terminologien nochmals zusammenzufassen und einen Satz an Ausdrücken festzulegen, der im Folgenden für diese Arbeit Anwendung findet.

-- evtl. Kurzüberblick (definition list) über die Ansätze, die jetzt kommen --

\subsection{Graspable User Interfaces}

\citeauthor{Fitzmaurice96} legt in jener Arbeit, in der es den Begriff des "Graspable User Inferfaces" prägt \citep{Fitzmaurice96}, auch Eigenschaften fest, anhand deren sich die "Graspability" einer Benutzungsschnittstelle zeigt und beurteilen lässt. Diese Beurteilung erfolgt auf einer generischen Skala mit Ausprägungen von "niedrig" bis "hoch", wobei "hohe" Werte in mehreren Eigenschaften auf eher hohe "Graspability" hinweist.

\subsubsection{Space Mulitplexing}
\subsubsection{Concurrency}
\subsubsection{Physical Form}
\subsubsection{Spartially aware}
\subsubsection{Spatial recofigurability}

\subsection{Tangible Bits}

\citep{Ishii97}

\subsection{Containers, Tokens und Tools}
\label{sub:containers_tokens_tools}

\citet{Holmquist99} legen ihre Arbeit als konzeptuelle Betrachtung von interaktiven Systemen an, in denen physische Objekte verwendet werden, um auf digitale Information zuzugreifen bzw. diese zu manipulieren. Das Einteilungsschema, das die Autoren vorschlagen, basiert auf der Art und Weise, in der Information an diese physischen Objekte gebunden ist. Grundsätzlich unterschieden sie zwischen \emph{Containern}, \emph{Tokens} und \emph{Tools}, wobei eine exakte Abgrenzung bzw. eindeutige Zuordung zu einer Kategorie nicht immer möglich und sinnvoll ist.

Der hier vorgestellte Ansatz fokussiert auf die physische Interaktion als Eingabemedium, auf den Aspekt der Informationsausgabe wird nicht eingegangen. Dies ist für das Verständnis der folgenden Beschreibungen im Kontext der späteren historischen Entwicklung wichtig und muss bei der Anwendung dieses Ansatzes berücksichtigt werden. 

\subsubsection{Containers}

Als Container werden alle jene Objekte bezeichnet, an die beliebige digitale Information gebunden werden kann. Eine Container ist also ein unspezifisches physisches Objekt in einem Tangible User Interface. Sein Aussehen oder andere physische Eigenschaften lasst keine Aussage über die Art der angebundenen Information bzw. die Information selbst zu. 

Beliebige physische Objekte können als Container agieren, sofern sie die Möglichkeit bieten, Information in bzw. auf ihnen abzulegen oder von einer Infrastruktur eindeutig identifizierbar sind, so dass Information über die eindeutige Identifikation an sie gebunden werden kann. Container agieren somit ausschließlich als physische Informationsträger und können als solche verwendet werden, um Information zwischen Systemen zu transportieren. 

Ein typisches Beispiel ist die Verwendung eines Füllfederhalters als Container, an den beliebige Information gebunden werden kann, um diese von einem Ort zum anderen transportieren zu können. Der Füllfederhalter steht in keinem direkten Zusammenhang mit der angebundenen Information, aus seinem Erscheinungsbild oder seinen Eigenschaften kann nicht auf die angebundene Information geschlossen werden.

\subsubsection{Tokens}

Tokens sind physische Objekte, deren äußeres Erscheinungsbild bzw. deren Eigenschaften in irgendeiner Weise mit der durch sie repräsentierten Information zusammenhängen. Das physische Objekte und die angebundene Information sind nicht mehr voneinander unabhängig sondern stehen in eine konzeptuellen Zusammenhang. Die äußere Form oder andere physische Eigenschaften dienen of als Hinweis auf die angebundenen Informationsart oder stehen sogar in Zusammenhang mit der konkrekt angebundenen Information.

Ein typisches Beispiel für ein Token wäre ein Buch, an das über eine eindeutige Identifikation (etwa ein \gls{RFID}-Tag) der jeweilige Text oder Zusatzmaterial gebunden wird. Das physische Buch steht dabei in dirketem Zusammenhang mit der angebundenen digitalen Information.

\subsubsection{Tools}
Tools sind physische Elemente, die nicht Information, sondern Funktionen repräsentieren. Die Anwendung von Tools hat dabei nicht unbedingt physische Auswirkungen, jene digitalen, virtuellen Objekte, auf die das Tool angewandt wird, werden aber entsprechend der Funktion des Tools manipuliert.

Beispiele für Tools sind physische Objekte, die zur Auswahl digitaler Objekte dienen oder Objekte wie Linsen, deren Anwendung zusätzliche Information zu anderen Containern oder Tokens abruft.

\subsubsection{Zugriff auf und Interaktion mit Tokens und Containern}

Der Zugriff auf die Information, die an ein Token oder einen Container gebunden ist, erfolgt über \emph{Information Faucets} (also "Informations-Zapfhähne" oder "-Armaturen"). Diese Faucets sind aktive Komponenten (im Gegensatz zu Tokens und Containern, die im Allgemeinen passive Komponenten sind, also keine dedizierte Elektronik enthalten), deren Aufgabe darin besteht, aus Tokens oder Containern, die in deren Reichweite gelangen, die angebundene Information zu extrahieren und auszugeben. Faucets können auch dazu verwendet werden, den Zugriff auf Information einzuschränken. So kann die Ausgabe von Information an eine bestimmte Kombination von Tokens oder Containern gebunden werden oder von einem bestimmten Aufenthaltsort abhängig gemacht werden.

Die Anbindung von Information an ein Token oder einen Container kann ebenfalls eingeschränkt sein, bzw. ist im Fall von Tokens per Definition durch den notwendigen Zusammenhang zwischen physischem Element und Information eingeschränkt. Neben dieser konzeptuell notwendigen Einschränkung können auch weitere Regeln geprüft werden oder z.B. die Bindung zwischen Objekt und Information statisch (d.h. unveränderbar) gespeichert werden. 

\subsection{Tangible Objects Meaning}
\citep{Underkoffler99}

\subsection{Das MCRpd Interaktions-Modell}
\citep{Ullmer00}

\subsection{Degree of Coherence}
\citep{Koleva03}

\subsection{Tokens und Constraints nach Shaer et al.}
\citep{Shaer04}

\subsection{Einteilung nach Klemmer, Li, Lin und Landay}
\citep{Klemmer04}

\subsection{Taxonomie nach Fishkin}
\label{sub:taxonomie_fishkin}

\citet{Fishkin04} versucht in seiner Arbeit, den Begriff des Tangible User Interfaces zu definieren und ein Kategorienschema zu schaffen, in das sich auf tangibler Interaktion beruhrende Systeme einordnen lassen. Sein Ziel ist es, ein Framework zur Verfügung zu stellen, auf Basis dessen sich Systeme vergleichen lassen und das das Design von Tangible Interfaces unterstützen kann.

\citeauthor{Fishkin04} fasst den Begriff des Tangible Interfaces sehr breit und definiert ein Interaktives System mit tangiblem Interface als eines, in dem die Eingabe über die Manipulation (im wörtlichen Sinn, also mit den Händen) von phyischen Objekten vorgenommen wird und die Ausgabe die physische Natur eines Objektes verändert. Diese Definition umfasst auch Systeme mit "herkömmlichen" Interfaces.

Nach dieser umfassenden Definition strukturiert \citeauthor{Fishkin04} den Raum möglicher Tangible Interfaces durch die Einführung zweier Analysedimensionen, anhand derer er seine  Taxonomie aufspannt. Diese beiden Dimensionen sind "Embodiment" und "Metaphor". Sie sind orthogonal zueinander und hohe Wert dieser beiden Dimensionen bezeichnen "tangiblere" Systeme. "Hohe Tangibilität" ist jedoch kein Qualitätskriterium sondern lediglich eine Eigenschaft, die ein System für einen bestimmten Anwendungsfall besser oder schlechter geeignet machen kann.

\subsubsection{Embodiment}
Die Dimension "Embodiment" beschreibt, wie eng die Eingabe am Interface mit der Ausgabe gekoppelt ist. Das Kriterium zu Einordnung ist hier der Ort der Wahrnehmbarkeit des Systemzustandes und der Systemaktivität. Je kohärenter die Ausgabe- und Eingabe-Kanäle sind, je näher sich die Informationsausgabe also bei der Eingabe befindet, desto höher ist die Ausprägung dieser Dimension. \citeauthor{Fishkin04} unterscheidet hier vier Ausprägungen:
\begin{description}
 \item[Full] Bei "full Embodiment" ist das Ausgabegerät gleichzeitig das Eingabegerät. Der Zustand des Geräts ist direkt in seinen physischen Eigenschaften abgebildet.
 \item[Nearby] "Nearby Embodiment" tritt auf, wenn die Ausgabe nahe dem Eingabeobjekt auftritt und eng an dieses gebunden ist, also in direktem, unmittelbaren Zusammenhang steht. 
 \item[Environmental] "Environmental Embodiment" ist gegeben, wenn die Ausgabe im unmittelbaren Umfeld des auftritt aber sich nicht unmittelbar am tangiblen Eingabeobjekt manifestiert. Typische Vertreter dieser Ausprägung sind akustische Ausgabekanäle.
 \item[Distant] Von "distant Embodiment" spricht man, wenn sich Ein- und Ausgabekanäle vollständig raumlich enthoppelt sind, der Fokus der Aufmerksamkeit der Benutzer also nicht gleichzeitig auf Ein- und Ausgabekanal liegen kann. 
 \end{description}

\subsubsection{Metaphor}

Die Dimension "Metaphor" bildet die Eigenschaft von Tangible Interfaces ab, auf eine Benutzerinteraktion so zu reagieren, wie die reale Welt auf eine entsprechende Aktion reagieren würde. Die Ausprägung in "Metaphor" ist also dann hoch, wenn das System analog zu realem, physiklisch begründbarem Verhalten reagiert. Hier sind grundsätzlich zwei Kategorien zu unterscheiden, in denen der Bezugspunkt der "Methaphor" verschieden ist. "Metaphor" kann sich entweder auf das Aussehen des jeweiligen Objektes beziehen oder auf die Bewegung des Objektes Bezug nehmen. Im ersten Fall spricht \citeauthor{Fishkin04} von "Metaphor of Noun", im zweiten Fall von "Metaphor of Verb". Die Ausprägungen auf der "Metaphor"-Dimension gruppieren sich dann wie Folgt:
\begin{description}
 \item[None] Die Interface-Objekte zeigen weder in Form noch Funktion eine Analogie zur Realität
 \item[Noun] Diese Analogie ist gegeben, wenn am Interface Objekte existieren, die eine reale Entsprechung haben, aber nicht wie diese manipuliert werden können. Ein klassisches Beispiel aus traditionellen interaktiven Systemen ist die "Fenster"- oder "Schreibtisch"-Metapher (sind analog zu realen Fenstern bzw. Schreibtischen ausgelegt, bieten aber andere Interaktionsmöglichkeiten). Bei Tangible User Interfaces ist diese Zuordnung dann gegeben, wenn ein Eingabeobjekt so aussieht wie ein Objekt der realen Welt, aber keine weiteren Eigenschaften mit diesem teilt.
 \item[Verb] Eine Zuordung zu dieser Kategorie erfolgt, wenn die Interaktion mit einem Objekt eine reale Entsprechung hat, dieses jedoch selbst keine Analogie zur realen Welt bildet. Diese Ausprägung tritt bei \glspl{TUI} unter anderem bei Gestensteuerung von Systemen auf.
 \item[Noun and Verb] Hier hat das betreffende Objekt selbst eine Entsprechung in der realen Welt und auch dessen Verwendung ist analog zu jener der realen Entsprechung. Die Objekte sind dennoch nach wie vor unterschiedlich, das reale Objekt kann nicht im Tangible Interface eingesetzt werden, umgekehrt bietet das \gls{TUI}-Objekt nicht die reale Funktionalität des realen Objektes. 
 \item[Full] In der höchsten Ausprägung existiert kein Unterschied zwischen \gls{TUI}-Objekt und realem Objekt - es gibt keine Analogie mehr, weil die Objekte identisch sind. Dieser Zustand ist erreicht, wenn Benutzer das TUI-Objekt manipulieren und sich die reale Welt entsprechend verändert. Beispiele für Systeme auf dieser Stufe sind zum Beispiel digital augmentierte Whiteboards, wo mit elektronischen Markern auf eine Oberfläche "geschrieben" wird, wobei die hinterlassene "Tinte" simultan projiziert wird.
 
 \end{description}

\subsubsection{Anwendung der Taxonomie}
\citeauthor{Fishkin04} wendet seine Taxonomie auf die oben bereits beschriebenen Ansätze von \citep{Holmquist99} und \citep{Underkoffler99} an und zeigt, dass sich diese einordnen lassen. Er ordnet nach einer umfassenden Literaturstudie außerdem über 60 konkrete Tangible Interfaces in seine Taxonomie ein und stellt diese anhand deren Ausprägungen gegenüber. Ein offener Punkt ist die Verbindung zum MCRpd-Framework \citep{Ullmer00}, dass \citeauthor{Fishkin04} als komplementär bezeichnet und das auf einer anderen Abstraktionsstufe operiere. 

\subsection{Tokens und Constraints nach Ullmer et al.}
\citep{Ullmer05}

\subsection{Tangible Bits: Beyond Pixels}
\citep{Ishii08}

\subsection{Zusammenfassung}

-- Grafik mit den Zusammenhängen und Bezugnahmen zwischen den einzelnen Arbeiten --
% section konzeptualisierungen_von_tangible_interfaces (end)

\section{Tangible Interfaces in kooperativer Verwendung} % (fold)
\label{sub:tangible_interfaces_in_kooperativer_verwendung}
\citep{Hornecker04}
% subsection tangible_interfaces_in_kooperativer_verwendung (end)

% section tangible_interfaces (end)

\section{Tabletop Interfaces} % (fold)
\label{sec:tabletop_interfaces}

Grundlagen

\subsection{Historische Entwicklung} % (fold)
\label{sub:historische_entwicklung_von_tabletop_interfaces}

\subsubsection{Sensetable} % (fold)
\label{subs:sensetable}
Der Sensetable \citep{Patten01}
% subsubsection sensetable (end)

\subsubsection{BUILD-IT} % (fold)
\label{par:build_it}
\citep{Fjeld01}
% subsubsection build_it (end)
% subsection historische_entwicklung_von_tabletop_interfaces (end)
% section tabletop_interface (end)

\section{Tangible Interfaces zur Modellbildung} % (fold)
\label{sub:tangible_interfaces_zur_modellbildung}

% subsection tangible_interfaces_zur_modellbildung (end)

\subsection{Aktuelle verwandte Ansätze} % (fold)
\label{sub:aktuelle_verwandte_ansätze}

% subsection aktuelle_verwandte_ansätze (end)
\begin{itemize}
	\item Historische Entwicklung von Tabletop Interfaces
	\begin{itemize}
		\item Sensetable
		\item Morten Fjeld
		\item ReacTable
		\item Eva Hornecker
	\end{itemize}
	\item Historische Entwicklung von Tangible Interfaces zur Modellbildung
	\begin{itemize}
		\item Sensetable Modeling Application
		\item Designer's Outpost (Klemmer)
	\end{itemize}
	\item Aktuelle verwandte Ansätze
	\begin{itemize}
		\item Antle (TEI Mail-Pointer)
		\item Sun (TEI Demo)
	\end{itemize}
\end{itemize}


% section grundlegende_&_verwandte_arbeiten (end)

% chapter implementierung_Überblick (end)