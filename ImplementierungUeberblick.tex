\chapter{Grundlagen der Implementierung} % (fold)
\label{cha:implementierung_Überblick}

Wie im Kapitel "Design" gefordert, wurde zur Umsetzung des Werkzeugs ein "Tangible Tabletop Interface" verwendet. Tabletop Interface zeichnen sich im Generellen dadurch aus, dass im Gegensatz zu handelsüblichen Rechnern nicht nur die Software sondern auch die Hardware applikationsspezifisch ist und nicht generisch eingesetzt werden kann. Die Hardware bildet dabei einen Teil oder die gesamte Benutzungsschnittstelle ab. Im speziellen Fall eines "Tangible Tabletop Interfaces" basiert der Benutzerinteraktion auf der Verwendung physischer Bausteine ("Tokens"), die auf der physischen Oberfläche des Interfaces manipuliert werden. Dieses Paradigma wird ergänzt von Tabletop Interfaces, die die Benutzerinteraktion ausschließlich auf Gesten bzw. Berührungen der Oberfläche abbilden (horizontal verbaute "Touch-" bzw. "Multi-Touch-Displays").

REFs!!! Die Entwicklung von Tabletop Interfaces begann Mitte der 1990er-Jahren mit den Arbeiten von Ishii \& Ullmer. Auch die erste Anwendung, die sich mit Modellierungs-Ansätzen mit Hilfe von Tabletop Interfaces konzentriert, stammt aus dieser Zeit. Mit dem fortschreiten der technologischen Entwicklung ist heute ein Status erreicht, in dem mit Hilfe generischer Identifikations-Frameworks schnell und ohne großen Aufwand Applikationen mit "tangiblen" Inputkanälen erstellt werden können. Zur Zeit noch im Prototypenstatus befinden sich Ansätze, die sich mit generischen Möglichkeiten des tangiblen Informationsoutputs beschäftigt. Der Rückkanal vom Rechner zum Benutzer wird heute zumeist mit der Projektion von Inhalten auf die Arbeitsoberfläche umgesetzt.

In den folgenden Abschnitten wird die historische Entwicklung von Tabletop Interfaces sowie der aktuelle Stand der Entwicklung im Anwendungsbereich dieser Arbeit betrachtet. Es werden dabei die grundlegenden Konzepte und Eigenschaften der jeweiligen Arbeiten betrachtet und das Potential hinsichtlich der Umsetzung von in Kapitel XY identifizierten Anforderungen an das hier entwickelte Werkzeug betrachtet. 

\section{Entwicklung von Tangible Interfaces} % (fold)
\label{sec:tangible_interfaces}

Der Begriff der Tangible bzw. Graspable Interfaces – also der "berührbaren" oder "begreifbaren" Benutzungsschnittstellen — stammt aus der Mitte der neunziger Jahre des zwanzigsten Jahrhunderts. \citet{Fitzmaurice95} werden im Allgemeinen als die ersten betrachtet, die den Begriff des "Graspable User Interfaces" prägen und damit die Manipulierbarkeit digitaler Information durch physische Mittel beschreiben. \citet{Fitzmaurice96} präzisiert später den Begriff durch die Abgrenzung zwischen (herkömmlichen, maus-, tastatur- und bildschirmbasierenden) zeitlich gemultiplexten Schnittstellen, bei denen der Informationsaustausch zwischen Benutzer und System über einen Kanal zeitlich hintereinander erfolgt und den (neuartigen, berührbaren) räumlich gemultiplexten Schnittstellen, bei denen mehrere Kanäle gleichzeitig zur Interaktion zwischen Benutzer und System verwendet werden können. 

Der Begriff des "Tangible User Interfaces" wurde kurz danach bzw. parallel dazu von \citet{Ishii97} eingeführt. \citeauthor{Ishii97} verfolgen dabei bei der Definition den umgekehrten Weg und sprechen von einer "Augmentation der realen Welt durch eine Kopplung von digitaler Information and physische Objekte"\footnote{\emph{“augment the real physical world by coupling digital information to everyday physical objects and environments”}\citep{Ishii97}}. 

\subsection{Ubiquitous Computing}

\subsection{Augmented Reality} 

% subsection tangibles_historischer_hintergrund (end)

\section{Konzeptualisierung und Einteilung von Tangible Interfaces} % (fold)
\label{sec:konzeptualisierungen_von_tangible_interfaces}

Die Entwicklung des Forschungsgebiets der "Tangible Interfaces" wurde von mehreren konzeptuellen Arbeiten maßgeblich beeinflusst. Die dort vorschlagenen Erklärungsmodelle definieren das Gebiet und grenzen es gegenüber anderen Forschungsbereichen ab. Sie dienen außerdem als Grundlage für Erklärung und Konzeption konkreter Tangible Interfaces. Im Folgenden wird die historische Entwicklung dieser konzeptuellen Modelle beschrieben und auf deren Spezifika eingegangen.

Zur struktrierten Betrachtung von Tangible Interfaces ist es außerdem notwendig, jene Dimensionen zu identifizieren, an denen sich einzelne Tangible Interfaces einordnen und unterscheiden lassen. Die Ausprägungen dieser Dimensionen liefern kombiniert ein Begriffssystem, dass bei der Aufbereitung von unterschiedlichen Ansätzen im Bereich der Tangible Interface sowie deren Vergleich helfen kann. Die hier vorgestellten Ansätze tragen unterschiedlich detailliert und aus unterschiedlichen Gesichtspunkten zu dieser Thematik bei. Die einzelnen Ansätze werden hier dargestellt und in Kapitel XY auf das in dieser Arbeit entwickelte System angewandt um so das System-Design aus konzeptueller Sicht zu reflektieren und potentielle Verbesserungs- und Erweiterungsmöglichkeiten zu identifizieren.

\subsection{Graspable User Interfaces}

\citeauthor{Fitzmaurice96} legt in jener Arbeit, in der es den Begriff des "Graspable User Inferfaces" prägt \citep{Fitzmaurice96}, auch Eigenschaften fest, anhand deren sich die "Graspability" einer Benutzungsschnittstelle zeigt und beurteilen lässt. Diese Beurteilung erfolgt auf einer generischen Skala mit Ausprägungen von "niedrig" bis "hoch", wobei "hohe" Werte in mehreren Eigenschaften auf eher hohe "Graspability" hinweist.

\subsubsection{Space Mulitplexing}
\subsubsection{Concurrency}
\subsubsection{Physical Form}
\subsubsection{Spartially aware}
\subsubsection{Spatial recofigurability}

\subsection{Tangible Bits}

\citep{Ishii97}

\subsection{Containers, Tokens und Tools}

\citep{Holmquist99} legen ihre Arbeit als konzeptuelle Betrachtung von interaktiven Systemen an, in denen physische Objekte verwendet werden, um auf digitale Information zuzugreifen bzw. diese zu manipulieren. Das Einteilungsschema, das die Autoren vorschlagen, basiert auf der Art und Weise, in der Information an diese physischen Objekte gebunden ist. 

\subsubsection{Containers}

\subsubsection{Tokens}

\subsubsection{Tools}

\subsection{Das MCRpd Interaktions-Modell}
\citep{Ullmer00}

\subsection{Degree of Coherence}
\citep{Koleva03}

\subsection{Tokens und Constraints nach Shaer et al.}
\citep{Shaer04}

\subsection{Einteilung nach Klemmer, Li, Lin und Landay}
\citep{Klemmer04}

\subsection{Taxonomie nach Fishkin}
\citep{Fishkin04}

\subsection{Tokens und Constraints nach Ullmer et al.}
\citep{Ullmer05}

\subsection{Tangible Bits: Beyond Pixels}
\citep{Ishii08}

% section konzeptualisierungen_von_tangible_interfaces (end)

\section{Tangible Interfaces in kooperativer Verwendung} % (fold)
\label{sub:tangible_interfaces_in_kooperativer_verwendung}
\citep{Hornecker04}
% subsection tangible_interfaces_in_kooperativer_verwendung (end)

% section tangible_interfaces (end)

\section{Tabletop Interfaces} % (fold)
\label{sec:tabletop_interfaces}

Grundlagen

\subsection{Historische Entwicklung} % (fold)
\label{sub:historische_entwicklung_von_tabletop_interfaces}

\subsubsection{Sensetable} % (fold)
\label{subs:sensetable}
Der Sensetable \citep{Patten01}
% subsubsection sensetable (end)

\subsubsection{BUILD-IT} % (fold)
\label{par:build_it}
\citep{Fjeld01}
% subsubsection build_it (end)
% subsection historische_entwicklung_von_tabletop_interfaces (end)
% section tabletop_interface (end)

\section{Tangible Interfaces zur Modellbildung} % (fold)
\label{sub:tangible_interfaces_zur_modellbildung}

% subsection tangible_interfaces_zur_modellbildung (end)

\subsection{Aktuelle verwandte Ansätze} % (fold)
\label{sub:aktuelle_verwandte_ansätze}

% subsection aktuelle_verwandte_ansätze (end)
\begin{itemize}
	\item Historische Entwicklung von Tabletop Interfaces
	\begin{itemize}
		\item Sensetable
		\item Morten Fjeld
		\item ReacTable
		\item Eva Hornecker
	\end{itemize}
	\item Historische Entwicklung von Tangible Interfaces zur Modellbildung
	\begin{itemize}
		\item Sensetable Modeling Application
		\item Designer's Outpost (Klemmer)
	\end{itemize}
	\item Aktuelle verwandte Ansätze
	\begin{itemize}
		\item Antle (TEI Mail-Pointer)
		\item Sun (TEI Demo)
	\end{itemize}
\end{itemize}


% section grundlegende_&_verwandte_arbeiten (end)

% chapter implementierung_Überblick (end)