\chapter{Implementierung – Überblick} % (fold)
\label{cha:implementierung_Überblick}

Wie im Kapitel "Design" gefordert, wurde zur Umsetzung des Werkzeugs ein "Tangible Tabletop Interface" verwendet. Tabletop Interface zeichnen sich im Generellen dadurch aus, dass im Gegensatz zu handelsüblichen Rechnern nicht nur die Software sondern auch die Hardware applikationsspezifisch ist und nicht generisch eingesetzt werden kann. Die Hardware bildet dabei einen Teil oder die gesamte Benutzungsschnittstelle ab. Im speziellen Fall eines "Tangible Tabletop Interfaces" basiert der Benutzerinteraktion auf der Verwendung physischer Bausteine ("Tokens"), die auf der physischen Oberfläche des Interfaces manipuliert werden. Dieses Paradigma wird ergänzt von Tabletop Interfaces, die die Benutzerinteraktion ausschließlich auf Gesten bzw. Berührungen der Oberfläche abbilden (horizontal verbaute "Touch-" bzw. "Multi-Touch-Displays").

\section{Grundlegende \& verwandte Arbeiten} % (fold)
\label{sec:grundlegende_&_verwandte_arbeiten}

REFs!!! Die Entwicklung von Tabletop Interfaces begann Mitte der 1990er-Jahren mit den Arbeiten von Ishii \& Ullmer. Auch die erste Anwendung, die sich mit Modellierungs-Ansätzen mit Hilfe von Tabletop Interfaces konzentriert, stammt aus dieser Zeit. Mit dem fortschreiten der technologischen Entwicklung ist heute ein Status erreicht, in dem mit Hilfe generischer Identifikationsframeworks schnell und ohne großen Aufwand Applikationen mit "tangiblen" Inputkanälen erstellt werden können. Źur Zeit noch im Prototypenstatus befinden sich Ansätze, die sich mit generischen Möglichkeiten des tangiblen Informationsoutputs beschäftigt. Der Rückkanal vom Rechner zum Benutzer wird heute zumeist mit der Projektion von Inhalten auf die Arbeitsoberfläche umgesetzt.

In den folgenden Abschnitten wird die historische Entwicklung von Tabletop Interfaces sowie der aktuelle Stand der Entwicklung im Anwendungsbereich dieser Arbeit betrachtet. Es werden dabei die grundlegenden Konzepte und Eigenschaften der jeweiligen Arbeiten betrachtet und das Potential hinsichlich der Umsetzung von in Kapitel XY identifizierten Anforderungen an das hier entwickelte Werkzeug betrachtet. 

\begin{itemize}
	\item Historische Entwicklung von Tabletop Interfaces
	\begin{itemize}
		\item Sensetable
		\item Morten Fjeld
		\item ReacTable
		\item Eva Hornecker
	\end{itemize}
	\item Historische Entwicklung von Tangible Interfaces zur Modellbildung
	\begin{itemize}
		\item Sensetable Modeling Application
		\item Designer's Outpost (Klemmer)
	\end{itemize}
	\item Aktuelle verwandte Ansätze
	\begin{itemize}
		\item Antle (TEI Mail-Pointer)
		\item Sun (TEI Demo)
	\end{itemize}
\end{itemize}


% section grundlegende_&_verwandte_arbeiten (end)

% chapter implementierung_Überblick (end)