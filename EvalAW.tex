\chapter{Evaluierung der durchgeführten Articulation Work} % (fold)
\label{cha:eval_aw}

Die Evaluierung der durchgeführten Articulation Work bildet den letzten Teil der empirischen Untersuchung und prüft die letztdenlich Eignung des Werkzeugs für den intendierten Verwendungszweck im Sinne der Zielsetzung. 

\section{Hypothesen} % (fold)
\label{sec:a_hypothesen}

In diesem Abschnitt werden die Hypothesen angeführt und begründet, die in diesem Teil der empirischen Untersuchung geprüft werden. Die im Folgenden beschriebenen Hypothesen gehen aber auf die Wirkung von „Articulation Work“ auf die reale Welt ein. Nicht Gegenstand der Untersuchung ist die Verwendung diagrammatischer Modelle zum Zwecke der Durchführung von „Articulation Work“ (siehe \ref{cha:eval_modell}) und die Wirkung des Werkzeugs bei der Durchführung (siehe \ref{cha:eval_werkzeug}).

\subsection{Konzeptuell begründete Hypothesen} % (fold)
\label{sub:a_konzeptionell_begründete_hypothesen}

Die folgenden Hypothesen sind unmittelbar aus der globalen Zielsetzung dieser Arbeit abgeleitet. Die Wirkung von „Articulation Work“ zeigt sich in der organisationalen Realität an der Durchführung der „Production Work“ \citet{Fujimura87}. Um die Wirkung der mit dem Werkzeug durchgeführten „Articulation Work“ zeigen zu können, muss deshalb neben dieser auch die „Production Work“ betrachtet werden. Die Überprüfung erfolgt dabei in zwei Schritten. 

Im ersten Schritt wird die Durchführung der „Articulation Work“ selbst betrachtet. Im Rahmen der Verwendung der Externalisierung von mentalen Modellen zum Zwecke der Durchführung von „Articulation Work“ ist es -- wie in Kapitel \ref{cha:methodik} ausgeführt und in Anforderung \ref{anf:kollaborative_und_unmittelbare_manipulierbarkeit_des_modells} abgebildet -- notwendig, eine kooperative Nutzung des unterstützenden Werkzeugs zu ermöglichen. Ein wesentlicher Schritt zur erfolgreichen Durchführung von „Articulation Work“ ist neben der eigentlichen Externalisierung (die in den ersten beiden Hypothesen des vorhergehenden Kapitels \ref{cha:eval_modell} abgebildet wurde) die Abstimmung der indviduellen mentalen Modelle der Beteiligten. „Abstimmung“ bedeutet hier einen Abgleich der indviduellen Verständnisse jener Arbeitsaspekte, die im Sinne von Kapitel \ref{cha:articulation_work} „problematisch“ sind bzw. enge Kooperation der Beteiligten in der „Production Work“ bedingen.

\begin{hyp}
	\label{hyp:abstimmung}
	Das Werkzeug unterstützt den Prozess der Abstimmung der individuellen Modelle zwischen Personen.
\end{hyp}

Der zweite Schritt der Untersuchung in diesem Teil der Evaluierung betrachtet letztendlich die durchgeführte Arbeit selbst. Das globale Ziel des hier vorgestellten Ansatzes ist die Unterstützung von Articulation Work. Ob diese tatsächlich erfolgreich durchgeführt wurde, zeigt sich an der Wirkung auf die zugehörige kooperativer Arbeit (die „Production Work“). Es ist also zu beurteilen, ob die Anwendung des Werkzeuges tatsächlich auf die jeweils betrachteten Arbeitsabläufe wirkt und welcher Natur diese Auswirkungen sind.

\begin{hyp}
	\label{hyp:wirkung}
	Die Anwendung des Werkzeugs hat Auswirkungen auf die Ergebnisse kooperativer Arbeit.
\end{hyp}

% subsection konzeptionell_begründete_hypothesen (end)

% section hypothesen (end)

\section{Untersuchungsdesign und Durchführung} % (fold)
\label{sec:a_untersuchungsdesign}

% section untersuchungsdesign (end)

\section{Ergebnisse} % (fold)
\label{sec:a_ergebnisse}

% section ergebnisse (end)

% chapter eval_aw (end)