\chapter{Evaluierung der durchgeführten Articulation Work} % (fold)
\label{cha:eval_aw}

Die Evaluierung der durchgeführten Articulation Work bildet den letzten Teil der empirischen Untersuchung und prüft die letztdenlich Eignung des Werkzeugs für den intendierten Verwendungszweck im Sinne der Zielsetzung. 

\section{Hypothesen} % (fold)
\label{sec:a_hypothesen}

In diesem Abschnitt werden die Hypothesen angeführt und begründet, die in diesem Teil der empirischen Untersuchung geprüft werden. Die im Folgenden beschriebenen Hypothesen gehen aber auf die Wirkung von Articulation Work auf die reale Welt ein. Nicht Gegenstand der Untersuchung ist die Durchführung der Articulation Work (siehe \ref{cha:eval_modell}) und die Wirkung des Werkzeugs bei der Durchführung (siehe \ref{cha:eval_werkzeug}).

\subsection{Konzeptuell begründete Hypothesen} % (fold)
\label{sub:a_konzeptionell_begründete_hypothesen}

Die folgenden Hypothesen sind unmittelbar aus der globalen Zielsetzung dieser Arbeit abgeleitet.

Die Wirkung von Articulation Work zeigt sich in der organisationalen Realität an der Durchführung der Production Work (REF Fujimura). Um die Wirkung der mit dem Werkzeug durchgeführten Articulation Work zeigen zu können, muss deshalb neben dieser auch die Production Work betrachtet werden. Die Überprüfung erfolgt dabei in zwei Schritten. Im ersten Schritt wird noch die Articulation Work selbst betrachtet. Der Weg zur Verbesserung der Production Work führt im hier vorgestellten Ansatz wie in Kapitel XY beschrieben über die Abstimmung der entsprechenden mentalen Modelle der beteiligten Personen. Zu beurteilen ist also, ob diese Abstimmung stattfindet und durch das Werkzeug unterstützt wird. 

\begin{hyp}
	\label{hyp:abstimmung}
	Das Werkzeug unterstützt den Prozess der Abstimmung zwischen Personen.
\end{hyp}

Der zweite Schritt der Untersuchung in diesem Teil der Evaluierung betrachtet letztendlich die durchgeführte Arbeit selbst. Das globale Ziel des hier vorgestellten Ansatzes ist die Unterstützung von Articulation Work. Ob diese tatsächlich erfolgreich durchgeführt wurde, zeigt sich an der Wirkung auf die zugehörige kooperativer Arbeit (die Production Work). Es ist also zu beurteilen, ob die Anwendung des Werkzeuges tatsächlich auf die jeweils betrachteten Arbeitsabläufe wirkt und welcher Natur diese Auswirkungen sind.

\begin{hyp}
	\label{hyp:wirkung}
	Die Anwendung des Werkzeugs hat Auswirkungen auf die Ergebnisse kooperativer Arbeit.
\end{hyp}

% subsection konzeptionell_begründete_hypothesen (end)

\subsection{Explorativ gebildete Hypothesen} % (fold)
\label{sub:a_explorativ_gebildete_hypothesen}

% subsection explorativ_gebildete_hypothesen (end)

% section hypothesen (end)

\section{Untersuchungsdesign und Durchführung} % (fold)
\label{sec:a_untersuchungsdesign}

% section untersuchungsdesign (end)

\section{Ergebnisse} % (fold)
\label{sec:a_ergebnisse}

% section ergebnisse (end)

% chapter eval_aw (end)