\chapter{Konzeptuelle Evaluierung} % (fold)
\label{cha:konzeptuelle_evaluierung}

\section{Betrachtung im Lichte des Tangible Bits Ansatzes} % (fold)
\label{sec:betrachtung_tangible_bits}

Grundlage der Betrachtungen in diesem Abschnitt ist das Konzept der Tangible Bits \citep{Ishii97}, der in Abschnitt \ref{sub:tangible_bits} beschrieben wird.

\subsection{Abbildung} % (fold)

Das hier vorgestellte Werkzeug kann hinsichtlich seiner Funktion als eine Instanz des Konzepts „Interactive Surface“ betrachtet werden. Die „Surface“ ist hierbei eine Tischoberfläche, auf der interagiert wird. Die im Rahmen der Beschreibung des „metaDESK“ \citep{Ullmer97} als Beispiel für eine „Interactive Surface“ eingeführten \gls{TUI}-Elemente finden zum Teil auch im hier vorgestellten Werkzeug Anwendung.

Die Modellierungstokens und einbettbaren Tokens des Werkzeugs sind \emph{Phicons}, also passive Träger von digitaler Information. Die Werkzeugtokens zur Manipulation des Modells entsprechen \emph{Phandles}, also Elemente, die dazu verwendet werden, digitale Information zu verändern bzw. festzulegen. Jene Werkzeugtokens, die der Steuerung der Systemfunktionen dienen, sind hingegen als \emph{Instruments} zu klassifizieren. \emph{Lenses} und \emph{Trays} kommen im Werkzeug nicht zum Einsatz.

Hinsichtlich der Metaphorik unterscheiden \citet{Ullmer97} zwischen unterschiedlichen Abstraktionsebenen von Phicons (\emph{generic} -- \emph{symbolic} -- \emph{model}), wobei im vorliegenden System ob der offenen Semantik die Modellierungstokens ausschließlich \emph{generic Phicons} sind bzw. sein können. Die Werkzeugtokens sind zumeist als \emph{symbolic Phicons}, im Falle des Löschtokens -- dem Radiergummi -- eher als \emph{model Phyicon} zu klassifizieren.

\subsection{Bewertung} % (fold)

Für die Bewertung des Werkzeugs ist vor allem dessen Gegenüberstellung zu den vorgeschlagenen Elementen einer „Interactive Surface“ von Interesse. Hier zeigt sich, das die unterschiedlichen Arten von Tokens, die im Werkzeug eingeführt wurden, feingranular auf die unterschiedlichen Element-Arten von \citep{Ishii97} abbildbar sind. Insbesondere die explizite Unterscheidung zwischen \emph{Phandles} und \emph{Instruments} ist eine Alleinstellungsmerkmal der hier vorgeschlagenen Systematik.

Eine mögliche Lücke, die Erweiterungspotential für das Werkzeug anzeigen könnte, ist die Abwesenheit von TUI-Elementen, die als \emph{Lenses} oder \emph{Trays} zu klassifizieren sind. Insbesondere \emph{Trays} erscheinen für die explizite Interaktion mit einzelnen Tokens -- etwa der Benennung oder der Einbettung von Zusatzinformation -- als geeignet. Die dazu notwendigen Interaktionsabläufe würden expliziter auf den Vorgang der Zuordnung von Information eingehen und sich stärken von anderen Interaktionen unterscheiden, die anderen Zwecken, z.B. der Herstellung von Verbindungen zwischen Modellierungstokens, dienen.

\section{Einordnung in das Ordnungssystem von Holmquist et al.}

Grundlage der Einordnung in diesem Abschnitt ist der Ansatz von \citep{Holmquist99}, der in Abschnitt \ref{sub:containers_tokens_tools} beschrieben wurde.

\subsection{Abbildung}

Die von \citeauthor{Holmquist99} verwendete Terminologie ist im Wesentlichen direkt auf jene abbildbar, die in dieser Arbeit verwendet wurde. Die Modellierungstokens und einbettbaren Tokens entsprechen im Wesentlichen \emph{Tokens}. Dies ist dadurch begründbar, dass die Art eines Modellierungstokens in einem Modell immer im gleichen Zusammenhang mit der Art der Information steht, die durch dieses repräsentiert wird. Eine Eigenschaft, die eher \emph{Containern} zuzuordnen ist, ist jedoch die dynamische Festlegbarkeit der Bedeutung einer Art von Modellierungstokens - die physischen Elemente ansich sind vor Beginn der Modellbildung generisch (also \emph{Container}), werden aber im Zuge der Modellierung mit Bedeutung belegt (die dann für alle Instanzen dieser Art von Modellierungstokens gilt) und sind dann eher als \emph{Tokens} zu klassifizieren. 

Die Werkzeugtokens des hier vorgestellten Systems entsprechen in ihrer Konzeption den \emph{Tools}. Sie manipulieren digitale Information, lösen Aktionen aus oder versetzten das System in einen anderen Zustand und entsprechen damit exakt der Definition von \emph{Tools}, die von den Autoren gegeben wird.

\emph{Information Faucets} sind im Kontext des hier vorgestellten Systems einerseits die Tischoberfläche, über die Information zu Modellierungstokens abgerufen werden kann, andererseits ist die Registrierungskamera ein klassisches Faucet im Sinne der Definition, da sie dem Abruf oder der Assoziation von Information an ein Token dient, sobald dieses in den Erfassungsbereich der Kamera gerät.

\subsection{Bewertung}

Die konzeptuellen Elemente des hier vorgestellten Systems sind also auf das Ordnungsstem von \citet{Holmquist99} abbildbar. Die Problematik der nicht eindeutigen Zuordnung von Modellierungstokens zur Kategorie \emph{Tokens} oder \emph{Constraints} ist einerseits auf eine der grundlegenden Design-Paradigmen des hier entwickelten Werkzeugs -- der Flexibiltät der Abbildung -- zurückzuführen, weist aber andererseits auch auf mögliches Verbesserungspotential hin.

Durch die Flexibilisierung nicht nur der Bindung zwischen physischen Elementen und digitaler Repräsentation sondern auch der Verwendung von unterschiedlichen physischen Elementen selbst könnten Modellierungstokens eher \emph{Token}-artiger werden. Indem Modellierende eigenen physische Elemente (auf ihrem Arbeitskontext) einbringen können, könnte die Erfassbarkeit der Bedeutung der physischen Repräsentation unter Umständen verbessert werden können.

\section{Einordnung in das Object-Meaning-Kontinuum} % (fold)
\label{sec:einordnung_in_das_object_meaning_kontinuum}

Grundlage der Einordnung in diesem Abschnitt ist der Ansatz von \citet{Underkoffler99}, der in Abschnitt \ref{sub:embodied_user_interfaces} beschrieben wurde.

\subsection{Abbildung} 

Das Object-Meaning-Kontinuum ist eines der ersten Ansätze, die die physischen Objekte eines \gls{TUI} nicht strikten Kategorien zuordnen sondern auf einem Kontinuum anordnen. Dabei wird kein kategorischer Unterschied zwischen informationsrepräsentierenden Objekten und Werkzeug-Objekten gemacht -- sowohl in der Mitte des Kontinuums als auch an den Enden verschwimmt die Grenzen zwischen Objekt als reiner Repräsentation und reinem Werkzeug. 

In der Folge werden die Objekte des Werkzeugs in das Kontinuum eingeordnet, wobei zur einfacheren Anwendbarkeit die entlang des Kontinuums von den Autoren definierten Ausprägungen verwendet werden.

Die Ausprägung \emph{object as noun} kommt im Werkzeug nicht zum Einsatz. Keines der physischen Elemente hat eine direkte Entsprechung in der realen Welt -- durch die geforderte Flexibilität der Abbildung wäre das auch nicht möglich. 

Die Elemente die zur Modellbildung verwendet werden -- also Modellierungstokens und einbettbare Tokens -- sind der Ausprägung \emph{object as attribute} zuzuordnen, da in die Farbe bzw. Form der Tokens Bedeutung (nämlich die Semantik des jeweiligen Art von Tokens) codiert ist. Bei Modellierungstokens ist zu beachten, dass die Zuordnung der Bedeutung dynamisch zu Laufzeit erfolgt, der physischen Eigenschaft des Objekts also erst durch die Benutzer konkret Bedeutung zugewiesen wird.

Keines der Objekte des Werkzeug ist aufgrund seines Designs der Ausprägung \emph{object as pure object} zuzuweisen. Diese Zuordnung kann jedoch dynamisch bei der Modellbildung für Modellierungstokens eintreten, wenn die Benutzer den unterschiedlichen Objektarten keine Bedeutung zuordnen und diese beliebig mit Information belegen. 

Die Tokens, die im Werkzeug nicht unmittelbar zur Modellbildung verwendet werden, verteilen sich zwischen den beiden funktional abstrahierten Ausprägungen des Kontinuums. Der Ausprägung \emph{Object as Verb} sind das Historiensteuerungs-Token und das Markierungstoken zur Herstellung einer gerichteten Verbindung zuzuordnen. Für das Historiensteuerungs-Token gilt dies, da dessen Drehbewegung zur zeitliche Navigation auf die Bewegungen eines Uhrzeigers abbildbar ist. Das Markierungstoken zur Herstellung einer gerichteten Verbindung weist eine Pfeilspitze als Grundfläche auf, wodurch eine physische Eigenschaft Hinweise auf die Funktion des Tokens gibt (hier allerdings grenzwertig, das die dreieckige Grundfläche nicht eindeutig als Pfeilspitze zu erkennen ist). Die übrigen Tokens sind eher im Bereich des \emph{object as reconfigurable tool} anzusiedeln, da ihrer äußere Form oder ander physische Eigenschaften keine Hinweise auf deren Funktionalität geben. Dies gilt für die allgemeinen Markierungstokens, das Snaphshot-Token und das Wieder\-herstellungs-Token. 

Einen Spezialfall bildet das Löschtoken, das mit dem Radiergummi als Repräsentation eine \emph{object as verb}-Einordnung suggeriert (Radiergummi zum Löschen von Verbindungen), tatsächlich das System aber lediglich in einen Löschmodus versetzt, in dem Verbindungen mittels anderer Interaktionsabläufe gelöscht werden können. Hinsichtlich seiner tatsächlichen Verwendung ist das Löschtoken also als \emph{object as reconfigurabe tool} zu klassifizieren.

\subsection{Bewertung} 

Das von den Autoren vorgeschlagene Kontinuum eignet sich, um die Elemente eines \gls{TUI} hinsichtlich deren Bedeutung und Verwendung einzuordnen. Diese Einordnung kann nützlich sein, um Elemente zu identifizieren, deren tatsächliche Verwendung im TUI nicht mit der wahrgenommenen Bedeutung übereinstimmt. Dazu müssen die Elemente unabhängig von der konkreten Implementierung klassifiziert werden (ggf. von nicht am Design und der Entwicklung beteiligten Personen) und das Ergebnis der umgesetzten Funktionalität gegenübergestellt werden.

Für das hier vorgestellte Werkzeug ist eine derartige Diskrepanz wie oben bereits beschrieben am Löschtoken zu erkennen. Dieses suggeriert eine Verwendbarkeit im Sinne von \emph{object as verb}, setzt aber tatsächlich die augenscheinliche Funktion (Löschen) nicht um (bzw. ist auf eine andere Funktion -- Löschmodus aktivieren -- abgebildet) und ist deshalb lediglich als \emph{object as reconfigurable tool} einzuordnen. Eine „Aufwertung“ des Löschtokens im Sinne einer Hinterlegung mit der tatsächlichen Lösch-Funkton würde eine erwartungskonforme Verwendbarkeit eher sicherstellen und so zur Verbesserung des Gesamtsystems beitragen.

% section einordnung_in_das_object_meaning_kontinuum (end)

\section{Betrachtung im Lichte des MCRpd-Modells} % (fold)
\label{sec:betrachtung_im_lichte_des_mcrpd_modells}

Grundlage der Einordnung in diesem Abschnitt ist der Ansatz von \citet{Ullmer00}, der in Abschnitt \ref{sub:mcrpd} beschrieben wurde.

\subsection{Abbildung}

Wird das erstellte Werkzeug dem \gls{MCRpd}-Modell gegenübergestellt, so ist erkennbar, das die Eigenschaften des Werkzeugs augenscheinlich nicht den Anforderungen des \gls{MCRpd}-Modells an ein Tangible User Interface genügen. Das Werkzeug verfügt über einen Ausgabekanal -- den Bildschirm -- der nicht an die physische Repräsentation gekoppelte ist. Bei näherer Betrachtung erscheint eine vollständige Einordnung jedoch argumentierbar. All jene Interaktionen, die mit der eigentlichen Modellierung zusammenhängen, genügen den Anforderungen des \gls{MCRpd}-Modells ohne Einschränkungen. Die Manipulation des \emph{Model} (im \gls{MCRpd}-Modell) erfolgt über die Tischoberfläche, die gleichzeitig dazu verwendet wird, den Systemzustand zu manifestieren. Dem \gls{MCRpd}-Modell entgegenzulaufen scheinen jene Interaktionsabläufe, die den sekundären Ausgabekanal einbeziehen. Dabei sind zwei Fälle zu unterscheiden. Bei der Einbettung von Information in Modellelemente wird die sekundäre Oberfläche zur Auswahl der anzubindenden Ressource und damit als \gls{GUI} benutzt. Eine Einordnung in das \gls{MCRpd}-Modell ist hier damit nicht möglich. Bei der Betrachtung der Modellierungshistorie wird die sekundäre Oberfläche als alleiniges Ausgabemedium benutzt, der Systemzustand wird durch das runde Navigationstoken auf der Oberfläche beeinflusst. Dies verletzt grundsätzlich den Aufbau des MCRpd-Modells, betrachtet man jedoch die daraus abgeleiteten Kern-Charakteristika von \glspl{TUI}, so kann festgestellt werden, das diese dennoch nicht verletzt sind. Das in Frage zu stellende Charakteristikum ist jene mit der Forderung nach Kopplung zwischen der physischen Repräsentation des \emph{Models} (\emph{REP-P}) und der intangiblen, digitalen Manifestation von Modellaspekten in der realen Welt (\emph{REP-D}). Die Autoren        fordern von dem Zusammenhand zwischen \emph{REP-P} und \emph{REP-D} jedoch ausschließlich, dass er \emph{„perceptually coupled“} sein müsse, die Kopplung also van den Benutzern als solche wahrgenommen werden müsse. Betrachtet man das runde Navigationstoken als \emph{REP-P} und die Ausgabe am sekundären Ausgabekanal als \emph{REP-D}, so ist diese Kopplung feststellbar, da sich \emph{REP-D} immer in Abhängigkeit von \emph{REP-P} verändert. Insofern ist das \gls{MCRpd}-Modell nicht verletzt, das Werkzeug weist die von den Autoren als Kern-Charakteristika von Tangible User Interfaces bezeichneten Eigenschaften auf.

Hinsichtlich der Kategorien von \glspl{TUI}, die von den Autoren festgelegt werden, ist das System der Kategorie \emph{relational} zuzuordnen. Das hier vorgestellte Werkzeug ist nicht \emph{spatial}, da die Position der verwendeten Tokens relativ zum Referenzrahmen (der Tischoberfläche) keine spezifische Bedeutung haben. Die Bedeutung ist viel mehr in den Beziehungen der Tokens untereinander codiert, was wiederum für ein \emph{relationales} System sprechen würde. Gleichzeitig kann damit die Kategorie \emph{associative} ausgeschlossen werden, da in System dieser Art keine Beziehungen zwischen Tokens berücksichtigt werden. Da die Beziehungen zwischen Tokens nur digital und nicht physisch abgebildet werden, ist die Bedingung für ein \emph{konstruierendes} System nicht erfüllt. \emph{Constructive} wäre das Werkzeug dann, wenn der Modellzustand vollständig durch physische Elemente und Verbindungen abgebildet wäre.

\subsection{Bewertung}

Das \emph{MCRpd}-Modell ist ein im Vergleich zu anderen Ansätzen eher abstraktes, konzeptuelles Modell zur Beschreibung eines Tangible User Interfaces. Trotzdem -- oder auch deswegen -- eignet es sich gut zur Reflexion der Eigenschaften eines \glspl{TUI} bzw. zur Prüfung der Konsistenz der vorgesehenen Benutzerinteraktionen.

Das Werkzeug konnte in die Logik des Modells eingeordnet werden, wobei bei der Beschreibung der Interaktion zur Steuerung der Modellierungshistorie verstärkter Argumentationsbedarf herrschte. Dies kann auf eine möglicherweise zu schwache Kopplung zwischen \emph{REP-P} und \emph{REP-D} hinweisen. Tatsächlich wird bei der Kontrolle der Modellierungshistorie auf der Tischoberfläche kein Feedback ausgegeben, ob das Steuerungs-Token erkannt wurde und in welchem Zustand es sich aktuell befindet. Die Kopplung könnte etwa in Form einer Darstellung des aktuell dargestellten Zeitpunkts in der Modellierungshistorie rund um das Kontroll-Token angezeigt werden, was die Kopplung zwischen den beiden Komponenten der Repräsentation verstärken würde.

% section betrachtung_im_lichte_des_mcrpd_modells (end)

\section{Einordnung in das Framework nach Koleva et al.} % (fold)
\label{sec:einordnung_in_das_framework_nach_koleva_et_al_}

Grundlage der Einordnung in diesem Abschnitt ist der Ansatzes von \citep{Koleva03}, der in Abschnitt \ref{sub:degree_of_coherence} beschrieben wurde.

\subsection{Abbildung} % (fold)
\label{sub:abbildung}

Das Framework eignet sich zur Einordnung einzelner Aspekte eines Tangible User Interfaces, aufgrund seiner Ausrichtung auf die Brücke zwischen realer und digitaler Welt insbesondere für die Betrachtung der eingesetzten Tokens und deren Verwendung zur Repräsentation und Manipulation des Systemzustandes. In Tabelle \ref{tab:degree_of_coherence} werden die Tokens in die Kategorien entlang des Kohärenz-Kontinuums eingeordnet und hinsichtlich der Eigenschaften ihrer Brückenfunktion in die digitale Welt betrachtet. 

\begin{table}[htbp]
	\centering
	\caption{Beurteilung des Werkzeugs hinsichtlich des Degree of Coherence}
	\begin{tabular}{| p{2cm} || p{2cm} | p{1,2cm} | p{2,5cm} | p{1,5cm} | p{1,2cm} | p{1,2cm} |} \hline
		Element & Kategorie & Trans\-for\-mation & Sensing of Inter\-action & Konfig\-urierbar\-keit & Lebens\-dauer & Auto\-nomie \\ \hline \hline
		Modell\-ierungs\-token  & Proxy & lit. & X-Y-Position und Rotation, Öffnungsstatus & fixiert & temp. & abh. \\ \hline
		einbett\-bares Token & Identifier & transf. & Präsenz, Container & fixiert & temp. & unabh. \\ \hline
		Mark\-ierungs\-token & Specialized Tool & transf. & X-Y-Position & fixiert & temp. & unabh. \\ \hline
		Löschtoken & Projection & transf. & Präsenz & fixiert & perm. & unabh. \\ \hline
		Snapshot\-token & Specialized Tool & transf. & Präsenz & fixiert & perm. & unabh. \\ \hline
		Historien\-navigations\-token & Specialized Tool & transf. & Rotation & fixiert & perm. & unabh. \\ \hline
		Wieder\-herstellungs\-token & Specialized Tool & transf. & Präsenz & fixiert & perm. & unabh. \\ \hline
	\end{tabular}
	\footnotesize lit. \ldots literally, transf. \ldots transformed, konfig. \ldots konfigurierbar, temp. \ldots temporär,\\ perm. \ldots permanent, abh. \ldots abhängig, unabh. \ldots unabhängig
	\label{tab:degree_of_coherence}
\end{table}

Die Kardinalität wurde hier nicht gesondert betrachtet, die die Kardinalität immer 1:1 ist, also eine eindeutige Zuordnung zwischen realem Objekt und digitaler Repräsentation gegeben ist. Im Übrigen verzichten auch \citet{Koleva03} auf die Einordnung in diese Kategorie, da sie generell nur geringen Unterscheidungswert hat. Hinsichtlich der Source of Link, die in der Tabelle ebenfalls nicht angegeben ist (und von den Autoren ebenfalls nicht verwendet wird), ist zu erwähnen, das das Werkzeug durchaus einen Aspekt aufweist, bei dem der Source of Link die digitale Welt ist. Im Rahmen der Wiederherstellungsunterstützung gibt das System Anweisungen zur Manipulation der realen Welt, wodurch sich der Informationsfluss umkehrt. Da jedoch kein physisches Element direkt manipuliert wird, ist eine Einordnung in das oben angeführte Schema nicht möglich (der Link ist lediglich indirekt vorhanden).

% subsection abbildung (end)

\subsection{Bewertung} % (fold)
\label{sub:bewertung}

Das von \citep{Koleva03} vorgeschlagene Framework ermöglicht die Klassifikation eines Tangible Interfaces über den Aspekt der Stärke der Bindung zwischen digitaler und realer Welt. Die Autoren nehmen damit eine zu diesem Zeitpunkt neue Perspektive ein, der noch keine große Aufmerksamkeit geschenkt wurde. Durch die Vernachlässigung der Interaktion am Tangible User Interface stellt eine Analyse unter Einsatz der im Framework vorgeschlagenen Kategorien und Merkmalen nur einen Teilaspekt des Gesamtsystems dar. Trotz dieser Einschränkung stellt das Framework ob seiner detaillieren Betrachtung der Eigenschaften der Verknüpfung von physischen Objekten mit digitaler Information einen potentiellen Mehrwert dar beim Design oder der Analyse von \glspl{TUI} dar. 

Insbesondere ermöglicht das Framework, nicht ausgeschöpftes Kohärenz-Potential zu identifizieren. Im konkreten Fall des hier vorgestellten Werkzeugs lässt sich das am Beispiel des Löschtokens zeigen. Dieses physisch durch einen Radiergummi repräsentierte Token wird im Moment lediglich als Schalter verwendet. Das System wird in den Löschmodus versetzt, sobald das Token auf der Oberfläche erkannt wird. Das Token ist als \emph{Projection} einzuordnen, die das Token mit der Information des aktivierten oder deaktivierten Löschmodus verbindet. Obwohl hoch kohärent, ist das Token trotzdem suboptimal eingesetzt, da es in der Praxis als Werkzeug wahrgenommen wird, das zum Löschen einer spezifischen Verbindung verwendet werden kann (\emph{Specialized Tool}). An diesem Beispiel lassen sich zwei Aspekte zeigen, die bei der Verwendung des Frameworks beachtet werden müssen. Zum einen ist hohe Kohärenz nicht für jeden Anwendungsfall anstrebenswert, da bei Werkzeugen im Allgemeinen eine nicht permanente Bindung verwendet wird. Zum anderen zeigt sich die Unvollständigkeit der Analyse mittels dem Framework, da die Metaphorik des physischen Elements, also seine Bedeutung in der Interaktion, nicht berücksichtigt wird. Beide Aspekte -- Kohärenz und Metaphorik -- berücksichtigt erst \citep{Fishkin04} in der von ihm vorgeschlagenen Taxonomie (siehe Abschnitt \ref{sub:taxonomie_fishkin} und \ref{sec:einordnung_in_die_taxonomie_von_fishkin}).

% subsection bewertung (end)
% section einordnung_in_das_framework_nach_koleva_et_al_ (end)

\section{Spezifikation des TAC-Schemas nach Shaer et al.} % (fold)
\label{sec:spezifikation_des_tac_schemas_nach_shaer_et_al_}

Grundlage der Einordnung in diesem Abschnitt ist der Ansatzes von \citet{Shaer04}, der in Abschnitt \ref{sub:tokens_und_constraints_nach_shaer_et_al_} beschrieben wurde.

\subsection{Abbildung} % (fold)

Das „Token and Constraints“-Schema (\gls{TAC}) erlaubt es, ein Tangible Interface sowohl hinsichtliche dessen Struktur als auch dessen Verwendung zu beschreiben. In Tabelle \ref{tab:tac} wird das Schema auf das hier vorgestellte Werkzeug angewandt. 

	\begin{longtable}{| p{0,8cm} || p{2,2cm} | p{2cm} || p{2cm} | p{2cm} | p{3cm} |} \caption{Spezifikation des Werkzeug mittels TAC-Schema}\label{tab:tac} \\ \hline	 
		TAC & \multicolumn{2}{|c||}{Struktur} & \multicolumn{3}{c|}{Verhalten} \\ 
		& Token & Constraint & Variable & Aktion & Feedback \\ \hline \hline
		\endfirsthead 
		\caption[]{(Fortsetzung)}\\ 
		\hline
			TAC & \multicolumn{2}{|c||}{Struktur} & \multicolumn{3}{c|}{Verhalten} \\ \hline 
			& Token & Constraint & Variable & Aktion & Feedback \\ \hline \hline
		\endhead
		\multirow{3}{*}{1} & Modell\-ierungs-  & Oberfläche & Modell\-element  & Auflegen & Modell\-element anzeigen \\ \cline{5-6} 
						   & token &			   &  & Bewegen & Modell\-element bewegen \\ \cline{5-6} 
						   & 	  &			   &		  & Entfernen & Modell\-element entfernen \\ \hline
		\multirow{2}{*}{2} & einbett\-bares Token & Modell\-ierungs-    & Modell\-element		  & Hinein\-legen & Daten einbetten \\ \cline{5-6}
						   &  & token   &  & Heraus\-nehmen & Container-Kopplung aufheben \\ \hline
		3 & einbett\-bares Token & Regist\-rierungs\-kamera & einbett\-bares Modell\-element & Vor die Kamera halten & ungebunden: Datenbindung auslösen; gebunden: Gebundene Daten anzeigen \\ \hline
		4 & Markierungs\-token & Modell\-ierungs\-token & Modell\-element & Neben Modell\-ierungs\-token platzieren	& Markierung anzeigen \\ \hline
		5 & Markierungs\-token & Modell\-ierungs\-token, markiertes Modell\-ierungs\-token & Verbindung & Neben unmarkiertem Modell\-ierungs\-token platzieren	& Verbindung herstellen und anzeigen \\ \hline  
		6 & Tastatur & markiertes Modell\-ierungs\-token & Modell\-element & Tastatur\-eingabe & Benennung des markierten Modellelements \\ \hline
		7 & Tastatur & Verbind\-ungen, kein markiertes Modell\-ierungs\-token & zuletzt hergestellte Verbindung & Tastatur\-eingabe & Benennung der zuletzt hergestellten Verbindung \\ \hline
		\multirow{2}{*}{8} & Lösch\-token & Oberfläche & Modell\-element & Auflegen & Löschmodus aktivieren \\ \cline{5-6}
		 				   &    	& 			 &  & Entfernen & Löschmodus deaktivieren \\ \hline
		9 & Snapshot\-token & Oberfläche & Modell\-ierungs\-historie & Auflegen & Aktuellen Modellzustand sichern, Blitz anzeigen \\ \hline
		\multirow{3}{*}{10} & Historien\-kontroll\-token & Oberfläche & Modell\-ierungs\-historie  & Auflegen & Letzten gespeicherten Snapshot anzeigen \\ \cline{5-6}
						   &   &			 &  & Drehen & Durch die gespeicherten Snapshots navigieren \\ \cline{5-6}
						   &   &			 &  & Entfernen & Aktuelles Modell anzeigen \\ \hline
		11 & Wieder\-herstellungs\-token & Oberfläche, vorhandenes Historien\-kontroll\-token & Modell\-zustand & Auflegen & Aktuell angezeigten Snapshot wiederherstellen \\ \hline

	\end{longtable}

Jene Interaktionsabläufe, bei denen das System die Aktionen der Benutzer anleitet (z.B. bei der Unterstützung der Wiederherstellung) können in diesem Schema nicht abgebildet werden, da die Constraints keine keine physischen Objekte sondern lediglich projizierte Information sind.

\subsection{Bewertung} % (fold)

Das \gls{TAC}-Schema eignet sich für eine umfassende Spezifikation des Struktur und des Verhaltens eines Tangible User Interfaces. Das vorgeschlagene Schema geht jedoch (wie die meisten anderen Ansätze auch) davon aus, dass das \gls{TUI} vor allem zur Informationseingabe verwendet wird und das sich der Systemzustand und dessen Manifestierung am Interface in Abhängigkeit dieser Eingaben ändern. Nicht abbildbar sind Interaktionen, die vom System ausgelöst bzw. kontrolliert werden, bei denen also die \emph{Variable} das aktive und nicht das manipulierte Element ist (im Gegensatz zum zuvor vorgestellten \emph{Degree of Coherence}-Ansatz der mit der \emph{Source of Link}-Eigenschaft explizit auf diesen Aspekt eingeht -- siehe Abschnitt \ref{sub:degree_of_coherence}).

Entsprechend dieser Einschränkung eignet sich das \gls{TAC}-Schema weitgehend für die Spezifikation des hier vorgeschlagenen Werkzeuges. Lediglich die Wiederherstellungsunterstützung kann nicht abgebildet werden, da sie vom System gesteuert wird. Beim Einsatz zur Spezifikation eines \gls{TUI} oder bei der Untersuchung desselben hinsichtlich möglichem Verbesserungspotential ist vor allem auf die möglichen Constraints eines Tokens zu achten. Dabei ist es hilfreich, unterschiedliche Constraints bezüglich der von ihnen vorgegebenen oder durch sie ermöglichten Aktionen zu betrachten. Als Beispiel im konkreten System kann wiederum das Löschtoken verwendet werden. Diese wird in der aktuellen Implementierung mit dem Constraint „Oberfläche“ verwendet, um den Löschmodus zu aktivieren (wenn es aufgelegt wird) bzw. zu deaktivieren (wenn es entfernt wird). Setzt man das Löschtoken nun in ein \gls{TAC} mit dem Constraint „Verbindung“, ergeben sich neue Möglichkeiten der Interaktion. Ein Aufsetzen des Löschtokens auf eine Verbindung könnte diese unmittelbar löschen und würde so den notwendigen Interaktionsablauf massiv vereinfachen. Das mit diesem Constraint auch die Metapher des verwendeten Radiergummis sinnbringend verwendet wird, ist ein Nebeneffekt, der jedoch im \gls{TAC}-Schema nicht repräsentiert wird. Vielmehr ist die Metaphorik Ausgangspunkt für eine sinnvolle und verständliche Auswahl möglicher Constraints für ein Token.


% section spezifikation_des_tac_schemas_nach_shaer_et_al_ (end)

\section{Einordnung in die Taxonomie von Fishkin} % (fold)
\label{sec:einordnung_in_die_taxonomie_von_fishkin}

Grundlage der Einordnung in diesem Abschnitt ist der Ansatzes von \citep{Fishkin04}, der in Abschnitt \ref{sub:taxonomie_fishkin} beschrieben wurde.

\subsection{Abbildung}
Das in dieser Arbeit entwickelte Werkzeug überspannt aufgrund seiner komplexen Struktur in beiden von \citeauthor{Fishkin04} vorgeschlagenen Dimensionen zur Klassifikation von Tangible Interfaces mehrere Ausprägungen. Um eine umfassende und ins Detail gehende Einordnung vornehmen zu können, werden im Folgenden Einzelaspekte des Systems betrachtet und eingeordnet. Während die Dimension "Embodiment" bereits in Kapitel \ref{cha:visualisierung} betrachtet wurde, um eine strukturierte Zuordnung der Ausgabekanäle vornehmen zu können, werden hier die einzelnen Funktionalitäten des Systems (siehe Abschnitt \ref{sec:benutzerinteraktion_mit_dem_werkzeug}) jeweils beiden Dimensionen zugeordnet (siehe Tabelle \ref{tab:einordnungFishkin})


	\begin{longtable}{| p{6cm} || p{3cm} | p{3cm} |} \hline
		 & Embodiment & Metaphor \\ \hline \hline
		Platzieren und Benennen von Modellelementen & distant, nearby (Tastatur), full (Haftnotiz) & verb (Tastatur), verb + noun (Haftnotiz) \\ \hline
		Erstellen von Verbindern & nearby & verb bis noun+verb (Werkzeugtokens), verb (räumliche Nähe)\\ \hline
		Löschen von Verbindern & environmental bis nearby & noun \\ \hline
		Einbetten von Information & full & noun + verb \\ \hline
		Abrufen von Information & distant & verb \\ \hline
		Erstellen von Snapshots & environmental bis nearby & none \\ \hline
		Navigation in der Modell-Historie & distant & verb \\ \hline
		Wiederherstellen eines Modell-Zustandes & nearby & noun + verb\\ \hline
			\caption{Einordnung des Systems in die Taxonomie nach Fishkin}
			\label{tab:einordnungFishkin}
		
	\end{longtable}

Beim \emph{Platzieren und Benennen von Modellelementen} ist die Benennung auf zwei Arten möglich, die unterschiedlich in die Taxonomie einzuordnen sind. Bei Benennung mittels Auswahl und Tastatur ist durch die Projektion der Benennung die Embodiment-Ausprägung "nearby" zu wählen. Der Vorgang der Auswahl und Benennung kann als analog zur realen Welt gesehen werden, die eingesetzten Werkzeuge sind aber generischer Natur -- Metaphor ist also als "verb" zu klassifizieren. Bei der Benennung mittels Haftnotitz ist durch die unmittelbar auf den Tokens angebrachten Benennungen Embodiment "full", Der Vorgang des Beschriftens wird analog zur realen Welt durchgeführt, auch die Informationsträger (Haftnotizen) entsprechen jenen der realen Welt, Metaphor ist also "verb + noun", wobei  der notwendige Vorgang der expliziten Erfassung einer Beschriftung durch das System eine Klassifikation "full" verhindert und sogar die Einstufung "noun + verb" etwas abschwächt (keine Analogie des Vorgangs zur realen Welt).

Zur \emph{Herstellung von Verbindern} existieren ebenfalls zwei Möglichkeiten. In beiden Fällen ist durch die Projektion der Verbindung die Ausprägung in Embodiment "nearby", sie unterscheiden sich jedoch hinsichtlich "Metaphor". Bei der Verwendung von Werkzeugtokens ist der Vorgang der Auswahl der Endpunkte analog zur realen Welt zu sehen und somit als "verb" einzustufen. Die Verwendung von spezifischen Werkzeugtokens zur Herstellung gerichteter Verbinder zeigt sogar Züge von "noun + verb", da die durch das Token dargestellte Pfeilspitze eine Analogie zur realen Welt bildet.

Das \emph{Löschen von Verbindern} wird durch das Lösch-Token vorgenommen. Dieses ist durch einen Radiergummi symbolisiert, der jedoch nicht als solche eingesetzt wird sondern das System nur in einen Löschmodus versetzt. Die Klassifikation in Metaphor ist demnach "noun". Die Visualisierung des Löschzustandes erfolgt unspezifisch durch die Umfärbung der gesamten Tischoberfläche, womit ein Embodiment von "nearby" oder "environmental" (aufgrund der Unspezifität) gerechtfertigt wäre.

\emph{Einbetten von Information} erfolgt durch die Verwendung der Modellierungstokens als Container und Hineinlegen von kleineren Tokens. Embodiment ist in diesem Fall "full", die die Einbettung physisch nachvollzogen wird. Metaphor ist durch die Analogie des "Hineinlegens" von Information in "Container" in die Ausprägung "noun + verb" einzuordnen.

Das \emph{Abrufen von Information} wird über den sekundären Ausgabekanal abgewickelt und ist daher in Embodiment als "distant" einzuordnen. Der Vorgang des Herausnehmens von Information aus einem Container existiert analog zur realen Welt, das bei diesem Vorgang im Zentrum stehende Objekt, das einbettbare Token, ist jedoch generisch und weist nicht auf die Art der eigebetteten Information hin. Eine Klassifikation von "verb" in Metaphor erscheint daher gerechtfertigt.

Beim \emph{Erstellen von Snapshots} wird die gesamte Tischoberfläche als Feedbackkanal genutzt. Insofern ist Embodiment wie im Falle des Löschens von Verbindern im Bereich "environmental" bis "nearby" anzusiedeln. Das Snapshot-Token selbst ist ein generisches Objekt, das keine Analogie zur realen Welt aufweist. Metaphor ist daher "none".

Die \emph{Navigation in der Modell-Hierarchie} erfolgt mit dem runden Navigations-Token. Zur Ausgabe der gespeicherten Modell-Zustände wird der sekundäre Ausgabekanal
verwendet. Embodiment ist deshalb "distant". Metaphor beschränkt sich auf "verb", da der Drehvorgang zur Navigation analog zum Einstellen einer Uhr erfolgt, das Token selbst aber bis auf seine runde Form generisch ist.

Das \emph{Wiederherstellen eines Modellzustandes} erfolgt durch spezifische Anweisungen auf der Modellierungsoberfläche. Embodiment ist also als "nearby" einzustufen. Der Vorgang der Wiederherstellung erfolgt durch Verschieben der Modellierungstokens, was im Wesentlichen analog zur realen Welt abläuft. Da unmittelbar die Objekte manipuliert werden, kann Metaphor als "noun + verb" eingestuft werden.

\subsection{Bewertung}

Die Taxonomie nach \citeauthor{Fishkin04} ermöglicht eine strukturierte Erfassung einzelner Aspekte eines Tangible User Interfaces. Eine aussagekräftige Gesamteinordnung ist nur bei einfachen \glspl{TUI} möglich, komplexe, mit vielen Interaktionsmöglichkeiten ausgestattete Systeme tendieren dazu, ein sehr breites Spektrum der Taxonomie abzudecken. Für die detaillierte Betrachtung eines komplexen Gesamtsystems erscheint die Taxonomie dennoch geeignet, da einerseits aus den einzelnen Teileinordnungen für den jeweiligen Anwendungsfall ggf. Verbesserungspotentiale abgeleitet werden können und andererseits (nach der Betrachtung des hier entwickelten Systems) scheint, als ob ein die Taxonomie breit abdeckendes Gesamtsystem potentiell Inkonsistenzen im Interaktionsdesign aufweist bzw. unterschiedliche Interaktionsparadigmen vermischt wurden. Vor allem "Ausreißer" aus einem vorwiegend einheitlichen Gesamtbild scheinen einer näheren Betrachtung hinsichtlich eines möglichen Redesigns wert.

Konkret können diese Vermutungen im vorliegenden System vor allem an der Konzeption des Lösch-Tokens und des Snapshot-Tokens festgemacht werden. Der Großteil der Interaktionen mit dem System beinhaltet in der Dimension Metaphor den "verb"-Aspekt (zu etwa gleichen Teilen ausschließlich und in der Kombination mit "noun"). Die Funktionalitäten, die die beiden erwähnten Tokens einbeziehen, laufen diesem Trend entgegen und zeigen in Metaphor die Ausprägung "noun" bzw. "none". Tatsächlich zeigt sich in der Praxis, das die die Anwendbarkeit dieser Tokens von Benutzern missverstanden bzw. nicht verstanden wird. Ein Redesign dieser Tokens mit expliziterer bzw. eher aktivitätsorientierter Metaphor erscheint deshalb untersuchenswert.

Zusammenfassend scheint die Taxonomie vor allem im Zusammenhang mit der Sicherung von konsistenter Interaktion an der Benutzungsschnittstelle sinnvoll anwendbar zu sein. Der Mehrwert des Ansatzes zeigt sich hier nicht so sehr in den absoluten Ausprägungen auf den beiden Dimensionen sondern vielmehr in den relativen Unterschieden, die zwischen den einzelnen Teilen des Tangible User Interfaces auftreten.

% section einordnung_in_die_taxonomie_von_fishkin (end)

\section{Zusammenfassung}

% chapter konzeptuelle_evaluierung (end)