\chapter{Konzeptuelle Evaluierung} % (fold)
\label{cha:konzeptuelle_evaluierung}

\section{Einordung in das Ordnungssystem von Holmquist et al.}

Grundlage der Einordung in diesem Abschnitt ist der Ansatzes von \citep{Holmquist99}, der in Abschnitt \ref{sub:containers_tokens_tools} beschrieben wurde.

\subsection{Abbildung}

Die von \citeauthor{Holmquist99} verwendete Terminologie ist im Wesentlichen direkt auf jene abbildbar, die in dieser Arbeit verwendet wurde. Die Modellierungstokens und einbettbaren Tokens entsprechen im Wesentlichen \emph{Tokens}. Dies ist dadurch begründbar, dass die Art eines Modellierungstokens in einem Modell immer im gleichen Zusammenhang mit der Art der Information steht, die durch dieses repräsentiert wird. Eine Eigenschaft, die eher \emph{Containern} zuzuordnen ist, ist jedoch die dynamische Festlegbarkeit der Bedeutung einer Art von Modellierungstokens - die physischen Elemente ansich sind vor Beginn der Modellbildung generisch (also \emph{Container}), werden aber im Zuge der Modellierung mit Bedeutung belegt (die dann für alle Instanzen dieser Art von Modellierungstokens gilt) und sind dann eher als \emph{Tokens} zu klassifizieren. 

Die Werkzeugtokens des hier vorgestellten Systems entsprechen in ihrer Konzeption den \emph{Tools}. Sie manipulieren digitale Information, lösen Aktionen aus oder versetzten das System in einen anderen Zustand und entsprechen damit exakt der Definition von \emph{Tools}, die von den Autoren gegeben wird.

\emph{Information Faucets} sind im Kontext des hier vorgestellten Systems einerseits die Tischoberfläche, über die Information zu Modellierungstokens abgerufen werden kann, andererseits ist die Registrierungskamera ein klassisches Faucet im Sinne der Definition, da sie dem Abruf oder der Assoziation von Information an ein Token dient, sobald dieses in den Erfassungsbereich der Kamera gerät.

\subsection{Bewertung}

Die konzeptuellen Elemente des hier vorgestellten Systems sind also auf das Ordnungsstem von \citet{Holmquist99} abbildbar. Die Problematik der nicht eindeutigen Zuordnung von Modellierungstokens zur Kategorie \emph{Tokens] oder \emph{Constraints} ist einerseits auf eine der grundlegenen Design-Paradigmen des hier entwickelten Werkzeugs -- der Flexibiltät der Abbildung -- zurückzuführen, weist aber andererseits auch auf mögliches Verbesserungspotential hin.

Durch die Flexibilisierung nicht nur der Bindung zwischen physischen Elementen und digitaler Repräsentation sondern auch der Verwendung von unterschiedlichen physischen Elementen selbst könnten Modellierungstokens eher \emph{Token}-artiger werden. Indem Modellierende eigenen physische Elemente (auf ihrem Arbeitskontext) einbringen können, könnte die Erfassbarkeit der Bedeutung der physischen Repräsentation unter Umständen verbessert werden können.

% chapter konzeptuelle_evaluierung (end)