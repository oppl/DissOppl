\chapter{Untersuchungsdesign} % (fold)
\label{cha:untersuchungsdesign}

Die Evaluierung des in der vorliegenden Arbeit beschriebenen Werkzeuges wurde entsprechend der in Kapitel \textbf{XY} vorgestellten konzeptuellen Zusammenhänge mit Fokus auf unterschiedliche Gesichtspunkte durchgeführt. Die beiden grundlegenden Untersuchungsfragen sind
\begin{itemize}
	\item Unterstützen Werkzeug und Methode Articulation Work?
	\item Ermöglichen und unterstützen die Teilwerkzeuge des Modellierungstisches Articulation Work?
\end{itemize}
Die erste Frage setzt im oberen Bereich der Argumentationskette an (\textbf{Bild einfügen}) und untersucht, ob Articulation Work bei Einsatz des Werkzeugs tatsächlich auftritt bzw. ob diese unterstützt wird. Die zweite Frage geht wiederum von der Unterstützung von Articulation Work aus, betrachtet hierbei jedoch den Beitrag der vorhandenen Teilwerkzeuge und zielt auf die Untersuchung der Übereinstimmung zwischen intendierter (bzw. aus den Anforderungen ableitbaren) und tatsächlicher Einsatzgebiete ab. Die diesen Fragen zugrunde liegenden Annahmen und die jeweiligen Ansätze zur Messung werden in den nächsten beiden Abschnitten behandelt. Der letzte Abschnitt dieses Kapitels beschreibt dann die konkrete Planung der Untersuchung und die im Einzelnen durchgeführten Untersuchungs-Phasen. 

\section{Frage 1 – Unterstützung von Articulation Work} % (fold)
\label{sec:frage_1_unterstützung_von_articulation_work}

Axiom 1:
Erfolgreiche Articulation Work zeigt sich an der Production Work (-> Ref. Strauss)

Axiom 1,5:
Erfolgreiche Production Work zeigt sich an der Zielerreichung (-> Erfolg) (-> Ref. Fujimura)

Axiom 2: 
(Geschäfts-)Erfolg steht in direktem Zusammenhang mit funktionierender Interaktion (-> Ref. 

Messung:
Die Unterstützung der Articulation Work kann an Ihren Auswirkungen auf die Production Work gemessen werden, dort im speziellen an der Qualität der Interaktion ("Work rests ultimately on Interaction"). Messpunkte können dabei die Akteure oder die Ergebnisse der kollaborativen Arbeit sein.

Misst man an den Akteuren, so kann die subjektive Zufriedenheit mit dem Arbeitsprozess (Verlauf der Interaktion) und/oder das beobachtbare Verhalten der Akteure als Merkmal herangezogen werden. Ersteres kann methodisch durch qualitative Interviews (ggf. unterstützt durch Modelle -> Herrmann, Jahnke 2008) beurteilt werden. Zweiteres kann durch Techniken der Interaktionsanalyse bewertet werden, wobei insbesondere die Gegenüberstellung der tatsächlichen zu den vereinbarten Interaktionsmodalitäten und die Veränderung dieser Vereinbarung über die Zeit von Interesse ist. Dazu ist eine Externalisierung der Vereinbarungen zu unterschiedlichen Zeitpunkten notwendig. 

Misst man an den Ergebnissen der kollaborativen Arbeit, so kann die subjektive Zufriedenheit mit dem Ergebnis und die "verobjektivierte" (Experten)-Beurteilung der Qualität des Ergebnisses als Merkmal verwendet werden. Ersters wird wiederum qualitativ in Befragungen zu erheben sein. Die Qualität des Ergebnisses wird von Experten an noch zu definierenden Merkmalen gemessen werden, an denen sich die Interaktion bei der Erstellung zeigt (etwa: Stilbrüche, etc.). Bei der Beurteilung der Qualität der Ergebnisse ist vor allem die Gegenüberstellung zu Ergebnissen von Interesse, die ohne Unterstützung der Articulation Work entstanden sind. Dementsprechend kann der Einsatz einer Kontrollgruppe sinnvoll sein.

\section{Frage 2 – Beitrag und Verwendung der Teilwerkzeuge} % (fold)
\label{sec:frage_2_beitrag_und_verwendung_der_teilwerkzeuge}

Axiom: Articulation Work kann durch Strukturlegetechniken unterstützt werden

Messung:
Die Verwendung der einzelnen Werkzeuge kann durch Beobachtung und Befragung modellierender Personen sowie der Untersuchung der Modellierungsergebnisse beurteilt werden. Messpunkte sind damit wiederum die Akture und die Ergebnisse der Artikulation. Bei der Messung ist in diesem Zusammenhang kein kollaboratives Setting notwendig, da nicht die Auswirkungen des Werkzeugs auf Interaktion sondern die Verwendung des Werkzeugs selbst untersucht werden soll. Die Modellierung wird also von einzelnen Personen durchgeführt (Selbst-Artikulation, Externalisierung eignere mentaler Modelle). Durch Auswertung der Modellierungen aus Evaluierung I lässt sich ein etwaiger Unterschied in der Verwendung der Werkzeuge bei kollaborativen Settings belegen.

% section frage_2_beitrag_und_verwendung_der_teilwerkzeuge (end)
% section frage_1_unterstützung_von_articulation_work (end)

\section{Untersuchungsablauf} % (fold)
\label{sec:untersuchungsablauf}

Die Evaluierung selbst wurde in drei Phasen gegliedert, deren Untersuchungsfokus auf 
\begin{itemize}
	\item der Verwendbarkeit des Werkzeugs an sich,  
	\item dessen Eignung zur Unterstützung von "Articulation Work" in Lehr- und Lern-Szenarien und
	\item dessen Eignung zur Unterstützung von "Articulation Work" beim Einsatz in Unternehmen
\end{itemize}
lag. Die erste Phase ist als Vorlauf zu sehen, der nicht unmittelbar der Beantwortung der Untersuchungsfragen diente, sondern auf die rein explorative Untersuchung der tatsächlichen Verwendbarkeit des Werkzeuges (im Sinne der Verständlichkeit und Robustheit) abzielte. Die beiden folgenden Phasen decken die Bearbeitung der eigentlichen Untersuchungsfragen ab, wobei durch den unterschiedlichen Einsatzkontext versucht wurde die verschiedenen Anwendungsgebiete des Werkzeugs in die Untersuchung einfließen zu lassen.

Im Zuge der Evaluierungen wurde das Werkzeug unter realen Einsatzbedingungen getestet. Unter "real" ist hier zu verstehen, dass die testenden Benutzer mit der Bedienung des Werkzeugs vorab nicht vertraut waren und dass die Aufgabenstellung stets einen konkreten Bezug zu einem für die jeweiligen Personen relevanten Arbeitsszenario aufwies. In den folgenden Abschnitten wird im Detail auf die Konzeption der einzelnen Phasen und den jeweiligen Beitrag zur Beantwortung der Untersuchungsfragen eingegangen.

\subsection{Phase 1 – Verwendbarkeit} % (fold)
\label{sub:phase_1_verwendbarkeit}

% subsection phase_1_verwendbarkeit (end)

\subsection{Phase 2 – Lehr- und Lern-Szenarien} % (fold)
\label{sub:phase_2_lehr_und_lern_szenarien}


% subsection phase_2_lehr_und_lern_szenarien (end)

\subsection{Phase 3 – Unternehmenseinsatz} % (fold)
\label{sub:phase_3_unternehmenseinsatz}

% subsection phase_3_unternehmenseinsatz (end)
% section untersuchungsablauf (end)

% chapter untersuchungsdesign (end)

\chapter{Untersuchungsergebnisse} % (fold)
\label{cha:untersuchungsergebnisse}

\section{Erhobene Daten} % (fold)
\label{sec:erhobene_daten}

\subsection{Phase 1} % (fold)
\label{sub:phase_1}

In Phase 1 wurden 9 Modellierungsdurchgänge mit insgesamt 18 Personen durchgeführt. An dem vorangegangenen Pretest nahmen 12 Personen teil.
% subsection phase_1 (end)

\subsection{Phase 2} % (fold)
\label{sub:phase_2}

In Phase 2 wurden Untersuchungen im Rahmen zweier Lehrveranstaltungen durchgeführt. An der ersten Untersuchung nahmen 18 Studierende der Wirtschaftsinformatik teil, die in Gruppen zu 2 Personen insgesamt 17 Modellierungsdurchgänge durchführten. An der zweiten Untersuchung nahmen 54 Studierende in Gruppen zu 3 Personen an insgesamt 18 Modellierungsdurchgängen teil.
% subsection phase_2 (end)

\subsection{Phase 3} % (fold)
\label{sub:phase_3}

% subsection phase_3 (end)
% section erhobene_daten (end)

\section{Auswertung \& Interpretation} % (fold)
\label{sec:auswertung_&_interpretation}

% section auswertung_&_interpretation (end)
% chapter untersuchungsergebnisse (end)