\part{Evaluierung des Instruments} % (fold)
\label{prt:evaluierung}

\section*{Einleitung} % (fold)
\label{sec:evaluierung_einleitung}
\thispagestyle{empty}

\markboth{Einleitung}{Evaluierung des Instruments}

Nach der nun erfolgten Beschreibung der Umsetzung des Werkzeugs wird in diesem Teil die Überprüfung der Verwendbarkeit des entwickelten Instruments und seiner Effekte behandelt. Ziel dieser Arbeit ist es, die effektive Durchführung explizite „Articulation Work“ zu unterstützen. Ziel dieses Teils ist, diese Anforderung hinsichtlich ihrer Erfüllung oder Nicht-Erfüllung zu überprüfen und damit die Beantwortung zweite in Kapitel \ref{cha:einführung} formulierte Forschungsfrage zu vervollständigen.

Wie in Kapitel \ref{cha:mentale_modelle} argumentiert, führt ein möglicher Weg zur Unterstützung expliziter „Articulation Work“ über die (kollaborative) Externalisierung mentaler Modelle. Die aus dieser Externalisierung resultierenden Strukturen sind ihrerseits diagrammatische Modelle. Die Qualität dieser Modelle ist vielschichtig bewertbar, im Kontext des hier verfolgten Verwendungszwecks sind aber einige Bewertungsdimensionen identifizierbar, die bei der Unterstützung von „Articulation Work“ relevant sind. Diese Dimensionen werden ebenfalls hinsichtlich ihrer Ausprägung im hier vorgestellten Werkzeug zu bewerten sein. 

Letztendlich wird die Externalisierung mentaler Modelle technisch durch ein Tabletop Interface unterstützt. Auch die technische Umsetzung bzw. deren Verständlichkeit und Verwendbarkeit hat Auswirkungen auf den Erfolg der Externalisierung und damit der durchgeführten „Articulation Work“. Das Werkzeug selbst und seine Nutzung muss also ebenfalls untersucht und im Kontext der Anforderungen in dieser Arbeit bewertet werden. 

Entsprechend dieser Ausführungen wurde die hier beschriebenen Untersuchung durchgeführt. Sie gliedert sich in drei Teile, die sich mit dem Werkzeug selbst, den erstellten Modellen und den Auswirkungen durchgeführten expliziten „Articulation Work“ auseinandersetzen. Die Struktur dieses Teiles spiegelt diese Aufteilung wieder. In Kapitel \ref{cha:konzeptuelle_evaluierung} wird das Werkzeug aus Sicht seiner Eigenschaften als Tabletop Interface theoretisch-konzeptuell betrachtet und aus den in diesen Bereich verfügbaren Analyse- und Beschreibungsframeworks mögliches Verbesserungspotential identifiziert. Kapitel \ref{cha:eval_ueberblick} beschreibt die Grundlagen der empirischen Untersuchung, in der das hier entwickelte Werkzeug hinsichtlich seiner Verwendbarkeit und Wirkung untersucht wurde. In Kapitel \ref{cha:eval_werkzeug} werden Design und Umsetzung jener Tests beschrieben, in denen die Benutzbarkeit und Verständlichkeit des Werkzeugs überprüft wurden. Die Überprüfung der Effekte des Werkzeugs beginnt im darauf folgenden Kapitel \ref{cha:eval_modell}, in dem die erstellten Modelle und deren Entstehungsprozess Gegenstand der Betrachtung sind. Letztendlich wird in Kapitel \ref{cha:eval_aw} auf die Wirkung des Werkzeugs auf „Articulation Work“ und damit letztendlich auf die im jeweiligen Anwendungsfall zu erzielenden Effekte eingegangen.

% section evaluierung_einleitung (end)

\automark[section]{chapter} 

\chapter{Konzeptuelle Evaluierung} % (fold)
\label{cha:konzeptuelle_evaluierung}

\section{Betrachtung im Lichte des Tangible Bits Ansatzes} % (fold)
\label{sec:betrachtung_tangible_bits}

Grundlage der Betrachtungen in diesem Abschnitt ist das Konzept der Tangible Bits \citep{Ishii97}, der in Abschnitt \ref{sub:tangible_bits} beschrieben wird.

\subsection{Abbildung} % (fold)

Das hier vorgestellte Werkzeug kann hinsichtlich seiner Funktion als eine Instanz des Konzepts „Interactive Surface“ betrachtet werden. Die „Surface“ ist hierbei eine Tischoberfläche, auf der interagiert wird. Die im Rahmen der Beschreibung des „metaDESK“ \citep{Ullmer97} als Beispiel für eine „Interactive Surface“ eingeführten \gls{TUI}-Elemente finden zum Teil auch im hier vorgestellten Werkzeug Anwendung.

Die Modellierungstokens und einbettbaren Tokens des Werkzeugs sind \emph{Phicons}, also passive Träger von digitaler Information. Die Werkzeugtokens zur Manipulation des Modells entsprechen \emph{Phandles}, also Elemente, die dazu verwendet werden, digitale Information zu verändern bzw. festzulegen. Jene Werkzeugtokens, die der Steuerung der Systemfunktionen dienen, sind hingegen als \emph{Instruments} zu klassifizieren. \emph{Lenses} und \emph{Trays} kommen im Werkzeug nicht zum Einsatz.

Hinsichtlich der Metaphorik unterscheiden \citet{Ullmer97} zwischen unterschiedlichen Abstraktionsebenen von Phicons (\emph{generic} -- \emph{symbolic} -- \emph{model}), wobei im vorliegenden System ob der offenen Semantik die Modellierungstokens ausschließlich \emph{generic Phicons} sind bzw. sein können. Die Werkzeugtokens sind zumeist als \emph{symbolic Phicons}, im Falle des Löschtokens -- dem Radiergummi -- eher als \emph{model Phyicon} zu klassifizieren.

\subsection{Bewertung} % (fold)

Für die Bewertung des Werkzeugs ist vor allem dessen Gegenüberstellung zu den vorgeschlagenen Elementen einer „Interactive Surface“ von Interesse. Hier zeigt sich, das die unterschiedlichen Arten von Tokens, die im Werkzeug eingeführt wurden, feingranular auf die unterschiedlichen Element-Arten von \citep{Ishii97} abbildbar sind. Insbesondere die explizite Unterscheidung zwischen \emph{Phandles} und \emph{Instruments} ist eine Alleinstellungsmerkmal der hier vorgeschlagenen Systematik.

Eine mögliche Lücke, die Erweiterungspotential für das Werkzeug anzeigen könnte, ist die Abwesenheit von TUI-Elementen, die als \emph{Lenses} oder \emph{Trays} zu klassifizieren sind. Insbesondere \emph{Trays} erscheinen für die explizite Interaktion mit einzelnen Tokens -- etwa der Benennung oder der Einbettung von Zusatzinformation -- als geeignet. Die dazu notwendigen Interaktionsabläufe würden expliziter auf den Vorgang der Zuordnung von Information eingehen und sich stärken von anderen Interaktionen unterscheiden, die anderen Zwecken, z.B. der Herstellung von Verbindungen zwischen Modellierungstokens, dienen.

\section{Einordnung in das Ordnungssystem von Holmquist et al.}

Grundlage der Einordnung in diesem Abschnitt ist der Ansatz von \citep{Holmquist99}, der in Abschnitt \ref{sub:containers_tokens_tools} beschrieben wurde.

\subsection{Abbildung}

Die von \citeauthor{Holmquist99} verwendete Terminologie ist im Wesentlichen direkt auf jene abbildbar, die in dieser Arbeit verwendet wurde. Die Modellierungstokens und einbettbaren Tokens entsprechen im Wesentlichen \emph{Tokens}. Dies ist dadurch begründbar, dass die Art eines Modellierungstokens in einem Modell immer im gleichen Zusammenhang mit der Art der Information steht, die durch dieses repräsentiert wird. Eine Eigenschaft, die eher \emph{Containern} zuzuordnen ist, ist jedoch die dynamische Festlegbarkeit der Bedeutung einer Art von Modellierungstokens - die physischen Elemente ansich sind vor Beginn der Modellbildung generisch (also \emph{Container}), werden aber im Zuge der Modellierung mit Bedeutung belegt (die dann für alle Instanzen dieser Art von Modellierungstokens gilt) und sind dann eher als \emph{Tokens} zu klassifizieren. 

Die Werkzeugtokens des hier vorgestellten Systems entsprechen in ihrer Konzeption den \emph{Tools}. Sie manipulieren digitale Information, lösen Aktionen aus oder versetzten das System in einen anderen Zustand und entsprechen damit exakt der Definition von \emph{Tools}, die von den Autoren gegeben wird.

\emph{Information Faucets} sind im Kontext des hier vorgestellten Systems einerseits die Tischoberfläche, über die Information zu Modellierungstokens abgerufen werden kann, andererseits ist die Registrierungskamera ein klassisches Faucet im Sinne der Definition, da sie dem Abruf oder der Assoziation von Information an ein Token dient, sobald dieses in den Erfassungsbereich der Kamera gerät.

\subsection{Bewertung}

Die konzeptuellen Elemente des hier vorgestellten Systems sind also auf das Ordnungsstem von \citet{Holmquist99} abbildbar. Die Problematik der nicht eindeutigen Zuordnung von Modellierungstokens zur Kategorie \emph{Tokens} oder \emph{Constraints} ist einerseits auf eine der grundlegenden Design-Paradigmen des hier entwickelten Werkzeugs -- der Flexibiltät der Abbildung -- zurückzuführen, weist aber andererseits auch auf mögliches Verbesserungspotential hin.

Durch die Flexibilisierung nicht nur der Bindung zwischen physischen Elementen und digitaler Repräsentation sondern auch der Verwendung von unterschiedlichen physischen Elementen selbst könnten Modellierungstokens eher \emph{Token}-artiger werden. Indem Modellierende eigenen physische Elemente (auf ihrem Arbeitskontext) einbringen können, könnte die Erfassbarkeit der Bedeutung der physischen Repräsentation unter Umständen verbessert werden können.

\section{Einordnung in die Taxonomie von Fishkin}

Grundlage der Einordnung in diesem Abschnitt ist der Ansatzes von \citep{Fishkin04}, der in Abschnitt \ref{sub:taxonomie_fishkin} beschrieben wurde.

\subsection{Abbildung}
Das in dieser Arbeit entwickelte Werkzeug überspannt aufgrund seiner komplexen Struktur in beiden von \citeauthor{Fishkin04} vorgeschlagenen Dimensionen zur Klassifikation von Tangible Interfaces mehrere Ausprägungen. Um eine umfassende und ins Detail gehende Einordnung vornehmen zu können, werden im Folgenden Einzelaspekte des Systems betrachtet und eingeordnet. Während die Dimension "Embodiment" bereits in Kapitel \ref{cha:visualisierung} betrachtet wurde, um eine strukturierte Zuordnung der Ausgabekanäle vornehmen zu können, werden hier die einzelnen Funktionalitäten des Systems (siehe Abschnitt \ref{sec:benutzerinteraktion_mit_dem_werkzeug}) jeweils beiden Dimensionen zugeordnet (siehe Tabelle \ref{tab:einordnungFishkin})

\begin{table}[htbp]
	\centering
	\begin{tabular}{| p{6cm} || p{3cm} | p{3cm} |} \hline
		 & Embodiment & Metaphor \\ \hline \hline
		Platzieren und Benennen von Modellelementen & distant, nearby (Tastatur), full (Haftnotiz) & verb (Tastatur), verb + noun (Haftnotiz) \\ \hline
		Erstellen von Verbindern & nearby & verb bis noun+verb (Werkzeugtokens), verb (räumliche Nähe)\\ \hline
		Löschen von Verbindern & environmental bis nearby & noun \\ \hline
		Einbetten von Information & full & noun + verb \\ \hline
		Abrufen von Information & distant & verb \\ \hline
		Erstellen von Snapshots & environmental bis nearby & none \\ \hline
		Navigation in der Modell-Historie & distant & verb \\ \hline
		Wiederherstellen eines Modell-Zustandes & nearby & noun + verb\\ \hline
	\end{tabular}
	\caption{Einordnung des Systems in die Taxonomie nach Fishkin}
	\label{tab:einordnungFishkin}
\end{table}

Beim \emph{Platzieren und Benennen von Modellelementen} ist die Benennung auf zwei Arten möglich, die unterschiedlich in die Taxonomie einzuordnen sind. Bei Benennung mittels Auswahl und Tastatur ist durch die Projektion der Benennung die Embodiment-Ausprägung "nearby" zu wählen. Der Vorgang der Auswahl und Benennung kann als analog zur realen Welt gesehen werden, die eingesetzten Werkzeuge sind aber generischer Natur -- Metaphor ist also als "verb" zu klassifizieren. Bei der Benennung mittels Haftnotitz ist durch die unmittelbar auf den Tokens angebrachten Benennungen Embodiment "full", Der Vorgang des Beschriftens wird analog zur realen Welt durchgeführt, auch die Informationsträger (Haftnotizen) entsprechen jenen der realen Welt, Metaphor ist also "verb + noun", wobei  der notwendige Vorgang der expliziten Erfassung einer Beschriftung durch das System eine Klassifikation "full" verhindert und sogar die Einstufung "noun + verb" etwas abschwächt (keine Analogie des Vorgangs zur realen Welt).

Zur \emph{Herstellung von Verbindern} existieren ebenfalls zwei Möglichkeiten. In beiden Fällen ist durch die Projektion der Verbindung die Ausprägung in Embodiment "nearby", sie unterscheiden sich jedoch hinsichtlich "Metaphor". Bei der Verwendung von Werkzeugtokens ist der Vorgang der Auswahl der Endpunkte analog zur realen Welt zu sehen und somit als "verb" einzustufen. Die Verwendung von spezifischen Werkzeugtokens zur Herstellung gerichteter Verbinder zeigt sogar Züge von "noun + verb", da die durch das Token dargestellte Pfeilspitze eine Analogie zur realen Welt bildet.

Das \emph{Löschen von Verbindern} wird durch das Lösch-Token vorgenommen. Dieses ist durch einen Radiergummi symbolisiert, der jedoch nicht als solche eingesetzt wird sondern das System nur in einen Löschmodus versetzt. Die Klassifikation in Metaphor ist demnach "noun". Die Visualisierung des Löschzustandes erfolgt unspezifisch durch die Umfärbung der gesamten Tischoberfläche, womit ein Embodiment von "nearby" oder "environmental" (aufgrund der Unspezifität) gerechtfertigt wäre.

\emph{Einbetten von Information} erfolgt durch die Verwendung der Modellierungstokens als Container und Hineinlegen von kleineren Tokens. Embodiment ist in diesem Fall "full", die die Einbettung physisch nachvollzogen wird. Metaphor ist durch die Analogie des "Hineinlegens" von Information in "Container" in die Ausprägung "noun + verb" einzuordnen.

Das \emph{Abrufen von Information} wird über den sekundären Ausgabekanal abgewickelt und ist daher in Embodiment als "distant" einzuordnen. Der Vorgang des Herausnehmens von Information aus einem Container existiert analog zur realen Welt, das bei diesem Vorgang im Zentrum stehende Objekt, das einbettbare Token, ist jedoch generisch und weist nicht auf die Art der eigebetteten Information hin. Eine Klassifikation von "verb" in Metaphor erscheint daher gerechtfertigt.

Beim \emph{Erstellen von Snapshots} wird die gesamte Tischoberfläche als Feedbackkanal genutzt. Insofern ist Embodiment wie im Falle des Löschens von Verbindern im Bereich "environmental" bis "nearby" anzusiedeln. Das Snapshot-Token selbst ist ein generisches Objekt, das keine Analogie zur realen Welt aufweist. Metaphor ist daher "none".

Die \emph{Navigation in der Modell-Hierarchie} erfolgt mit dem runden Navigations-Token. Zur Ausgabe der gespeicherten Modell-Zustände wird der sekundäre Ausgabekanal
verwendet. Embodiment ist deshalb "distant". Metaphor beschränkt sich auf "verb", da der Drehvorgang zur Navigation analog zum Einstellen einer Uhr erfolgt, das Token selbst aber bis auf seine runde Form generisch ist.

Das \emph{Wiederherstellen eines Modellzustandes} erfolgt durch spezifische Anweisungen auf der Modellierungsoberfläche. Embodiment ist also als "nearby" einzustufen. Der Vorgang der Wiederherstellung erfolgt durch Verschieben der Modellierungstokens, was im Wesentlichen analog zur realen Welt abläuft. Da unmittelbar die Objekte manipuliert werden, kann Metaphor als "noun + verb" eingestuft werden.

\subsection{Bewertung}

Die Taxonomie nach \citeauthor{Fishkin04} ermöglicht eine strukturierte Erfassung einzelner Aspekte eines Tangible User Interfaces. Eine aussagekräftige Gesamteinordnung ist nur bei einfachen \glspl{TUI} möglich, komplexe, mit vielen Interaktionsmöglichkeiten ausgestattete Systeme tendieren dazu, ein sehr breites Spektrum der Taxonomie abzudecken. Für die detaillierte Betrachtung eines komplexen Gesamtsystems erscheint die Taxonomie dennoch geeignet, da einerseits aus den einzelnen Teileinordnungen für den jeweiligen Anwendungsfall ggf. Verbesserungspotentiale abgeleitet werden können und andererseits (nach der Betrachtung des hier entwickelten Systems) scheint, als ob ein die Taxonomie breit abdeckendes Gesamtsystem potentiell Inkonsistenzen im Interaktionsdesign aufweist bzw. unterschiedliche Interaktionsparadigmen vermischt wurden. Vor allem "Ausreißer" aus einem vorwiegend einheitlichen Gesamtbild scheinen einer näheren Betrachtung hinsichtlich eines möglichen Redesigns wert.

Konkret können diese Vermutungen im vorliegenden System vor allem an der Konzeption des Lösch-Tokens und des Snapshot-Tokens festgemacht werden. Der Großteil der Interaktionen mit dem System beinhaltet in der Dimension Metaphor den "verb"-Aspekt (zu etwa gleichen Teilen ausschließlich und in der Kombination mit "noun"). Die Funktionalitäten, die die beiden erwähnten Tokens einbeziehen, laufen diesem Trend entgegen und zeigen in Metaphor die Ausprägung "noun" bzw. "none". Tatsächlich zeigt sich in der Praxis, das die die Anwendbarkeit dieser Tokens von Benutzern missverstanden bzw. nicht verstanden wird. Ein Redesign dieser Tokens mit expliziterer bzw. eher aktivitätsorientierter Metaphor erscheint deshalb untersuchenswert.

Zusammenfassend scheint die Taxonomie vor allem im Zusammenhang mit der Sicherung von konsistenter Interaktion an der Benutzungsschnittstelle sinnvoll anwendbar zu sein. Der Mehrwert des Ansatzes zeigt sich hier nicht so sehr in den absoluten Ausprägungen auf den beiden Dimensionen sondern vielmehr in den relativen Unterschieden, die zwischen den einzelnen Teilen des Tangible User Interfaces auftreten.

\section{Zusammenfassung}

% chapter konzeptuelle_evaluierung (end)

% final draft
% todo: Interkapitel-Referenzen, Kontextgrafik


\chapter{Überblick über die empirische Untersuchung}
\label{cha:eval_ueberblick}

In diesem Kapitel wird ein Überblick über die in dieser Arbeit durchgeführte empirische Untersuchung gegeben. Dabei wird auf die einzelnen zu untersuchenden Aspekte, deren theoretische Grundlagen und die Durchführung der Untersuchung gegeben. Abbildung \ref{fig:img_Kontextgrafiken_k11} stellt dieses Kapitel und dessen Aufbau im Kontext der anderen inhaltlich vor- und nachgelagerten Kapitel dar.


\begin{figure}[htbp]
	\centering
		\includegraphics[scale=0.6]{img/Kontextgrafiken/k11.png}
	\caption{Kapitel „Überblick über die empirische Untersuchung“ im Gesamtzusammenhang}
	\label{fig:img_Kontextgrafiken_k11}
\end{figure}

Die im Rahmen der empirischen Evaluierung zu untersuchenden Aspekte sind Gegenstand des ersten Abschnitts. Neben einer wiederholenden grundlegenden Betrachtung werden hier die jeweiligen Untersuchungsfragen festgelegt. Eine nähere Betrachtung der einzelnen Aspekte, die Festlegung der Methodik und deren Operationalisierung im Rahmen des konkreten Untersuchungsdesigns erfolgt im Rahmen der übrigen Kapitel in diesem Teil der Arbeit.

Im zweiten Abschnitt wird ein Überblick über das globale Untersuchungsdesign gegeben. Auf Basis der zu evaluierenden Aspekte werden die konkret durchgeführten Teile der Evaluation (im Folgenden: "Evaluierungsblöcke") beschrieben und den Aspekten zugeordnet. Diese Evaluierungsblöcke werden überblicksweise hinsichtlich der intendierten Ziele, der Aufgabenstellung und der jeweiligen Anzahl der Teilnehmer beschrieben. Die Beschreibung bildet die Grundlage für die Beschreibung der Evaluierung der zu prüfenden Aspekte in den folgenden Kapiteln.

\section{Zu untersuchende Aspekte} % (fold)
\label{sec:untersuchungsaspekte}

Ziel dieser Arbeit ist die Unterstützung von expliziter Articulation Work. Eine Möglichkeit, explizite Articulation Work zu unterstützen, ist die Externalisierung und Abstimmung der mentalen Modelle über den betreffenden Arbeitsvorgang, die den Handlungen der beteiligten Personen zugrunde liegen (siehe Abschnitt \ref{sec:articulation_work_und_mentale_modelle}). Die Externalisierung mentaler Modelle ist mittels unterschiedlicher Methoden möglich, wobei sich Ansätze, die auf der Abbildung mentaler Modelle in diagrammatischen Strukturen basieren, als gut geeignet erwiesen haben (siehe Abschnitt \ref{sec:externalisierung_mentaler_modelle}). Zwei derartige Methoden sind Concept Mapping und Strukturlegetechniken, die beide Vor- und Nachteil hinsichtlich des Einsatzes in kollaborativen Szenarien zeigen (siehe \ref{cha:methodik}). In dieser Arbeit wird deshalb versucht, die Vorteile der beiden Ansätze methodisch zu vereinigen und zur Vermeidung der Nachteile durch ein Tabletop Interface zu unterstützen (siehe Kapitel \ref{cha:methodik} sowie \ref{cha:anforderungen}).

Anhand dieser Argumentationskette zeigt sich, dass zwischen der Zielformulierung und dem konkreten Werkzeug zur Zielerreichung einige argumentative Schritte liegen, die vorerst lediglich (aus der Literatur begründete) Annahmen darstellen. Im Zuge der Evaluation der Ergebnisse dieser Arbeit müssen nun diese Schritte einzeln betrachtet werden und hinsichtlich der jeweiligen Zielerreichung überprüft werden. Untersuchungsgegenstand ist dabei jeweils das erstellte Werkzeug, die betrachteten Aspekte unterscheiden sich je nach Argumentationsschritt. Die Untersuchungsfragen, die die Argumentationsschritte abdecken sind:
\begin{itemize}
 \item Sind das Werkzeug und dessen Komponenten verständlich und wie intendiert einsetzbar? (Aspekt: Werkzeug)
 \item Erlauben Werkzeug und Methode die Abbildung semantisch offener diagrammatischer Modelle? (Aspekt: Modell)
 \item Unterstützen Werkzeug und Methode Articulation Work? (Aspekt: Articulation Work)
\end{itemize}

Diese Fragen decken die Aspekte der oben beschriebene Argumentationskette ab, die Detaillierung der Fragestellungen ist in den folgenden Abschnitten beschrieben. Die Beschreibung der zu prüfenden Hypothesen sowie die Operationalisierung der Untersuchungsfragen erfolgt in den Kapitel \ref{cha:eval_werkzeug} bis \ref{cha:eval_aw}.

\subsection{Evaluierung des Werkzeugs}
\label{sub:eval_werkzeug}

Die Evaluierung des Werkzeugs an sich beschäftigt sich mit der Beantwortung der ersten Untersuchungsfrage. Diese zielt auf die Verständlichkeit des Werkzeugs im weiteren Sinn ab. Unter Verständlichkeit im weiteren Sinn ist hier zu verstehen, dass einerseits geprüft werden muss, ob die Bedeutung und grundlegende Verwendung der Komponenten des Werkzeugs von Benutzern erfasst und verstanden werden und ob andererseits die Interaktionsabläufe, die zur Auslösung bzw. Abwicklung einer Funktion des Werkzeugs führen, für Benutzer verständlich und nachvollziehbar sind.

Neben der quantitativen Bewertung anhand dieser Metriken ist bei der Untersuchung dieses Aspektes vor allem auch das qualitative Feedback der Benutzer notwendig, um Ansatzpunkte zur Verbesserung der Verwendbarkeit des Werkzeugs zu erhalten. Diese Anregungen können im Sinne eines iterativen Designprozesses umgesetzt und deren Auswirkungen erneut einer Evaluierung unterzogen werden. Neben der Erhebung dieser zusätzlich funktionalen Anforderungen für einen iterativen Designprozess sind in diesem Zusammenhang auch Hinweise hinsichtlich nicht-funktionaler Aspekte des Systems zu berücksichtigen, die der Verwendbarkeit negativ beeinflussen bzw. auch unkritisch sein können.

Die Verwendbarkeit des Werkzeugs kann nicht entkoppelt von der Anwendungsdomäne betrachtet werden, muss also im Kontext der Aufgabe, für die es eingesetzt wird, gesehen werden. Das Werkzeug ist zwar grundsätzlich für die Repräsentation beliebiger diagrammatischer Modelle ausgelegt, eignet sich aufgrund der unterschiedlichen Anforderungen jedoch nicht gleich gut für alle möglichen Anwendungsfälle (so sind z.B. ausschließlich Verbindungen mit zwei Endpunkten erstellbar, Verbindungen mit mehr Endpunkten werden nicht unterstützt). Die Prüfung der Verwendbarkeit des Werkzeugs kann hier fokussiert auf die in dieser Arbeit verfolgten Anwendungsfälle durchgeführt werden, die im Bereich der konzeptionellen Netze (im Wesentlichen Varianten von Concept Maps) und im Bereich der Abbildung von Arbeitsvorgängen (im Wesentlichen kausale Zusammenhänge mit Kontextinformation) zu finden sind. Die Unterstützung anderer Anwendungsfälle ist möglich und unter Umständen erstrebenswert, stellt jedoch kein Beurteilungskriterium dar.

\subsection{Evaluierung der Modellrepräsentationen}
\label{sub:eval_modell}

Der zweite zu evaluierende Aspekt sind die mit dem Werkzeug erstellten Modelle, die als Mittel zur Durchführung expliziter Articulation Work dienen. Eine wesentliche Eigenschaft, die Modelle dabei aufweisen müssen, ist die Adäquatheit der Modellierungssprache hinsichtlich der durch die Benutzer zu repräsentierenden Information. Diese Eigenschaft wird in der vorliegenden Arbeit durch die in Kapitel \ref{cha:mentale_modelle} beschriebene Anforderung der semantischen Offenheit abgedeckt, der jedoch vor allem hinsichtlich der intersubjektiven Verständlichkeit der Modelle und deren Eindeutigkeit nicht nur Vorteile bringt. Grundlegende ist in dieser Phase zu evaluieren, ob die erstellten Modelle den im Rahmen des Einsatzes zur Unterstützung von Articulation Work intendierten Zweck erfüllen. Dabei sind sowohl das Modell als auch das (hier von den Benutzern festgelegte) Metamodell zu betrachten. Anhaltspunkte zur Identifikation der zu evaluierenden Objekte sowie zum Vorgehen bieten hier der Ansatz der „Interactive Process Models“ \citep{Jorgensen04} und die „Grundsätze der ordnungsgemäßen Modellierung“ \citep{Becker00} sowie von diesen Arbeiten abgeleitete Ansätze.

Die eben beschriebenen Ansatzpunkte erlauben eine Evaluierung der erstellten Modelle hinsichtlich der Abbildbarkeit der Kernaspekte von „Articulation Work“ im engeren Sinne (Strauss' „salient dimensions“: \emph{„who, where, when, what and how“} \citep{Fjuk97}), decken also im Wesentlichen eine an organisationalen Abläufen orientierten Sicht auf Modelle ab. Im Sinne der Offenheit der Abbildung müssen aber auch Modelle berücksichtigt werden, die nicht diese „salient dimensions“ zur Grundlage haben, also „Concept Maps“ \citep{Novak06} im allgemeinen Sinn sind und damit die Abbildung mentaler Modelle nicht nur über unmittelbare Arbeitsaspekte sondern über beliebige Sachverhalte erlauben \citep{Ifenthaler06}. Dabei sind Metriken notwendig, die die erstellten Modelle selbst betrachten und deren Eigenschaften und Verwendung beim Concept Mapping bzw. im Rahmen von Strukturlegetechniken berücksichtigen.

Wie bereits im letzten Abschnitt angeführt, ist auch bei diesem Aspekt der Evaluierung der in dieser Arbeit verfolgte Anwendungszweck des Werkzeugs (bzw. hier: der Modelle) zu berücksichtigen. Dies ist insofern ein einschränkender Faktor, als dass hier Modelle lediglich im Kontext der Externalisierung mentaler Modelle und zur Unterstützung von Articulation Work berücksichtigt werden. Das Werkzeug selbst erlaubt auch die Erstellung von Modellen zu anderen Anwendungszwecken, die jedoch hier nicht weiter berücksichtigt werden.  

\subsection{Evaluierung der Articulation Work}
\label{sub:eval_articulation_work}

Letztendlich muss auch die durchgeführte Articulation Work selbst beurteilt werden. In der Literatur zum Thema „Articulation Work“ werden zumeist lediglich das Phänomen „Articulation Work“ und dessen konkrete Ausprägungen beschrieben (siehe Kapitel \ref{cha:articulation_work}), Ansätze zur Bewertung des Erfolgs von „Articulation Work“ sind jedoch selten zu finden. Aus der Verschränkung zwischen „Articulation Work“ und „Production Work“, also jenem Anteil der Arbeit, der unmittelbar der Zielerreichung dient, die von mehreren Autoren, unter anderem \citet{Fujimura87} und \citet{Strauss93}, erwähnt wird, lassen sich jedoch Ansatzpunkte ableiten.

„Articulation Work“ tritt immer dann auf, wenn eine Zielerreichung in der Production Work aufgrund von Unklarheiten oder Problemen zwischen den beteiligen Individuen nicht möglich ist. Ein erfolgreicher Abschuss der „Production Work“ bei am Beginn oder während der Arbeit bestehenden Unklarheiten weißt also unter Umständen auf erfolgreich durchgeführte Articulation Work hin. „Articulation Work“ manifestiert sich im Arbeitsprozess auf unterschiedliche Arten, so dass bei der Evaluierung hinsichtlich der Auswirkungen des Werkzeugs diese von den übrigen Einflussfaktoren (also auf anderen Wegen durchgeführte „Articulation Work“) getrennt werden muss. Dazu ist eine Betrachtung des gesamten Arbeitsablaufs unter Berücksichtigung von Production und Articulation Work notwendig. Metriken, die bei der Bewertung des Erfolgs von „Articulation Work“ zu berücksichtigen sind, sind also einerseits im Ergebnis des Arbeitsprozesses, andererseits auch im Arbeitsprozess selbst zu finden.

Ein zweiter Ansatzpunkt zur Bewertung des Erfolgs von „Articulation Work“ liegt in den Aussagen von \citet{Strauss93} hinsichtlich der wahrgenommenen "Problematik" einer Arbeitssituation, die „Articulation Work“ notwendig macht. Diese Wahrnehmung ist individueller Natur, d.h. „Articulation Work“ ist dann notwendig, wenn zumindest einer am Arbeitsablauf beteiligten Person Aspekte der Arbeit unklar sind oder problematisch erscheinen. Im Gegenzug ist keine „Articulation Work“ notwendig bzw. diese abgeschlossen, wenn alle beteiligten Personen die Situation als unproblematisch empfinden bzw. mit den im Rahmen der (expliziten) „Articulation Work“ erzielten Ergebnissen zufrieden sind. Hier liegt der Ansatzpunkt für eine Evaluierung des Erfolgs der durchgeführten „Articulation Work“, der diese auf Basis der individuellen Wahrnehmungen der beteiligten Personen beurteilt.
% section untersuchungsaspekte (end)

\section{Globales Untersuchungsdesign}
\label{sec:globales_untersuchungsdesign}

Die oben beschriebenen Aspekte müssen nun im Rahmen einer empirischen Untersuchung geprüft werden. Während das detaillierte Untersuchungsdesigns in den folgenden Kapiteln, die sich jeweils einem der drei zu evaluierenden Aspekte widmen, beschrieben wird, wird an dieser Stelle ein Überblick über das globale Untersuchungsdesign und die im Rahmen der Evaluierung durchgeführten Anwendungen des Werkzeugs gegeben.

Im ursprünglichen globalen Untersuchungsdesign war vorgesehen, jedem der zu untersuchenden Aspekte einen Block an Anwendungen des Werkzeugs mit einer auf den jeweiligen Aspekt abgestimmten Aufgabenstellung zuzuordnen. Nach Durchführung der ersten beiden Blöcke wurde offensichtlich, dass sich aus der Anwendung des Werkzeugs heraus zusätzliche Hypothesen ableiten ließen, die -- um sie in der Evaluierung berücksichtigen zu können -- in einem späteren Block geprüft werden mussten. Außerdem wurde offensichtlich, dass vor allem zur Evaluierung des Werkzeugs in allen Blöcken Verbesserungspotential identifiziert werden konnte bzw. Anregungen der Anwender rückgemeldet wurden, die zum Teil im Rahmen des iterativen Entwicklungsprozesses in das Werkzeug einflossen und deren Wirkung in einem späteren Block erneut geprüft werden musste. 

Letztendlich wurden die Blöcke für die Evaluierung mehrerer bzw. aller Aspekte herangezogen, sofern die jeweilige Aufgabenstellung geeignet war. Bei der nun folgenden Beschreibung der Anwendungs-Blöcke wird deshalb jeweils angegeben und begründet, inwieweit diese in die Evaluierung welcher Aspekte einfließen. Ein Überblick über das globale Untersuchungsdesign mit einer überblicksweisen Zuordnung zwischen den zu evaluierenden Aspekten und den Anwendungsblöcken wird in Abschnitt \ref{sec:eval_ueberblick_zusammenfassung} gegeben.

\subsection{Block 1: Technische Evaluierung}
\label{sub:eval_1}

Die Intention von Block 1 war die grundlegende Verständlichkeit und Verwendbarkeit des Werkzeugs zu prüfen. Fokus dieses Blocks an Anwendungen des Werkzeugs war also die Untersuchung der Eigenschaften des Werkzeugs selbst. Zusätzlich wurde hier explorativ die Wirkung des Werkzeugs auf die Modellierungstätigkeit und Kooperation der Anwender untersucht.

\subsubsection{Kontext} % (fold)
\label{ssub:1_kontext}

Die Untersuchung wurde im Rahmen einer Diplomarbeit durchgeführt (\cite{Bohninger10}), wobei die Untersuchungen in keinen einheitlichen realen Arbeitskontext eingebettet waren. Allerdings war die Aufgabenstellung so formuliert, dass die erstellten Modelle aus den Arbeitskontexten der jeweiligen Teilnehmer stammten.

% subsubsection kontext (end)

\subsubsection{Aufgabenstellung und Ablauf} % (fold)
\label{ssub:1_aufgabenstellung}

Den modellierenden Teilnehmern wurde mitgeteilt, dass sie einen Aspekt aus ihrem täglichen Arbeits- oder Privatleben abbilden sollten, der regelmäßig auftritt oder bereits mehrmals für Probleme sorgte. Die bewusste Offenheit der Aufgabenstellung sollte dabei bewirken, dass sich die Teilnehmer nicht zu sehr auf den abzubildenden Sachverhalt, sondern eher auf den Abbildungsprozess selbst fokussierten. Die Modellbildung erfolgte jeweils individuell.

Nur die Hälfte der Teilnehmer erstellte tatsächlich Modelle. Die zweite Hälfte wurde zur Überprüfung der Verständlichkeit der Modelle sowie der Verwendbarkeit des Werkzeugs zur kooperativen Modellierung herangezogen. Dazu wurde nach Abschluss einer Modellbildung jeweils ein nicht modellierender Teilnehmer an die Modellierungsoberfläche gebeten und aufgefordert, die Abbildung zu interpretieren. Die Beurteilung der Adäquatheit dieser Interpretation erfolgte durch den ursprünglich modellierenden Teilnehmer.

In einer dritten Phase wurden beide Teilnehmer aufgefordert, dass Modell gemeinsam zu reflektieren und gegebenenfalls zu verändern, um es den Ergebnissen der Reflexion anzupassen. In dieser Phase war das vorrangige Ziel, die Verwendung des Werkzeugs bei der Veränderung von Modellen und dessen kollaborativer Anwendung zu testen. 

Entsprechend dieser Beschreibung ist die Phase 1 dieses Blocks dem Anwendungsszenario „Verfeinerung mentaler Modelle“ (siehe Abschnitt \ref{sub:verfeinerung_individueller_mentaler_modelle}) zuzuordnen. Die Phasen 2 und 3 sind dem Anwendungsszenario „Wissenstransfer“ (siehe Abschnitt \ref{sub:wissenstransfer}) zuzuordnen.

% subsubsection aufgabenstellung (end)

\subsubsection{Anwendungen und Teilnehmer} % (fold)
\label{ssub:1_teilnehmer}

Insgesamt wurden neun Anwendungen des Werkzeug wie oben beschrieben durchgeführt. Zusätzlich wurde das Untersuchungsdesign im Rahmen von drei Anwendungen getestet (Pretest), woraus hinsichtlich der technischen Eigenschaften des Werkzeugs ebenfalls bereits Erkenntnisse gewonnen werden konnten. Insgesamt nahmen also 24 Personen an diesem Block von Anwendungen teil, 6 davon in der Pretest-Phase.

Die Teilnehmer (exkl. Pretest) stammten aus unterschiedlichen beruflichen Hintergründen und unterschieden sich auch in Art der höchsten abgeschlossenen Ausbildung (7 Universität/FH, 7 Matura, 4 Lehrabschluss). Die Altersspanne lag zwischen 19 und 43 Jahren, 13 Teilnehmer waren weiblich, 11 männlich.

Die Modellierungsphasen (exkl. Pretest) dauerten im Schnitt 8 Minuten ($SD=2:13$), die kürzeste Modellbildung dauerte 5 Minuten, die längste 12 Minuten. Die Interpretations- und Reflexionsphasen (nicht separat aufschlüsselbar, da zum Großteil ineinander übergehend) dauerten im Schnitt 5 Minuten ($SD=1:45$). 

% subsubsection teilnehmer (end)

\subsubsection{Verwendung der Ergebnisse} % (fold)
\label{ssub:1_verwendung_der_ergebnisse}

Die Ergebnisse dieses Blocks flossen in die Evaluierung des Werkzeugs und in die Hypothesenbildung hinsichtlich der erstellten Modelle ein. Für die Evaluierung der Modelle konnten erste Erkenntnisse hinsichtlich der Verständlichkeit der mit offener Semantik gewonnen werden. Keine Ergebnisse brachte dieser Block für die Evaluierung der durchgeführten „Articulation Work“.

% subsubsection verwendung_der_ergebnisse (end)

\subsection{Block 2: Aushandlung von Zusammenarbeit 1}
\label{sub:eval_2}

In Block 2 lag der Fokus der Evaluation erstmals auf der Unterstützung von Articulation Work. In diesem Rahmnen wurden auch die Verwendbarkeit des Werkzeugs im praktischen Anwendungskontext und die Eigenschaften der erstellten Modelle sowie deren Rolle im Prozess der expliziten Articulation Work untersucht.

\subsubsection{Kontext} % (fold)
\label{ssub:2_kontext}

Block 2 wurde im Rahmen eines Seminars aus Wirtschaftsinformatik mit Studierenden dieser Studienrichtung durchgeführt. Die im Seminar zu erstellenden wissenschaftlichen Arbeiten wurden von den Studierenden in Gruppen zu 2-3 Personen ausgearbeitet. Die Gruppen wurden so gebildet, dass sich die Teilnehmer nicht persönlich kannten oder zumindest nicht bereits in anderen Kontexten zusammengearbeitet hatten. Ziel dieser Maßnahme war die Vermeidung der Verfälschung der Untersuchungsergebnissse durch bereits eingespielte Gruppen (Erfahrungen in Seminaren der Vorjahre zeigen tendentiell schlechtere Ergebnisse bei der Zusammenarbeit von einander nicht persönlich bekannten bzw. nicht eingespielten Teilnehmern).

Im Rahmen des Seminars wurden sechs Forschungsgebiete ausgewählt, die in Zusammenhang mit der Erstellung und Verwendung sozio-technischer Systeme stehen (konkret: Organisationales Lernen, eLearning, \gls{CSCW}, Mentale Modelle, Articulation Work und semantische Contentanreicherung). Den Gruppen wurden jeweils zufällig zwei dieser Themen zugewiesen, die Aufgabe für die wissenschaftliche Arbeit war das Finden und Beschreiben einer möglichen Verknüpfung oder eines möglichen Zusammenhanges zwischen diesen Themen. Dieser Zusammenhang sollte im Zentrum der Seminararbeit stehen und aus beiden Grundlagen-Themen argumentiert sein. Ziel dieser Maßnahme war es, die Seminararbeit so offen wie möglich zu gestalten und einen Themenfindungs- bzw. -konkretisierungsprozess in den Ablauf zu integrieren. Außerdem wurde so ein Szenario geschaffen, in dem sich eine strikte Arbeitsteilung der Gruppenteilnehmer ohne weitere Zusammenarbeit während der Ausarbeitung der Inhalte („Production Work“) potentiell auf das Ergebnis auswirkt und sich konkret der fehlenden oder schwachen Verknüpfung der Grundlagen-Themen zeigt.

% subsubsection kontext (end)

\subsubsection{Aufgabenstellung und Ablauf} % (fold)
\label{ssub:2_aufgabenstellung}

Das Werkzeug wurde im Rahmen des Seminars für jede Gruppe zweimal eingesetzt. Die erste Anwendung fand zu Beginn des Seminars nach der Themenzuteilung statt. Die Aufgabe war die Aushandlung der Modalitäten der Zusammenarbeit mit der Zielsetzung, das an der resultierenden wissenschaftlichen Arbeit die Ko-Autorenschaft nicht mehr zu erkennen sein sollte (etwa durch plötzlich wechselnde Schreibstile oder Brüche in der Argumentationskette). Den Teilnehmern wurde das Werkzeug und dessen Funktionen vorgestellt und ohne weitere Vorgaben zur Verfügung gestellt (insbesondere wurden weder Vorgaben hinsichtlich der Topologie des zu erstellenden Modells oder der Bedeutung der Modellierungselemente gemacht).

In der zweiten Anwendung wurde der Zusammenarbeitsprozess reflektiert und gegebenenfalls eine Adaption vereinbart. Die zweite Anwendung fand in der Mitte des Semesters nach Abschluss der Literaturrecherche und der Grobkonzeption, aber vor der Erstellung der eigentlichen wissenschaftlichen Arbeit statt. Konkrete Zielsetzung für die Teilnehmer war hier, auf Basis der bisherigen Erfahrungen die weitere Zusammenarbeit zu vereinbaren. Das Werkzeug wurde ohne neuerliche Vorstellung und ohne Vorgaben hinsichtlich der Verwendung zur Verfügung gestellt.

Entsprechend dieser Beschreibung ist die erste Anwendung des Werkzeugs in diesem Block dem Anwendungsszenario „Aushandlung mentaler Modelle“ (siehe Abschnitt \ref{sub:aushandlung_individueller_mentaler_modelle}) zuzuordnen. Die zweite Anwendung ist in das Anwendungsszenario „Abstimmung mentaler Modelle“ (siehe Abschnitt \ref{sub:abstimmung_individueller_mentaler_modelle}) einzuordnen.
% subsubsection aufgabenstellung (end)

\subsubsection{Anwendungen und Teilnehmer} % (fold)
\label{ssub:2_teilnehmer}

Insgesamt nahmen an diesem Block 19 Personen in 9 Gruppen zu 2 bzw. einmalig 3 Personen teil. Jede der Gruppen setzte das Werkzeug zweimal ein, wodurch insgesamt 18 Anwendungen die Grundlage für die Auswertung der Ergebnisse bilden.

Die Teilnehmer waren allesamt Studierende der Wirtschaftsinformatik im zweiten Studienabschnitt. 18 Personen waren männlich, eine weiblich. Vier Personen hatten insofern Erfahrung mit wissenschaftlichen Arbeiten bzw. den konkreten Anforderungen in der betreffenden Lehrveranstaltungen, als dass sie bereits zuvor eine Lehrveranstaltung gleichen Typs besucht hatten.

In der ersten Runde dauerten die Anwendungen durchschnittlich 20 Minuten 50 Sekunden ($SD=4:18$), in der zweiten Runde lediglich 9 Minuten 49 Sekunden ($SD=5:20$).

% subsubsection teilnehmer (end)

\subsubsection{Verwendung der Ergebnisse} % (fold)
\label{ssub:2_verwendung_der_ergebnisse}

Die in diesem Block erhobenen Daten fließen in die Auswertung alle drei zu evaluierenden Aspekte ein. Zur Auswertung hinsichtlich des Erfolgs von Articulation Work liegen neben den Aufnahmen der Modellierungsvorgänge und den erstellten Modellen selbst auch Prozessreflexionen der Teilnehmer über den Erstellungsprozess der Seminararbeiten sowie die Seminararbeit an sich vor. Die Auswirkungen von „Articulation Work“ können also am Ergebnis (im Vergleich zu Ergebnissen auf Lehrveranstaltungen mit identischem Konzept) und am subjektiv wahrgenommenen Verlauf des Erstellungsprozesses der Arbeit bewertet werden.

Hinsichtlich der Auswertung des Modell-Aspektes wird durch diesen Block die Betrachtung von Modellen ermöglicht, die im Kontext der Arbeitsabstimmung erstellt wurden, also im Wesentlichen der Definition von Vorgehen und Schnittstellen dienen. Untersucht werden hier Aufbau und Inhalt der Modelle, wobei besonderes Augenmerk auf der Prozess und Ergebnis der Bedeutungszuweisung zu den Modellelementen liegt.

Im Rahmen der Werkzeug-Evaluation bringt dieser Block die ersten Hinweise auf die Anforderungen an das Werkzeug bei der Verwendung desselben im Rahmen einer realen Aufgabenstellung. Außerdem wurde in diesem Block erstmals ein durchgängig kollaboratives Szenario eingesetzt, bei dem immer mindestens zwei Personen gleichzeitig das Werkzeug verwenden.

% subsubsection verwendung_der_ergebnisse (end)

\subsection{Block 3: Concept Mapping 1}
\label{sub:eval_3}

Der Fokus von Block 3 lag auf der Erstellung von semantisch vernetzten Strukturen im Allgemeinen, wobei das Konzept der Concept Maps als ein etabliertes Werkzeug zur Externalisierung mentaler Modelle eingesetzt wurde. Inhaltlich fokussierte dieser Block nicht auf die Unterstützung von Articulation Work im engeren Sinne, wohl aber auf die Externalisierung und Abstimmung mentaler Modelle, was wie in Kapitel \ref{cha:mentale_modelle} beschrieben ein Mittel zur Unterstützung expliziter Articulation Work ist. Im Zentrum der Aufmerksamkeit steht in diesem Block also die Evaluierung der erstellten Modelle und der Nutzen des Werkzeugs zur Aushandlung einer einheitlichen auf einen gegebenen Sachverhalt.

\subsubsection{Kontext} % (fold)
\label{ssub:3_kontext}

Der dritte Block wurde im Rahmen einer Lehrveranstaltung zur Schulung von Methoden der Prozess- und Kommunikationsmodellierung durchgeführt. Diese Lehrveranstaltung ist Teil der im Curriculum definierten Basiskompetenz Wirtschaftsinformatik und wird von Studierenden im zweiten bis dritten Studiensemester besucht.

Im Rahmen der Lehrveranstaltung wurden drei unterschiedliche Prozessmodellierungssprachen (SeeMe \citep{Herrmann04a}, Subjekt-orientierte Modellierung mittels JPass \citep{Fleischmann07} und \gls{EPK}s aus dem ARIS-Konzept \citep{Scheer00}) eingeführt und praktisch an einem durchgängigen Beispiel angewandt. Diese Sprachen unterscheiden sich sowohl im Anwendungsgebiet, in den abgebildeten Aspekten des realen Prozesses sowie in der Darstellungsform des Modells. Ziel der letzten Teilaufgabe, die unter Einsatz des hier vorgestellten Werkzeugs durchgeführt wurde, war bei den Studierenden ein Verständnis für die Unterschiede und Gemeinsamkeiten zwischen diesen Sprachen zu erzeugen und sie in die Lage zu versetzen, für einen gegebenen Anwendungsfall eine adäquate Sprache auszuwählen.  

% subsubsection kontext (end)

\subsubsection{Aufgabenstellung und Ablauf} % (fold)
\label{ssub:3_aufgabenstellung}

Die Aufgabe zur Erstellung der Concept Map umfasste zwei Teile, wobei im zweiten Teil das Tabletop Interface eingesetzt wurde. Die Aufgabenstellung lautete in beiden Teilen, eine Concept Map zu erstellen, die die wesentlich erscheinenden Eigenschaften der vorgestellten Sprachen sowie deren Gemeinsamkeiten und Unterschiede darstellt. In der ersten Phase war diese Aufgabe von den Studierenden individuell zu lösen, wobei die Concept Map auf Papier oder mit Hilfe des Werkzeugs CMapTools\footnote{http://cmap.ihmc.us} \citep{Canas04} am Rechner erstellt werden konnte. 

In der zweiten Phase wurden Gruppen zu je drei Teilnehmern gebildet, die nun ihre individuellen Sichten konsolidieren und jeweils eine gemeinsame Concept Map zur gleichen Aufgabenstellung unter Einsatz des hier vorgestellten Werkzeugs erstellen sollten. Die Gruppen wurden zufällig zusammengesetzt, den Teilnehmern war während der individuellen Phase die Zuteilung nicht bekannt, so dass eine Abstimmung vor Anwendung des Werkzeugs weitgehend ausgeschlossen werden kann.

Der zweite Teil der Aufgabenstellung in diesem Block, in dem das Werkzeug zur Anwendung gebracht wurde, entspricht damit einer Ausprägung des Anwendungsszenarios „Abstimmung mentaler Modelle“ (siehe Abschnitt \ref{sub:abstimmung_individueller_mentaler_modelle}).

% subsubsection aufgabenstellung (end)

\subsubsection{Anwendungen und Teilnehmer} % (fold)
\label{ssub:3_teilnehmer}

An den Anwendungen, die in diesem Block durchgeführt wurden, nahmen insgesamt 54 Personen teil, die in 18 Gruppen einmalig mit dem Werkzeug arbeiteten. Alle Teilnehmer waren Studierende der Wirtschaftsinformatik im ersten Studienabschnitt (1-4 Semester), 8 waren weiblich, 46 männlich. Keinem der Teilnehmer war der Ansatz des Concept Mapping vor Beginn der betreffenden Aufgabe bekannt, Erfahrungen mit Prozessmodellierungssprachen (also dem Gegenstand der Concept Map) sammelten alle Teilnehmer erstmals im Rahmen der Lehrveranstaltung, in der dieser Evaluierungs-Block durchgeführt wurde.

Den Teilnehmern wurde das Werkzeug vor Beginn der Anwendung demonstriert und in sämtlichen Anwendungsaspekten erklärt. Die Anwendungen selbst dauerten durchschnittlich 32 Minuten 32 Sekunden ($SD=10:07$), wobei die kürzeste Anwendung 14 Minuten, die längste 45 Minuten dauerte.
% subsubsection teilnehmer (end)

\subsubsection{Verwendung der Ergebnisse} % (fold)
\label{ssub:3_verwendung_der_ergebnisse}

Die Daten, die aus diesem Block gewonnen werden konnten, gehen in die Evaluierung des Modell-Aspekts ein. Hier können einerseits wiederum die erstellten Modelle hinsichtlich Struktur, Inhalt und semantischen Zuweisungen untersucht werden. Der Modellierungsgegenstand ist in diesem Fall jedoch anders gelagert als im vorhergehenden Fall, anstelle eines Arbeitsabstimmung ist hier ein Vergleich von Konzepten durchzuführen. Andererseits können hier die Abstimmungsprozesse der individuellen mentalen Modelle insofern betrachtet werden, als dass für jede Gruppe neben dem kooperativ erstellten Ergebnis auch noch die individuellen Concept Maps vorliegen und ausgewertet werden können.

Wie bereits in den zuvor beschriebenen Blöcken können auch hier wieder Erkenntnisse hinsichtlich der Verwendung des Werkzeugs gewonnen werden. Aufgrund der der Aufgabe innewohnenden Relevanz der Verbindungen zwischen Konzepten ist vor allem deren Verwendung bzw. der Vorgang deren Erstellung zu betrachten.

Der Aspekt Articulation Work bleibt in diesem Block insofern außen vor, als dass kein Arbeitskontext vorliegt, keine aufzulösende Problematik vorliegt und keine Zusammenarbeit auszuhandeln ist. Insofern wird dieser Aspekt in diesem Block nicht explizit behandelt. Aufgrund der Durchführung sämtlicher Schritte, die zur Unterstützung expliziter Articulation Work notwendig sind (Externalisierung, Abstimmung) können aber die einzelnen Anwendungen zur Hypothesenbildung für den Evaluierungs-Aspekt Articulation Work herangezogen werden.

% subsubsection verwendung_der_ergebnisse (end)

\subsection{Block 4: Aushandlung von Zusammenarbeit 2}
\label{sub:eval_4}

Block 4 deckt die erste Anwendung des Werkzeugs im realen Unternehmenskontext ab. Im Rahmen einer Diplomarbeit \citep{Wahlmuller10} wurde das Werkzeug zur Offenlegung unmittelbar relevanter bzw. urgenter Fragestellungen eingesetzt, die im Rahmen eines Workshops zu den Abläufen in und zur Struktur der IT-Abteilung einer Unternehmensgruppe aus dem Bildungsbereich auftraten. Fokus dieses Blocks war die Untersuchung der Einsetzbarkeit des Werkzeugs im praktischen Kontext und dessen tatsächlicher Unterstützungsleistung für Articulation Work. Dazu wurde neben der Begleitung der eigentlichen Modellierungssession in zeitlichem Abstand auch eine Erhebung der wahrgenommenen Wirkungen auf die Arbeitspraxis durchgeführt.

\subsubsection{Kontext} % (fold)
\label{ssub:4_kontext}

Das Werkzeug wird im Kontext einer österreichweit tätigen Unternehmensgruppe im Aus- und Weiterbildungsbereich eingesetzt. Konkret kam das Werkzeug bei einem Workshop zum Einsatz, der von der Abteilung für technisches Produkt- und Service-Management in der konzernweiten IT-Abteilung abgehalten wurde. Die Abteilung hat rund 30 Mitarbeiter, die sich in insgesamt 5 Unterabteilungen gliedern. Zusätzlich ist ein Mitarbeiter abteilungsweit für die Qualitätssicherung der Arbeitsabläufe verantwortlich. Dieser leitete die Workshops, bei denen das Werkzeug zum Einsatz kam und führte in Abstimmung mit den jeweils betroffenen Kollegen die Themenauswahl durch.

% subsubsection kontext (end)

\subsubsection{Aufgabenstellung und Ablauf} % (fold)
\label{ssub:4_aufgabenstellung}

In unterschiedlichen Konstellationen mit Gruppengrößen von 2 bis 6 Personen wurden an zwei Workshop-Terminen Themen aus dem täglichen Arbeitskontext behandelt. Dabei wurden zum Einen Unterabteilungs-interne oder -übergreifende Arbeitsabläufe abgebildet und ausgehandelt, die als potentiell problematisch oder neu einzurichten wahrgenommen wurden. Zum Anderen wurde die wahrgenommene Struktur einer Unterabteilung selbst und deren Außenbeziehungen abgebildet, reflektiert und zwischen den Mitgliedern derselben abgestimmt.

Bei Aufgaben der ersten Kategorie begann die Bearbeitung jeweils mit der kooperativen Repräsentation des Ist-Standes und damit einem Abgleich der individuellen Sichten auf den aktuellen Arbeitsablauf. In weiterer Folge wurde anhand des Modells mögliches Optimierungspotential diskutiert und das Modell ggf. dementsprechend adaptiert.

Bei der Darstellung der Struktur einer Unterabteilung wurden im ersten Schritt die relevanten organisationalen Einheiten und Rollen gesammelt und auf der Oberfläche platziert. In weiterer Folge war die Aufgabe die Zusammenhänge innerhalb der Unterabteilung und deren Beziehungen nach außen durch räumliche Anordnung der definierten Einheiten sowie deren Kommunikationskanäle explizit durch Assoziationen darzustellen. Ziel war eine Repräsentation des Ist-Zustands der Abteilung, die soweit abgestimmt wurde, dass alle Teilnehmer ihre individuelle Sicht auf das Modell abbilden konnten.

Die unterschiedlichen Anwendungen in diesem Block sind entsprechend der obigen Beschreibung als Ausprägungen des Anwendungsszenarios „Abstimmung mentaler Modelle“ (siehe Abschnitt \ref{sub:abstimmung_individueller_mentaler_modelle}) zu betrachten.
% subsubsection aufgabenstellung (end)

\subsubsection{Anwendungen und Teilnehmer} % (fold)
\label{ssub:4_teilnehmer}

Am ersten Workshop-Tag nahmen insgesamt 6 Teilnehmer an 5 Modellierungsdurchgängen in Gruppen von 2 bis 5 Personen teil. Beim zweiten Workshop nahmen insgesamt 8 Teilnehmer an ebenfalls 5 Modellierungsdurchgängen teil. Insgesamt beschäftigten sich 8 Aufgaben mit konkreten Arbeitsabläufen, 2 Aufgaben widmeten sich der Struktur von Unterabteilungen. Die Gruppengröße variierte zwischen 3 und 6 Personen. 10 Teilnehmer nahmen an mehr als einem Modellierungsdurchgang teil, eine Person war an beiden Workshop-Tagen beteiligt.

Durch die Einbindung aller Unterabteilungen kamen Teilnehmer mit unterschiedlichem fachlichen Hintergrund zu Einsatz. Etwa die Hälfte der Teilnehmer war der Gruppe der Techniker oder Softwareentwickler zuzuordnen. Die andere Hälfte setzte sich aus Mitarbeiter im Support, Verkauf, Einkauf sowie der internen Verrechnung zusammen. Eine Teilnehmerin war weiblich, alle anderen Teilnehmer waren männlich.

Allen Teilnehmern wurde einmalig das Werkzeug und dessen Bedienung vorgestellt. Die Modellierungsdurchgänge dauerten zwischen 25 Minuten und etwa 1,5 Stunden. Sämtliche Teilnehmer wurden nach ihrer letzten Teilnahme an einem Durchgang mittels einem Fragebogen sowohl nach der Nützlichkeit des Werkzeugs als auch nach dem wahrgenommenen Nutzen des inhaltlichen Ergebnisses befragt. Um die mittelfristigen Auswirkungen der durchgeführten Modellierungsdurchgänge beurteilen zu können, wurde acht Wochen nach dem zweiten Workshop erneut eine Befragung durchgeführt, in der die wahrgenommene Auswirkungen thematisiert wurden. 

% subsubsection teilnehmer (end)

\subsubsection{Verwendung der Ergebnisse} % (fold)
\label{ssub:4_verwendung_der_ergebnisse}

Die Daten, die das Ergebnis dieses Blocks bilden, werden zur Evaluierung des Aspekts „Articulation Work“ eingesetzt. Betrachtet werden dabei die wahrgenommenen und beobachtbaren Veränderungen am Arbeitsprozess, der unter Einsatz des Werkzeugs reflektiert wurde.

Neben diesem Aspekt werden auch die erstellten Modelle, das in diesem Fall wieder aus der Domäne der Arbeitsabstimmung stammen, betrachtet und hinsichtlich ihrer Struktur und Semantik ausgewertet. 

Der Werkzeug-Aspekt wird in diesem Teil der Untersuchung nicht gesondert betrachtet, Verbesserungs- und Erweiterungspotential wird nur bei Erwähnung oder offensichtlichen Bedienungsfehlern bzw. Verständnisschwierigkeiten explizit identifiziert.

% subsubsection verwendung_der_ergebnisse (end)

\subsection{Block 5: Concept Mapping 2}
\label{sub:eval_5}

In Block 5 wird im Wesentlich der Evaluierungs-Blocks 3 (siehe Abschnitt \ref{sub:eval_3}) inhaltlich erneut durchgeführt (die Modellierungsaufgabe ist identisch). Im Gegensatz zu Block 3, wo die grundlegende Eignung des Werkzeugs zum Concept Mapping im Mittelpunkt stand, wird in Block 5 eine vergleichende Studie durchgeführt, die die Eignung des Tabletop Interfaces zum kollaborativen Concept Mapping mit jener der rechner-basierten CMapTools \citep{Canas04} vergleicht.

\subsubsection{Kontext} % (fold)
\label{ssub:5_kontext}

Die Anwendungssituation ist in diesem Block identisch mit dem in Abschnitt \ref{ssub:3_kontext} beschriebenen Kontext (Lehrveranstaltung im Curriculum Wirtschaftsinformatik zur Schulung von Ansätzen in der Prozess- und Kommunikationsmodellierung).

Der Ablauf der Lehrveranstaltung unterschied sich nur insofern von jenem in Block 3, als dass für jede Modellierungssprache separat eine Reflexion in Gruppen zu zwei Studierenden durchgeführt wurde. In diesen Reflexionen wurden die eigenen Anwendungen der jeweiligen Sprache mit einer Musterlösung gegenübergestellt und hinsichtlich ihrer Korrektheit und dem Vorgehen bei der Modellierung betrachtet.

% subsubsection kontext (end)

\subsubsection{Aufgabenstellung und Ablauf} % (fold)
\label{ssub:5_aufgabenstellung}

Die Aufgabenstellung ist identisch mit jener in Block 3. Ziel ist es, drei in der Lehrveranstaltung vorgestellte Prozessmodellierungssprachen hinsichtlich ihrer als wesentlich empfundenen Eigenschaften und deren Gemeinsamkeiten und Unterschiede zu betrachten und in einer Concept Map abzubilden. Das Vorgehen unterscheiden sich jedoch wegen der unterschiedlichen Zielsetzung der Untersuchung von jenem in Block 3.

Nach Abschluss der letzten Reflexionsphase (also nach drei Modellierungsphasen und drei Reflexionsphasen) wurde eine Gruppeneinteilung für die kollaborative Erstellung der Concept Map vorgenommen. Die Gruppen wurden aus jeweils zwei zufällig ausgewählten Studierenden gebildet. In der Untersuchung erhielt die Hälfte der Gruppen den Aufgabe, die Aufgabenstellung unter Verwendung des Tabletop Interfaces durchzuführen, die andere Hälfte verwendete das rechner-basierte Werkzeug CMapTools \citep{Canas04}, um die Concept Map zu erstellen. Die Gruppen wurden zufällig einem Werkzeug zugeordnet und führten die Aufgabenstellung in beiden Fällen kollaborativ in einer kontrollierten Umgebung durch. Im Gegensatz zu Block 3 entfiel hier die explizit geforderte individuelle Vorbereitungsphase, um eine stärkere inhaltliche Auseinandersetzung mit den Inhalten während der Modellierung zu fördern. 

Wie bereits in Block 3 ist diese Aufgabenstellung eine Ausprägung des Anwendungsszenarios „Abstimmung mentaler Modelle“ (siehe Abschnitt \ref{sub:abstimmung_individueller_mentaler_modelle}).

% subsubsection aufgabenstellung (end)

\subsubsection{Anwendungen und Teilnehmer} % (fold)
\label{ssub:5_teilnehmer}

An der Untersuchung nahmen 49 Studierende in 23 Gruppen teil, wobei 11 Gruppen die Aufgabenstellung unter Verwendung des hier vorgestellen Werkzeugs und 12 Gruppen unter Verwendung der CMapTools durchführten. Die Teilnehmer waren allesamt Studierende der Wirtschaftsinformatik in der ersten Phase des Bakkelauratsstudiums (erstes bis drittes Semester), 40 Teilnehmer waren männlich, 9 weiblich. Keiner der Teilnehmer hatte Vorkenntnisse in der Prozessmodellierung oder im Concept Mapping.

Den Teilnehmern wurde das Werkzeug vor Beginn der Anwendung demonstriert und in sämtlichen Anwendungsaspekten erklärt. Die Anwendungen selbst dauerten im Fall der Durchführung mittels CMapTools durchschnittlich 41 Minuten 10 Sekunden ($SD=8:34$), wobei die kürzeste Anwendung 30 Minuten 16 Sekunden, die längste 54 Minuten dauerte. Im Fall der Durchführung am hier vorgestellten System betrug die Modellierungsdauer im Schnitt 34 Minuten 18 Sekunden ($SD=9:11$), wobei die kürzeste Anwendung 21 Minuten, die längster 54 Minuten dauerte.

% subsubsection teilnehmer (end)

\subsubsection{Verwendung der Ergebnisse} % (fold)
\label{ssub:5_verwendung_der_ergebnisse}

Die in diesem Block erhobenen Daten fließen in vorrangig in den Modell-Aspekt der Evaluierung ein. Hier wird eine vergleichende Studie durchgeführt, die das Ziel hat, die Eignung der beiden verwendeten Ansätze für die Externalisierung von mentalen Modellen gegenüberzustellen. Grundlage dieser Beurteilung ist das erstellte Modell, außerdem wird der auch Modellierungsprozess in der Auswertung berücksichtigt.

Hinsichtlich des Werkzeug-Aspekts wird in diesem Block neben der Identifikation von Verbesserungspotential und Verständnisschwierigkeiten auch die Zufriedenheit mit dem Werkzeug bzw. dessen Akzeptanz bei den Benutzern explizit erhoben. 

Der Aspekt „Articulation Work“ wird hier wie schon in Block 3 und aus den dort angeführten Gründen (siehe Abschnitt \ref{ssub:3_verwendung_der_ergebnisse}) nicht weiter berücksichtigt.

% subsubsection verwendung_der_ergebnisse (end)

\section{Eingesetzte Werkzeuge und Verfahren} % (fold)
\label{sec:eingesetzte_werkzeuge_und_verfahren}

Für die Erfassung und Auswertung der erhobenen Daten kamen unterschiedliche Werkzeuge zum Einsatz. Grundsätzlich wurden sämtliche Anwendungen des Tabletop Interface auf Video erfasst, um eine nachträgliche quantitative und qualitative Auswertung zu ermöglichen. In den Evaluierungsblöcken 1, 4 und 5 kamen aufgrund der Fragestellung außerdem Fragebögen zur Erhebung des Vorwissens bzw. der Erfahrungen bei der Benutzung des Werkzeugs zum Einsatz. Die Auswahl bzw. das Design dieser Fragebögen ist abhängig von den jeweils zu testenden Hypothesen und wird dementsprechend im Rahmen der Beschreibung des Untersuchungsdesigns in den folgenden Kapiteln beschrieben. In allen Anwendungen wurden zudem die Modellierungsergebnisse graphisch festgehalten.

Sämtliche erfassten Daten wurden vor der Auswertung digital aufbereitet. Videos wurden als Mediendateien im MPEG4-Format abgelegt, Fragebögen wurden gescannt und im PDF-Format abgelegt, die graphischen Repräsentationen der Modellierungsergebnisse liegen als Bilddateien im PNG- bzw. JPEG-Format vor.

Die Rohdaten wurden einer quantitativen sowie qualitativen Auswertung unterzogen. Die dazu eingesetzten technischen Werkzeuge und methodischen Ansätze werden in den folgenden Abschnitten näher betrachtet.

\subsection{Werkzeuge} % (fold)
\label{sub:werkzeuge}

Zur Erfassung der quantitativen Daten wurde Microsoft Excel 2007\footnote{http://office.microsoft.com/excel} verwendet. Die deskriptiven statistischen Parameter wurden ebenfalls mit Microsoft Excel sowie mit dem Statistik-Paket R\footnote{http://www.r-project.org/} in der Version 2.9.0 berechnet. Die Parameter der schließenden Statistik wurden ebenfalls mit R berechnet. Die im Bereich der schließenden Statistik eingesetzten Methoden sind Gegenstand der folgenden Abschnitte. Zur Visualisierung der deskriptiven Parameter wurde neben R auch die Software OmniGraphSketcher\footnote{http://www.omnigroup.com/applications/omnigraphsketcher} eingesetzt.

Im Bereich der qualitativen Auswertung war vor allem die Benutzung des Tabletop Interface und die Interaktion der Modellierenden untereinander von Interesse. Zur Auswertung kam dabei einerseits offene Fragen in den eingesetzten Fragebögen und andererseits die von \citet{Hornecker04} vorgeschlagene Variante der Interaktionsanalyse nach \citet{Jordan95} zum Einsatz (siehe Abschnitt \ref{sub:interaktionsanalyse}). Die im Zuge der Durchführung zu erstellenden Transkripte der Interkationsabläufe wurden ohne spezifische Werkzeugunterstützung in einem Texteditor erstellt.
% subsection werkzeuge (end)

\subsection{Signifikanztests} % (fold)
\label{sub:signifikanztests}

Signifikanztests werden verwendet, um zu ermitteln, ob die Unterschiede zwischen zwei Stichproben tatsächlich signifikant sind, d.h. ob sich mit einer gegebenen Irrtumswahrscheinlichkeit (etwa $p<0.05$) auch die beiden den Stichproben zugrundeliegenden Grundgesamtheiten unterscheiden \citep[][S. 496]{Bortz03}. Signifikanztests sind deshalb ein zentraler Bestandteil der quantitativen Hypothesenprüfung.

Je nach Eigenschaften der zugrundeliegenden Grundgesamtheiten und Umfang der Stichprobe müssen unterschiedliche Verfahren zur Signifikanzprüfung eingesetzt werden. In den folgenden Unterabschnitten werden die hier verwendeten Tests kurz beschrieben und die Voraussetzungen für deren Einsatz angeführt. Eine umfassende Beschreibung der Methoden würde den Umfang dieser Arbeit sprengen. Als Grundlage für die Auswahl dienten in dieser Arbeit das einführende Werk von \citet{Bortz03} sowie die Website „Using R for statistical analyses“\footnote{http://gardenersown.co.uk/Education/Lectures/R}. Für eine umfassendere Beschreibung der Methoden sei hier auf diese Quellen verwiesen.

\subsubsection{t-Test} % (fold)
\label{ssub:t_test}

Der t-Test nach Student prüft in der Grundvariante anhand einer Stichprobe, ob der erwartete Mittelwert der entsprechenden Grundgesamtheit gleich, kleiner oder größer einem gegebenen Wert ist. In der -- hier eingesetzten -- Variante für zwei Stichproben prüft der Test, ob der erwartete Mittelwert der der ersten Stichprobe zugrundeliegenden Grundgesamtheit gleich, kleiner oder größer ist als jener der Grundgesamtheit zur zweiten Stichprobe. Für mehr als zwei Stichproben kann der t-Test nicht eingesetzt werden, alternativ kann der Kruskal-Wallis-Test (siehe unten) zur Anwendung gebracht werden.

Der t-Test geht von einer intervallskalierten, normalverteilten Grundgesamtheit aus. Bei Grundgesamtheiten, deren Verteilung unbekannt ist, kann bei ausreichender Stichprobengröße (häufig: $n>30$) auf Grund des zentralen Grenzwertsatzes von einer Normalverteilung ausgegangen werden und der t-Test wiederum eingesetzt werden. Bei kleineren Stichproben kann der t-Test nur dann verwendet werden, wenn eine Normalverteilung der Grundgesamtheit zu erwarten ist. Dies kann auch für kleine Stichproben mit dem Sharpiro-Wilk-Test (siehe unten) überprüft werden. Eine weitere Bedingung für den Einsatz des t-Tests ist, dass die Varianz der Grundgesamtheiten der beiden Stichproben identisch ist. Dies kann mit dem F-Test (siehe unten) festgestellt werden.

% subsubsection t_test (end)

\subsubsection{Wilcoxon-Test} % (fold)
\label{ssub:wilcoxon_text}
Der Wilcoxon-Rangsummentest (oder alternativ: Mann-Whitney-U-Test) ist ein Verfahren zur Überprüfung ob zwei Verteilungen signifikant übereinstimmen. Die Verteilungen müssen im Gegensatz zum t-Test nicht normalverteilt sein, sollten aber eine ähnliche Form aufweisen. Der Wilcoxon-Test ist auch für kleine Stichproben geeignet.

Aufgrund der Stichprobengrößen in den vorliegenden Untersuchungen ist der Wilcoxon-Test dem t-Test hier im Allgemeinen vorzuziehen. 

% % subsubsection wilcoxon_text (end)

\subsubsection{Sharpiro-Wilk-Test} % (fold)
\label{ssub:sharpiro_wilk_test}

Der Sharpiro-Wilk-Test \citep{Shapiro65} testet eine Verteilung auf „Nicht-Normaliät“ (d.h. die Nullhypothese ist, dass die Verteilung nicht normalverteilt ist). Mit einer Irrtumswahrscheinlichkeit von $p<0.05$ kann daher bei Ablehnung der Nullhypothese davon ausgegangen werden, dass die geprüfte Verteilung nicht normalverteilt ist. Dieser Test eignet sich auch für kleine Stichproben (ab $n>3$).

Er wird hier eingesetzt, um zu prüfen, ob der t-Test eingesetzt werden kann oder nicht (da dieser eine Normalverteilung der Parameter voraussetzt).

% subsubsection sharpiro_wilk_test (end)

\subsubsection{F-Test} % (fold)
\label{ssub:f_test}

Der F-Test (oder: Varianzquotienten-Test) überprüft, ob die Varianzen zweier normalverteilter Grundgesamtheiten signifikant übereinstimmen. Er wird hier eingesetzt, um die entsprechende Voraussetzung für den Einsatz des t-Tests zu überprüfen. Muss die Nullhypothese verworfen werden, so muss anstelle des t-Tests der Welch-Test eingesetzt werden, auf den hier nicht näher eingeangen wird.

% subsubsection f_test (end)

\subsubsection{Kruskal-Wallis-Test} % (fold)
\label{ssub:kruskal_wallis_test}

Der Kruskal-Wallis-Test ist wie der Wilcoxon-Test ein Verfahren, mit dem die Übereinstimmung von Verteilungen auf Signifikanz überprüft werden kann. Wie dieser setzt er keine Normalverteilung voraus und eignet sich auch für kleine Stichproben. 

Der Kruskal-Wallis-Test kann jedoch Gegensatz zu den anderen Verfahren auch für die Überprüfung von mehr als zwei Verteilungen gleichzeitig eingesetzt werden. Die Nullhypothese ist, dass sich die Verteilungen nicht unterscheiden. Werden detailliertere Hypothesen benötigt, so muss eine paarweise Überprüfung der Verteilungen mit einem der oben beschriebenen Verfahren vorgenommen werden.

Der Kruskal-Wallis-Test kann ab drei Verteilungen mit einer Stichprobengröße von jeweils mindestens 6 sinnvoll eingesetzt werden. In dieser Arbeit kommt er nur selten zum Einsatz, da großteils lediglich die Signifikanz der Übereinstimmung zweier Verteilungen überprüft werden muss.

% subsubsection kruskal_wallis_test (end)

% subsection signifikanztests (end)

\subsection{Interaktionsanalyse} % (fold)
\label{sub:interaktionsanalyse}

Die Interaktionsanalyse nach \citet{Jordan95} dient der qualitativen Auswertung von Interaktionsabläufen zwischen unterschiedlichen Individuen und den dazu eingesetzten Hilfsmitteln. Grundsätzlich wird die Interaktion aufgezeichnet und transkripiert. Das Transkipt enthält dabei nicht nur die verbale Interaktion sondern auch eine exakte Beschreibung der non-verbalen Aktivitäten der Beteiligten. Insbesondere wurde hier auf die Erfassung der Verwendung der verfügbaren Hilfsmittel geachtet. Die Interaktionsanalyse wurde von \citet{Hornecker04} zur Beschreibung der Wirkung von Tangible Interfaces auf Kooperation zwischen Individuen eingesetzt. Die in diesem Kontext vorgeschlagenen vereinfachte Variation der ursprünglichen Methode (v.a. der Verzicht auf interdisziplinär zusammengesetzte Analysegruppen) wurde in dieser Arbeit übernommen. 

Der Fokus der Analyse lag auf der Nutzung und Wirkung des eingesetzten Werkzeugs, weshalb zur Transkription jene Szenen ausgewählt wurden, in denen diese sichtbar wird. Transkripiert wurde jene Szenen, aus denen im Sinne der festgelegten Auswertungsebenen Schlüsse auf die Verwendung des Werkzeugs, die Wirkung bei der Modellbildung oder den Einfluss auf die Interaktion zwischen den Beteiligten gezogen werden können. Dies umfasst Szenen, in denen
\begin{itemize}
  \item das Werkzeug nicht in der in der Beschreibung des Interaktionsdesigns (siehe Abschnitt \ref{sec:benutzerinteraktion_mit_dem_werkzeug}) festgelegten Art verwendet wurde,
  \item die Funktionalität oder Bedienung des Werkzeugs missverstanden wurde,
  \item das Modell als Referenz verwendet wurde um Sachverhalte individuell zu reflektieren,
  \item das Modell von einem Teilnehmer als Referenz verwendet wurde um anderen Teilnehmern Sachverhalte zu erklären,
  \item das Modell als Mittel zur Fokussierung der inhaltlichen Diskussion zwischen den Teilnehmern verwendet wurde.
\end{itemize}

Bei der Transkription kam das von \citet{Hornecker04} vorgeschlagene Codierungsschema zum Einsatz, das im Folgenden wiedergegeben ist: \emph{„Zeilenweise Transkription nach einfachem, sequentiellem Schema. Zeitlicher Ablauf wird durch die Nummerierung vorne wiedergegeben. Fehlt eine Zeilennummer, so findet das Beschriebene mehr oder minder parallel mit dem Geschehen der darüberstehenden Zeile statt. Zusätzlich alle zehn Sekunden Zeitstempel in eigener Zeile. Sprünge in Zeilennummern zeigen Auslassungen in der Transkription an. Nur angedeutet werden Betonung und Zeitverhalten des Sprechens. Die Zeitdauer von Gestik und manueller Handlung ergibt sich aus den Beschreibungen.“} 

In der Darstellung werden dieses Codierungsschema wiefolgt dargestellt (ebenfalls angelehnt an \citet{Hornecker04}):
\begin{transkript}
	\emph{Zusammenhang, Auslöser der Situation}\\
	Schritt. \textbf{Teilnehmer:} Aussage\\
	\textbf{Teilnehmer:} simultan getätigte Aussage \emph{bzw. Handlung}\\
	Zeitstempel\\	
	Schritt. \textbf{Teilnehmer:} Aussage \emph{(gleichzeitig mit der Aussage ausgeführte Tätigkeit, eingefügt an jener Stelle, an der die Aktivität beginnt)} Aussage (Fortsetzung)\\
	\emph{Interaktion zwischen Teilnehmern oder der Teilnehmer mit dem System}\\
	Schritt. \textbf{Teilnehmer:} Aussage \textbf{Teil der Interaktion bzw. Aussage, der für die Prüfung der jeweiligen Hypothese relevant ist} Aussage (Fortsetzung)\\
\end{transkript}


Beispielhaft dargestellt kann ein Transkript wiefolgt aussehen:
\begin{transkript}
	\emph{Teilnehmer versuchen mit dem Radiergummi und nur einem anderen Marker einen Verbinder zu entfernen.}\\
	1. \textbf{B:} Können wir die nicht so auch einfach löschen?\\
	2. \textbf{C:} Ja mit dem Radiergummi.\\
	3. \textbf{B:} Muss ich den jetzt zuerst so \emph{(Hält den Radiergummi zur Kamera)} hinhalten?\\
	4. \textbf{A:} Nein, ich glaube, \textbf{den musst du einfach da \emph{(zeigt auf den Verbinder)} drauf legen.}\\
	10s\\
	5. \emph{B legt den Radiergummi auf den vom System automatisch erstellten Verbinder.}\\
	20s\\
	6. \textbf{A:} Und jetzt muss man \emph{(legt ein Markierungtoken auf den Verbinder)} Nein.\\
	\emph{Der Verbinder lässt sich auf diese Art nicht löschen und die Teilnehmer entscheiden sich den Fehler mittels der Wiederherstellungsfunktion zu beseitigen.}
\end{transkript}

% subsection interaktionsanalyse (end)
% section eingesetzte_werkzeuge_und_verfahren (end)

% section eingesetzte_verfahren (end)
\section{Zusammenfassung}
\label{sec:eval_ueberblick_zusammenfassung}

In diesem Kapitel wurde das globale Untersuchungsdesign zur Evaluierung der hier vorgestellten Arbeit beschrieben. In den ersten Abschnitten wurden die zu evaluierenden Aspekte identifiziert und beschrieben. Im Rahmen dieser Beschreibungen wurden auch mögliche Ansatzpunkte für die konkrete Untersuchung angeführt, die die Basis für die detaillierte Konzeption der Evaluierung dieser Aspekte in den Kapiteln \ref{cha:eval_werkzeug} bis \ref{cha:eval_aw} bildet. 

Im folgenden Abschnitt wurden die einzelnen im Rahmen der Evaluierung durchgeführten Untersuchungen angeführt. Diese Untersuchungen fokussieren jeweils auf einen der zu evaluierenden Aspekte. Ihnen liegt jeweils ein konkretes Szenario zu Grunde, das in einer Reihe von Anwendungen des Werkzeugs durch verschiedene Benutzer in Modelle umgesetzt wird. Je nach Fokus der Untersuchung werden vor- und nachgelagerte bzw. parallel ablaufende Aktivitäten in die Untersuchung mit einbezogen.

Die ursprüngliche Zuordnung zwischen den zu evaluierenden Aspekten und den einzelnen Evaluierungs-Blöcken ist in Tabelle \ref{tab:evaluierungsMatrixOriginal} nochmals überblicksweise angeführt. Die Zuordnung hatte jeweils Einfluss auf das Szenario, in dem das Werkzeug angewandt wurde sowie auf das Untersuchungsdesign.

\begin{table}[htbp]
	\centering
	\caption{Ursprüngliches globales Untersuchungsdesign}
	\begin{tabular}{| p{3cm} || p{2cm} | p{2cm} | p{2cm} |} \hline
		 & Werkzeug & Modell & Articulation Work \\ \hline \hline
		 Block 1 & x &  &   \\ \hline
		 Block 2 &  &  & x  \\ \hline
		 Block 3 &  & x &   \\ \hline
		 Block 4 &  &  & x  \\ \hline
		 Block 5 &  & x &   \\ \hline
	\end{tabular}
	\label{tab:evaluierungsMatrixOriginal}
\end{table}

Im Zuge der Durchführung der Evaluierung erwies sich die strikte Zuordnung eines Blocks zu genau einem zu evaluierenden Aspekt als nicht durchführbar. Tatsächlich liefern Untersuchungen zu einem (im Sinne der Zielhierarchie) übergeordneten Aspekte (von „unten“ nach „oben“: Werkzeug -- Modell -- Articulation Work) immer auch Erkenntnisse zu den untergeordneten zu evaluierenden Aspekten. Die Zuordnung der Evaluierungs-Blöcke zu den Aspekten verändert sich also wie in Tabelle \ref{tab:evaluierungsMatrix} angegeben. Diese Zuordnung liegt auch den oben angeführten Beschreibungen der Blöcke zugrunde, in denen jeweils die Beiträge eines Blocks zu den zu evaluierenden Aspekten angegeben wurden.

\begin{table}[htbp]
	\centering
	\caption{Einfluss der Untersuchungen auf die zu evaluierenden Aspekte}
	\begin{tabular}{| p{3cm} || p{2cm} | p{2cm} | p{2cm} |} \hline
		 & Werkzeug & Modell & Articulation Work \\ \hline \hline
		 Block 1 & \textbf{x} &  &   \\ \hline
		 Block 2 & x & x & \textbf{x}  \\ \hline
		 Block 3 & x & \textbf{x} &   \\ \hline
		 Block 4 & x & x & \textbf{x}  \\ \hline
		 Block 5 & x & \textbf{x} &   \\ \hline
	\end{tabular}
	\label{tab:evaluierungsMatrix}
\end{table}

In den folgenden Kapiteln wird nun die Evaluierung der einzelnen Aspekte über die Evaluierungs-Blöcke hinweg im Detail beschrieben. Dabei werden die Hypothesenbildung bzw. die Entwicklung der Hypothesen über die Zeit, die möglichen Ansätze zur Evaluierung der jeweiligen Hypothesen sowie das Untersuchungsdesign, das die Prüfung der Hypothesen ermöglicht, beschrieben. Die Kapitel schließen jeweils mit einer Zusammenfassung der Ergebnisse der Hypothesenprüfung und einer Bewertung dieser Ergebnisse im Kontext der globalen Zielsetzung, also der Unterstützung von expliziter Articulation Work.

% section globales_untersuchungsdesign (end)
% chapter eval_ueberblick (end)

\chapter{Evaluierung der Verwendbarkeit des Werkzeugs} % (fold)
\label{cha:eval_werkzeug}

Im ersten empirischen Teil der Evaluierung wurde die grundlegende Verständlichkeit und Verwendbarkeit des Werkzeug geprüft. Ziel war es hier, konzeptionelle und technische Eigenschaften bzw. Verhaltensweisen des Werkzeugs zu identifizieren, die den Modellierungsprozess behindern oder unterbrechen. Darunter fällt grundsätzlich jede Eigenschaft und jede Verhaltensweise, die die Benutzer zwingt, sich mit dem technischen System an sich zu beschäftigen und von der Erfüllung der eigentlichen Aufgabe ablenkt bzw. diese unterbricht. Abbildung \ref{fig:img_Kontextgrafiken_k12} stellt dieses Kapitel und dessen Aufbau im Kontext der anderen inhaltlich vor- und nachgelagerten Kapitel dar.

\begin{figure}[htbp]
	\centering
		\includegraphics[scale=0.6]{img/Kontextgrafiken/k12.png}
	\caption{Kapitel „Evaluierung der Verwendbarkeit des Werkzeugs“ im Gesamtzusammenhang}
	\label{fig:img_Kontextgrafiken_k12}
\end{figure}

Die Untersuchung wurde daneben auch genutzt, um explorativ die inhaltliche Verwendung des Systems zu untersuchen (d.h. wie es für seinen eigentlichen Verwendungszweck, die Modellierung, eingesetzt wurde) und Hypothesen abzuleiten, die in weiteren Schritten getestet werden konnten.

\section{Hypothesen} % (fold)
\label{sec:hypothesen}

In diesem Abschnitt werden die Hypothesen angeführt und begründet, die in diesem Teil der empirischen Untersuchung geprüft werden. Die hier angegebenen Hypothesen gehen auf die Eigenschaften des Werkzeugs in der Verwendung durch die Benutzer ein. Bei der Hypothesenbildung wird auf den Verwendungszweck des Werkzeugs, die Unterstützung der Bildung diagrammatischer Modelle, Rücksicht genommen -- die Modelle selbst sind jedoch nicht Gegenstand der Betrachtung, sondern werden erst im nächsten Kapitel behandelt. Nicht berücksichtigt wird außerdem die Verwendung zur Unterstützung von Articulation Work -- die Implikationen des Werkzeugs auf diese sind Gegenstand von Kapitel \ref{cha:eval_aw}.

\subsection{Konzeptionell begründete Hypothesen} % (fold)
\label{sub:konzeptionell_begründete_hypothesen}

Die folgenden Hypothesen wurden aus der Aufgabenstellung (siehe Kapitel \ref{cha:einführung}) sowie den Anforderungen an das Werkzeug (siehe Kapitel \ref{cha:anforderungen}) abgeleitet. Neben der Formulierung der Hypothese ist jeweils die Begründung aus der Konzeption des Werkzeugs angeführt.

Der grundlegende Anspruch des Werkzeugs ist es, explizite Articulation Work zu unterstützen. Wie in Teil \ref{prt:grundlagen} dieser Arbeit beschrieben, wird dies hier über die Externalisierung und Aushandlung von mentalen Modellen realisiert. Ein gängiges Mittel, um mentale Modelle zu repräsentieren, sind diagrammatische Modelle, worunter die Ergebnisse der vorgeschlagenen Methoden zur Externalisierung -- Concept Mapping und Strukturlegetechniken -- fallen. Das Werkzeug muss also die Repräsentation diagrammatischer Modelle unterstützen. Die Prüfung dieser Hypothese ermöglicht die Beurteilung der Erfüllung der Anforderung \ref{anf:physische_abbildung_legen_beliebiger_diagrammatischer_modelle} (siehe Seite \pageref{anf:physische_abbildung_legen_beliebiger_diagrammatischer_modelle}). 

\begin{hyp}
	\label{hyp:diagmodelle}
	Das Werkzeug ermöglicht die Repräsentation diagrammatische Modelle.
\end{hyp}

„Articulation Work“ ist immer in einen kooperativen Arbeitszusammenhang eingebettet. Die Kollaboration findet dabei nicht nur im produktiven Teil der Arbeit statt, sondern hat immer auch Auswirkungen auf die „Articulation Work“. Jede Unterstützung von „Articulation Work“ muss damit auch in kooperativen Szenarien einsetzbar sein. Dies gilt auch für das hier vorgestellte Werkzeug, das die kooperative Bearbeitung einer Aufgaben (hier: der Externalisierung und Abstimmung mentaler Modelle) ermöglichen muss. Die Prüfung dieser Hypothese ermöglicht die Beurteilung der Erfüllung der Anforderung \ref{anf:kollaborative_und_unmittelbare_manipulierbarkeit_des_modells} (siehe Seite \pageref{anf:kollaborative_und_unmittelbare_manipulierbarkeit_des_modells}).

\begin{hyp}
	\label{hyp:kollaborativ}
	Das Werkzeug ermöglicht kooperatives Arbeiten an einer Aufgabe.
\end{hyp}

Die Aspekte von Arbeit, die im Rahmen von „Articulation Work“ abzustimmen sind, sind unterschiedlicher Natur. Naheliegend ist eine Abstimmung der Abläufe und Schnittstellen zwischen Personen, aber auch nicht-prozedurale Information wie das Verständnis der Struktur und Elemente eines Arbeitszusammenhangs kann Gegenstand von Articulation Work sein. Gleiches gilt für die im Rahmen der Articulation Work abzustimmenden mentalen Modelle -- diese bilden die Basis für Handlungsentscheidungen, umfassen aber im Allgemeinen (in Abgrenzung zu Schemata) nicht nur handlungsleitende Information sondern auch Kontextinformation, die die Bewertung der wahrgenommenen Situation ermöglicht. Demensprechend muss ein Werkzeug zu Unterstützung von expliziter Articulation Work und damit der Externalisierung von mentalen Modellen die Verwendung in unterschiedlichen Kontexten, d.h. für unterschiedliche zu externalisierenden Informationsstrukturen, die in mentalen Modellen abgebildet sind, ermöglichen. Die Prüfung dieser Hypothese ermöglicht die Beurteilung der Erfüllung der Anforderung \ref{anf:nicht_vorgegebene_semantik_der_modellierungselemente} (siehe Seite \pageref{anf:nicht_vorgegebene_semantik_der_modellierungselemente}).

\begin{hyp}
	\label{hyp:kontexte}
	Das Werkzeug ist gleichwertig für Modellierungsaufgaben in unterschiedlichen Kontexten einsetzbar.
\end{hyp}

Die ersten drei hier formulierten Hypothesen sind unmittelbar aus der globalen Zielsetzung abgeleitet und bilden die grundlegenden Anforderungen an das Werkzeug bei der Unterstützung von Articulation Work ab. Die nun folgenden Hypothesen sind konzeptionell nicht mehr direkt auf die globale Zielsetzung ausgerichtet sondern stellen auf Funktionalität des Werkzeugs ab, die den Modellbildungsprozess unterstützen soll. 

Auf Basis der Möglichkeit zur Navigation durch die Entstehungsgeschichte des Modells besteht auch die Möglichkeit, vergangene Modellzustände wiederherzustellen. Das Werkzeug unterstützt dabei die Benutzer durch die Ausgabe von schrittweisen Anweisungen, die den aktuellen Modellzustand in den wiederherzustellenden Zustand überführen. Allgemein bietet diese Funktionalität die Möglichkeit, erkannte Fehler im Modell zu korrigieren, ohne dabei bereits repräsentierte Information zu verlieren. Im kollaborativen Einsatz ermöglicht diese Funktionalität, alternative, individuelle Sichten auf den abzustimmenden Sachverhalt zu repräsentieren und dabei die Möglichkeit bieten, einen für alle Beteiligten akzeptablen Ausgangspunkt wiederherzustellen. Die Prüfung dieser Hypothese ermöglicht die Beurteilung der Erfüllung der Anforderung \ref{anf:ermöglichung_experimenteller_veränderungen_am_modell} (siehe Seite \pageref{anf:ermöglichung_experimenteller_veränderungen_am_modell}).

\begin{hyp}
	\label{hyp:wiederherstellung}
	Die Möglichkeit der Wiederherstellung vergangener Modellzustände fördert die Bereitschaft alternative Repräsentationen auszuprobieren.
\end{hyp}

Die letzten beiden Hypothesen dieses Abschnitts sind ausschließlich auf die Verwendung des Werkzeugs an sich ausgerichtet und stehen nicht im Kontext von Articulation Work oder der Unterstützung der Externalisierung mentaler Modelle. Hypothese \ref{hyp:behinderung} steht für den in der Zielsetzung formulierten Anspruch, dass das Werkzeug in den Hintergrund treten muss und die Beschäftigung mit der eigentlichen Aufgabe nicht behindern darf. Dabei wird hier nicht auf den konkreten Anwendungsfall -- die Erstellung von Modellen -- eingegangen sondern lediglich die allgemeine Funktionsfähigkeit und Bedienbarkeit des Werkzeugs betrachtet. Ersteres ist Gegenstand der Evaluierung der erstellten Modelle, die in Kapitel \ref{cha:eval_modell} beschrieben werden. Die Prüfung dieser Hypothese ermöglicht die Beurteilung der Erfüllung der Anforderung \ref{anf:physische_abbildung_legen_beliebiger_diagrammatischer_modelle} (siehe Seite \pageref{anf:physische_abbildung_legen_beliebiger_diagrammatischer_modelle}).

\begin{hyp}
	\label{hyp:behinderung}
	Das Werkzeug behindert die Modellbildung nicht.
\end{hyp}

Hypothese \ref{hyp:gewöhnung} geht davon aus, dass bei wiederholten Verwendung des Werkzeugs Lern- und Gewöhnungseffekte auftreten, die die Verwendung erleichtern, beschleunigen und zu weniger Fehlbedienung führen. Dies ist ein Effekt, der bei jedem Werkzeug zu erwarten ist, dessen zugrunde liegenden Konzepte den Benutzern bewusst sind. Von dieser Voraussetzung kann durch die inhaltliche Einführung der Benutzer in die das Werkzeug prägenden und motivierenden Ideen ausgegangen werden. Damit wäre zu erwarten, dass das Werkzeug bei wiederholtem Einsatz in den späteren Anwendungen effizienter (im Sinne von schneller und Fehlbedienungen vermeidend) verwendet wird. Die Prüfung dieser Hypothese ermöglicht die Beurteilung der Erfüllung der Anforderung \ref{anf:physische_abbildung_legen_beliebiger_diagrammatischer_modelle} (siehe Seite \pageref{anf:physische_abbildung_legen_beliebiger_diagrammatischer_modelle}).

\begin{hyp}
	\label{hyp:gewöhnung}
	Wiederholte Verwendung des Werkzeugs führt zu schnellerer Modellbildung und weniger Fehlbedienungen.
\end{hyp}

Hinsichtlich der in Kapitel \ref{cha:anforderungen} formulierten Anforderungen können die hier formulierten Hypothesen zusammenfassend wie in Tabelle \ref{hyp:eval_tui} dargestellt eingeordnet werden:

\begin{table}[htbp]
	\centering
	\caption{Hypothesen zur Werkzeugbenutzung und deren Bezug zu den Anforderungen an das Werkzeug}
\begin{tabular}{|c|c|}
  \hline
   Hypothese & Anforderung \\ \hline
   \ref{hyp:diagmodelle} & \ref{anf:physische_abbildung_legen_beliebiger_diagrammatischer_modelle} \\
   \ref{hyp:kollaborativ} & \ref{anf:kollaborative_und_unmittelbare_manipulierbarkeit_des_modells} \\
   \ref{hyp:kontexte} & \ref{anf:nicht_vorgegebene_semantik_der_modellierungselemente} \\
   \ref{hyp:wiederherstellung} & \ref{anf:ermöglichung_experimenteller_veränderungen_am_modell} \\
   \ref{hyp:behinderung} & \ref{anf:physische_abbildung_legen_beliebiger_diagrammatischer_modelle} \\
   \ref{hyp:gewöhnung} & \ref{anf:physische_abbildung_legen_beliebiger_diagrammatischer_modelle} \\ \hline
\end{tabular} 
	\label{hyp:eval_tui}
\end{table}

% subsection konzeptionell_begründete_hypothesen (end)

\subsection{Explorativ gebildete Hypothesen} % (fold)
\label{sub:explorativ_gebildete_hypothesen}

Neben den aus der Aufgabenstellung abgeleiteten Hypothesen wurden einige Hypothesen auch während der Durchführung der einzelnen Evaluierungs-Blöcke gebildet. Diese Hypothesen sind spezifischer auf einzelne Aspekte des Werkzeugs abgestellt und decken beobachtete Auffälligkeiten und Missverständnisse in der Verwendung des Werkzeugs ab. 

Die erste in diesem Zusammenhang beobachtete Auffälligkeit betrifft die Herstellung von Verbindern zwischen einzelnen Modellelementen. Wie in Abschnitt \ref{sub:verbinden_von_modellelementen} beschrieben, existieren zwei Möglichkeiten, diese Funktion auszuführen. Einerseits können die beiden Modellelemente, die verbunden werden sollen, mit Markierungs-Tokens ausgewählt werden, worauf hin eine Verbindung hergestellt werden. Andererseits können Verbinder auch durch das Zusammenführen der zu verbindenden Blöcke (bis sich deren Breitseiten berühren) hergestellt werden. In der ersten Implementierung des Werkzeugs, die im Evaluierungs-Block 1 und im ersten Teil des zweiten Blocks verwendet wurde, war lediglich die erste Variante verfügbar. Die Möglichkeit zur Herstellung von Verbindern wurde in den in diesen Blöcken durchgeführten Anwendungen kaum eingesetzt. Dies führte einerseits zur Bildung der Hypothese \ref{hyp:keine_verbinder} (siehe Abschnitt \ref{sub:m_explorativ_gebildete_hypothesen}), andererseits wurde bei ersten Auswertungen der Beobachtungen der im Verhältnis zum übrigen Modellierungs-Prozess hohe Zeit-Aufwand bei der Herstellung von Verbindern offensichtlich. Dieser Aufwand ist den Maßnahmen zur Stabilisierung der Erkennungsleistung des Werkzeugs geschuldet und kann mit den eingesetzten Interaktionsablauf nicht reduziert werden. Aufgrund einer Anregung eines Untersuchungsteilnehmers wurde deshalb die oben beschriebene zusätzliche Möglichkeit zur Herstellung von Verbindungen implementiert. Zu untersuchen ist nun, ob diese Maßnahme die Nutzung von Verbindern bei der Modellbildung tatsächlich erhöht.

\begin{hyp}
	\label{hyp:verbinder}
	Die Einführung der alternativen Möglichkeit zur Verbindungsherstellung erhöht die Nutzung von Verbindern bei der Modellerstellung.
\end{hyp}

Die zweite hier aufgestellte Hypothese betrifft eine Auffälligkeit bei der Verwendung des Löschtokens. Das Löschtoken wird verwendet, um das Werkzeug in einen Modus zu versetzen, in dem Verbinder gelöscht werden können. Schon die konzeptionelle Einordnung des Werkzeugs in Kapitel \ref{cha:konzeptionelle_evaluierung} zeigte Potential für Missverständnisse in der Verwendung dieses Tokens (siehe z.B. die Abschnitte \ref{sec:spezifikation_des_tac_schemas_nach_shaer_et_al_} und \ref{sec:einordnung_in_die_taxonomie_von_fishkin}). Zusammengefasst liegt die aus der Theorie ableitbare Problematik darin, dass durch die äußere Form des Tokens -- einem Radiergummi -- eine Metapher für dessen Verwendung („ausradieren“ von Elementen) suggeriert wird, die in dieser Form im Werkzeug nicht umgesetzt ist, da das Token lediglich als Schalter fungiert. Erste Beobachtungen deuteten darauf hin, dass die Verwendung des Löschtoken tatsächlich unverständlich oder missverständlich zu sein scheint. Die zugehörige Hypothese ist positiv formuliert, zu erwarten wäre demnach, dass sie verworfen werden muss.

\begin{hyp}
	\label{hyp:radierer}
	Das Löschtoken ermöglicht intuitives Löschen von Modellelementen.
\end{hyp}

% subsection explorativ_gebildete_hypothesen (end)
% section hypothesen (end)

\section{Untersuchungsdesign und Durchführung} % (fold)
\label{sec:untersuchungsdesign}

In diesem Abschnitt wird auf Basis der oben formulierten Hypothesen das Untersuchungsdesign abgeleitet und die Durchführung der Untersuchung beschrieben. Der erste Teil des Abschnitts beschreibt die Operationalisierung der Hypothesen und damit die Festlegung wie diese konkret geprüft werden können. Im zweiten Teil des Abschnitts wird die Durchführung der Prüfung beschrieben. Hier erfolgt neben der Zuordnung der einzelnen Evaluierungsblöcke (siehe Abschnitt \ref{sec:globales_untersuchungsdesign}) auch die Darstellung rein beschreibender Parameter der Werkzeugverwendung, die nicht unmittelbar in die Prüfung der Hypothesen eingehen. 

\subsection{Operationalisierung} % (fold)
\label{sub:operationalisierung}

In diesem Abschnitt wird für jede Hypothese identifiziert, in welcher Form sie geprüft werden kann. Dies umfasst die Festlegung der Messpunkte sowie der jeweiligen Mess- und Auswertungsmethode (letzte bezugnehmend auf den in Abschnitt \ref{sec:eingesetzte_werkzeuge_und_verfahren} beschriebenen Verfahren). Zudem werden jene Evaluationsblöcke festgelegt, die für die jeweilige Untersuchung herangezogen wurden.

Für jede Hypothese wird also spezifiziert, anhand welcher Aspekte diese geprüft werden kann (= abhängige Variablen). Zudem wird festgelegt welche Ausgangssituation bei der Anwendung gewählt werden muss, um die Prüfung durchführen zu können (= unabhängige Variable) und welche Faktoren die Beurteilung ggf. ungewollt beeinflussen können (= Störvariablen).

\subsubsection{Repräsentation diagrammatischer Modelle} % (fold)
\label{ssub:repräsentation}

Gegenstand dieses Abschnitts ist die Prüfung der Hypothese \ref{hyp:diagmodelle}. Diese bezieht sich auf die Eignung des Werkzeugs für die Repräsentation diagrammatischer Modelle.

Voraussetzung für die Prüfung der Hypothese ist der Einsatz von Modellierungsaufgaben, die so formuliert sind, dass es grundsätzlich möglich ist, sie durch die Beschreibung in einem diagrammatischen Modell zu erfüllen. Keinen Einfluss auf die Untersuchung haben die eingesetzte Methodik sowie eventuell vorhandene Modellierungsvorkenntnisse, da die grundsätzlich Möglichkeit der Erstellung diagrammatischer Modelle unabhängig von der Art der Verwendung und von der Kompetenz der Benutzer ist. 

Geprüft wird die Hypothese hier an der Repräsentation, die mit Hilfe des Werkzeugs erstellt wurde. Ein diagrammatisches Modell zeichnet nach \citep{Larkin87} aus, dass in ihm Konzepte und deren Zusammenhänge visuell-graphisch dargestellt werden können (in Abgrenzung zu textuellen Beschreibungen). Zur Bewertung der Hypothese werden deshalb die erstellten Repräsentationen herangezogen und überprüft, ob sie den Anforderungen an ein diagrammatisches Modell -- das Vorhandensein von Konzepten und Beziehungen zwischen diesen -- erfüllen.

% subsubsection repräsentation (end)

\subsubsection{Kooperatives Arbeiten} % (fold)
\label{ssub:kollaboratives_arbeiten}

Gegenstand dieses Abschnitts ist die Prüfung der Hypothese \ref{hyp:kollaborativ}. Dabei wird überprüft, ob das Werkzeug kooperatives Arbeiten an einer Modellierungsaufgabe erlaubt.

Dazu muss eine Modellierungsaufgabe gewählt werden, in der die kooperatives Erstellung des Modells vorgesehen ist. Etwaige Modellierungsvorkenntnisse haben keinen Einfluss auf die Beurteilung der hier betrachteten Hypothese.

Zur Beurteilung eignen sich in diesem Fall die Zeitverteilung der Beteiligung der einzelnen Benutzer am Modellierungsvorgang, das Verhalten der Benutzer bei simultaner Manipulation eines Modells auf der Modellierungsoberfläche sowie der subjektive Eindruck der Benutzer über deren Kooperation untereinander. Der erstgenannte Aspekt kann quantitativ gemessen werden, wobei eine tendenziell zeitlich gleichverteilte Einbindung der Beteiligten in die Modellbildung für die Annahme der Hypothese spricht. Zusätzlich kann mittels dem zweiten und dritten Aspekt qualitativ beurteilt werden, ob und wie eine kooperative Manipulation des Modells durch mehrere Benutzer gleichzeitig möglich ist.

% subsubsection kollaboratives_arbeiten (end)

\subsubsection{Einsetzbarkeit in unterschiedlichen Kontexten} % (fold)
\label{ssub:einsetzbarkeit_in_unterschiedlichen_kontexten}

Gegenstand dieses Abschnitts ist die Operationalisierung der Hypothese \ref{hyp:kontexte}. Diese Hypothese zielt dabei auf die Eignung des Werkzeugs zur Modellbildung in unterschiedlichen Kontexten, d.h. für unterschiedliche Modellierungsaufgaben. 

Zur Beurteilung dieser Hypothese muss die Modellierungsaufgabe entsprechend den unterschiedlichen Einsatzkontexten variiert werden. Etwaige Modellierungsvorkenntnisse können die individuelle Beurteilung insofern beeinflussen, als das sie Werkzeugs für eine bestimmte Aufgabe als besser oder schlechter geeignet erscheinen lassen.

Zur Prüfung der Hypothese bieten sich sind in diesem Fall die Wahrnehmung der Eignung durch die Benutzer, die qualitativ beurteilt wird, und die Korrelation der Größe der erstellten Modelle mit der benötigten Modellierungsdauer an. Korrelliert die Modellgröße positiv mit der Modellierungsdauer, so ist der Zeitanteil, der zu Beschäftigung mit dem Werkzeug selbst (und nicht mit der Modellierungsaufgabe) tendenziell stabil. Daraus kann abgeleitet werden, dass das Werkzeug die verglichenen Modellierungsaufgaben gleich gut (oder schlecht) unterstützt.

% subsubsection einsetzbarkeit_in_unterschiedlichen_kontexten (end)

\subsubsection{Wiederherstellung vergangener Modellzustände} % (fold)
\label{ssub:wiederherstellung_vergangener_modellzustände}

Gegenstand dieses Abschnitts ist die Operationalisierung der Hypothese \ref{hyp:wiederherstellung}. Gegenstand der Überprüfung ist die Verwendung der Wiederherstellungsfunktionalität zum Zwecke der versuchsweisen Veränderung des Modells.

Zur Prüfung dieser Hypothese muss die Modellierungsaufgabe so gestaltet, dass sinnvoll unterschiedliche Repräsentationen gebildet werden können. Modellierungsvorkenntnisse haben keine Auswirkungen auf diese Untersuchung.

Zur Beurteilung dieser Hypothese wird ist die \emph{Anzahl der Verwendungen der Wiederherstellungsfunktionalität zur Korrektur inhaltlich verworfener Repräsentationen} herangezogen. Werte über 0 deuten hier auf eine Annahme der Hypothese hin. Zusätzlich können qualitative Aussagen zur Nutzung dieser Funktionalität und deren \emph{wahrgenommenen Nutzen} zur Beurteilung verwendet werden. 

% subsubsection wiederherstellung_vergangener_modellzustände (end)

\subsubsection{Nicht-Behinderung} % (fold)
\label{ssub:nicht_behinderung}

Gegenstand dieses Abschnitts ist die Operationalisierung der Hypothese \ref{hyp:behinderung}. Dabei wird überprüft, ob bei der Verwendung des Werkzeugs dieses in den Aufmerksamkeitsfokus der Benutzer tritt oder sich diese auf die eigentliche Modellierungsaufgabe konzentrieren können. 

Die Modellierungsaufgabe hat keinen Einfluss auf die Überprüfung dieser Hypothese, lediglich etwaig vorhandene \emph{Modellierungsvorkenntnisse} können als \textbf{Störvariable} wirken, da sie Einfluss auf die erwartete Funktionalität des Werkzeugs haben kann.

Zur Beurteilung, ob bzw. inwieweit das Werkzeug die Modellbildung behindert, werden sowohl quantitativ als auch qualitative beurteilbare Metriken herangezogen. Die Anzahl von \emph{Fehlfunktionen des Werkzeugs} bzw. das \emph{Auftreten von Systemabstürzen} kann als Indikator für eine behindernde Wirkung des Werkzeugs herangezogen werden. Das Auftreten von Missverständnissen und daraus resultierende Fehlbedienungen können ebenfalls eine Behinderung des Modellierungsvorgangs interpretiert werden. Zudem werden Aussagen der Benutzer hinsichtlich hinderlicher Faktoren bei der Werkzeugbenutzung als Maß für die wahrgenommene Behinderung durch das Werkzeug herangezogen. Der Einfluss von Modellierungsvorkenntnissen kann in diesem Fall nicht mit statistischen Maßnahmen kompensiert werden. Etwaige Vorkenntnisse werden dementsprechend bei der Auswertung angeführt und müssen bei der Diskussion der Hypothese berücksichtigt werden.

% subsubsection nicht_behinderung (end)

\subsubsection{Gewöhnung an das Werkzeug} % (fold)
\label{ssub:gewöhnung_an_das_werkzeug}

Gegenstand dieses Abschnitts ist die Operationalisierung der Hypothese \ref{hyp:gewöhnung}. Dabei wird überprüft, ob wiederholte Benutzung des Werkzeugs Auswirkung auf die Qualität der Interaktion hat. Eine Erhöhung der Qualität äußert sich in schnellerer Modellbildung und weniger Fehlbedienung.

Bei der Prüfung der Hypothese muss eine etwaige veränderte Funktionalität des Werkzeugs zwischen den verglichenen Evaluierungsblöcken berücksichtigt werden, die die Interaktion einerseits erleichtern kann, andererseits aber auch zu Fehlbedienung aufgrund von unbekannten Interaktionsmustern führen kann. Auch unterschiedliche Modellierungsaufgaben, die ein Individuum in den aufeinander folgenden Anwendungen bearbeitet, können die Beurteilung erschweren, weil potentiell andere (noch unbekannte) Funktionen des Werkzeugs zum Einsatz kommen können.

Zur Beurteilung der Qualität der Interaktion sind einerseits die Anzahl der Fehlbedienungen des Werkzeugs pro Zeiteinheit und andererseits die Arbeitsdauer am Werkzeug\footnote{Die Arbeitsdauer am Werkzeug ist im Gegensatz zur gesamten Modellierungsdauer um jenen Zeitanteil reduziert, in dem die Teilnehmer interagieren, ohne am Werkzeug zu arbeiten.} in Abhängigkeit der Modellgröße heranzuziehen. Die Normierung der Arbeitsdauer ist notwendig, um vergleichbare Werte für unterschiedliche Werkzeug-Anwendungen zu erhalten. Sinken beide Werte zwischen zwei Evaluierungsblöcken, die auf der gleichen Stichprobe aufbauen, signifikant, so kann die Hypothese bestätigt werden. Um eine Vergleichbarkeit zwischen den Anwendungen herzustellen, ist es sinnvoll, in beiden Blöcken eine identische Modellierungsaufgabe zu stellen und die Funktionalität des Werkzeugs nicht zu verändern. Identische Modellierungsaufgaben können durch die wiederholte inhaltliche Beschäftigung mit der Aufgabe zu schnellerer Arbeit bzw. zu kompakteren Modellen führen. Dies kann wiederum durch die Berücksichtigung der reinen Arbeitszeit am Werkzeug sowie der Normierung derselben in Abhängigkeit der Modellgröße kompensiert werden.

% subsubsection gewöhnung_an_das_werkzeug (end)

\subsubsection{Herstellung von Verbindern} % (fold)
\label{ssub:herstellung_von_verbindern}

Gegenstand dieses Abschnitts ist die Operationalisierung der Hypothese \ref{hyp:verbinder}. Mit Hilfe dieser Hyothese soll überprüft werden, ob die Einführung der alternativen Möglichkeit zur Herstellung von Verbindern deren Verwendung signifikant gesteigert hat.

Bei der Messung muss der Einfluss der Modellierungsaufgabe (da sie die Anzahl der benötigten Verbinder beeinflussen kann) und eventuell vorhandene Modellierungsvorkenntnisse (da diese Einfluss auf die Struktur des Modells haben können) berücksichtigt werden. Um den Einfluss dieser Aspekte zu reduzieren, wird die Beurteilung in zwei Evaluierungsblöcken vorgenommen, in denen die gleiche Stichprobe mit der gleichen Aufgabenstellung das Werkzeug mit der gleichen Methodik anwandte. Lediglich die Funktionalität des Werkzeugs wurde zwischen den beiden Anwendungen um den alternativen Weg zur Herstellung von Verbindern erweitert.  

Zur Beurteilung des Ausmaßes der Verwendung von Verbindern kann die \emph{Connectedness} des Modells herangezogen werden. Die Connectedness ist das Verhältnis zwischen der Anzahl der im Modell verwendeten Verbinder und der Anzahl der verwendeten Knoten (Modellierungselemente). Hier ist zu prüfen, ob die Connectedness in jenem Evaluierungs-Block, in dem der alternative Weg zur Herstellung von Verbindungen verfügbar war, signifikant höher ist als in jenem Block, in dem sie nicht verfügbar war.

% subsubsection herstellung_von_verbindern (end)

\subsubsection{Verwendung des Löschtokens} % (fold)
\label{ssub:löschtoken}

Gegenstand dieses Abschnitts ist die Operationalisierung der Hypothese \ref{hyp:radierer}. Dabei wird überprüft, ob das Löschtoken intuitiv korrekt verwendet wird oder ob es zu Fehlinterpretationen kommt.

Die Verwendbarkeit des Löschtokens ist unabhängig von der Modellierungsaufgabe, der angewandten Methodik und auch von eventuell vorhandenen Modellierungsvorkenntnissen. Hinsichtlich des Anwendungskontext des Werkzeugs sind also keine Voraussetzungen zu beachten.

Zur Beurteilung der intuitiven Verwendbarkeit werden quantitative und qualitative Merkmale der Werkzeugverwendung herangezogen. Quantitativ beurteilbar ist der Anteil der Fehlbedienungen des Löschtokens in Bezug auf alle Anwendungen des Werkzeugs, in denen es grundsätzlich verwendet wurde. Qualitativ wird die Art des Missverständnisses, das zu den jeweiligen Fehlbedienungen führt, beurteilt.

Zur Messung der quantitativen Variablen wird für jede Anwendung die Anzahl der Fehlbedienungen erhoben, die durch das Löschtoken verursacht wurden. Dieser Wert wird in Bezug zur Gesamtanzahl der Fehlbedienungen gesetzt, so dass der Anteil der durch das Löschtoken verursachten Fehlbedienungen berechnet werden kann. Bei "gleich guter" intuitiver Bedienbarkeit aller Werkzeuge wäre eine Gleichverteilung der Fehler zu erwarten. Ist der Anteil der durch das Löschtoken verursachten Fehlbedienungen höher als der Anteil, der bei Gleichverteilung zu erwarten wäre, so deutet dies auf eine Ablehnung der Hypothese hin.

Qualitativ werden Modellierungssituationen betrachtet, in denen das Löschtoken zum Einsatz kommt. Auf Basis von Transkripten der Interaktion zwischen den Benutzern und dem Werkzeug, bei denen es zu Fehlbedierungen kam, werden die aufgetretenen Missverständnisse explizit identifziert.

% subsubsection löschtoken (end)

% subsection operationalisierung (end)

\subsection{Durchführung} % (fold)
\label{sub:durchführung}

In diesem Abschnitt werden die für diesen Evaluierungs-Teil relevanten deskriptiven Parameter der berücksichtigten Anwendungs-Blöcke angeführt.
Als Grundlage der Überprüfung der Hypothesen werden hier die Evaluierungs-Blöcke 1 bis 5 verwendet. Dabei wurden für die quantitativ zu prüfenden Variablen die Blöcke 2 und 3 herangezogen, da in diesen die größten Stichproben zur Verfügung standen. In die qualitative Auswertung der Ergebnisse wurden hingegen alle Blöcke (1-5) mit einbezogen.

\subsubsection{Stichprobe} % (fold)

Für die Untersuchung der Hypothesen in diesem Kapitel wurden die Evaluierungsblöcke 1 bis 5 herangezogen. Die Stichprobe setzt sich wie in Tabelle \ref{tab:stichprobe_tui} beschrieben zusammen.

\begin{table}[htbp]
	\centering
	\caption{Stichproben der Evaluierung zur Werkzeugverwendung}

		\begin{tabular}{| l || c | c |}
		\hline
			Evaluierungsblock & $n_{Anwendungen}$ & $n_{Teilnehmende}$ \\ \hline
			technische Evaluierung		  &  9 & 18 \\
			Aushandlung 1 (1. Durchgang)  &  9 & 19 \\
			Aushandlung 1 (2. Durchgang)  &  9 & 18 \\
			Concept Mapping 1			  & 18 & 54 \\
			Aushandlung 2				  & 10 & 13 \\
			Concept Mapping 2 (Tisch)     & 11 & 24 \\  \hline
			Gesamt						  & 66 & 146 \\ \hline
	\end{tabular}
	\label{tab:stichprobe_tui}
\end{table}

\subsubsection{Dauer der Werkzeugverwendung} % (fold)

Die Dauer der Werkzeug-Verwendung wurde den Blöcken 2 („Aushandlung“) und 3 („Concept Mapping“) erhoben. Die Bearbeitungszeit ist wie in Tabelle \ref{tab:dauer_werkzeugverwendung} dargestellt verteilt (siehe auch Abbildung \ref{fig:img_Evaluierung_usageTimeOverview}\footnote{In allen Boxplots gilt folgende Notation: 
\begin{itemize}
	\item breite horizontale Linie: Bereich zwischen 25\%- und 75\%-Quantil
	\item breite vertikale Linie: Median
	\item linke schmale Linie: Bereich zwischen 2,5\%- und 25\%-Quantil
	\item rechte schmale Linie: Bereich zwischen 75\%- und 97,5\%-Quantil
	\item Kreuze: Ausreißer (außerhalb 2,5\%- und 97,5\%-Quantil)
\end{itemize}
}):

\begin{table}[htbp]
	\centering
	\caption{Dauer der Werkzeugverwendung}
\begin{tabular}{| p{1cm} || p{3cm} | p{3cm} | p{3cm} |}
  \hline
   & Aushandlung (1. Durchgang) & Aushandlung (2. Durchgang) & Concept Mapping \\ \hline
   $t_{min}$ & 11m 54s & 2m 5s & 14m 1s \\ 
   $\overline{t}$ & 20m 53s & 9m 49s & 32m 32s \\ 
   $s(t)$ & 4m 18s & 5m 20s & 10m 7s \\
   $t_{max}$ & 27m 30s & 19m 29s & 45m 0s \\ \hline
\end{tabular} 
	\label{tab:dauer_werkzeugverwendung}
\end{table}

\begin{figure}[htbp]
	\centering
		\includegraphics[width=15cm]{img/Evaluierung/usageTimeOverview.png}
	\caption{Dauer der Werkzeugverwendung -- Überblick}
	\label{fig:img_Evaluierung_usageTimeOverview}
\end{figure}

Die erhobene Dauer der Werkzeug-Verwendung teilt sich ein einen Anteil, an dem tatsächlich mit dem Werkzeug interagiert wird und einen Anteil, der anderen Tätigkeiten (wie inhaltlicher Diskussion, Bedeutungsaushandlung, \ldots) gewidmet ist. Diese beiden Anteile sind in den einzelnen Blöcken wie folgt verteilt (siehe auch die Abbildungen \ref{fig:img_Evaluierung_usageTimeConceptMapping} und \ref{fig:img_Evaluierung_usageTimeNegotiation}):

\begin{figure}[htbp]
	\centering
		\includegraphics[width=15cm]{img/Evaluierung/usageTimeConceptMapping.png}
	\caption{Dauer der Werkzeugverwendung -- Concept Mapping}
	\label{fig:img_Evaluierung_usageTimeConceptMapping}
\end{figure}

\begin{figure}[htbp]
	\centering
		\includegraphics[width=15cm]{img/Evaluierung/usageTimeNegotiation.png}
	\caption{Dauer der Werkzeugverwendung -- Aushandlung}
	\label{fig:img_Evaluierung_usageTimeNegotiation}
\end{figure}

% subsection durchführung (end)
% section untersuchungsdesign (end)

\section{Ergebnisse} % (fold)
\label{sec:ergebnisse}

\subsection{Repräsentation diagrammatischer Modelle} % (fold)
\label{sub:repräsentation_diagrammatischer_modelle}

Gegenstand der hier beschriebenen Untersuchung ist Hypothese \ref{hyp:diagmodelle} („Das Werkzeug ermöglicht die Repräsentation diagrammatische Modelle.“). Als Grundlage dieser Untersuchung dienen die Ergebnisse aller Evaluierungsblöcke, da die Aufgaben in allen Fällen auf die Erstellung einer Repräsentation in Form eines diagrammatischen Modells gefordert war.

Ausgewertet wird hier, ob die Ergebnisse der Modellierung jeweils als diagrammatisches Modell zu klassifizieren sind. Ein diagrammatisches Modell zeichnet nach \citep{Larkin87} aus, dass in ihm Konzepte und deren Zusammenhänge visuell-graphisch dargestellt werden. Eine Darstellung von Beziehungen kann durch die explizite Darstellung von Verbindungen zwischen Konzepten oder durch andere graphische Mittel wie Gruppierung von Konzepten in räumlicher Nähe erfolgen. Um eine eindeutige Auswertbarkeit gewährleisten zu können, wird hier auf die explizite Darstellung von Verbindungen eingeschränkt. 

\subsubsection{Auswertung} % (fold)

In allen vorliegenden Modellen wurden Konzepte als Grundelemente des diagrammatischen Modells verwendet. Das Kriterium zur Klassifizierung als diagrammatisches Modell ist im Folgenden also das Vorhandensein von Verbindungen. Bei der Auswertung ergab sich die in Tabelle \ref{tab:modelle_mit_verbindern} dargestellte Verteilung.

\begin{table}[htbp]
	\centering
	\caption{Anzahl der Modelle mit Verbindern}

\begin{tabular}{| p{3cm} || p{3cm} | p{3cm} |}
  \hline
   Block & Modelle gesamt & Modelle mit Verbindern \\ \hline
   1 & 9 & 0 \\ 
   2 & 18 & 9 \\ 
   3 & 18 & 17 \\ 
   4 & 10 & 10 \\ 
   5 & 11 & 11 \\ \hline
   Gesamt & 66 & 47 \\ \hline
\end{tabular}
	\label{tab:modelle_mit_verbindern}
\end{table}

Insgesamt sind in 66 Modellen, die als Ergebnis vorliegen, 47 Modelle zu identifizieren, in denen explizit Verbindungen zur Darstellung von Beziehungen zwischen Konzepten verwendet werden ($71,2\%$). Eine implizite Darstellung von Beziehungen ist jedoch in allen vorliegenden Modellen zu erkennen. Nicht explizit durch Verbindungen abgebildete Beziehungen werden in allen Fällen durch die räumliche Konfiguration der Konzepte zueinander dargestellt.

\subsubsection{Diskussion} % (fold)

Legt man das Kriterium des Vorhandenseins von Verbindungen zwischen Konzepten an, so sind $71,2\%$ der betrachteten Modelle als diagrammatische Modelle zu klassifizieren. Dies erscheint vordergründig eine geringe Zahl zu sein, die gegen die allgemeine Gültigkeit der Hypothese sprechen würde. Allerdings sind in allen Modelle implizite Verbindungen zwischen Konzepten eindeutig zu identifizieren. Außerdem ist zu erkennen, dass der Anteil an diagramatischen Modellen über die Evaluierungsblöcke (und damit die Weiterentwicklung des Werkzeugs über die Zeit) hinweg stetig ansteigt, bis er in den letzten beiden Blöcken jeweils $100\%$ erreicht. Dies ist durch technische Fehlfunktionen zu erklären, die es in ersten Evaluierungsblöcken schwer bzw. teilweise unmöglich machten, explizite Verbindungen intentional zu erstellen. Unter Anbetracht dieser Erkenntnisse erscheint die Annahme der Hypothese \ref{hyp:diagmodelle} als gerechtfertigt.

Die Abbildung von Verbindungen durch räumliche Konfiguration ist Gegenstand der Prüfung von Hypothese \ref{hyp:keine_verbinder} in Kapitel \ref{cha:eval_modell} und wird dort einer näheren Betrachtung unterzogen.

\subsubsection{Ergebnis} % (fold)

\textbf{Hypothese \ref{hyp:diagmodelle} kann auf Basis der Untersuchung bestätigt werden.} Die Abbildung von Konzepten und Beziehungen zwischen diesen wurde in allen vorliegenden Modellen erfolgreich umgesetzt, wenngleich die Modellierung von expliziten Verbindungen in den ersten beiden Evaluierungsblöcken aufgrund von technischen Unzulänglichkeiten nicht durchgeführt wurde.

% subsection repräsentation_diagrammatischer_modelle (end)

\subsection{Kooperatives Arbeiten} % (fold)
\label{sub:kollaboratives_arbeiten}

Gegenstand der hier beschriebenen Untersuchung ist Hypothese \ref{hyp:kollaborativ} („Das Werkzeug ermöglicht kooperatives Arbeiten an einer Aufgabe.“). Zur Untersuchung der quantitativ beurteilbaren Aspekte wurden die Werkzeuganwendungen aus den Evaluierungsblöcken 2 ($n=9$) und 3 ($n=18$) herangezogen, wobei in Block 2 in Gruppen zu zwei Personen modelliert wurde (in einem Fall drei Personen), in Block 3 in Gruppen zu drei Personen (in drei Fällen nur zwei Personen). Zusätzlich wurden zur qualitative Beurteilung Daten aus Block 4 verwendet.

In Evaluierungsblock 4 wurde hinsichtlich der subjektiven Wahrnehmung der Kooperation eine Befragung der Teilnehmer mittels eines Fragebogens durchgeführt (diese umfasste auch weitere Aspekte, die in späteren Abschnitten besprochen werden). Die Fragestellungen zur Kooperation wurde in 4 geschlossenen Items codiert, die auf einer 7-teiligen Likert-Skala zu beantworten waren. Zusätzlich wurden offene Fragen hinsichtlich der Nützlichkeit der Werkzeugs eingesetzt, die an dieser Stelle ebenfalls hinsichtlich Aussagen zur Kooperation zwischen den Teilnehmern ausgewertet werden.

\subsubsection{Auswertung} % (fold)

Grundlage des ersten Teils der Auswertung ist die Verteilung der Modellierungsdauer zwischen den Teilnehmern. Um die unterschiedliche Gesamt-Modellierungsdauer in den einzelnen Anwendungen zu kompensieren, wurden die Berechnungen auf Basis der prozentuellen Zeitanteile der einzelnen Teilnehmer durchgeführt. Die einzelnen Datensätze wurden so sortiert, dass die anteilsmäßige Modellierungsdauer von Teilnehmer A bis Teilnehmer C (bzw. B) abnimmt. In den einzelnen Evaluierungsblöcken ergeben sich die in Abbildung \ref{fig:img_Evaluierung_timeDist} dargestellten Verteilungen.

\begin{figure}[htbp]
	\centering
		\includegraphics[height=2.5in]{img/Evaluierung/timeDistSE1.png}
		\includegraphics[height=2.5in]{img/Evaluierung/timeDistSE2.png}
		\includegraphics[height=2.5in]{img/Evaluierung/timeDistUE.png}
	\caption{Zeitverteilung zwischen den Teilnehmern}
	\label{fig:img_Evaluierung_timeDist}
\end{figure}

\todo Zu prüfen ist hier, ob die Zeit-Anteile der einzelnen Teilnehmer signifikant unterschiedlich sind. Dazu wird für jeden Block die Signifikanz zwischen den Verteilung der einzelnen Teilnehmerklassen berechnet (eine Teilnehmerklasse setzt sich aus all jenen Teilnehmern zusammen, die am längsten, am zweitlängsten bzw. am drittlängsten aktiv waren).  Aufgrund der geringen Stichprobengröße kommt zur Prüfung der Signifikanz der t-Test nicht in Frage, es wird der \emph{Wilcoxon-Test} herangezogen. Der t-Test setzt außerdem Normalverteilung der Prüfgrößen voraus, was zumindest bei einer der Verteilungen nicht der Fall ist (Sharpiro-Wilk-Test für $conn_{B22}$: $p=6.29e^{-5}$, damit ist von Nicht-Normalverteilung auszugehen).

Im zweiten Teil der Auswertung wurde in einem Fragebogen in 4 Items aggregiert die Frage nach kooperativen Aspekten bei der Modellbildung gestellt. Diese wurden im Schnitt als sehr hoch oder hoch beurteilt ($M = 1.79$, $SD = 0.56$, $t4(13) = -14.28$, $p<.001$). Dieses Ergebnis steht in Übereinstimmung mit den qualitativen Aussagen der Benutzer, von denen 10 explizit auf die kooperationsfördende Wirkung des Werkzeugs hinwiesen. Auch in Auswertungen der  Videoaufnahmen der betreffenden Modellierungsdurchgänge ist zu erkennen, dass zwischen $40$ und $70\%$ der gesamten Modellierungsdauer der Interaktion zwischen den Teilnehmern zuzurechnen ist.

\subsubsection{Diskussion} % (fold)

\subsubsection{Ergebnis} % (fold)


% subsection kollaboratives_arbeiten (end)

\subsection{Einsetzbarkeit in unterschiedlichen Kontexten} % (fold)
\label{sub:einsetzbarkeit_in_unterschiedlichen_kontexten}

Gegenstand der hier beschriebenen Untersuchung Hypothese \ref{hyp:kontexte} („Das Werkzeug ist gleichwertig für Modellierungsaufgaben in unterschiedlichen Kontexten einsetzbar.“). Als Grundlage dieser Untersuchung dienen die Ergebnisse der Evaluierungsblöcke 2 und 4.

\subsubsection{Auswertung} 

\subsubsection{Diskussion} 

\subsubsection{Ergebnis} 

% subsection einsetzbarkeit_in_unterschiedlichen_kontexten (end)

\subsection{Wiederherstellung vergangener Modellzustände} % (fold)
\label{sub:wiederherstellung_vergangener_modellzustände}

Gegenstand der hier beschriebenen Untersuchung ist Hypothese \ref{hyp:wiederherstellung} („Die Möglichkeit der Wiederherstellung vergangener Modellzustände fördert die Bereitschaft alternative Repräsentationen auszuprobieren.“). Als Grundlage dieser Untersuchung dienen die Ergebnisse der Evaluierungsblöcke 2 bis 5, da die Funktion zur Wiederherstellung vergangener Modellzustände erst in diesen Blöcken funktionsfähig zur Verfügung stand.

\subsubsection{Auswertung} 

Für alle Anwendungen des Werkzeugs in den Evaluierungsblöcken 2 bis 5 wurde hier untersucht, wie oft die Möglichkeit zur Wiederherstellung vergangener Modellzustände eingesetzt wurde, um alternative Modellierungswege auszuprobieren. Nicht berücksichtigt wurden Einsätze derselben Funktion, die zur Korrektur von Modellierungsfehlern durch Fehlerkennungen des Systems verwenden wurden (verstärkt in den Evaluierungsblöcken 2 und 3 aufgetreten, in 4 und 5 durch Stabilisierung der Erkennungsleistung nicht mehr relevant). Die Verteilung des Einsatzes der Funkion ist in absoluten Zahlen in Tabelle \ref{tab:anzahl_wiederherstellung} für jeden Evaluierungsblock angeführt

\begin{table}[htbp]
	\centering
	\caption{Anzahl des Einsatzes der Wiederherstellungsfunkion}
\begin{tabular}{| c || c | c | c | c || c | c | c | c |}
  \hline
   EB    & 0 E. & 1 E. & 2 E. & 3+ E. \\ \hline
   2     & 18 & 0 & 0 & 0 \\ 
   3     & 14 & 4 & 0 & 0 \\ 
   4     & 10 & 0 & 0 & 0 \\ 
   5     & 10 & 1 & 0 & 0 \\ \hline
   Ges.  & 52 & 5 & 0 & 0 \\ \hline
\end{tabular} \\
\footnotesize EB \ldots Evaluierungsblock, x E.\ldots x Einsätze der Wiederherstellungsfunktion
	\label{tab:anzahl_wiederherstellung}
\end{table}

Die Wiederherstellungsfunktion wurde also insgesamt in $8.77\%$ der Fälle ($n=57$) eingesetzt und kam maximal einmal je Anwendung zum Einsatz.  Aus den Videoanalysen ist außerdem erkennbar, dass die Wiederherstellungsfunktion -- falls ihre Verwendung überhaupt in Betracht gezogen wird -- in den meisten Fällen lediglich zur Fehlerkorrektur eingesetzt wird. (in 52 Anwendungen wurde die Wiederherstellungsfunkion in 37 Fällen -- $71.2\%$ -- mindestens einmal zur Korrektur von Erkennungsfehlern und 5 mal zur Korrektur von inhaltlich verworfenen Modellierungswegen verwendet).

Bei der in den Blöcken 1, 4 und 5 durchgeführten Befragung der Teilnehmer hinsichtlich der Erfahrungen mit dem Werkzeug wurde unter anderem nach als besonders nützlich empfundenen Funktionen bzw. Eigenschaften des Werkzeugs gefragt. Die Wiederherstellungsfunktion wurde in diesem Zusammenhang von keinem Teilnehmer ($n=55$) erwähnt. 

\subsubsection{Diskussion} 

Die Ergebnisse der Auswertung der Untersuchung zu dieser Hypothese zeigt ein geringes Ausmaß der Verwendung der Wiederherstellungsfunktion zum Zwecke der Erstellung von Modellalternativen. Die Funktion wurde in $71.2\%$ der Anwendungen verwendet, was für ein hohes Bewusstsein über deren Existenz spricht. Lediglich in $8.77\%$ der Anwendungen wurde die Funktion zur Verfolgung alternativer Modellierungswege eingesetzt, in $61.5\%$ der Anwendungen wurde sie lediglich zur Fehlerkorrektur verwendet. Auch in der qualitativen Erhebung der als nützlich wahrgenommenen Werkzeugfunktionalitäten wurde die Wiederherstellungsfunktion in keinem Fall genannt. Auf Basis dieser Ergebnisse kann die Hypothese nicht bestätigt werden. 

\subsubsection{Ergebnis} 

\textbf{Hypothese \ref{hyp:wiederherstellung} kann auf Basis der Untersuchung nicht bestätigt werden.} Die Wiederherstellungsfunktion wird nur in unter $10\%$ der untersuchten Anwendungen  zur Verfolgung alternativer Modellierungswege genutzt. Die Funktion wird außerdem von den Anwendern bei der Frage nach den als nützlich wahrgenommene Funktionen nicht genannt.

% subsection wiederherstellung_vergangener_modellzustände (end)

\subsection{Nicht-Behinderung} % (fold)
\label{sub:nicht_behinderung}

Gegenstand der hier beschriebenen Untersuchung ist Hypothese \ref{hyp:behinderung} („Das Werkzeug behindert die Modellbildung nicht.“). Als Grundlage dieser Untersuchung dienen die Ergebnisse der Evaluierungsblöcke 2 bis 5, da sich das Werkzeug erst in diesen Blöcken hinsichtlich der Funktionalität in vollständigem Zustand befand. Zu berücksichtigen ist bei der Auswertung, dass im Laufe der Evaluierungsblöcken 4 und 5 eine Überarbeitung der Implementierung vorgenommen wurde, mittels der das Auftreten von Fehlerkennungen verringert werden konnte und deren Korrektur weniger aufwändig wurde. Befragungen der Modellierenden hinsichtlich einer etwaigen Behinderung durch das Werkzeug wurden in den Blöcken 1, 4 und 5 durchgeführt, wobei lediglich die Anmerkungen aus den letzen beiden Blöcken für den aktuellen Entwicklungsstand des Werkzeugs relevant sind.

\subsubsection{Auswertung} 

In Tabelle \ref{tab:fehlfunktionen} wird gegliedert nach Evaluierungsblocken dargestellt, wie oft es in einer einzelnen Anwendung zu Fehlfunktionen in der Erkennung kam, die den Modellierungsfluss unterbrachen. Als Fehlerkennungen wurde das Verschwinden von Blöcken oder Fehlzuordnungen von Benennungen sowie die unbeabsichtigte oder von System eigenständig vorgenommene Erstellung oder Entfernung von Verbindern bzw. Richtungspfeilen eingeordnet. Zusätzlich wurden Systemabstürze als massive Unterbrechung, die zum Gesamtverlust des bis zum Zeitpunkt des Absturzes erstellten Modells führten, separat ausgewertet.

\begin{table}[htbp]
	\centering
	\caption{Fehlfunktionen und Abstürze des Werkzeugs}
\begin{tabular}{| c || c || c | c | c | c || c |}
  \hline
   EB    & Anw. & 0 Ff. & 1-3 Ff. & 4-6 Ff. & 7+ Ff. & Systemabstürze \\ \hline
   2     & 18 & 0 &  8 &  5 &  5 &  4 \\ 
   3     & 18 & 1 & 10 &  4 &  3 &  5 \\ 
   4     & 10 & 0 &  2 &  2 &  5 &  1 \\ 
   5     & 11 & 0 &  3 &  3 &  4 &  5 \\ \hline
   Ges.  & 57 & 1 & 23 & 14 & 17 & 15 \\ \hline
\end{tabular} \\
\footnotesize EB \ldots Evaluierungsblock, Anw. \ldots Anzahl der Anwendungen, x Ff.\ldots x Fehlfunktionen
	\label{tab:fehlfunktionen}
\end{table}

In der Gesamtheit der betrachteten Anwendungen ($n=57$) ergibt sich folgende Verteilung der Anzahl der Fehlerkennungen je Anwendung, die auch in Abbildung \ref{fig:img_Evaluierung_fehlerkennungen} graphisch dargestellt ist. In $1.75\%$ der Fälle ($n_{0}=1$) trat keine Fehlerkennung während der Anwendung auf. In $40.35\%$ der Fälle ($n_{1-3}=23$) traten zwischen 1 und 3 Fehlerkennungen auf. 4-6 Fehlerkennungen konnten in $24.56\%$ der Fälle ($n_{4-6}=14$) festgestellt werden. 7 oder mehr Fehlerkennungen traten in $29.82\%$ der Fälle ($n_{7+}=17$) auf. In $26.32\%$ der Fälle ($n_{Absturz}=15$) kam es zu Systemabstürzen, wobei diese in 10 Fällen nach Ende des eigentlichen Modellierungsvorgangs auftraten.

\begin{figure}[htbp]
	\centering
		\includegraphics[width=10cm]{img/Evaluierung/fehlerkennungen.png}
	\caption{Verteilung der Anzahl der Fehlerkennungen je Anwendung -- Übersicht}
	\label{fig:img_Evaluierung_fehlerkennungen}
\end{figure}

\todo qualitative Daten

\subsubsection{Diskussion} 

\todo Die Daten der quantitativen Auswertung zeigen, dass es in annähernd allen betrachteten Fällen zu zumindest einer Fehlerkennung kam. Es ist davon auszugehen, dass jede Fehlerkennung den Modellierungsfluss unterbricht, da das dann inkorrekte Modelle korrigiert werden muss. Insofern 

\todo Bestätigung und Relativierung durch die Überarbeitung der Interaktion durch die qualitativen Ergebnisse.

Der hohe Anteil von Systemabstürzen ist insofern zu relativieren, als dass diese in zwei Drittel der Fälle nach Abschluss der eigentlichen Modellierungstätigkeit auftraten und somit die Modellerstellung selbst nicht mehr unterbrachen. Abstürze traten durchgängig vor allem in langen Modellierungsdurchgängen etwa ab Minute 40 auf, da ab diesem Zeitpunkt der Speicherbedarf der Historie tendenziell an die Grenzen des verfügbaren Arbeitsspeichers stößt. Alternativ kam es an Tagen mit starker Modellierungstätigkeit ab etwa 5 Stunden durchgängiger Betriebsdauer zu Überhitzungen des Rechners, auf dem die Software ausgeführt wurde, was zum Gesamtabsturz des Betriebssystems führte. Lediglich in 5 Fällen war der Absturz auf fehlerhaftes Programmverhalten (abgesehen von der Speicherproblematik) zurückzuführen. Diese Fälle traten in den Evaluierungsblöcken 2 und 3 auf. Trotzdem sind auch Systemabstürze in der Endphase der Anwendung nach der Modellierung durch den auftretenden Datenverlust nicht akzeptabel und sprechen somit gegen die Annahme der Hypothese.

Insgesamt kann die hier geprüfte Hypothese aus den angeführten Gründen nicht bestätigt werden.

\subsubsection{Ergebnis} 

\textbf{Hypothese \ref{hyp:behinderung} kann auf Basis der Untersuchung nicht bestätigt werden.}


% subsection nicht_behinderung (end)

\subsection{Gewöhnung an das Werkzeug} % (fold)
\label{sub:gewöhnung_an_das_werkzeug}

Gegenstand der hier beschriebenen Untersuchung ist Hypothese \ref{hyp:gewöhnung} („Wiederholte Verwendung des Werkzeugs führt zu schnellerer Modellbildung und weniger Fehlbedienungen.“). Als Grundlage dieser Untersuchung dienen die Ergebnisse des Evaluierungsblocks 2, da in diesem für jede Teilnehmerzusammenstellung jeweils zwei Anwendungen des Werkzeugs durchgeführt wurden.

\subsubsection{Auswertung} 

Zur Auswertung der Modellierungsgeschwindigkeit (hinsichlich des Hypothesenteils „schnellere Modellbildung“) wurde die reine Modellierungszeit jeder Anwendung (ohne Diskussionszeit) mit der jeweiligen Modellgröße normiert. In Tabelle 	\ref{tab:normierte_zeiten} sind die Anwendungszeiten und Modellgrößen sowie die daraus errechneten normierten Werte für beide Anwendungen der Gruppen in Evaluierungsblock 2 angegeben. 

\begin{table}[htbp]
	\centering
	\caption{Modellierungszeiten in Abhängigkeit der Modellgröße in Evaluierungsblock 2}
\begin{tabular}{| c || c | c | c || c | c | c |}
  \hline
   Gruppe    & $t_{1}$ & $n_{1}$ & $t'_{1}$ & $t_{2}$ & $n_{2}$ & $t'_{2}$ \\ \hline
   1     & 620 & 20 & 31.0 & 300 &  6 & 50.0 \\ 
   2     & 450 & 13 & 34.6 & 420 &  7 & 60.0 \\ 
   3     & 240 & 12 & 20.0 & 285 &  6 & 47.5 \\ 
   4     & 215 & 10 & 21.5 & 420 & 13 & 32.3 \\ 
   5     & 577 & 12 & 48.1 & 270 &  7 & 38.6 \\ 
   6     & 339 & 10 & 33.9 & 330 &  8 & 41.3 \\ 
   7     & 348 &  8 & 43.5 & 110 &  5 & 22.0 \\ 
   8     & 855 & 16 & 53.4 & 510 & 12 & 42.5 \\ 
   9     & 735 &  8 & 91.9 & 195 & 12 & 16.3 \\ \hline
\end{tabular} \\
\footnotesize $t_{x}$ \ldots Modellierungsdauer in Sekunden, $n_{x}$ \ldots Anzahl der Elemente, $t'_{1}$ \ldots normierte Modellierungdauer in Sekunden
	\label{tab:normierte_zeiten}
\end{table}

Zusammenfassend ist zwischen der ersten Anwendung (normierte Modellierungsdauer: $M=42.0, SD=21.8, n=9$) und der zweiten Anwendung (normierte Modellierungsdauer: $M=38.9, SD=13.7, n=9$) keine signifikante Verringerung der normierten Modellierungsdauer zu erkennen (einseitiger Wilcoxon-Test für gepaarte Stichproben: $V=21, p=0.590$\footnote{Aufgrund der beiden kleinen Stichproben und der Nicht-Normalverteilung der beiden Stichproben (Shapiro-Wilk-Test 1. Anwendungsdurchgang: $W=0.853, p=0.081$, 2. Anwendungsdurchgang: $W=0.972, p=0.910$) kann der t-Test ($t=0.286, df=8, p=0.391$) trotz der gleichen Varianz der Stichproben (F-Test: $F=2.53, p=0.211$) nicht angewandt werden.}).

Die Anzahl der Fehlbedienungen ist die Anwendungen in beiden Modellierungsdurchgängen in Evalierungsblock 2 in Tabelle \ref{tab:fehlbedienungen} angegeben. Als Fehlbedienungen wurden all jene Interaktionen mit dem Werkzeug eingestuft, in denen die Bedienung nicht dem intendierten Interaktionsdesign folgte. Fehlfunktionen des Werkzeugs wurden nicht berücksichtigt.

\begin{table}[htbp]
	\centering
	\caption{Anzahl der Fehlbedienungen in Evaluierungsblock 2}
\begin{tabular}{| c || c | c |}
  \hline
   Gruppe    & $FB_{1}$ & $FB_{2}$ \\ \hline
   1     & 1 & 0 \\ 
   2     & 4 & 1 \\ 
   3     & 2 & 1 \\ 
   4     & 0 & 0 \\ 
   5     & 0 & 0 \\ 
   6     & 6 & 1 \\ 
   7     & 3 & 1 \\ 
   8     & 6 & 1 \\ 
   9     & 4 & 2 \\ \hline
\end{tabular} \\
\footnotesize $FB_{x}$ \ldots Anzahl der Fehlbedienungen
	\label{tab:fehlbedienungen}
\end{table}

Zusammenfassend konnte hier gezeigt werden, dass die Anzahl der Fehlbedienungen zwischen Anwendung 1 ($M=2.89, SD=2.32, n=9$) und Anwendung 2 ($M=0.78, SD=0.67, n=9$) signifikant geringer geworden ist (einseitiger Wilcoxon-Test für gepaarte Stichproben: $V=28, p=0.0109$\footnote{Aufgrund der beiden kleinen Stichproben und der Nicht-Normalverteilung der ersten Stichprobe (Shapiro-Wilk-Test 1. Anwendungsdurchgang: $W=0.9144, p=0.348$, 2. Anwendungsdurchgang: $W=0.813, p=0.0284$)   sowie der unterschiedlichen Varianz der Stichproben (F-Test: $F=12.06, p=0.00199$) kann der t-Test ($t=3.333, df=8, p=0.00517$) nicht angewandt werden.}).

\subsubsection{Diskussion} 

Eine signifikante Beschleunigung der Modellierungsgeschwindigkeit konnte in obiger Untersuchung nicht festgestellt werden. Die mit der Modellgröße normierte Modellierungszeit verringerte sich zwischen den beiden Anwendungen im Schnitt nur geringfügig. Dieses Ergebnis kann somit nicht als Indikator für die Bestätigung der Hypothese gesehen werden. In den anderen Evaluierungsblöcken (3, 4 und 5) liegt die durchschnittliche normierte Modellierungsdauer in ähnlichen Bereichen wie in den beiden Durchgängen von Evaluierungsblock 2. Bei Anwendung des Werkzeugs durch den Entwickler selbst ist die normierte Modellierungsdauer hingegen auf ungefähr den halben Wert reduziert. Benutzer ohne tiefgehende und mehrfach wiederholte Anwendungserfahrungen scheinen also keinen signifikant messbaren Beschleunigungseffekt bei der Bedienung des Werkzeugs zu erfahren.

Hingegen ist die Anzahl der Fehlbedienungen in den jeweils zweiten Anwendungen des Werkzeugs im Vergleich zur jeweils ersten Anwendung signifikant gesunken. Dies spricht für die Bestätigung der hier geprüften Hypothese. Betrachtet man die Fehlbedienungen detaillierter, so ist ein Großteil der aufgetretenen Fälle sowohl in der ersten als auch in der zweiten Anwendung auf Verständnisschwierigkeiten bei der Bedienung des Löschtokens (zum Zeitpunkt der Evaluierung noch mit dem zustandsbehafteten Interaktionsdesign implementiert, siehe Abschnitt \ref{sub:verwendung_des_löschtokens}) und der Verwendung der Wiederherstellungsfunktion zurückzuführen. Das Interaktionsdesign beider Aspekte wäre also zu hinterfragen (bzw. wurde im Falle des Löschtokens hinterfragt).

Insgesamt kann die hier untersuchte Hypothese nur zum Teil bestätigt werden, da der vermutete Beschleunigungseffekt nicht nachzuweisen war.

\subsubsection{Ergebnis} 

\textbf{Hypothese \ref{hyp:gewöhnung} kann auf Basis der vorliegenden Daten teilweise bestätigt werden.} Während kein signifikanter Beschleunigungseffekt bei wiederholter Verwendung des Werkzeugs gemessen werden konnte, war eine signifikante Verringerung der Anzahl der Fehlbedienungen des Werkzeugs bei wiederholtem Einsatz feststellbar.

% subsection gewöhnung_an_das_werkzeug (end)

\subsection{Herstellung von Verbindern} % (fold)
\label{sub:herstellung_von_verbindern}

Gegenstand der hier beschriebenen Untersuchung ist Hypothese \ref{hyp:verbinder} („Die Einführung der alternativen Möglichkeit zur Verbindungsherstellung erhöht die Nutzung von Verbindern bei der Modellerstellung.“). Zur Untersuchung herangezogen wurden die Werkzeuganwendungen aus Evaluierungsblock 2 ($n=9$). Dieser wurde gewählt, da in diesem Block alle Teilnehmer das Werkzeug zweimal mit der gleichen Aufgabenstellung anwandten, wobei in der ersten Anwendungsrunde lediglich die ursprüngliche Funktionalität zur Herstellung von Verbindern verfügbar war, in der zweiten Runde aber bereits der alternative Funktionalität implementiert war. Zur weiteren Überprüfung der Ergebnisse werden außerdem die Ergebnisse aus Block 3 ($n=17$) herangezogen, bei dessen Durchführung ebenfalls bereits die alternative Funktionalität verfügbar war.

\subsubsection{Auswertung} % (fold)

Grundlage der Auswertung ist das Modellmerkmal „Connectedness“, worunter hier das Verhältnis zwischen der Anzahl der in einem Modell verwendeten Verbindern und den verwendeten Modellelementen verstanden wird. In den einzelnen Evaluierungsblöcken verteilt sich die Connectedness wie in den Abbildungen \ref{fig:img_Evaluierung_connectednessAushandlung1}, \ref{fig:img_Evaluierung_connectednessAushandlung2} und \ref{fig:img_Evaluierung_connectednessConceptMapping} dargestellt.

\begin{figure}[htbp]
	\centering
		\includegraphics[height=2in]{img/Evaluierung/connectednessAushandlung1.png}
	\caption{Connectedness in Evaluierungsblock 2 - Durchgang 1}
	\label{fig:img_Evaluierung_connectednessAushandlung1}
\end{figure}

\begin{figure}[htbp]
	\centering
		\includegraphics[height=2in]{img/Evaluierung/connectednessAushandlung2.png}
	\caption{Connectedness in Evaluierungsblock 2 - Durchgang 2}
	\label{fig:img_Evaluierung_connectednessAushandlung2}
\end{figure}

\begin{figure}[htbp]
	\centering
		\includegraphics[height=2in]{img/Evaluierung/connectednessConceptMapping.png}
	\caption{Connctedness in Evaluierungsblock 3}
	\label{fig:img_Evaluierung_connectednessConceptMapping}
\end{figure}

Zu prüfen ist, ob die Connectedness in jenem Evaluierungs-Blöcken bzw. -Durchgängen, in denen die alternative Funktionalität zur Verbindungs-Herstellung verfügbar war, signifikant höher ist, als in jenen, in denen dies nicht der Fall war. Berechnet wird die Signifikanz zwischen den Ergebnissen der beiden Durchgänge von Block 2 ($conn_{B21}$ und $conn_{B22}$) sowie zwischen den Ergebnissen ersten Durchgang von Block 2 und den Ergebnissen von Block 3 ($conn_{B3}$). Im zweiten Fall ist zu beachten, dass die Aufgabenstellung nicht identisch war und somit eine potentielle Störvariable wirksam wird. Aufgrund der geringen Stichprobengröße kommt zur Prüfung der Signifikanz der t-Test nicht in Frage, es wird der \emph{Wilcoxon-Test} herangezogen. Der t-Test setzt außerdem Normalverteilung der Prüfgrößen voraus, was zumindest bei einer der Verteilungen nicht der Fall ist (Sharpiro-Wilk-Test für $conn_{B22}$: $p=6.29e^{-5}$, damit ist von Nicht-Normalverteilung auszugehen).

Die Null-Hypothese des Wilcoxon-Tests ist, das die beiden Verteilungen identisch verteilt sind. Entsprechend dem erwarteten Ergebnis (dass die mit der alternativen Funktionalität durchgeführten Anwendungen höhere Connectedness aufweisen) wurde die Alternativ-Hypothese so festgelegt, dass sie angenommen wird, wenn die Verteilung des zweiten Blocks gegenüber dem ersten Block nach rechts verschoben (also wertemäßig höher) ist.

Der Wilcoxon-Test für ungepaarte Stichproben ergibt für $conn_{B21}$ und $conn_{B22}$ und der eben beschriebenen Alternativ-Hypothese $p=0.9854$ -- die Alternativ-Hypothese ist damit anzunehmen, die zweite Verteilung (jene mit Einsatz der alternativen Funktionalität der Verbindungsherstellung) weist eine signifikant höhere Connectedness auf als die erste Verteilung (ohne diese Funktionalität). 

Für $conn_{B21}$ und $conn_{B3}$ ergibt der Wilcoxon-Test für ungepaarte Stichproben mit der gleichen Alternativ-Hypothese $p=0.98$ -- auch hier ist die Alternativ-Hypothese anzunehmen.

Für $conn_{B22}$ und $conn_{B3}$ ergibt der Wilcoxon-Test für ungepaarte Stichproben mit der gleichen Alternativ-Hypothese $p=0.7586$ -- auch hier ist die Alternativ-Hypothese anzunehmen.

\subsubsection{Diskussion} % (fold)

Aufgrund der Ergebnisse der berechneten Signifikanztests ist die Hypothese anzunehmen. Mit der Einführung der alternativen Möglichkeit zur Herstellung von Verbindungen war in den einzelnen Anwendungen des Werkzeugs eine Zunahme der Verwendung von Verbindern zu beobachten. Während die Benutzer bei der ursprünglichen Funktion zur Herstellung von Verbindungen zum Großteil auf diese verzichteten (auch bereits in Evaluierungsblock 1), wurden Verbinder unabhängig von der Aufgabenstellung mit der Einführung der alternativen Funktionalität verstärkt eingesetzt.

Die Connectedness eignet sich als Parameter zur vergleichenden Beurteilung des Ausmaßes der Verwendung von Verbindern, da durch die Einbeziehung der Größe des Modells (repräsentiert durch die Anzahl der verwendeten Modellelemente) in die Berechnung den Wert für unterschiedliche Modelle vergleichbar macht. 

Einfluss auf die Höhe der Connectedness hat aber die Aufgabenstellung, die zur Bildung des Modells führt. Unterschiedliche Modellierungsaufgaben führen zu unterschiedlichen Modell-Topologien, die sich wiederum in der Anzahl der verwendeten Verbinder auswirkt. Dies zeigt sich am Ergebnis des Wilcoxon-Tests für $conn_{B22}$ und $conn_{B3}$ -- in beiden Fällen stand die alternative Möglichkeit zur Verbindungsherstellung zur Verfügung $conn_{B3}$ ist trotzdem signifikant höher als $conn_{B22}$. Die kann darin begründet liegen, dass die Concept-Mapping-Aufgabe aus $conn_{B3}$ eher zu stärker verbundenen Modellen führt als eher zur ablauforientierten Modellen führende Arbeitsabstimmungs-Aufgabe aus $conn_{B22}$. Während bei Concept Mapping beliebige Konzepte in Beziehung stehen können, stehen Elemente bei ablauf-orientierten Modellen vor allem mit ihren kausalen Vorgängern und Nachfolgern in Beziehung, was die Anzahl der Verbinder einschränkt.

Aufgrund der großen Rolle der Aufgabenstellung ist bei der Überprüfung der Hypothese wichtig, diese Störvariable möglichst auszuschalten. Zur Beurteilung wird deswegen ausschließlich der Wilcoxon-Test zwischen $conn_{B21}$ und $conn_{B22}$ herangezogen, da in diese beiden Verteilungen mit der gleichen Aufgabenstellung und identischer Stichprobe (jedoch in zeitlichem Abstand von ca. einem Monat) zustande gekommen sind (da die Messungen unabhängig voneinander entstanden, wird ein Wilcoxon-Test für ungepaarte Variablen verwendet). Das Resultat des Wilcoxon-Tests spricht stark für die Annahme der Alternativhypothese des Tests und damit für die Annahme von Hypothese \ref{hyp:verbinder}. Zu berücksichtigen ist hier jedoch die geringe Stichprobengröße, die die Aussagekraft des Ergebnisses wieder in Frage stellt.

\subsubsection{Ergebnis} % (fold)

Die Auswertung zeigt eine signifikant höhere Verwendung von Verbindern bei Verfügbarkeit der alternativen Funktionalität zur Verbindungs-Herstellung. Auch die Natur der Aufgabenstellung scheint hohen Einfluss auf die Verwendung von Verbindern zu haben (siehe dazu auch die Diskussion von Hypothese \ref{hyp:keine_verbinder} in Abschnitt \ref{sub:abbildung_von_zusammenhängen_ohne_verbinder}). \textbf{Hypothese \ref{hyp:verbinder} kann auf Basis der vorliegenden Daten bestätigt werden.}

% subsection herstellung_von_verbindern (end)

\subsection{Verwendung des Löschtokens} % (fold)
\label{sub:verwendung_des_löschtokens}

In diesem Abschnitt werden die Ergebnisse der Überprüfung der Hypothese \ref{hyp:radierer} („Das Löschtoken ermöglicht intuitives Löschen von Modellelementen.“) vorgestellt.

\subsubsection{Auswertung} % (fold)

\begin{transkript}
	\emph{Die Teilnehmer möchten einen Block umbenennen.}\\
	\textbf{A:} Wie haben wir jetzt gesagt \emph{(markiert den roten Baustein)} keine Modellierungsvorgabe \emph{(gibt Bezeichnung ein)}\\
	\emph{System übernimmt die neue Beschriftung für den Baustein nicht.}\\
	\textbf{A:} Wo wurde das hingeschrieben? \emph{(Pause)} Radiergummi? Glaubst du kann man das wegradieren?\\
	\textbf{B:} Probiere es aus.\\
	\textbf{\emph{A legt Radiergummi zum Block mit der Absicht die Beschriftung zu löschen}}\\
	\textbf{B:} Nein! Du löscht alles. Hör auf! \\
	\textbf{A:} Ok wie war das zuerst? Lassen wir das mal weg. \emph{(legt Baustein zur Seite)}\\
	\emph{A legt den Block zur Seite.} 
\end{transkript}

Ein ähnliches Missverständnis zeigt sich auch in folgender Situation:

\begin{transkript}
	\emph{TLN A und B stellen jeweils ihren Marker zu den Blöcken, die verbunden werden sollen. Dabei wird eine gerichtete Verbindung erstellt.}\\
	\textbf{C:} Jetzt haben wir aber einen Pfeil gebastelt.\\
	\textbf{B:} Ja stimmt. Interessant.\\
	\textbf{A:} Wie war das mit dem Radiergummi. \emph{(nimmt Radiergummi und legt ihn auf die Verbindung)}\\
	\textbf{B:} Nein\\
	\textbf{C:} Nein, mit dem Glas! Du löscht alles!\\
	\textbf{A:} Nein nur die Verbindung. \textbf{\emph{(Macht Radierbewegungen auf der Verbindung)}}\\
	\textbf{C:} Ich glaube dass wir das Glas nehmen müssen.\\
	\emph{A schiebt die Blöcke zwischen denen die Verbindung gelöscht werden soll zusammen.}\\
	\textbf{A:} Da es funktioniert. \emph{(schiebt die Blöcke weiter auseinander und bemerkt dass die Verbindung nicht gelöscht wurde)} Nein.\\
	\textbf{B:} Ich glaube der Radiergummi vernichtet alles.\\
	\textbf{A:} Nein der Radiergummi vernichtet nur Verbindungen. Nur welche? \emph{(schiebt beide Blöcke wieder zusammen – nimmt Radiergummi weg und schiebt Blöcke in die Ausgangsposition)}
\end{transkript}

\begin{transkript}
	\emph{Es wird eine falsche Beschriftung eingefügt. Die Teilnehmer wollen diese löschen, verwenden den Radiergummi allerdings falsch.}\\
	\textbf{B:} Aber irgendwie steht jetzt Ereignisse nicht bei dem Ding \emph{(zeigt auf gelben Block)} sondern dort \emph{(zeigt auf beschriftete Verbindung)}.\\
	\emph{A verrückt den gelben Block ein wenig.}\\
	\textbf{B:} Normal ist das nicht oder?\\
	\textbf{C:} Nein.\\
	\emph{A nimmt den Radiergummi.}\\
	\textbf{A:} Ich glaube das. \emph{(setzt den Radiergummi auf die Arbeitsfläche)}\\
	\textbf{C:} Aber nicht alles!\\
	\emph{A nimmt Radiergummi wieder weg. System erstellt eine Verbindung zwischen zwei roten Blöcken. Teilnehmer lachen. \textbf{A legt Radiergummi auf die erstellte Verbindung, und nimmt ihn wieder weg.} A nimmt die beiden verbundenen Blöcke und verschiebt sie.}\\
	\textbf{A:} Vielleicht so. \emph{(führt die Blöcke zusammen)}
\end{transkript}

\begin{transkript}
	\emph{In der Szene erstellt das System einen ungewollten Verbinder, die Teilnehmer versuchen auf verschiedene Arten den Verbinder zu löschen.}\\
	\textbf{B:} Und wie kann ich die Verbindungen löschen?\\
	\textbf{B:} Warte einmal, da gibt es irgendwo das mit dem Radiergummi.\\
	\textbf{A:} murmelt zustimmend \\
	\emph{\textbf{B nimmt den Radiergummi und platziert ihn direkt auf dem Verbinder}}\\
	\emph{Das System färbt den Tisch rot}\\
	\textbf{A:} Nein, warte. Da löscht du Alles!\\
	\emph{\textbf{B verschiebt den Radiergummi auf dem Tisch, hebt ihn an und platziert ihn direkt auf einem Block.}}\\
	\emph{Sobald der Radiergummi von der Oberfläche auf den Block gelegt wurde, entfernt das System die rote Färbung.}\\
	\textbf{A:} Ich glaube da löscht du Alles.\\
	\emph{B legt den Radiergummi an mehreren Stellen trotz der Warnung von TN A auf die Oberfläche}\\
	\textbf{B:} Nein, es will eh nicht.\\
\end{transkript}

\begin{transkript}
	\emph{C versucht die Benennung eines Verbinders mittels Radiergummi zu entfernen.}\\
	\textbf{B:} Aber irgendwie steht jetzt Ereignisse nicht bei dem Ding \emph{(deutet auf einen Block)} sondern dort \emph{(deutet auf einen Verbinder)}. Das wollen wir nicht oder?\\
	\textbf{A:} Nein.\\
	\textbf{C:} Ich glaube das. \emph{\textbf{(nimmt den Radiergummi und legt ihn auf den Verbinder den die Teilnehmer entfernen wollen.)}}\\
	\textbf{A:} Aber nicht alles.\\
	\emph{C entfernt den Radiergummi wieder von der Modellierungsoberfläche. In diesem Moment erstellt das System automatisch einen neuen Verbinder. C versucht den neuen Verbinder mittels Radiergummi zu entfernen.}\\
	\textbf{A:} Oh Gott.\\
	\textbf{C:} Vielleicht so \emph{(schiebt die beiden betroffenen Blöcke zusammen)}, nein.\\
	\textbf{B:} Nein.\\
	\textbf{A:} Oh Gott oh Gott oh Gott.\\
	\textbf{B:} Gehen wir einen Prozessschritt zurück.\\
	\textbf{C:} Genau.\\
\end{transkript}

\begin{transkript}
	\emph{Teilnehmer versuchen mit dem Radiergummi und nur einem anderen Marker einen Verbinder zu entfernen.}\\
	\textbf{B:} Können wir die nicht so auch einfach löschen?\\
	\textbf{C:} Ja mit dem Radiergummi.\\
	\textbf{B:} Muss ich den jetzt zuerst so \emph{(Hält den Radiergummi zur Kamera)} hinhalten?\\
	\textbf{A:} Nein, ich glaube, \textbf{den musst du einfach da \emph{(zeigt auf den Verbinder)} drauf legen.}\\
	\emph{B legt den Radiergummi auf den vom System automatisch erstellten Verbinder.}\\
	\textbf{A:} Und jetzt muss man \emph{(legt ein Markierungtoken auf den Verbinder)} Nein.\\
	\emph{Der Verbinder lässt sich auf diese Art nicht löschen und die Teilnehmer entscheiden sich den Fehler mittels der Wiederherstellungsfunktion zu beseitigen.}
\end{transkript}

\subsubsection{Diskussion} % (fold)

\subsubsection{Ergebnis} % (fold)

% subsection verwendung_des_löschtokens (end)
% section ergebnisse (end)

% chapter eval_tui (end) 
\chapter{Evaluierung der erstellten Modelle} % (fold)
\label{cha:eval_modell}

% chapter eval_modell (end)
\chapter{Evaluierung der durchgeführten Articulation Work} % (fold)
\label{cha:eval_aw}

\section{Hypothesen} % (fold)
\label{sec:a_hypothesen}

\subsection{Konzeptuell begründete Hypothesen} % (fold)
\label{sub:a_konzeptuell_begründete_hypothesen}

\begin{hyp}
	Das Werkzeug verbessert den Prozess der Abstimmung zwischen Personen.
\end{hyp}

\begin{hyp}
	Die Anwendung des Werkzeugs verbessert die Ergebnisse kollaborativer Arbeit.
\end{hyp}

% subsection konzeptuell_begründete_hypothesen (end)

\subsection{Explorativ gebildete Hypothesen} % (fold)
\label{sub:a_explorativ_gebildete_hypothesen}

% subsection explorativ_gebildete_hypothesen (end)

% section hypothesen (end)

\section{Untersuchungsdesign und Durchführung} % (fold)
\label{sec:a_untersuchungsdesign}

% section untersuchungsdesign (end)

\section{Ergebnisse} % (fold)
\label{sec:a_ergebnisse}

% section ergebnisse (end)

% chapter eval_aw (end)

%\input{Untersuchungsdesign}
%\chapter{Untersuchungsergebnisse} % (fold)
\label{cha:untersuchungsergebnisse}

\section{Erhobene Daten} % (fold)
\label{sec:erhobene_daten}

\subsection{Phase 1} % (fold)
\label{sub:phase_1}

In Phase 1 wurden 9 Modellierungsdurchgänge mit insgesamt 18 Personen durchgeführt. An dem vorangegangenen Pretest nahmen 12 Personen teil.
% subsection phase_1 (end)

\subsection{Phase 2} % (fold)
\label{sub:phase_2}

In Phase 2 wurden Untersuchungen im Rahmen zweier Lehrveranstaltungen durchgeführt. An der ersten Untersuchung nahmen 18 Studierende der Wirtschaftsinformatik teil, die in Gruppen zu 2 Personen insgesamt 17 Modellierungsdurchgänge durchführten. An der zweiten Untersuchung nahmen 54 Studierende in Gruppen zu 3 Personen an insgesamt 18 Modellierungsdurchgängen teil.
% subsection phase_2 (end)

\subsection{Phase 3} % (fold)
\label{sub:phase_3}

% subsection phase_3 (end)
% section erhobene_daten (end)

\section{Auswertung \& Interpretation} % (fold)
\label{sec:auswertung_&_interpretation}

% section auswertung_&_interpretation (end)
% chapter untersuchungsergebnisse (end)


% part evaluierung (end)
