\chapter{Externalisierung mentaler Modelle} % (fold)
\label{cha:externalisierung}



\section{Concept Mapping} % (fold)
\label{sec:concept_mapping}

% section concept_mapping (end)

\section{Strukturlegetechniken} % (fold)
\label{sec:strukturlegetechniken}

Strukturlegetechniken sind Ansätze, in denen gelegte Strukturen zur Repräsentation von "Wissen" eingesetzt werden. Die gelegten Strukturen (die im Wesentlichen aus Knoten und Kanten unterschiedlicher Bedeutung bestehen) bilden dabei die Zusammenhänge einzelner Konstrukte ab, wie sie die legende Person wahrnimmt. Der Prozess des Legens ist eine \emph{"Rekonstruktion subjektiver Theorien"} \citep{Dann92} und stellt eine \emph{"[\ldots] verstehende Beschreibung von Handlungen nicht aus der Perspektive eines außenstehenden Beobachters, sondern aus Sicht der handelnden Person, des Akteurs selber"} \citep{Dann92} dar. 

% section strukturlegetechniken (end)

\section{Herausforderungen bei der Anwendung}
\label{sec:herausforderungen_bei_der_anwendung}

\subsection{Kollaborative Anwendbarkeit}

pro Strukturlegetechniken 

\subsection{Nachhaltige Verwendung der Information}

pro Concept Mapping

\subsection{Zusammenführung}

Offenheit des Conceptmapping

Strukturlegetechniken mit IT-Unterstützung

Nicht SLT in den Computer sondern Computer zur Unterstützung von SLT
%chapter externalisierung (end)