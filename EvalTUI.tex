\chapter{Evaluierung der Verwendbarkeit des Werkzeugs} % (fold)
\label{cha:eval_tui}

Im ersten empirischen Teil der Evaluierung wurde die grundlegende Verständlichkeit und Verwendbarkeit des Werkzeug geprüft. Ziel war es hier, konzeptuelle und technische Eigenschaften bzw. Verhaltensweisen des Werkzeugs zu identifizieren, die den Modellierungsprozess behindern oder unterbrechen. Darunter fällt grundsätzlich jede Eigenschaft und jede Verhaltensweise, die die Benutzer zwingt, sich mit dem technischen System an sich zu beschäftigen und von der Erfüllung der eigentlichen Aufgabe ablenkt bzw. diese unterbricht. Die Hypothesen die dabei geprüft wurden, können grundsätzlich in zwei Kategorien eingeteilt werden:
\begin{itemize}
 \item Hypothesen, die jene Designentscheidungen abbilden, die im Zuge der Erstellung des Werkzeugs getroffen wurden und auf die Verständlichkeit der notwendigen Interaktionen abzielen.
 \item Hypothesen, die die grundsätzliche Verwendbarkeit des Werkzeugs abdecken.
\end{itemize}

Die Untersuchung wurde daneben auch genutzt, um explorativ die inhaltliche Verwendung des Systems zu untersuchen (d.h. wie es für seinen eigentlichen Verwendungszweck, die Modellierung, eingesetzt wurde) und Hypothesen abzuleiten, die in weiteren Schritten getestet werden konnten.

\section{Untersuchungsdesign}

Störvariable: technische Vorkenntnisse 

\section{Durchführung}

In die Durchführung dieses Teils flossen Ergebnisse aus in mehreren Blöcken durchgeführte und zeitlich insgesamt über ein Jahr verteilte Modellierungssessions ein. Als eine Modellierungssession wird hier jeweils eine Anwendung des Werkzeugs mit einer spezifischen Aufgabenstellung bezeichnet. Zwischen den Blöcken wurden im Sinne eines iterativen Entwicklungsprozesses auf dem jeweiligen Feedback der vorhergehenden Sessions Überarbeitungen am und Erweiterungen des Werkzeugs vorgenommen. Auf jene Hypothesen, auf deren Erfüllung derartige Maßnahmen Einfluss hatten, werden im nächsten Abschnitt gekennzeichnet und entsprechend beschrieben.

Der erste Block der Sessions wurde in einem inhaltlich offenen Setting durchgeführt.

Der zweite Block wurde im Rahmen 

\section{Ergebnisse}


% chapter eval_tui (end) 