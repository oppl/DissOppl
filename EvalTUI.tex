\chapter{Evaluierung der Verwendbarkeit des Werkzeugs} % (fold)
\label{cha:eval_werkzeug}

Im ersten empirischen Teil der Evaluierung wurde die grundlegende Verständlichkeit und Verwendbarkeit des Werkzeug geprüft. Ziel war es hier, konzeptionelle und technische Eigenschaften bzw. Verhaltensweisen des Werkzeugs zu identifizieren, die den Modellierungsprozess behindern oder unterbrechen. Darunter fällt grundsätzlich jede Eigenschaft und jede Verhaltensweise, die die Benutzer zwingt, sich mit dem technischen System an sich zu beschäftigen und von der Erfüllung der eigentlichen Aufgabe ablenkt bzw. diese unterbricht. Abbildung \ref{fig:img_Kontextgrafiken_k12} stellt dieses Kapitel und dessen Aufbau im Kontext der anderen inhaltlich vor- und nachgelagerten Kapitel dar.

\begin{figure}[htbp]
	\centering
		\includegraphics[scale=0.6]{img/Kontextgrafiken/k12.png}
	\caption{Kapitel „Evaluierung der Verwendbarkeit des Werkzeugs“ im Gesamtzusammenhang}
	\label{fig:img_Kontextgrafiken_k12}
\end{figure}

Die Untersuchung wurde daneben auch genutzt, um explorativ die inhaltliche Verwendung des Systems zu untersuchen (d.h. wie es für seinen eigentlichen Verwendungszweck, die Modellierung, eingesetzt wurde) und Hypothesen abzuleiten, die in weiteren Schritten getestet werden konnten.

\section{Hypothesen} % (fold)
\label{sec:hypothesen}

In diesem Abschnitt werden die Hypothesen angeführt und begründet, die in diesem Teil der empirischen Untersuchung geprüft werden. Die hier angegebenen Hypothesen gehen auf die Eigenschaften des Werkzeugs in der Verwendung durch die Benutzer ein. Bei der Hypothesenbildung wird auf den Verwendungszweck des Werkzeugs, die Unterstützung der Bildung diagrammatischer Modelle, Rücksicht genommen -- die Modelle selbst sind jedoch nicht Gegenstand der Betrachtung, sondern werden erst im nächsten Kapitel behandelt. Nicht berücksichtigt wird außerdem die Verwendung zur Unterstützung von Articulation Work -- die Implikationen des Werkzeugs auf diese sind Gegenstand von Kapitel \ref{cha:eval_aw}.

\subsection{Konzeptionell begründete Hypothesen} % (fold)
\label{sub:konzeptionell_begründete_hypothesen}

Die folgenden Hypothesen wurden aus der Aufgabenstellung (siehe Kapitel \ref{cha:einführung}) sowie den Anforderungen an das Werkzeug (siehe Kapitel \ref{cha:anforderungen}) abgeleitet. Neben der Formulierung der Hypothese ist jeweils die Begründung aus der Konzeption des Werkzeugs angeführt.

Der grundlegende Anspruch des Werkzeugs ist es, explizite Articulation Work zu unterstützen. Wie in Teil \ref{prt:grundlagen} dieser Arbeit beschrieben, wird dies hier über die Externalisierung und Aushandlung von mentalen Modellen realisiert. Ein gängiges Mittel, um mentale Modelle zu repräsentieren, sind diagrammatische Modelle, worunter die Ergebnisse der vorgeschlagenen Methoden zur Externalisierung -- Concept Mapping und Strukturlegetechniken -- fallen. Das Werkzeug muss also die Repräsentation diagrammatischer Modelle unterstützen. Die Prüfung dieser Hypothese ermöglicht die Beurteilung der Erfüllung der Anforderung \ref{anf:physische_abbildung_legen_beliebiger_diagrammatischer_modelle} (siehe Seite \pageref{anf:physische_abbildung_legen_beliebiger_diagrammatischer_modelle}). 

\begin{hyp}
	\label{hyp:diagmodelle}
	Das Werkzeug ermöglicht die Repräsentation diagrammatische Modelle.
\end{hyp}

„Articulation Work“ ist immer in einen kooperativen Arbeitszusammenhang eingebettet. Die Kollaboration findet dabei nicht nur im produktiven Teil der Arbeit statt, sondern hat immer auch Auswirkungen auf die „Articulation Work“. Jede Unterstützung von „Articulation Work“ muss damit auch in kooperativen Szenarien einsetzbar sein. Dies gilt auch für das hier vorgestellte Werkzeug, das die kooperative Bearbeitung einer Aufgaben (hier: der Externalisierung und Abstimmung mentaler Modelle) ermöglichen muss. Die Prüfung dieser Hypothese ermöglicht die Beurteilung der Erfüllung der Anforderung \ref{anf:kollaborative_und_unmittelbare_manipulierbarkeit_des_modells} (siehe Seite \pageref{anf:kollaborative_und_unmittelbare_manipulierbarkeit_des_modells}).

\begin{hyp}
	\label{hyp:kollaborativ}
	Das Werkzeug ermöglicht kooperatives Arbeiten an einer Aufgabe.
\end{hyp}

Die Aspekte von Arbeit, die im Rahmen von „Articulation Work“ abzustimmen sind, sind unterschiedlicher Natur. Naheliegend ist eine Abstimmung der Abläufe und Schnittstellen zwischen Personen, aber auch nicht-prozedurale Information wie das Verständnis der Struktur und Elemente eines Arbeitszusammenhangs kann Gegenstand von Articulation Work sein. Gleiches gilt für die im Rahmen der Articulation Work abzustimmenden mentalen Modelle -- diese bilden die Basis für Handlungsentscheidungen, umfassen aber im Allgemeinen (in Abgrenzung zu Schemata) nicht nur handlungsleitende Information sondern auch Kontextinformation, die die Bewertung der wahrgenommenen Situation ermöglicht. Demensprechend muss ein Werkzeug zu Unterstützung von expliziter Articulation Work und damit der Externalisierung von mentalen Modellen die Verwendung in unterschiedlichen Kontexten, d.h. für unterschiedliche zu externalisierenden Informationsstrukturen, die in mentalen Modellen abgebildet sind, ermöglichen. Die Prüfung dieser Hypothese ermöglicht die Beurteilung der Erfüllung der Anforderung \ref{anf:nicht_vorgegebene_semantik_der_modellierungselemente} (siehe Seite \pageref{anf:nicht_vorgegebene_semantik_der_modellierungselemente}).

\begin{hyp}
	\label{hyp:kontexte}
	Das Werkzeug ist gleichwertig für Modellierungsaufgaben in unterschiedlichen Kontexten einsetzbar.
\end{hyp}

Die ersten drei hier formulierten Hypothesen sind unmittelbar aus der globalen Zielsetzung abgeleitet und bilden die grundlegenden Anforderungen an das Werkzeug bei der Unterstützung von Articulation Work ab. Die nun folgenden Hypothesen sind konzeptionell nicht mehr direkt auf die globale Zielsetzung ausgerichtet sondern stellen auf Funktionalität des Werkzeugs ab, die den Modellbildungsprozess unterstützen soll. 

Auf Basis der Möglichkeit zur Navigation durch die Entstehungsgeschichte des Modells besteht auch die Möglichkeit, vergangene Modellzustände wiederherzustellen. Das Werkzeug unterstützt dabei die Benutzer durch die Ausgabe von schrittweisen Anweisungen, die den aktuellen Modellzustand in den wiederherzustellenden Zustand überführen. Allgemein bietet diese Funktionalität die Möglichkeit, erkannte Fehler im Modell zu korrigieren, ohne dabei bereits repräsentierte Information zu verlieren. Im kollaborativen Einsatz ermöglicht diese Funktionalität, alternative, individuelle Sichten auf den abzustimmenden Sachverhalt zu repräsentieren und dabei die Möglichkeit bieten, einen für alle Beteiligten akzeptablen Ausgangspunkt wiederherzustellen. Die Prüfung dieser Hypothese ermöglicht die Beurteilung der Erfüllung der Anforderung \ref{anf:ermöglichung_experimenteller_veränderungen_am_modell} (siehe Seite \pageref{anf:ermöglichung_experimenteller_veränderungen_am_modell}).

\begin{hyp}
	\label{hyp:wiederherstellung}
	Die Möglichkeit der Wiederherstellung vergangener Modellzustände fördert die Bereitschaft alternative Repräsentationen auszuprobieren.
\end{hyp}

Die letzten beiden Hypothesen dieses Abschnitts sind ausschließlich auf die Verwendung des Werkzeugs an sich ausgerichtet und stehen nicht im Kontext von Articulation Work oder der Unterstützung der Externalisierung mentaler Modelle. Hypothese \ref{hyp:behinderung} steht für den in der Zielsetzung formulierten Anspruch, dass das Werkzeug in den Hintergrund treten muss und die Beschäftigung mit der eigentlichen Aufgabe nicht behindern darf. Dabei wird hier nicht auf den konkreten Anwendungsfall -- die Erstellung von Modellen -- eingegangen sondern lediglich die allgemeine Funktionsfähigkeit und Bedienbarkeit des Werkzeugs betrachtet. Ersteres ist Gegenstand der Evaluierung der erstellten Modelle, die in Kapitel \ref{cha:eval_modell} beschrieben werden. Die Prüfung dieser Hypothese ermöglicht die Beurteilung der Erfüllung der Anforderung \ref{anf:physische_abbildung_legen_beliebiger_diagrammatischer_modelle} (siehe Seite \pageref{anf:physische_abbildung_legen_beliebiger_diagrammatischer_modelle}).

\begin{hyp}
	\label{hyp:behinderung}
	Das Werkzeug behindert die Modellbildung nicht.
\end{hyp}

Hypothese \ref{hyp:gewöhnung} geht davon aus, dass bei wiederholten Verwendung des Werkzeugs Lern- und Gewöhnungseffekte auftreten, die die Verwendung erleichtern, beschleunigen und zu weniger Fehlbedienung führen. Dies ist ein Effekt, der bei jedem Werkzeug zu erwarten ist, dessen zugrunde liegenden Konzepte den Benutzern bewusst sind. Von dieser Voraussetzung kann durch die inhaltliche Einführung der Benutzer in die das Werkzeug prägenden und motivierenden Ideen ausgegangen werden. Damit wäre zu erwarten, dass das Werkzeug bei wiederholtem Einsatz in den späteren Anwendungen effizienter (im Sinne von schneller und Fehlbedienungen vermeidend) verwendet wird. Die Prüfung dieser Hypothese ermöglicht die Beurteilung der Erfüllung der Anforderung \ref{anf:physische_abbildung_legen_beliebiger_diagrammatischer_modelle} (siehe Seite \pageref{anf:physische_abbildung_legen_beliebiger_diagrammatischer_modelle}).

\begin{hyp}
	\label{hyp:gewöhnung}
	Wiederholte Verwendung des Werkzeugs führt zu schnellerer Modellbildung und weniger Fehlbedienungen.
\end{hyp}

Hinsichtlich der in Kapitel \ref{cha:anforderungen} formulierten Anforderungen können die hier formulierten Hypothesen zusammenfassend wie in Tabelle \ref{hyp:eval_tui} dargestellt eingeordnet werden:

\begin{table}[htbp]
	\centering
	\caption{Hypothesen zur Werkzeugbenutzung und deren Bezug zu den Anforderungen an das Werkzeug}
\begin{tabular}{|c|c|}
  \hline
   Hypothese & Anforderung \\ \hline
   \ref{hyp:diagmodelle} & \ref{anf:physische_abbildung_legen_beliebiger_diagrammatischer_modelle} \\
   \ref{hyp:kollaborativ} & \ref{anf:kollaborative_und_unmittelbare_manipulierbarkeit_des_modells} \\
   \ref{hyp:kontexte} & \ref{anf:nicht_vorgegebene_semantik_der_modellierungselemente} \\
   \ref{hyp:wiederherstellung} & \ref{anf:ermöglichung_experimenteller_veränderungen_am_modell} \\
   \ref{hyp:behinderung} & \ref{anf:physische_abbildung_legen_beliebiger_diagrammatischer_modelle} \\
   \ref{hyp:gewöhnung} & \ref{anf:physische_abbildung_legen_beliebiger_diagrammatischer_modelle} \\ \hline
\end{tabular} 
	\label{hyp:eval_tui}
\end{table}

% subsection konzeptionell_begründete_hypothesen (end)

\subsection{Explorativ gebildete Hypothesen} % (fold)
\label{sub:explorativ_gebildete_hypothesen}

Neben den aus der Aufgabenstellung abgeleiteten Hypothesen wurden einige Hypothesen auch während der Durchführung der einzelnen Evaluierungs-Blöcke gebildet. Diese Hypothesen sind spezifischer auf einzelne Aspekte des Werkzeugs abgestellt und decken beobachtete Auffälligkeiten und Missverständnisse in der Verwendung des Werkzeugs ab. 

Die erste in diesem Zusammenhang beobachtete Auffälligkeit betrifft die Herstellung von Verbindern zwischen einzelnen Modellelementen. Wie in Abschnitt \ref{sub:verbinden_von_modellelementen} beschrieben, existieren zwei Möglichkeiten, diese Funktion auszuführen. Einerseits können die beiden Modellelemente, die verbunden werden sollen, mit Markierungs-Tokens ausgewählt werden, worauf hin eine Verbindung hergestellt werden. Andererseits können Verbinder auch durch das Zusammenführen der zu verbindenden Blöcke (bis sich deren Breitseiten berühren) hergestellt werden. In der ersten Implementierung des Werkzeugs, die im Evaluierungs-Block 1 und im ersten Teil des zweiten Blocks verwendet wurde, war lediglich die erste Variante verfügbar. Die Möglichkeit zur Herstellung von Verbindern wurde in den in diesen Blöcken durchgeführten Anwendungen kaum eingesetzt. Dies führte einerseits zur Bildung der Hypothese \ref{hyp:keine_verbinder} (siehe Abschnitt \ref{sub:m_explorativ_gebildete_hypothesen}), andererseits wurde bei ersten Auswertungen der Beobachtungen der im Verhältnis zum übrigen Modellierungs-Prozess hohe Zeit-Aufwand bei der Herstellung von Verbindern offensichtlich. Dieser Aufwand ist den Maßnahmen zur Stabilisierung der Erkennungsleistung des Werkzeugs geschuldet und kann mit den eingesetzten Interaktionsablauf nicht reduziert werden. Aufgrund einer Anregung eines Untersuchungsteilnehmers wurde deshalb die oben beschriebene zusätzliche Möglichkeit zur Herstellung von Verbindungen implementiert. Zu untersuchen ist nun, ob diese Maßnahme die Nutzung von Verbindern bei der Modellbildung tatsächlich erhöht.

\begin{hyp}
	\label{hyp:verbinder}
	Die Einführung der alternativen Möglichkeit zur Verbindungsherstellung erhöht die Nutzung von Verbindern bei der Modellerstellung.
\end{hyp}

Die zweite hier aufgestellte Hypothese betrifft eine Auffälligkeit bei der Verwendung des Löschtokens. Das Löschtoken wird verwendet, um das Werkzeug in einen Modus zu versetzen, in dem Verbinder gelöscht werden können. Schon die konzeptionelle Einordnung des Werkzeugs in Kapitel \ref{cha:konzeptionelle_evaluierung} zeigte Potential für Missverständnisse in der Verwendung dieses Tokens (siehe z.B. die Abschnitte \ref{sec:spezifikation_des_tac_schemas_nach_shaer_et_al_} und \ref{sec:einordnung_in_die_taxonomie_von_fishkin}). Zusammengefasst liegt die aus der Theorie ableitbare Problematik darin, dass durch die äußere Form des Tokens -- einem Radiergummi -- eine Metapher für dessen Verwendung („ausradieren“ von Elementen) suggeriert wird, die in dieser Form im Werkzeug nicht umgesetzt ist, da das Token lediglich als Schalter fungiert. Erste Beobachtungen deuteten darauf hin, dass die Verwendung des Löschtoken tatsächlich unverständlich oder missverständlich zu sein scheint. Die zugehörige Hypothese ist positiv formuliert, zu erwarten wäre demnach, dass sie verworfen werden muss.

\begin{hyp}
	\label{hyp:radierer}
	Das Löschtoken ermöglicht intuitives Löschen von Modellelementen.
\end{hyp}

% subsection explorativ_gebildete_hypothesen (end)
% section hypothesen (end)

\section{Untersuchungsdesign und Durchführung} % (fold)
\label{sec:untersuchungsdesign}

In diesem Abschnitt wird auf Basis der oben formulierten Hypothesen das Untersuchungsdesign abgeleitet und die Durchführung der Untersuchung beschrieben. Der erste Teil des Abschnitts beschreibt die Operationalisierung der Hypothesen und damit die Festlegung wie diese konkret geprüft werden können. Im zweiten Teil des Abschnitts wird die Durchführung der Prüfung beschrieben. Hier erfolgt neben der Zuordnung der einzelnen Evaluierungsblöcke (siehe Abschnitt \ref{sec:globales_untersuchungsdesign}) auch die Darstellung rein beschreibender Parameter der Werkzeugverwendung, die nicht unmittelbar in die Prüfung der Hypothesen eingehen. 

\subsection{Operationalisierung} % (fold)
\label{sub:operationalisierung}

In diesem Abschnitt wird für jede Hypothese identifiziert, in welcher Form sie geprüft werden kann. Dies umfasst die Festlegung der Messpunkte sowie der jeweiligen Mess- und Auswertungsmethode (letzte bezugnehmend auf den in Abschnitt \ref{sec:eingesetzte_werkzeuge_und_verfahren} beschriebenen Verfahren). Zudem werden jene Evaluationsblöcke festgelegt, die für die jeweilige Untersuchung herangezogen wurden.

Für jede Hypothese wird also spezifiziert, anhand welcher Aspekte diese geprüft werden kann (= abhängige Variablen). Zudem wird festgelegt welche Ausgangssituation bei der Anwendung gewählt werden muss, um die Prüfung durchführen zu können (= unabhängige Variable) und welche Faktoren die Beurteilung ggf. ungewollt beeinflussen können (= Störvariablen).

\subsubsection{Repräsentation diagrammatischer Modelle} % (fold)
\label{ssub:repräsentation}

Gegenstand dieses Abschnitts ist die Prüfung der Hypothese \ref{hyp:diagmodelle}. Diese bezieht sich auf die Eignung des Werkzeugs für die Repräsentation diagrammatischer Modelle.

Voraussetzung für die Prüfung der Hypothese ist der Einsatz von Modellierungsaufgaben, die so formuliert sind, dass es grundsätzlich möglich ist, sie durch die Beschreibung in einem diagrammatischen Modell zu erfüllen. Keinen Einfluss auf die Untersuchung haben die eingesetzte Methodik sowie eventuell vorhandene Modellierungsvorkenntnisse, da die grundsätzlich Möglichkeit der Erstellung diagrammatischer Modelle unabhängig von der Art der Verwendung und von der Kompetenz der Benutzer ist. 

Geprüft wird die Hypothese hier an der Repräsentation, die mit Hilfe des Werkzeugs erstellt wurde. Ein diagrammatisches Modell zeichnet nach \citep{Larkin87} aus, dass in ihm Konzepte und deren Zusammenhänge visuell-graphisch dargestellt werden können (in Abgrenzung zu textuellen Beschreibungen). Zur Bewertung der Hypothese werden deshalb die erstellten Repräsentationen herangezogen und überprüft, ob sie den Anforderungen an ein diagrammatisches Modell -- das Vorhandensein von Konzepten und Beziehungen zwischen diesen -- erfüllen.

% subsubsection repräsentation (end)

\subsubsection{Kooperatives Arbeiten} % (fold)
\label{ssub:kollaboratives_arbeiten}

Gegenstand dieses Abschnitts ist die Prüfung der Hypothese \ref{hyp:kollaborativ}. Dabei wird überprüft, ob das Werkzeug kooperatives Arbeiten an einer Modellierungsaufgabe erlaubt.

Dazu muss eine Modellierungsaufgabe gewählt werden, in der die kooperatives Erstellung des Modells vorgesehen ist. Etwaige Modellierungsvorkenntnisse haben keinen Einfluss auf die Beurteilung der hier betrachteten Hypothese.

Zur Beurteilung eignen sich in diesem Fall die Zeitverteilung der Beteiligung der einzelnen Benutzer am Modellierungsvorgang, das Verhalten der Benutzer bei simultaner Manipulation eines Modells auf der Modellierungsoberfläche sowie der subjektive Eindruck der Benutzer über deren Kooperation untereinander. Der erstgenannte Aspekt kann quantitativ gemessen werden, wobei eine tendenziell zeitlich gleichverteilte Einbindung der Beteiligten in die Modellbildung für die Annahme der Hypothese spricht. Zusätzlich kann mittels dem zweiten und dritten Aspekt qualitativ beurteilt werden, ob und wie eine kooperative Manipulation des Modells durch mehrere Benutzer gleichzeitig möglich ist.

% subsubsection kollaboratives_arbeiten (end)

\subsubsection{Einsetzbarkeit in unterschiedlichen Kontexten} % (fold)
\label{ssub:einsetzbarkeit_in_unterschiedlichen_kontexten}

Gegenstand dieses Abschnitts ist die Operationalisierung der Hypothese \ref{hyp:kontexte}. Diese Hypothese zielt dabei auf die Eignung des Werkzeugs zur Modellbildung in unterschiedlichen Kontexten, d.h. für unterschiedliche Modellierungsaufgaben. 

Zur Beurteilung dieser Hypothese muss die Modellierungsaufgabe entsprechend den unterschiedlichen Einsatzkontexten variiert werden. Etwaige Modellierungsvorkenntnisse können die individuelle Beurteilung insofern beeinflussen, als das sie Werkzeugs für eine bestimmte Aufgabe als besser oder schlechter geeignet erscheinen lassen.

Zur Prüfung der Hypothese bieten sich sind in diesem Fall die Wahrnehmung der Eignung durch die Benutzer, die qualitativ beurteilt wird, und die Korrelation der Größe der erstellten Modelle mit der benötigten Modellierungsdauer an. Korrelliert die Modellgröße positiv mit der Modellierungsdauer, so ist der Zeitanteil, der zu Beschäftigung mit dem Werkzeug selbst (und nicht mit der Modellierungsaufgabe) tendenziell stabil. Daraus kann abgeleitet werden, dass das Werkzeug die verglichenen Modellierungsaufgaben gleich gut (oder schlecht) unterstützt.

% subsubsection einsetzbarkeit_in_unterschiedlichen_kontexten (end)

\subsubsection{Wiederherstellung vergangener Modellzustände} % (fold)
\label{ssub:wiederherstellung_vergangener_modellzustände}

Gegenstand dieses Abschnitts ist die Operationalisierung der Hypothese \ref{hyp:wiederherstellung}. Gegenstand der Überprüfung ist die Verwendung der Wiederherstellungsfunktionalität zum Zwecke der versuchsweisen Veränderung des Modells.

Zur Prüfung dieser Hypothese muss die Modellierungsaufgabe so gestaltet, dass sinnvoll unterschiedliche Repräsentationen gebildet werden können. Modellierungsvorkenntnisse haben keine Auswirkungen auf diese Untersuchung.

Zur Beurteilung dieser Hypothese wird ist die \emph{Anzahl der Verwendungen der Wiederherstellungsfunktionalität zur Korrektur inhaltlich verworfener Repräsentationen} herangezogen. Werte über 0 deuten hier auf eine Annahme der Hypothese hin. Zusätzlich können qualitative Aussagen zur Nutzung dieser Funktionalität und deren \emph{wahrgenommenen Nutzen} zur Beurteilung verwendet werden. 

% subsubsection wiederherstellung_vergangener_modellzustände (end)

\subsubsection{Nicht-Behinderung} % (fold)
\label{ssub:nicht_behinderung}

Gegenstand dieses Abschnitts ist die Operationalisierung der Hypothese \ref{hyp:behinderung}. Dabei wird überprüft, ob bei der Verwendung des Werkzeugs dieses in den Aufmerksamkeitsfokus der Benutzer tritt oder sich diese auf die eigentliche Modellierungsaufgabe konzentrieren können. 

Die Modellierungsaufgabe hat keinen Einfluss auf die Überprüfung dieser Hypothese, lediglich etwaig vorhandene \emph{Modellierungsvorkenntnisse} können als \textbf{Störvariable} wirken, da sie Einfluss auf die erwartete Funktionalität des Werkzeugs haben kann.

Zur Beurteilung, ob bzw. inwieweit das Werkzeug die Modellbildung behindert, werden sowohl quantitativ als auch qualitative beurteilbare Metriken herangezogen. Die Anzahl von \emph{Fehlfunktionen des Werkzeugs} bzw. das \emph{Auftreten von Systemabstürzen} kann als Indikator für eine behindernde Wirkung des Werkzeugs herangezogen werden. Das Auftreten von Missverständnissen und daraus resultierende Fehlbedienungen können ebenfalls eine Behinderung des Modellierungsvorgangs interpretiert werden. Zudem werden Aussagen der Benutzer hinsichtlich hinderlicher Faktoren bei der Werkzeugbenutzung als Maß für die wahrgenommene Behinderung durch das Werkzeug herangezogen. Der Einfluss von Modellierungsvorkenntnissen kann in diesem Fall nicht mit statistischen Maßnahmen kompensiert werden. Etwaige Vorkenntnisse werden dementsprechend bei der Auswertung angeführt und müssen bei der Diskussion der Hypothese berücksichtigt werden.

% subsubsection nicht_behinderung (end)

\subsubsection{Gewöhnung an das Werkzeug} % (fold)
\label{ssub:gewöhnung_an_das_werkzeug}

Gegenstand dieses Abschnitts ist die Operationalisierung der Hypothese \ref{hyp:gewöhnung}. Dabei wird überprüft, ob wiederholte Benutzung des Werkzeugs Auswirkung auf die Qualität der Interaktion hat. Eine Erhöhung der Qualität äußert sich in schnellerer Modellbildung und weniger Fehlbedienung.

Bei der Prüfung der Hypothese muss eine etwaige veränderte Funktionalität des Werkzeugs zwischen den verglichenen Evaluierungsblöcken berücksichtigt werden, die die Interaktion einerseits erleichtern kann, andererseits aber auch zu Fehlbedienung aufgrund von unbekannten Interaktionsmustern führen kann. Auch unterschiedliche Modellierungsaufgaben, die ein Individuum in den aufeinander folgenden Anwendungen bearbeitet, können die Beurteilung erschweren, weil potentiell andere (noch unbekannte) Funktionen des Werkzeugs zum Einsatz kommen können.

Zur Beurteilung der Qualität der Interaktion sind einerseits die Anzahl der Fehlbedienungen des Werkzeugs pro Zeiteinheit und andererseits die Arbeitsdauer am Werkzeug\footnote{Die Arbeitsdauer am Werkzeug ist im Gegensatz zur gesamten Modellierungsdauer um jenen Zeitanteil reduziert, in dem die Teilnehmer interagieren, ohne am Werkzeug zu arbeiten.} in Abhängigkeit der Modellgröße heranzuziehen. Die Normierung der Arbeitsdauer ist notwendig, um vergleichbare Werte für unterschiedliche Werkzeug-Anwendungen zu erhalten. Sinken beide Werte zwischen zwei Evaluierungsblöcken, die auf der gleichen Stichprobe aufbauen, signifikant, so kann die Hypothese bestätigt werden. Um eine Vergleichbarkeit zwischen den Anwendungen herzustellen, ist es sinnvoll, in beiden Blöcken eine identische Modellierungsaufgabe zu stellen und die Funktionalität des Werkzeugs nicht zu verändern. Identische Modellierungsaufgaben können durch die wiederholte inhaltliche Beschäftigung mit der Aufgabe zu schnellerer Arbeit bzw. zu kompakteren Modellen führen. Dies kann wiederum durch die Berücksichtigung der reinen Arbeitszeit am Werkzeug sowie der Normierung derselben in Abhängigkeit der Modellgröße kompensiert werden.

% subsubsection gewöhnung_an_das_werkzeug (end)

\subsubsection{Herstellung von Verbindern} % (fold)
\label{ssub:herstellung_von_verbindern}

Gegenstand dieses Abschnitts ist die Operationalisierung der Hypothese \ref{hyp:verbinder}. Mit Hilfe dieser Hyothese soll überprüft werden, ob die Einführung der alternativen Möglichkeit zur Herstellung von Verbindern deren Verwendung signifikant gesteigert hat.

Bei der Messung muss der Einfluss der Modellierungsaufgabe (da sie die Anzahl der benötigten Verbinder beeinflussen kann) und eventuell vorhandene Modellierungsvorkenntnisse (da diese Einfluss auf die Struktur des Modells haben können) berücksichtigt werden. Um den Einfluss dieser Aspekte zu reduzieren, wird die Beurteilung in zwei Evaluierungsblöcken vorgenommen, in denen die gleiche Stichprobe mit der gleichen Aufgabenstellung das Werkzeug mit der gleichen Methodik anwandte. Lediglich die Funktionalität des Werkzeugs wurde zwischen den beiden Anwendungen um den alternativen Weg zur Herstellung von Verbindern erweitert.  

Zur Beurteilung des Ausmaßes der Verwendung von Verbindern kann die \emph{Connectedness} des Modells herangezogen werden. Die Connectedness ist das Verhältnis zwischen der Anzahl der im Modell verwendeten Verbinder und der Anzahl der verwendeten Knoten (Modellierungselemente). Hier ist zu prüfen, ob die Connectedness in jenem Evaluierungs-Block, in dem der alternative Weg zur Herstellung von Verbindungen verfügbar war, signifikant höher ist als in jenem Block, in dem sie nicht verfügbar war.

% subsubsection herstellung_von_verbindern (end)

\subsubsection{Verwendung des Löschtokens} % (fold)
\label{ssub:löschtoken}

Gegenstand dieses Abschnitts ist die Operationalisierung der Hypothese \ref{hyp:radierer}. Dabei wird überprüft, ob das Löschtoken intuitiv korrekt verwendet wird oder ob es zu Fehlinterpretationen kommt.

Die Verwendbarkeit des Löschtokens ist unabhängig von der Modellierungsaufgabe, der angewandten Methodik und auch von eventuell vorhandenen Modellierungsvorkenntnissen. Hinsichtlich des Anwendungskontext des Werkzeugs sind also keine Voraussetzungen zu beachten.

Zur Beurteilung der intuitiven Verwendbarkeit werden quantitative und qualitative Merkmale der Werkzeugverwendung herangezogen. Quantitativ beurteilbar ist der Anteil der Fehlbedienungen des Löschtokens in Bezug auf alle Anwendungen des Werkzeugs, in denen es grundsätzlich verwendet wurde. Qualitativ wird die Art des Missverständnisses, das zu den jeweiligen Fehlbedienungen führt, beurteilt.

Zur Messung der quantitativen Variablen wird für jede Anwendung die Anzahl der Fehlbedienungen erhoben, die durch das Löschtoken verursacht wurden. Dieser Wert wird in Bezug zur Gesamtanzahl der Fehlbedienungen gesetzt, so dass der Anteil der durch das Löschtoken verursachten Fehlbedienungen berechnet werden kann. Bei "gleich guter" intuitiver Bedienbarkeit aller Werkzeuge wäre eine Gleichverteilung der Fehler zu erwarten. Ist der Anteil der durch das Löschtoken verursachten Fehlbedienungen höher als der Anteil, der bei Gleichverteilung zu erwarten wäre, so deutet dies auf eine Ablehnung der Hypothese hin.

Qualitativ werden Modellierungssituationen betrachtet, in denen das Löschtoken zum Einsatz kommt. Auf Basis von Transkripten der Interaktion zwischen den Benutzern und dem Werkzeug, bei denen es zu Fehlbedierungen kam, werden die aufgetretenen Missverständnisse explizit identifziert.

% subsubsection löschtoken (end)

% subsection operationalisierung (end)

\subsection{Durchführung} % (fold)
\label{sub:durchführung}

In diesem Abschnitt werden die für diesen Evaluierungs-Teil relevanten deskriptiven Parameter der berücksichtigten Anwendungs-Blöcke angeführt.
Als Grundlage der Überprüfung der Hypothesen werden hier die Evaluierungs-Blöcke 1 bis 5 verwendet. Dabei wurden für die quantitativ zu prüfenden Variablen die Blöcke 2 und 3 herangezogen, da in diesen die größten Stichproben zur Verfügung standen. In die qualitative Auswertung der Ergebnisse wurden hingegen alle Blöcke (1-5) mit einbezogen.

\subsubsection{Stichprobe} % (fold)

Für die Untersuchung der Hypothesen in diesem Kapitel wurden die Evaluierungsblöcke 1 bis 5 herangezogen. Die Stichprobe setzt sich wie in Tabelle \ref{tab:stichprobe_tui} beschrieben zusammen.

\begin{table}[htbp]
	\centering
	\caption{Stichproben der Evaluierung zur Werkzeugverwendung}

		\begin{tabular}{| l || c | c |}
		\hline
			Evaluierungsblock & $n_{Anwendungen}$ & $n_{Teilnehmende}$ \\ \hline
			technische Evaluierung		  &  9 & 18 \\
			Aushandlung 1 (1. Durchgang)  &  9 & 19 \\
			Aushandlung 1 (2. Durchgang)  &  9 & 18 \\
			Concept Mapping 1			  & 18 & 54 \\
			Aushandlung 2				  & 10 & 13 \\
			Concept Mapping 2 (Tisch)     & 11 & 24 \\  \hline
			Gesamt						  & 66 & 146 \\ \hline
	\end{tabular}
	\label{tab:stichprobe_tui}
\end{table}

\subsubsection{Dauer der Werkzeugverwendung} % (fold)

Die Dauer der Werkzeug-Verwendung wurde den Blöcken 2 („Aushandlung“) und 3 („Concept Mapping“) erhoben. Die Bearbeitungszeit ist wie in Tabelle \ref{tab:dauer_werkzeugverwendung} dargestellt verteilt (siehe auch Abbildung \ref{fig:img_Evaluierung_usageTimeOverview}\footnote{In allen Boxplots gilt folgende Notation: 
\begin{itemize}
	\item breite horizontale Linie: Bereich zwischen 25\%- und 75\%-Quantil
	\item breite vertikale Linie: Median
	\item linke schmale Linie: Bereich zwischen 2,5\%- und 25\%-Quantil
	\item rechte schmale Linie: Bereich zwischen 75\%- und 97,5\%-Quantil
	\item Kreuze: Ausreißer (außerhalb 2,5\%- und 97,5\%-Quantil)
\end{itemize}
}):

\begin{table}[htbp]
	\centering
	\caption{Dauer der Werkzeugverwendung}
\begin{tabular}{| p{1cm} || p{3cm} | p{3cm} | p{3cm} |}
  \hline
   & Aushandlung (1. Durchgang) & Aushandlung (2. Durchgang) & Concept Mapping \\ \hline
   $t_{min}$ & 11m 54s & 2m 5s & 14m 1s \\ 
   $\overline{t}$ & 20m 53s & 9m 49s & 32m 32s \\ 
   $s(t)$ & 4m 18s & 5m 20s & 10m 7s \\
   $t_{max}$ & 27m 30s & 19m 29s & 45m 0s \\ \hline
\end{tabular} 
	\label{tab:dauer_werkzeugverwendung}
\end{table}

\begin{figure}[htbp]
	\centering
		\includegraphics[width=15cm]{img/Evaluierung/usageTimeOverview.png}
	\caption{Dauer der Werkzeugverwendung -- Überblick}
	\label{fig:img_Evaluierung_usageTimeOverview}
\end{figure}

Die erhobene Dauer der Werkzeug-Verwendung teilt sich ein einen Anteil, an dem tatsächlich mit dem Werkzeug interagiert wird und einen Anteil, der anderen Tätigkeiten (wie inhaltlicher Diskussion, Bedeutungsaushandlung, \ldots) gewidmet ist. Diese beiden Anteile sind in den einzelnen Blöcken wie folgt verteilt (siehe auch die Abbildungen \ref{fig:img_Evaluierung_usageTimeConceptMapping} und \ref{fig:img_Evaluierung_usageTimeNegotiation}):

\begin{figure}[htbp]
	\centering
		\includegraphics[width=15cm]{img/Evaluierung/usageTimeConceptMapping.png}
	\caption{Dauer der Werkzeugverwendung -- Concept Mapping}
	\label{fig:img_Evaluierung_usageTimeConceptMapping}
\end{figure}

\begin{figure}[htbp]
	\centering
		\includegraphics[width=15cm]{img/Evaluierung/usageTimeNegotiation.png}
	\caption{Dauer der Werkzeugverwendung -- Aushandlung}
	\label{fig:img_Evaluierung_usageTimeNegotiation}
\end{figure}

% subsection durchführung (end)
% section untersuchungsdesign (end)

\section{Ergebnisse} % (fold)
\label{sec:ergebnisse}

\subsection{Repräsentation diagrammatischer Modelle} % (fold)
\label{sub:repräsentation_diagrammatischer_modelle}

Gegenstand der hier beschriebenen Untersuchung ist Hypothese \ref{hyp:diagmodelle} („Das Werkzeug ermöglicht die Repräsentation diagrammatische Modelle.“). Als Grundlage dieser Untersuchung dienen die Ergebnisse aller Evaluierungsblöcke, da die Aufgaben in allen Fällen auf die Erstellung einer Repräsentation in Form eines diagrammatischen Modells gefordert war.

Ausgewertet wird hier, ob die Ergebnisse der Modellierung jeweils als diagrammatisches Modell zu klassifizieren sind. Ein diagrammatisches Modell zeichnet nach \citep{Larkin87} aus, dass in ihm Konzepte und deren Zusammenhänge visuell-graphisch dargestellt werden. Eine Darstellung von Beziehungen kann durch die explizite Darstellung von Verbindungen zwischen Konzepten oder durch andere graphische Mittel wie Gruppierung von Konzepten in räumlicher Nähe erfolgen. Um eine eindeutige Auswertbarkeit gewährleisten zu können, wird hier auf die explizite Darstellung von Verbindungen eingeschränkt. 

\subsubsection{Auswertung} % (fold)

In allen vorliegenden Modellen wurden Konzepte als Grundelemente des diagrammatischen Modells verwendet. Das Kriterium zur Klassifizierung als diagrammatisches Modell ist im Folgenden also das Vorhandensein von Verbindungen. Bei der Auswertung ergab sich die in Tabelle \ref{tab:modelle_mit_verbindern} dargestellte Verteilung.

\begin{table}[htbp]
	\centering
	\caption{Anzahl der Modelle mit Verbindern}

\begin{tabular}{| p{3cm} || p{3cm} | p{3cm} |}
  \hline
   Block & Modelle gesamt & Modelle mit Verbindern \\ \hline
   1 & 9 & 0 \\ 
   2 & 18 & 9 \\ 
   3 & 18 & 17 \\ 
   4 & 10 & 10 \\ 
   5 & 11 & 11 \\ \hline
   Gesamt & 66 & 47 \\ \hline
\end{tabular}
	\label{tab:modelle_mit_verbindern}
\end{table}

Insgesamt sind in 66 Modellen, die als Ergebnis vorliegen, 47 Modelle zu identifizieren, in denen explizit Verbindungen zur Darstellung von Beziehungen zwischen Konzepten verwendet werden ($71,2\%$). Eine implizite Darstellung von Beziehungen ist jedoch in allen vorliegenden Modellen zu erkennen. Nicht explizit durch Verbindungen abgebildete Beziehungen werden in allen Fällen durch die räumliche Konfiguration der Konzepte zueinander dargestellt.

\subsubsection{Diskussion} % (fold)

Legt man das Kriterium des Vorhandenseins von Verbindungen zwischen Konzepten an, so sind $71,2\%$ der betrachteten Modelle als diagrammatische Modelle zu klassifizieren. Dies erscheint vordergründig eine geringe Zahl zu sein, die gegen die allgemeine Gültigkeit der Hypothese sprechen würde. Allerdings sind in allen Modelle implizite Verbindungen zwischen Konzepten eindeutig zu identifizieren. Außerdem ist zu erkennen, dass der Anteil an diagramatischen Modellen über die Evaluierungsblöcke (und damit die Weiterentwicklung des Werkzeugs über die Zeit) hinweg stetig ansteigt, bis er in den letzten beiden Blöcken jeweils $100\%$ erreicht. Dies ist durch technische Fehlfunktionen zu erklären, die es in ersten Evaluierungsblöcken schwer bzw. teilweise unmöglich machten, explizite Verbindungen intentional zu erstellen. Unter Anbetracht dieser Erkenntnisse erscheint die Annahme der Hypothese \ref{hyp:diagmodelle} als gerechtfertigt.

Die Abbildung von Verbindungen durch räumliche Konfiguration ist Gegenstand der Prüfung von Hypothese \ref{hyp:keine_verbinder} in Kapitel \ref{cha:eval_modell} und wird dort einer näheren Betrachtung unterzogen.

\subsubsection{Ergebnis} % (fold)

\textbf{Hypothese \ref{hyp:diagmodelle} kann auf Basis der Untersuchung bestätigt werden.} Die Abbildung von Konzepten und Beziehungen zwischen diesen wurde in allen vorliegenden Modellen erfolgreich umgesetzt, wenngleich die Modellierung von expliziten Verbindungen in den ersten beiden Evaluierungsblöcken aufgrund von technischen Unzulänglichkeiten nicht durchgeführt wurde.

% subsection repräsentation_diagrammatischer_modelle (end)

\subsection{Kooperatives Arbeiten} % (fold)
\label{sub:kollaboratives_arbeiten}

Gegenstand der hier beschriebenen Untersuchung ist Hypothese \ref{hyp:kollaborativ} („Das Werkzeug ermöglicht kooperatives Arbeiten an einer Aufgabe.“). Zur Untersuchung der quantitativ beurteilbaren Aspekte wurden die Werkzeuganwendungen aus den Evaluierungsblöcken 2 ($n=9$) und 3 ($n=18$) herangezogen, wobei in Block 2 in Gruppen zu zwei Personen modelliert wurde (in einem Fall drei Personen), in Block 3 in Gruppen zu drei Personen (in drei Fällen nur zwei Personen). Zusätzlich wurden zur qualitative Beurteilung Daten aus Block 4 verwendet.

In Evaluierungsblock 4 wurde hinsichtlich der subjektiven Wahrnehmung der Kooperation eine Befragung der Teilnehmer mittels eines Fragebogens durchgeführt (diese umfasste auch weitere Aspekte, die in späteren Abschnitten besprochen werden). Die Fragestellungen zur Kooperation wurde in 4 geschlossenen Items codiert, die auf einer 7-teiligen Likert-Skala zu beantworten waren. Zusätzlich wurden offene Fragen hinsichtlich der Nützlichkeit der Werkzeugs eingesetzt, die an dieser Stelle ebenfalls hinsichtlich Aussagen zur Kooperation zwischen den Teilnehmern ausgewertet werden.

\subsubsection{Auswertung} % (fold)

Grundlage des ersten Teils der Auswertung ist die Verteilung der Modellierungsdauer zwischen den Teilnehmern. Um die unterschiedliche Gesamt-Modellierungsdauer in den einzelnen Anwendungen zu kompensieren, wurden die Berechnungen auf Basis der prozentuellen Zeitanteile der einzelnen Teilnehmer durchgeführt. Die einzelnen Datensätze wurden so sortiert, dass die anteilsmäßige Modellierungsdauer von Teilnehmer A bis Teilnehmer C (bzw. B) abnimmt. In den einzelnen Evaluierungsblöcken ergeben sich die in Abbildung \ref{fig:img_Evaluierung_timeDist} dargestellten Verteilungen.

\begin{figure}[htbp]
	\centering
		\includegraphics[height=2.5in]{img/Evaluierung/timeDistSE1.png}
		\includegraphics[height=2.5in]{img/Evaluierung/timeDistSE2.png}
		\includegraphics[height=2.5in]{img/Evaluierung/timeDistUE.png}
	\caption{Zeitverteilung zwischen den Teilnehmern}
	\label{fig:img_Evaluierung_timeDist}
\end{figure}

\todo Zu prüfen ist hier, ob die Zeit-Anteile der einzelnen Teilnehmer signifikant unterschiedlich sind. Dazu wird für jeden Block die Signifikanz zwischen den Verteilung der einzelnen Teilnehmerklassen berechnet (eine Teilnehmerklasse setzt sich aus all jenen Teilnehmern zusammen, die am längsten, am zweitlängsten bzw. am drittlängsten aktiv waren).  Aufgrund der geringen Stichprobengröße kommt zur Prüfung der Signifikanz der t-Test nicht in Frage, es wird der \emph{Wilcoxon-Test} herangezogen. Der t-Test setzt außerdem Normalverteilung der Prüfgrößen voraus, was zumindest bei einer der Verteilungen nicht der Fall ist (Sharpiro-Wilk-Test für $conn_{B22}$: $p=6.29e^{-5}$, damit ist von Nicht-Normalverteilung auszugehen).

Im zweiten Teil der Auswertung wurde in einem Fragebogen in 4 Items aggregiert die Frage nach kooperativen Aspekten bei der Modellbildung gestellt. Diese wurden im Schnitt als sehr hoch oder hoch beurteilt ($M = 1.79$, $SD = 0.56$, $t4(13) = -14.28$, $p<.001$). Dieses Ergebnis steht in Übereinstimmung mit den qualitativen Aussagen der Benutzer, von denen 10 explizit auf die kooperationsfördende Wirkung des Werkzeugs hinwiesen. Auch in Auswertungen der  Videoaufnahmen der betreffenden Modellierungsdurchgänge ist zu erkennen, dass zwischen $40$ und $70\%$ der gesamten Modellierungsdauer der Interaktion zwischen den Teilnehmern zuzurechnen ist.

\subsubsection{Diskussion} % (fold)

\subsubsection{Ergebnis} % (fold)


% subsection kollaboratives_arbeiten (end)

\subsection{Einsetzbarkeit in unterschiedlichen Kontexten} % (fold)
\label{sub:einsetzbarkeit_in_unterschiedlichen_kontexten}

Gegenstand der hier beschriebenen Untersuchung Hypothese \ref{hyp:kontexte} („Das Werkzeug ist gleichwertig für Modellierungsaufgaben in unterschiedlichen Kontexten einsetzbar.“). Als Grundlage dieser Untersuchung dienen die Ergebnisse der Evaluierungsblöcke 2 und 4.

\subsubsection{Auswertung} 

\subsubsection{Diskussion} 

\subsubsection{Ergebnis} 

% subsection einsetzbarkeit_in_unterschiedlichen_kontexten (end)

\subsection{Wiederherstellung vergangener Modellzustände} % (fold)
\label{sub:wiederherstellung_vergangener_modellzustände}

Gegenstand der hier beschriebenen Untersuchung ist Hypothese \ref{hyp:wiederherstellung} („Die Möglichkeit der Wiederherstellung vergangener Modellzustände fördert die Bereitschaft alternative Repräsentationen auszuprobieren.“). Als Grundlage dieser Untersuchung dienen die Ergebnisse der Evaluierungsblöcke 2 bis 5, da die Funktion zur Wiederherstellung vergangener Modellzustände erst in diesen Blöcken funktionsfähig zur Verfügung stand.

\subsubsection{Auswertung} 

Für alle Anwendungen des Werkzeugs in den Evaluierungsblöcken 2 bis 5 wurde hier untersucht, wie oft die Möglichkeit zur Wiederherstellung vergangener Modellzustände eingesetzt wurde, um alternative Modellierungswege auszuprobieren. Nicht berücksichtigt wurden Einsätze derselben Funktion, die zur Korrektur von Modellierungsfehlern durch Fehlerkennungen des Systems verwenden wurden (verstärkt in den Evaluierungsblöcken 2 und 3 aufgetreten, in 4 und 5 durch Stabilisierung der Erkennungsleistung nicht mehr relevant). Die Verteilung des Einsatzes der Funkion ist in absoluten Zahlen in Tabelle \ref{tab:anzahl_wiederherstellung} für jeden Evaluierungsblock angeführt

\begin{table}[htbp]
	\centering
	\caption{Anzahl des Einsatzes der Wiederherstellungsfunkion}
\begin{tabular}{| c || c | c | c | c || c | c | c | c |}
  \hline
   EB    & 0 E. & 1 E. & 2 E. & 3+ E. \\ \hline
   2     & 18 & 0 & 0 & 0 \\ 
   3     & 14 & 4 & 0 & 0 \\ 
   4     & 10 & 0 & 0 & 0 \\ 
   5     & 10 & 1 & 0 & 0 \\ \hline
   Ges.  & 52 & 5 & 0 & 0 \\ \hline
\end{tabular} \\
\footnotesize EB \ldots Evaluierungsblock, x E.\ldots x Einsätze der Wiederherstellungsfunktion
	\label{tab:anzahl_wiederherstellung}
\end{table}

Die Wiederherstellungsfunktion wurde also insgesamt in $8.77\%$ der Fälle ($n=57$) eingesetzt und kam maximal einmal je Anwendung zum Einsatz.  Aus den Videoanalysen ist außerdem erkennbar, dass die Wiederherstellungsfunktion -- falls ihre Verwendung überhaupt in Betracht gezogen wird -- in den meisten Fällen lediglich zur Fehlerkorrektur eingesetzt wird. (in 52 Anwendungen wurde die Wiederherstellungsfunkion in 37 Fällen -- $71.2\%$ -- mindestens einmal zur Korrektur von Erkennungsfehlern und 5 mal zur Korrektur von inhaltlich verworfenen Modellierungswegen verwendet).

Bei der in den Blöcken 1, 4 und 5 durchgeführten Befragung der Teilnehmer hinsichtlich der Erfahrungen mit dem Werkzeug wurde unter anderem nach als besonders nützlich empfundenen Funktionen bzw. Eigenschaften des Werkzeugs gefragt. Die Wiederherstellungsfunktion wurde in diesem Zusammenhang von keinem Teilnehmer ($n=55$) erwähnt. 

\subsubsection{Diskussion} 

Die Ergebnisse der Auswertung der Untersuchung zu dieser Hypothese zeigt ein geringes Ausmaß der Verwendung der Wiederherstellungsfunktion zum Zwecke der Erstellung von Modellalternativen. Die Funktion wurde in $71.2\%$ der Anwendungen verwendet, was für ein hohes Bewusstsein über deren Existenz spricht. Lediglich in $8.77\%$ der Anwendungen wurde die Funktion zur Verfolgung alternativer Modellierungswege eingesetzt, in $61.5\%$ der Anwendungen wurde sie lediglich zur Fehlerkorrektur verwendet. Auch in der qualitativen Erhebung der als nützlich wahrgenommenen Werkzeugfunktionalitäten wurde die Wiederherstellungsfunktion in keinem Fall genannt. Auf Basis dieser Ergebnisse kann die Hypothese nicht bestätigt werden. 

\subsubsection{Ergebnis} 

\textbf{Hypothese \ref{hyp:wiederherstellung} kann auf Basis der Untersuchung nicht bestätigt werden.} Die Wiederherstellungsfunktion wird nur in unter $10\%$ der untersuchten Anwendungen  zur Verfolgung alternativer Modellierungswege genutzt. Die Funktion wird außerdem von den Anwendern bei der Frage nach den als nützlich wahrgenommene Funktionen nicht genannt.

% subsection wiederherstellung_vergangener_modellzustände (end)

\subsection{Nicht-Behinderung} % (fold)
\label{sub:nicht_behinderung}

Gegenstand der hier beschriebenen Untersuchung ist Hypothese \ref{hyp:behinderung} („Das Werkzeug behindert die Modellbildung nicht.“). Als Grundlage dieser Untersuchung dienen die Ergebnisse der Evaluierungsblöcke 2 bis 5, da sich das Werkzeug erst in diesen Blöcken hinsichtlich der Funktionalität in vollständigem Zustand befand. Zu berücksichtigen ist bei der Auswertung, dass im Laufe der Evaluierungsblöcken 4 und 5 eine Überarbeitung der Implementierung vorgenommen wurde, mittels der das Auftreten von Fehlerkennungen verringert werden konnte und deren Korrektur weniger aufwändig wurde. Befragungen der Modellierenden hinsichtlich einer etwaigen Behinderung durch das Werkzeug wurden in den Blöcken 1, 4 und 5 durchgeführt, wobei lediglich die Anmerkungen aus den letzen beiden Blöcken für den aktuellen Entwicklungsstand des Werkzeugs relevant sind.

\subsubsection{Auswertung} 

In Tabelle \ref{tab:fehlfunktionen} wird gegliedert nach Evaluierungsblocken dargestellt, wie oft es in einer einzelnen Anwendung zu Fehlfunktionen in der Erkennung kam, die den Modellierungsfluss unterbrachen. Als Fehlerkennungen wurde das Verschwinden von Blöcken oder Fehlzuordnungen von Benennungen sowie die unbeabsichtigte oder von System eigenständig vorgenommene Erstellung oder Entfernung von Verbindern bzw. Richtungspfeilen eingeordnet. Zusätzlich wurden Systemabstürze als massive Unterbrechung, die zum Gesamtverlust des bis zum Zeitpunkt des Absturzes erstellten Modells führten, separat ausgewertet.

\begin{table}[htbp]
	\centering
	\caption{Fehlfunktionen und Abstürze des Werkzeugs}
\begin{tabular}{| c || c || c | c | c | c || c |}
  \hline
   EB    & Anw. & 0 Ff. & 1-3 Ff. & 4-6 Ff. & 7+ Ff. & Systemabstürze \\ \hline
   2     & 18 & 0 &  8 &  5 &  5 &  4 \\ 
   3     & 18 & 1 & 10 &  4 &  3 &  5 \\ 
   4     & 10 & 0 &  2 &  2 &  5 &  1 \\ 
   5     & 11 & 0 &  3 &  3 &  4 &  5 \\ \hline
   Ges.  & 57 & 1 & 23 & 14 & 17 & 15 \\ \hline
\end{tabular} \\
\footnotesize EB \ldots Evaluierungsblock, Anw. \ldots Anzahl der Anwendungen, x Ff.\ldots x Fehlfunktionen
	\label{tab:fehlfunktionen}
\end{table}

In der Gesamtheit der betrachteten Anwendungen ($n=57$) ergibt sich folgende Verteilung der Anzahl der Fehlerkennungen je Anwendung, die auch in Abbildung \ref{fig:img_Evaluierung_fehlerkennungen} graphisch dargestellt ist. In $1.75\%$ der Fälle ($n_{0}=1$) trat keine Fehlerkennung während der Anwendung auf. In $40.35\%$ der Fälle ($n_{1-3}=23$) traten zwischen 1 und 3 Fehlerkennungen auf. 4-6 Fehlerkennungen konnten in $24.56\%$ der Fälle ($n_{4-6}=14$) festgestellt werden. 7 oder mehr Fehlerkennungen traten in $29.82\%$ der Fälle ($n_{7+}=17$) auf. In $26.32\%$ der Fälle ($n_{Absturz}=15$) kam es zu Systemabstürzen, wobei diese in 10 Fällen nach Ende des eigentlichen Modellierungsvorgangs auftraten.

\begin{figure}[htbp]
	\centering
		\includegraphics[width=10cm]{img/Evaluierung/fehlerkennungen.png}
	\caption{Verteilung der Anzahl der Fehlerkennungen je Anwendung -- Übersicht}
	\label{fig:img_Evaluierung_fehlerkennungen}
\end{figure}

\todo qualitative Daten

\subsubsection{Diskussion} 

\todo Die Daten der quantitativen Auswertung zeigen, dass es in annähernd allen betrachteten Fällen zu zumindest einer Fehlerkennung kam. Es ist davon auszugehen, dass jede Fehlerkennung den Modellierungsfluss unterbricht, da das dann inkorrekte Modelle korrigiert werden muss. Insofern 

\todo Bestätigung und Relativierung durch die Überarbeitung der Interaktion durch die qualitativen Ergebnisse.

Der hohe Anteil von Systemabstürzen ist insofern zu relativieren, als dass diese in zwei Drittel der Fälle nach Abschluss der eigentlichen Modellierungstätigkeit auftraten und somit die Modellerstellung selbst nicht mehr unterbrachen. Abstürze traten durchgängig vor allem in langen Modellierungsdurchgängen etwa ab Minute 40 auf, da ab diesem Zeitpunkt der Speicherbedarf der Historie tendenziell an die Grenzen des verfügbaren Arbeitsspeichers stößt. Alternativ kam es an Tagen mit starker Modellierungstätigkeit ab etwa 5 Stunden durchgängiger Betriebsdauer zu Überhitzungen des Rechners, auf dem die Software ausgeführt wurde, was zum Gesamtabsturz des Betriebssystems führte. Lediglich in 5 Fällen war der Absturz auf fehlerhaftes Programmverhalten (abgesehen von der Speicherproblematik) zurückzuführen. Diese Fälle traten in den Evaluierungsblöcken 2 und 3 auf. Trotzdem sind auch Systemabstürze in der Endphase der Anwendung nach der Modellierung durch den auftretenden Datenverlust nicht akzeptabel und sprechen somit gegen die Annahme der Hypothese.

Insgesamt kann die hier geprüfte Hypothese aus den angeführten Gründen nicht bestätigt werden.

\subsubsection{Ergebnis} 

\textbf{Hypothese \ref{hyp:behinderung} kann auf Basis der Untersuchung nicht bestätigt werden.}


% subsection nicht_behinderung (end)

\subsection{Gewöhnung an das Werkzeug} % (fold)
\label{sub:gewöhnung_an_das_werkzeug}

Gegenstand der hier beschriebenen Untersuchung ist Hypothese \ref{hyp:gewöhnung} („Wiederholte Verwendung des Werkzeugs führt zu schnellerer Modellbildung und weniger Fehlbedienungen.“). Als Grundlage dieser Untersuchung dienen die Ergebnisse des Evaluierungsblocks 2, da in diesem für jede Teilnehmerzusammenstellung jeweils zwei Anwendungen des Werkzeugs durchgeführt wurden.

\subsubsection{Auswertung} 

Zur Auswertung der Modellierungsgeschwindigkeit (hinsichlich des Hypothesenteils „schnellere Modellbildung“) wurde die reine Modellierungszeit jeder Anwendung (ohne Diskussionszeit) mit der jeweiligen Modellgröße normiert. In Tabelle 	\ref{tab:normierte_zeiten} sind die Anwendungszeiten und Modellgrößen sowie die daraus errechneten normierten Werte für beide Anwendungen der Gruppen in Evaluierungsblock 2 angegeben. 

\begin{table}[htbp]
	\centering
	\caption{Modellierungszeiten in Abhängigkeit der Modellgröße in Evaluierungsblock 2}
\begin{tabular}{| c || c | c | c || c | c | c |}
  \hline
   Gruppe    & $t_{1}$ & $n_{1}$ & $t'_{1}$ & $t_{2}$ & $n_{2}$ & $t'_{2}$ \\ \hline
   1     & 620 & 20 & 31.0 & 300 &  6 & 50.0 \\ 
   2     & 450 & 13 & 34.6 & 420 &  7 & 60.0 \\ 
   3     & 240 & 12 & 20.0 & 285 &  6 & 47.5 \\ 
   4     & 215 & 10 & 21.5 & 420 & 13 & 32.3 \\ 
   5     & 577 & 12 & 48.1 & 270 &  7 & 38.6 \\ 
   6     & 339 & 10 & 33.9 & 330 &  8 & 41.3 \\ 
   7     & 348 &  8 & 43.5 & 110 &  5 & 22.0 \\ 
   8     & 855 & 16 & 53.4 & 510 & 12 & 42.5 \\ 
   9     & 735 &  8 & 91.9 & 195 & 12 & 16.3 \\ \hline
\end{tabular} \\
\footnotesize $t_{x}$ \ldots Modellierungsdauer in Sekunden, $n_{x}$ \ldots Anzahl der Elemente, $t'_{1}$ \ldots normierte Modellierungdauer in Sekunden
	\label{tab:normierte_zeiten}
\end{table}

Zusammenfassend ist zwischen der ersten Anwendung (normierte Modellierungsdauer: $M=42.0, SD=21.8, n=9$) und der zweiten Anwendung (normierte Modellierungsdauer: $M=38.9, SD=13.7, n=9$) keine signifikante Verringerung der normierten Modellierungsdauer zu erkennen (einseitiger Wilcoxon-Test für gepaarte Stichproben: $V=21, p=0.590$\footnote{Aufgrund der beiden kleinen Stichproben und der Nicht-Normalverteilung der beiden Stichproben (Shapiro-Wilk-Test 1. Anwendungsdurchgang: $W=0.853, p=0.081$, 2. Anwendungsdurchgang: $W=0.972, p=0.910$) kann der t-Test ($t=0.286, df=8, p=0.391$) trotz der gleichen Varianz der Stichproben (F-Test: $F=2.53, p=0.211$) nicht angewandt werden.}).

Die Anzahl der Fehlbedienungen ist die Anwendungen in beiden Modellierungsdurchgängen in Evalierungsblock 2 in Tabelle \ref{tab:fehlbedienungen} angegeben. Als Fehlbedienungen wurden all jene Interaktionen mit dem Werkzeug eingestuft, in denen die Bedienung nicht dem intendierten Interaktionsdesign folgte. Fehlfunktionen des Werkzeugs wurden nicht berücksichtigt.

\begin{table}[htbp]
	\centering
	\caption{Anzahl der Fehlbedienungen in Evaluierungsblock 2}
\begin{tabular}{| c || c | c |}
  \hline
   Gruppe    & $FB_{1}$ & $FB_{2}$ \\ \hline
   1     & 1 & 0 \\ 
   2     & 4 & 1 \\ 
   3     & 2 & 1 \\ 
   4     & 0 & 0 \\ 
   5     & 0 & 0 \\ 
   6     & 6 & 1 \\ 
   7     & 3 & 1 \\ 
   8     & 6 & 1 \\ 
   9     & 4 & 2 \\ \hline
\end{tabular} \\
\footnotesize $FB_{x}$ \ldots Anzahl der Fehlbedienungen
	\label{tab:fehlbedienungen}
\end{table}

Zusammenfassend konnte hier gezeigt werden, dass die Anzahl der Fehlbedienungen zwischen Anwendung 1 ($M=2.89, SD=2.32, n=9$) und Anwendung 2 ($M=0.78, SD=0.67, n=9$) signifikant geringer geworden ist (einseitiger Wilcoxon-Test für gepaarte Stichproben: $V=28, p=0.0109$\footnote{Aufgrund der beiden kleinen Stichproben und der Nicht-Normalverteilung der ersten Stichprobe (Shapiro-Wilk-Test 1. Anwendungsdurchgang: $W=0.9144, p=0.348$, 2. Anwendungsdurchgang: $W=0.813, p=0.0284$)   sowie der unterschiedlichen Varianz der Stichproben (F-Test: $F=12.06, p=0.00199$) kann der t-Test ($t=3.333, df=8, p=0.00517$) nicht angewandt werden.}).

\subsubsection{Diskussion} 

Eine signifikante Beschleunigung der Modellierungsgeschwindigkeit konnte in obiger Untersuchung nicht festgestellt werden. Die mit der Modellgröße normierte Modellierungszeit verringerte sich zwischen den beiden Anwendungen im Schnitt nur geringfügig. Dieses Ergebnis kann somit nicht als Indikator für die Bestätigung der Hypothese gesehen werden. In den anderen Evaluierungsblöcken (3, 4 und 5) liegt die durchschnittliche normierte Modellierungsdauer in ähnlichen Bereichen wie in den beiden Durchgängen von Evaluierungsblock 2. Bei Anwendung des Werkzeugs durch den Entwickler selbst ist die normierte Modellierungsdauer hingegen auf ungefähr den halben Wert reduziert. Benutzer ohne tiefgehende und mehrfach wiederholte Anwendungserfahrungen scheinen also keinen signifikant messbaren Beschleunigungseffekt bei der Bedienung des Werkzeugs zu erfahren.

Hingegen ist die Anzahl der Fehlbedienungen in den jeweils zweiten Anwendungen des Werkzeugs im Vergleich zur jeweils ersten Anwendung signifikant gesunken. Dies spricht für die Bestätigung der hier geprüften Hypothese. Betrachtet man die Fehlbedienungen detaillierter, so ist ein Großteil der aufgetretenen Fälle sowohl in der ersten als auch in der zweiten Anwendung auf Verständnisschwierigkeiten bei der Bedienung des Löschtokens (zum Zeitpunkt der Evaluierung noch mit dem zustandsbehafteten Interaktionsdesign implementiert, siehe Abschnitt \ref{sub:verwendung_des_löschtokens}) und der Verwendung der Wiederherstellungsfunktion zurückzuführen. Das Interaktionsdesign beider Aspekte wäre also zu hinterfragen (bzw. wurde im Falle des Löschtokens hinterfragt).

Insgesamt kann die hier untersuchte Hypothese nur zum Teil bestätigt werden, da der vermutete Beschleunigungseffekt nicht nachzuweisen war.

\subsubsection{Ergebnis} 

\textbf{Hypothese \ref{hyp:gewöhnung} kann auf Basis der vorliegenden Daten teilweise bestätigt werden.} Während kein signifikanter Beschleunigungseffekt bei wiederholter Verwendung des Werkzeugs gemessen werden konnte, war eine signifikante Verringerung der Anzahl der Fehlbedienungen des Werkzeugs bei wiederholtem Einsatz feststellbar.

% subsection gewöhnung_an_das_werkzeug (end)

\subsection{Herstellung von Verbindern} % (fold)
\label{sub:herstellung_von_verbindern}

Gegenstand der hier beschriebenen Untersuchung ist Hypothese \ref{hyp:verbinder} („Die Einführung der alternativen Möglichkeit zur Verbindungsherstellung erhöht die Nutzung von Verbindern bei der Modellerstellung.“). Zur Untersuchung herangezogen wurden die Werkzeuganwendungen aus Evaluierungsblock 2 ($n=9$). Dieser wurde gewählt, da in diesem Block alle Teilnehmer das Werkzeug zweimal mit der gleichen Aufgabenstellung anwandten, wobei in der ersten Anwendungsrunde lediglich die ursprüngliche Funktionalität zur Herstellung von Verbindern verfügbar war, in der zweiten Runde aber bereits der alternative Funktionalität implementiert war. Zur weiteren Überprüfung der Ergebnisse werden außerdem die Ergebnisse aus Block 3 ($n=17$) herangezogen, bei dessen Durchführung ebenfalls bereits die alternative Funktionalität verfügbar war.

\subsubsection{Auswertung} % (fold)

Grundlage der Auswertung ist das Modellmerkmal „Connectedness“, worunter hier das Verhältnis zwischen der Anzahl der in einem Modell verwendeten Verbindern und den verwendeten Modellelementen verstanden wird. In den einzelnen Evaluierungsblöcken verteilt sich die Connectedness wie in den Abbildungen \ref{fig:img_Evaluierung_connectednessAushandlung1}, \ref{fig:img_Evaluierung_connectednessAushandlung2} und \ref{fig:img_Evaluierung_connectednessConceptMapping} dargestellt.

\begin{figure}[htbp]
	\centering
		\includegraphics[height=2in]{img/Evaluierung/connectednessAushandlung1.png}
	\caption{Connectedness in Evaluierungsblock 2 - Durchgang 1}
	\label{fig:img_Evaluierung_connectednessAushandlung1}
\end{figure}

\begin{figure}[htbp]
	\centering
		\includegraphics[height=2in]{img/Evaluierung/connectednessAushandlung2.png}
	\caption{Connectedness in Evaluierungsblock 2 - Durchgang 2}
	\label{fig:img_Evaluierung_connectednessAushandlung2}
\end{figure}

\begin{figure}[htbp]
	\centering
		\includegraphics[height=2in]{img/Evaluierung/connectednessConceptMapping.png}
	\caption{Connctedness in Evaluierungsblock 3}
	\label{fig:img_Evaluierung_connectednessConceptMapping}
\end{figure}

Zu prüfen ist, ob die Connectedness in jenem Evaluierungs-Blöcken bzw. -Durchgängen, in denen die alternative Funktionalität zur Verbindungs-Herstellung verfügbar war, signifikant höher ist, als in jenen, in denen dies nicht der Fall war. Berechnet wird die Signifikanz zwischen den Ergebnissen der beiden Durchgänge von Block 2 ($conn_{B21}$ und $conn_{B22}$) sowie zwischen den Ergebnissen ersten Durchgang von Block 2 und den Ergebnissen von Block 3 ($conn_{B3}$). Im zweiten Fall ist zu beachten, dass die Aufgabenstellung nicht identisch war und somit eine potentielle Störvariable wirksam wird. Aufgrund der geringen Stichprobengröße kommt zur Prüfung der Signifikanz der t-Test nicht in Frage, es wird der \emph{Wilcoxon-Test} herangezogen. Der t-Test setzt außerdem Normalverteilung der Prüfgrößen voraus, was zumindest bei einer der Verteilungen nicht der Fall ist (Sharpiro-Wilk-Test für $conn_{B22}$: $p=6.29e^{-5}$, damit ist von Nicht-Normalverteilung auszugehen).

Die Null-Hypothese des Wilcoxon-Tests ist, das die beiden Verteilungen identisch verteilt sind. Entsprechend dem erwarteten Ergebnis (dass die mit der alternativen Funktionalität durchgeführten Anwendungen höhere Connectedness aufweisen) wurde die Alternativ-Hypothese so festgelegt, dass sie angenommen wird, wenn die Verteilung des zweiten Blocks gegenüber dem ersten Block nach rechts verschoben (also wertemäßig höher) ist.

Der Wilcoxon-Test für ungepaarte Stichproben ergibt für $conn_{B21}$ und $conn_{B22}$ und der eben beschriebenen Alternativ-Hypothese $p=0.9854$ -- die Alternativ-Hypothese ist damit anzunehmen, die zweite Verteilung (jene mit Einsatz der alternativen Funktionalität der Verbindungsherstellung) weist eine signifikant höhere Connectedness auf als die erste Verteilung (ohne diese Funktionalität). 

Für $conn_{B21}$ und $conn_{B3}$ ergibt der Wilcoxon-Test für ungepaarte Stichproben mit der gleichen Alternativ-Hypothese $p=0.98$ -- auch hier ist die Alternativ-Hypothese anzunehmen.

Für $conn_{B22}$ und $conn_{B3}$ ergibt der Wilcoxon-Test für ungepaarte Stichproben mit der gleichen Alternativ-Hypothese $p=0.7586$ -- auch hier ist die Alternativ-Hypothese anzunehmen.

\subsubsection{Diskussion} % (fold)

Aufgrund der Ergebnisse der berechneten Signifikanztests ist die Hypothese anzunehmen. Mit der Einführung der alternativen Möglichkeit zur Herstellung von Verbindungen war in den einzelnen Anwendungen des Werkzeugs eine Zunahme der Verwendung von Verbindern zu beobachten. Während die Benutzer bei der ursprünglichen Funktion zur Herstellung von Verbindungen zum Großteil auf diese verzichteten (auch bereits in Evaluierungsblock 1), wurden Verbinder unabhängig von der Aufgabenstellung mit der Einführung der alternativen Funktionalität verstärkt eingesetzt.

Die Connectedness eignet sich als Parameter zur vergleichenden Beurteilung des Ausmaßes der Verwendung von Verbindern, da durch die Einbeziehung der Größe des Modells (repräsentiert durch die Anzahl der verwendeten Modellelemente) in die Berechnung den Wert für unterschiedliche Modelle vergleichbar macht. 

Einfluss auf die Höhe der Connectedness hat aber die Aufgabenstellung, die zur Bildung des Modells führt. Unterschiedliche Modellierungsaufgaben führen zu unterschiedlichen Modell-Topologien, die sich wiederum in der Anzahl der verwendeten Verbinder auswirkt. Dies zeigt sich am Ergebnis des Wilcoxon-Tests für $conn_{B22}$ und $conn_{B3}$ -- in beiden Fällen stand die alternative Möglichkeit zur Verbindungsherstellung zur Verfügung $conn_{B3}$ ist trotzdem signifikant höher als $conn_{B22}$. Die kann darin begründet liegen, dass die Concept-Mapping-Aufgabe aus $conn_{B3}$ eher zu stärker verbundenen Modellen führt als eher zur ablauforientierten Modellen führende Arbeitsabstimmungs-Aufgabe aus $conn_{B22}$. Während bei Concept Mapping beliebige Konzepte in Beziehung stehen können, stehen Elemente bei ablauf-orientierten Modellen vor allem mit ihren kausalen Vorgängern und Nachfolgern in Beziehung, was die Anzahl der Verbinder einschränkt.

Aufgrund der großen Rolle der Aufgabenstellung ist bei der Überprüfung der Hypothese wichtig, diese Störvariable möglichst auszuschalten. Zur Beurteilung wird deswegen ausschließlich der Wilcoxon-Test zwischen $conn_{B21}$ und $conn_{B22}$ herangezogen, da in diese beiden Verteilungen mit der gleichen Aufgabenstellung und identischer Stichprobe (jedoch in zeitlichem Abstand von ca. einem Monat) zustande gekommen sind (da die Messungen unabhängig voneinander entstanden, wird ein Wilcoxon-Test für ungepaarte Variablen verwendet). Das Resultat des Wilcoxon-Tests spricht stark für die Annahme der Alternativhypothese des Tests und damit für die Annahme von Hypothese \ref{hyp:verbinder}. Zu berücksichtigen ist hier jedoch die geringe Stichprobengröße, die die Aussagekraft des Ergebnisses wieder in Frage stellt.

\subsubsection{Ergebnis} % (fold)

Die Auswertung zeigt eine signifikant höhere Verwendung von Verbindern bei Verfügbarkeit der alternativen Funktionalität zur Verbindungs-Herstellung. Auch die Natur der Aufgabenstellung scheint hohen Einfluss auf die Verwendung von Verbindern zu haben (siehe dazu auch die Diskussion von Hypothese \ref{hyp:keine_verbinder} in Abschnitt \ref{sub:abbildung_von_zusammenhängen_ohne_verbinder}). \textbf{Hypothese \ref{hyp:verbinder} kann auf Basis der vorliegenden Daten bestätigt werden.}

% subsection herstellung_von_verbindern (end)

\subsection{Verwendung des Löschtokens} % (fold)
\label{sub:verwendung_des_löschtokens}

In diesem Abschnitt werden die Ergebnisse der Überprüfung der Hypothese \ref{hyp:radierer} („Das Löschtoken ermöglicht intuitives Löschen von Modellelementen.“) vorgestellt.

\subsubsection{Auswertung} % (fold)

\begin{transkript}
	\emph{Die Teilnehmer möchten einen Block umbenennen.}\\
	\textbf{A:} Wie haben wir jetzt gesagt \emph{(markiert den roten Baustein)} keine Modellierungsvorgabe \emph{(gibt Bezeichnung ein)}\\
	\emph{System übernimmt die neue Beschriftung für den Baustein nicht.}\\
	\textbf{A:} Wo wurde das hingeschrieben? \emph{(Pause)} Radiergummi? Glaubst du kann man das wegradieren?\\
	\textbf{B:} Probiere es aus.\\
	\textbf{\emph{A legt Radiergummi zum Block mit der Absicht die Beschriftung zu löschen}}\\
	\textbf{B:} Nein! Du löscht alles. Hör auf! \\
	\textbf{A:} Ok wie war das zuerst? Lassen wir das mal weg. \emph{(legt Baustein zur Seite)}\\
	\emph{A legt den Block zur Seite.} 
\end{transkript}

Ein ähnliches Missverständnis zeigt sich auch in folgender Situation:

\begin{transkript}
	\emph{TLN A und B stellen jeweils ihren Marker zu den Blöcken, die verbunden werden sollen. Dabei wird eine gerichtete Verbindung erstellt.}\\
	\textbf{C:} Jetzt haben wir aber einen Pfeil gebastelt.\\
	\textbf{B:} Ja stimmt. Interessant.\\
	\textbf{A:} Wie war das mit dem Radiergummi. \emph{(nimmt Radiergummi und legt ihn auf die Verbindung)}\\
	\textbf{B:} Nein\\
	\textbf{C:} Nein, mit dem Glas! Du löscht alles!\\
	\textbf{A:} Nein nur die Verbindung. \textbf{\emph{(Macht Radierbewegungen auf der Verbindung)}}\\
	\textbf{C:} Ich glaube dass wir das Glas nehmen müssen.\\
	\emph{A schiebt die Blöcke zwischen denen die Verbindung gelöscht werden soll zusammen.}\\
	\textbf{A:} Da es funktioniert. \emph{(schiebt die Blöcke weiter auseinander und bemerkt dass die Verbindung nicht gelöscht wurde)} Nein.\\
	\textbf{B:} Ich glaube der Radiergummi vernichtet alles.\\
	\textbf{A:} Nein der Radiergummi vernichtet nur Verbindungen. Nur welche? \emph{(schiebt beide Blöcke wieder zusammen – nimmt Radiergummi weg und schiebt Blöcke in die Ausgangsposition)}
\end{transkript}

\begin{transkript}
	\emph{Es wird eine falsche Beschriftung eingefügt. Die Teilnehmer wollen diese löschen, verwenden den Radiergummi allerdings falsch.}\\
	\textbf{B:} Aber irgendwie steht jetzt Ereignisse nicht bei dem Ding \emph{(zeigt auf gelben Block)} sondern dort \emph{(zeigt auf beschriftete Verbindung)}.\\
	\emph{A verrückt den gelben Block ein wenig.}\\
	\textbf{B:} Normal ist das nicht oder?\\
	\textbf{C:} Nein.\\
	\emph{A nimmt den Radiergummi.}\\
	\textbf{A:} Ich glaube das. \emph{(setzt den Radiergummi auf die Arbeitsfläche)}\\
	\textbf{C:} Aber nicht alles!\\
	\emph{A nimmt Radiergummi wieder weg. System erstellt eine Verbindung zwischen zwei roten Blöcken. Teilnehmer lachen. \textbf{A legt Radiergummi auf die erstellte Verbindung, und nimmt ihn wieder weg.} A nimmt die beiden verbundenen Blöcke und verschiebt sie.}\\
	\textbf{A:} Vielleicht so. \emph{(führt die Blöcke zusammen)}
\end{transkript}

\begin{transkript}
	\emph{In der Szene erstellt das System einen ungewollten Verbinder, die Teilnehmer versuchen auf verschiedene Arten den Verbinder zu löschen.}\\
	\textbf{B:} Und wie kann ich die Verbindungen löschen?\\
	\textbf{B:} Warte einmal, da gibt es irgendwo das mit dem Radiergummi.\\
	\textbf{A:} murmelt zustimmend \\
	\emph{\textbf{B nimmt den Radiergummi und platziert ihn direkt auf dem Verbinder}}\\
	\emph{Das System färbt den Tisch rot}\\
	\textbf{A:} Nein, warte. Da löscht du Alles!\\
	\emph{\textbf{B verschiebt den Radiergummi auf dem Tisch, hebt ihn an und platziert ihn direkt auf einem Block.}}\\
	\emph{Sobald der Radiergummi von der Oberfläche auf den Block gelegt wurde, entfernt das System die rote Färbung.}\\
	\textbf{A:} Ich glaube da löscht du Alles.\\
	\emph{B legt den Radiergummi an mehreren Stellen trotz der Warnung von TN A auf die Oberfläche}\\
	\textbf{B:} Nein, es will eh nicht.\\
\end{transkript}

\begin{transkript}
	\emph{C versucht die Benennung eines Verbinders mittels Radiergummi zu entfernen.}\\
	\textbf{B:} Aber irgendwie steht jetzt Ereignisse nicht bei dem Ding \emph{(deutet auf einen Block)} sondern dort \emph{(deutet auf einen Verbinder)}. Das wollen wir nicht oder?\\
	\textbf{A:} Nein.\\
	\textbf{C:} Ich glaube das. \emph{\textbf{(nimmt den Radiergummi und legt ihn auf den Verbinder den die Teilnehmer entfernen wollen.)}}\\
	\textbf{A:} Aber nicht alles.\\
	\emph{C entfernt den Radiergummi wieder von der Modellierungsoberfläche. In diesem Moment erstellt das System automatisch einen neuen Verbinder. C versucht den neuen Verbinder mittels Radiergummi zu entfernen.}\\
	\textbf{A:} Oh Gott.\\
	\textbf{C:} Vielleicht so \emph{(schiebt die beiden betroffenen Blöcke zusammen)}, nein.\\
	\textbf{B:} Nein.\\
	\textbf{A:} Oh Gott oh Gott oh Gott.\\
	\textbf{B:} Gehen wir einen Prozessschritt zurück.\\
	\textbf{C:} Genau.\\
\end{transkript}

\begin{transkript}
	\emph{Teilnehmer versuchen mit dem Radiergummi und nur einem anderen Marker einen Verbinder zu entfernen.}\\
	\textbf{B:} Können wir die nicht so auch einfach löschen?\\
	\textbf{C:} Ja mit dem Radiergummi.\\
	\textbf{B:} Muss ich den jetzt zuerst so \emph{(Hält den Radiergummi zur Kamera)} hinhalten?\\
	\textbf{A:} Nein, ich glaube, \textbf{den musst du einfach da \emph{(zeigt auf den Verbinder)} drauf legen.}\\
	\emph{B legt den Radiergummi auf den vom System automatisch erstellten Verbinder.}\\
	\textbf{A:} Und jetzt muss man \emph{(legt ein Markierungtoken auf den Verbinder)} Nein.\\
	\emph{Der Verbinder lässt sich auf diese Art nicht löschen und die Teilnehmer entscheiden sich den Fehler mittels der Wiederherstellungsfunktion zu beseitigen.}
\end{transkript}

\subsubsection{Diskussion} % (fold)

\subsubsection{Ergebnis} % (fold)

% subsection verwendung_des_löschtokens (end)
% section ergebnisse (end)

% chapter eval_tui (end) 