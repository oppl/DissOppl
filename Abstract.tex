\begin{abstract}

Der Erfolg kooperativer Arbeit beruht auf auf einem gemeinsamen Verständnis der betroffenen Abläufe durch die beteiligten Personen. Dieses gemeinsame Verständnis wird durch die ständige und unbewusste Durchführung von Articulation Work sichergestellt. In Situationen, die von den beteiligten als komplex und problematisch wahrgenommen werden, ist implizite, arbeitsbegleitende Durchführung von Articulation Work unter Umständnen nicht mehr ausreichend. Es ist dann notwendig, sich explizit mit der Abstimmung der individuellen Sichtweisen und der Bildung eines gemeinsamen Verständnisses zu beschäftigen.

Um explizite Articulation Work zu unterstützen, wird in dieser Arbeit versucht, die Interaktion der Beteiligten durch die koopertive Bildung und Diskussion diagrammatischer Modelle zu ermöglichen bzw. zu erleichtern. Dieser Zugang ist aus der Theorie der Bildung und Veränderung mentaler Modelle abgeleitet. Die Externalisierung der mentalen Modelle in Form von diagrammatischen Modellen wird dort als adäquates Mittel zur Refelexion und Kommunikation derselben identifiziert. Methodisch baut die Arbeit dabei auf Sturkturlegetechniken und Concept Mapping auf. Die dort vorgeschlagenen Methoden und Anforderungen an einer Werkzeugunterstützung werden unter Bezugnahme auf Articulation Work zusammengeführt. Die resultierende Methodik wird durch ein Tabletop Interface -- eine horizontale Interaktionsoberfläche mit rechnerbasierten Unterstützungsfunktionen -- unterstützt.

Das Werkzeug selbst wird hinsichtlich seiner Umsetzung in Hard- und Software beschrieben und einer empirischen Untersuchung unterzogen. Dabei wird die Verwendbarkeit des Werkzeugs selbst, dessen Nutzen bei der Abstimmung mentaler Modelle sowie letztendlich die Auswirkungen bei der Durchführung von Articulation Work untersucht. Die Ergebnisse deuten darauf hin, dass das Werkzeug den Anforderungen genügt und sowohl bei der Abstimmung mentaler Modelle als auch zum Teil im Kontext der durchgeführten Articulation Work zu den intendierten Wirkungen führt.
\end{abstract}