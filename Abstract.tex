 %\begin{abstract}

\section*{Kurzfassung}
Der Erfolg von (kooperativer) Arbeit beruht auf einem gemeinsamen Verständnis der betroffenen Abläufe durch die beteiligten Personen. Dieses gemeinsame Verständnis wird der Theorie von Strauss zufolge durch die ständige und unbewusste Durchführung von Tätigkeiten zur Abstimmung mit anderen Individuen erreicht. Beim Auftreten von Situationen, die von den Beteiligten als komplex und problematisch wahrgenommen werden, müssen nach Strauss bewusst dezidierte Aktivitäten der Abstimmung und zum Erreichen einer gemeinsamen Sichtweise durchgeführt werden. Sowohl die Identifikation der Notwendigkeit von Abstimmungsaktivitäten als auch deren Durchführung werden maßgeblich von den individuellen Wahrnehmungen der beteiligten Personen beeinflusst. Auf diesen Aspekt geht Strauss nicht ein, so dass auch Arbeiten, die sich bei der Entwicklung von Instrumenten der Unterstützung der Abstimmung auf dessen Arbeiten beziehen, die individuelle Dimension nicht explizit berücksichtigen. Wird diese individuelle Dimension ignoriert, so hat dies negative Auswirkungen auf das Arbeitsergebnis. Ziel dieser Arbeit ist deshalb die Unterstützung der Abstimmungsprozesse über Arbeitsabläufe unter expliziter Berücksichtigung der Bedürfnisse der beteiligten Individuen. Zu diesem Zweck werden Methoden aus der Theorie der mentalen Modelle nach Johnson-Laird mit den Anforderungen aus der Abstimmung von Arbeitsabläufen zusammengeführt.

Um die Abstimmung zu unterstützen, setzt der hier vorgestellte Ansatz die kooperative Bildung und Diskussion diagrammatischer Modelle ein. Dieser Zugang ist aus der Theorie der Bildung und Veränderung mentaler Modelle abgeleitet. Die Externalisierung der mentalen Modelle in Form von diagrammatischen Modellen ist nach Seel ein Weg zur Reflexion und Kommunikation derselben und ermöglicht so die Entwicklung einer gemeinsamen Sichtweise auf den kooperativen Arbeitsablauf. Methodisch baut die Arbeit auf Strukturlegetechniken und Concept Mapping auf, welche sich zur Externalisierung mentaler Modelle eignen. Die dort vorgeschlagenen Methoden werden unter Bezugnahme auf die Abstimmung von individuellen Sichtweisen auf Arbeitsabläufe zusammengeführt. Wesentlich für die kooperative Anwendung ist deren Durchführung auf einer durch mehrere Personen unmittelbar und gleichzeitig manipulierbaren Modellierungsoberfläche. Die entwickelte Methodik wird deshalb durch ein Tabletop Interface -- eine horizontale Interaktionsoberfläche mit rechnerbasierten Unterstützungsfunktionen -- zu einem Instrument ergänzt, mit dem die Durchführung von Abstimmungsaktivitäten unterstützt werden kann. 

Das Tabletop Interface ermöglicht die kooperative Bildung von Modellen mittels physischen Bausteinen, die auf der Interaktionsoberfläche platziert werden. Das Modell kann so unmittelbar und simultan von mehreren Personen erfasst und manipuliert werden. Technologisch basiert das System auf der Identifikation der Bausteine mittels Markern, die durch eine Kamera in Echtzeit erfasst werden. Die erfasste Information wird durch das System interpretiert, so dass Aktivitäten zur Modellbildung identifiziert werden können. Die Darstellung von Information zum erstellten Modell erfolgt durch Rückprojektion auf die Interaktionsoberfläche und einen Bildschirm, der als erweiterter Ausgabekanal für nicht auf der Oberfläche darstellbare Information dient. Durch zusätzliche Rechnerunterstützung werden kooperationsunterstützende Maßnahmen wie die Wiederherstellung vergangener Modellzustände ermöglicht. Die persistente Ablage der erstellten Modelle erfolgt als Topic Map, einem standardisierten Datenformat zur flexiblen Repräsentation semantischer Netze, das eine Wieder- und Weiterverwendbarkeit der erstellten Modelle gewährleistet.

Die Effektivität der Unterstützung von Abstimmungsaktivitäten durch das System wird im Rahmen einer empirischen Untersuchung untersucht. Dabei wird die Verwendbarkeit des interaktiven Systems selbst, dessen Nutzen bei der Abstimmung mentaler Modelle sowie letztendlich die Auswirkungen bei der Durchführung von Abstimmungsaktivitäten in Arbeitsprozessen untersucht. Die Ergebnisse zeigen, dass das Werkzeug verständlich und benutzbar ist und das Instrument in seiner Gesamtheit sowohl positive Wirkungen auf die Kooperation zwischen den beteiligten Personen hat als auch die Bildung einer gemeinsamen Sichtweise auf den betrachteten Arbeitsablauf hat.
%\end{abstract}

\newpage
\textcolor{white}
.
\textcolor{black}
\newpage
 
\cleardoublepage

\section*{Abstract}
Successful (cooperative) work requires that the involved workers develop a common understanding of the modalities of their interaction. According to Strauss, common understanding emerges from continuously and unconsciously conducted activities for alignment of understanding. In situation perceived to be complex or problematic by the involved persons, Strauss suggests that alignment activities have to triggered and conducted deliberately. Individual perceptions affect both, the identification of the need for alignment and alignment itself. Strauss does not explicitly address this aspect in his theory. Approaches that support alignment based upon Strauss' work thus also largely ignore the individual, cognitive dimension of alignment. Ignoring the individual dimension, however, has negative impact on the success of work processes. Accordingly, this work aims at extending the scope of alignment support by explicitly considering the perceptions and needs of individuals. The theory of mental models here is used to extend Strauss' concepts and develop effective support for developing a common understanding of work processes.

Following the theory of mental model development by Seel, the cooperative creation of diagrammatic models as representations of mental models can aid their alignment and the development of a common understanding. Suitable methods for building representations of mental models include structure elaboration techniques and concept mapping. Both methods have properties that are support the cooperative creation of models. In this work, they are integrated to form a method that is useable in the context of the alignment of cooperative work. The main feature for cooperation support is that modeling takes places on a simultaneously accessible and physically manipulable modeling surface. The method thus is complemented with a tabletop interface -- a horizontally mounted interaction surface that is augmented with computer support -- to effectively support the alignment of individual views on cooperative work processes.

Tangible tokens are used to cooperatively build models on the interaction surface. By physically placing the tokens, the model can be manipulated simultaneously by several people. Token identification is based on visual markers that are tracked by a camera in real time. The gathered information is interpreted by the system to identify modeling activities. Model information is displayed by back-projecting it onto the surface from underneath. An traditional screen is provided as an additional output channel for information  that cannot be displayed directly on the interaction surface. Cooperation is further supported by additional features like reconstruction support for former model states. Persistent model representation is based upon the standardized XML Topic Map format, which allows for a reusable, self-contained representation of generic semantic networks.

The systems's effectiveness in supporting the alignment of work is tested in an empirical study. In three steps, the system's usability, its effects on the alignment of mental models and the effectiveness in supporting the development of a common understanding of work processes are examined. The results of the study show that the system is comprehensible and useable. Positive effects on both, the cooperation among people during modeling and the alignment of individual views of cooperative work, have been observed.