\chapter{Ausgabe} % (fold)
\label{cha:visualisierung}

In diesem Kapitel wird die konzeptuelle Ausrichtung und technische Umsetzung jenes Teils des Werkzeugs behandelt, der sich mit der Ausgabe von Information an die Benutzer beschäftigt. Im Bereich der Tangible Interface erfolgt die Ausgabe von Information zumeist kohärent mit dem Eingabemedium, eine physische Trennung zwischen Eingabe- und Ausgabekanälen wie in der herkömmlichen Mensch-Maschine-Interaktion liegt nicht vor \citep{Ullmer00}. \citet{Fishkin04} relativiert die strikte Forderung in seiner Taxonomie für Tangible Interfaces (wie in Abschnitt \ref{sub:tangibles_taxonomien} beschrieben und klassifiziert Benutzungsschnittstellen unter anderem nach dem Grad deren Ein- und Ausgabe-Kohärenz. Dementsprechend sind nicht nur jene Ausgabekanäle Gegenstand dieses Kapitels, die Information in direkter Verbindung mit den Eingabemedien zurückspiegeln, sondern auch jene, die Information auf anderen, nicht-kohärenten Wegen ausgeben.

Im ersten Abschnitt dieses Kapitels werden auf Basis der in Kapitel XY genannten Anforderung an das Werkzeug die den Benutzern mitzuteilenden Informationen identifiziert, noch ohne konkret auf die technologische Realisierung der Ausgabekanäle einzugehen. Im darauf folgenden Abschnitt die technologischen Möglichkeiten zur Ausgabe von Information betrachtet und im Anschluss hinsichtlich ihrer Eignung für die im konkreten Anwendungsfall auszugebende Information bewertet und entsprechend zugeordnet.

Im Anschluss werden auf Basis dieser grundsätzlichen Technologieentscheidung Software-Frameworks beschrieben, die die Realisierung der gewählten Ausgabekanäle ermöglichen. Die Entscheidung für ein konkretes Framework wird auf Basis der funktionalen und nicht-funktionalen Anforderungen an die Ausgabe und deren Umsetzung getroffen. Der letzte Abschnitt beschreibt die eigentliche Umsetzung der Ausgabekanäle mittels der gewählten Technologie und geht die spezifischen Eigenschaften und Implementierungsentscheidungen der vorgestellten Lösung ein.

\section{Auszugebende Information} % (fold)
\label{sec:auszugebende_information}

Welche Information ist auszugeben?

% section auszugebende_information (end)

\section{Technologische Grundlage der Ausgabe} % (fold)
\label{sec:technologische_grundlage_der_visualisierung}

\subsection{Ansätze zur kohärenten Ausgabe} % (fold)
\label{sub:kohärente_ausgabe}
Projektion

\subsubsection{Mögliche Ansätze} % (fold)
\label{ssub:mögliche_ansätze}

% subsubsection mögliche_ansätze (end)

\paragraph{Projektion} % (fold)
\label{par:projektion}

% paragraph projektion (end)

\paragraph{Aktive Anzeige auf Tokens} % (fold)
\label{par:aktive_anzeige_auf_tokens}

% paragraph aktive_anzeige_auf_tokens (end)

\paragraph{Aktuatoren} % (fold)
\label{par:aktuatoren}

% paragraph aktuatoren (end)

\subsubsection{Entscheidung} % (fold)
\label{ssub:output_ansatz_entscheidung}

% subsubsection output_ansatz_entscheidung (end)
% subsection kohärente_ausgabe (end)

\subsection{Frameworks zur Ausgabe} % (fold)
\label{sub:frameworks_zur_ausgabe}

\subsubsection{In Frage kommende Frameworks} % (fold)
\label{ssub:in_frage_kommende_frameworks}

--> JHotDraw + evtl. andere Frameworks aus KnowIT-Diplomarbeit

\paragraph{JHotDraw} % (fold)
\label{par:jhotdraw}

% paragraph jhotdraw (end)

\subsubsection{Framework-Entscheidung} % (fold)
\label{ssub:output:framework_entscheidung}

% subsubsection output_framework_entscheidung (end)
% subsubsection in_frage_kommende_frameworks (end)

% subsection subsection_name (end)
% subsubsection jhotdraw (end)

% subsection frameworks_zur_ausgabe (end)

% section technologische_grundlage_der_visualisierung (end)

\section{Ausgabe von Information} % (fold)
\label{sec:ausgabe_von_information}

\subsection{Konzept} % (fold)
\label{sub:ausgabe_konzept}

Dual View
% subsection ausgabe_konzept (end)

\subsection{Architekur} % (fold)
\label{sub:architekur}

Modularchitektur beeinflusst von den Vorgaben der Entwurfsmuster in JHotDraw
% subsection architekur (end)

\subsection{Ausgabe von Information zum Modell} % (fold)
\label{sub:ausgabe_von_information_zum_modell}

\subsubsection{Information zur Modellelemeenten} % (fold)
\label{ssub:information_zur_modellelemeenten}

% subsubsection information_zur_modellelemeenten (end)

\subsubsection{Information zu Verbindern} % (fold)
\label{ssub:information_zu_verbindern}

% subsubsection information_zu_verbindern (end)
% subsection ausgabe_von_information_zum_modell (end)

\subsection{Ausgabe zur Kontrolle des Systems} % (fold)
\label{sub:ausgabe_zur_kontrolle_des_systems}

\subsubsection{Zustandsmeldungen} % (fold)
\label{ssub:zustandsmeldungen}
Löschmodus etc.

% subsubsection zustandsmeldungen (end)

\subsubsection{Wiederherstellungsunterstützung} % (fold)
\label{ssub:wiederherstellungsunterstützung}

% subsubsection wiederherstellungsunterstützung (end)
% subsection ausgabe_zur_kontrolle_des_systems (end)

% section ausgabe_von_information (end)

\section{Umsetzung der Ausgabe mit Software} % (fold)
\label{sec:umsetzung_der_ausgabe_mit_software}

\subsection{Ausgabe des Modellzustands} % (fold)
\label{sub:einsatz_von_jhotdraw}

Bildschirm und Projektor

\subsubsection{Kalibrierung der Ausgabe} % (fold)
\label{ssub:kalibrierung_der_ausgabe}

% subsubsection kalibrierung_der_ausgabe (end)

\subsection{Weitere Ausgabekanäle} % (fold)
\label{sub:weitere_ausgabekanäle}

Verteilter Viewer

% subsection weitere_ausgabekanäle (end)
% section umsetzung_der_ausgabe_mit_software (end)

% chapter visualisierung (end)