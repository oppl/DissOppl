\chapter{Ausgabe} % (fold)
\label{cha:visualisierung}

In diesem Kapitel wird die konzeptuelle Ausrichtung und technische Umsetzung jenes Teils des Werkzeugs behandelt, der sich mit der Ausgabe von Information an die Benutzer beschäftigt. Im Bereich der Tangible Interface erfolgt die Ausgabe von Information zumeist kohärent mit dem Eingabemedium, eine physische Trennung zwischen Eingabe- und Ausgabekanälen wie in der herkömmlichen Mensch-Maschine-Interaktion liegt nicht vor \citep{Ullmer00}. \citet{Fishkin04} relativiert die strikte Forderung in seiner Taxonomie für Tangible Interfaces (wie in Abschnitt \ref{sub:tangibles_taxonomien} beschrieben und klassifiziert Benutzungsschnittstellen unter anderem nach dem Grad deren Ein- und Ausgabe-Kohärenz. Dementsprechend sind nicht nur jene Ausgabekanäle Gegenstand dieses Kapitels, die Information in direkter Verbindung mit den Eingabemedien zurückspiegeln, sondern auch jene, die Information auf anderen, nicht-kohärenten Wegen ausgeben.

Im ersten Abschnitt dieses Kapitels werden auf Basis der in Kapitel XY genannten Anforderung an das Werkzeug die den Benutzern mitzuteilenden Informationen identifiziert, noch ohne konkret auf die technologische Realisierung der Ausgabekanäle einzugehen. Im darauf folgenden Abschnitt die technologischen Möglichkeiten zur Ausgabe von Information betrachtet und im Anschluss hinsichtlich ihrer Eignung für die im konkreten Anwendungsfall auszugebende Information bewertet und entsprechend zugeordnet.

Im Anschluss werden auf Basis dieser grundsätzlichen Technologieentscheidung Software-Frameworks beschrieben, die die Realisierung der gewählten Ausgabekanäle ermöglichen. Die Entscheidung für ein konkretes Framework wird auf Basis der funktionalen und nicht-funktionalen Anforderungen an die Ausgabe und deren Umsetzung getroffen. Der letzte Abschnitt beschreibt die eigentliche Umsetzung der Ausgabekanäle mittels der gewählten Technologie und geht die spezifischen Eigenschaften und Implementierungsentscheidungen der vorgestellten Lösung ein.

\section{Auszugebende Information} % (fold)
\label{sec:auszugebende_information}

Den Benutzern des Systems müssen während der Modellierung unterschiedliche Information zur Verfügung gestellt werden. Einerseits ist dies Information, die das Modell selbst betrifft, andererseits muss auch Information ausgegeben werden, die Aspekte des Modellierungsablaufs beschreibt oder unterstützt.

Die das Modell betreffende Information muss folgende Aspekte abdecken:
\begin{itemize}
 \item Die Modellelemente betreffende Information (Art, Position, Benennung)
 \item Die Verbinder betreffende Information (Art, Endpunkte, Benennung)
 \item Einbettete Elemente betreffene Information (Art, Inhalt, Container)
\end{itemize}

Zur Unterstützung des Modellierungsablaufs müssen folgende Aspekte zur Verfügung gestellt werden:
\begin{itemize}
 \item Information über vergangene Modellzustände
 \item Information zur die Wiederherstellung von Modellzuständen
\end{itemize}

Hier wird bewusst noch nicht auf die technische Umsetzung dieser Ausgabe eingegangen. In den folgenen Abschnitten wird erörtert, welche grundlegenden Technologien in Frage kommen, bevor auf Basis deren Eignung und den Vorgaben aus den Technologieentscheidungen zur Informationseingabe eine konkrete Lösung ausgewählt wird.

% section auszugebende_information (end)

\section{Technologische Grundlage der Ausgabe} % (fold)
\label{sec:technologische_grundlage_der_visualisierung}

Bei der Ausgabe von Information muss im Falle von Tangible Interfaces zwischen Ansätzen mit unterschiedlich stark ausgeprägter Kohärenz mit den Eingabekanälen unterschieden werden. Unter Kohärenz ist hier zu verstehen, das jene Artefakte, die zur Eingabe verwendet werden gleichzeitig auch die Reaktion des Systems -- also die Ausgabe -- wiederspiegeln. In der von \citet{Fishkin04} vorgeschlagenen Taxonomie (siehe Abschnitt XY) werden in der Dimension "Embodiment" auch Werkzeuge als Tangible Interfaces klassifiziert, bei denen die Ausgabe vollständig von der Eingabe entkoppelt ist. Diesem Verständnis folgt auch diese Arbeit.

Bei Tabletop Interfaces bietet sich die Tischoberfläche als Ausgabemedium an, um kohärente Informationsausgabe zu gewährleisten. Die Tischoberfläche dient hier wie in Kapitel \ref{cha:input_&_interpretation} beschrieben der Eingabe und kann durch unterschiedliche technologische Maßnahme auch zur Ausgabe genutzt werden. Eine weitere Möglichkeit, die Ausgabekohärenz bei Tabletop Interfaces sicherzustellen bzw. zu steigern, ist die Verwendung der zur Interaktion mit dem System verwendeten Tokens als Ausgabemedium. Je nach verfolgtem Ansatz (bzw. einer Kombination) sind unterschiedliche technische Maßnahmen zu setzen. In den folgenden Abschnitten werden die hier erwähnten grundsätzlich in Frage kommenden Ansätze betrachtet und im Anschluss hinsichtlich ihrer Eignung für das hier entwickelte System beurteilt. Basierend auf der grundsätzlichen Technologieentscheidung werden im Anschluss unterschiedliche technische Lösungen zur Erfüllung der Anforderungen beschrieben.

\subsection{Ansätze zur kohärenten Ausgabe} % (fold)
\label{sub:kohärente_ausgabe}

In diesem Abschnitt werden Ausgabeansätze behandelt, die nach \citep{Fishkin04} in der Embodiment-Dimension den Ausprägungen "full" oder "nearby" zuzuordnen sind. Die Ausgabe erfolgt bei den hier vorgestellten Ansätzen also direkt über die Eingabetokens ("full") oder ist räumlich unmittelbar in der Umgebung der Tokens angesiedelt.

\subsubsection{Darstellung auf der Tischoberfläche} % (fold)
\label{ssub:darstellung_auf_der_tischoberfläche}

Bei Tabletop-Interfaces ist die Nutzung der Tischoberfläche ein naheliegender und gängiger Ansatz zur Realisierung der Ausgabekanäle. Die Ausgabe erfolgt hierbei visuell, also durch die Darstellung der auszugebenden Information. Ein derartig ausgestaltetes Interface ist hinsichtlich seiner Ausprägung in der Embodiment-Dimension als "nearby" zu klassifizieren. Technologisch kommen zur Darstellung horizontal eingesetzte Bildschirme oder Oberflächen, auf die projiziert werden kann, in Frage.

Bei der Verwendung von Bildschirmen sind die Größe der zur Anzeige verwendbaren Oberfläche sowie die zur Anzeige verfügbare Auflösung (also indirekt die Größe eines Bildpunktes) wesentliche Kriterien. Bei heute verfügbaren LCD-Modulen mit Größen bis zu 132 cm in der Diagonale und Auflösungen von 1920 x 1080 Bildpunkten ist die Technologie soweit ausgereift und verfügbar, das dieser Ansatz grundsätzlich für den Einsatz in Tabletop Interfaces in Frage kommt. Vorteile sind die geringe Bauhöhe der Ausgabeeinheit (im Vergleich zu den im Folgenden vorgestellten Projektions-Lösungen). Nachteile sind die relative geringe Leuchtstärke, die einen Einsatz bei Tageslichtbedingungen schwierig machen sowie die Blickwinkelabhängigkeit, die bei horizontalem Einbau der Anzeigeeinheit stärker zum Tragen kommt als bei herkömmlicher vertikaler Verwendung.

Als Alternative zur Verwendung von aktiven Anzeigeeinheiten können Projektoren verwendet werden, die die darzustellende Information auf die Tischoberfläche projizieren. Hier ist zwischen Lösungen zu unterscheiden, bei denen die Projektion von oben erfolgt und jenen, die von unten auf eine durchscheinende Tischoberfläche projizieren. Bei ersteren muss der Projektor in ausreichender Höhe über der Tischoberfläche angebracht werden, um das projizierte Bild die notwendige Fläche abdecken zu lassen. Bei beschränkter Höhe nach oben kann ein Projektor mit Weitwinkelobjektiv oder ein Umlenkspiegel benutzt werden, der durch eine Vergrößerung des Abstands zwischen Projektor und Oberfläche auf einer nicht vertikalen Achse die notwendige Bauhöhe reduziert. Der größte Nachteil dieser Lösung ist die Abschattung der projizierten Information bei Manpulationen der Tokens auf der Oberfläche. Um dies zu vermeiden kann auch von unten auf eine durchscheinende Oberfläche projiziert werden. Bei diesem Lösungsansatz kommt es zu keinerlei Abschattungen, die Information wird wie bei Einsatz eines Bildschirms ständig angezeigt. Nachteilig wirkt sich hier der durch den Einsatz einer durchscheinenden Oberfläche verursachte Leuchtkraftverlust aus. Bei dieser Form der Projektion wird immer ein Teil des durch den Projektor ausgestrahlten Lichts von der Oberfläche nach unten zurück reflektiert. Im Gegensatz zur Projektion von oben ist dadurch der Kontrast der Darstellung wesentlich geringer. Für den Einsatz unter Tageslichtbedingungen erscheint also der erstgenannte Ansatz generell besser geeignet. Kritischer ist bei der Projektion von unten der notwendige Abstand zwischen Projektor und Tisch, da sich dieser direkt auf die Höhe des Tisches auswirkt. Um eine akzeptable Bauhöhe zu erreichen -- also den Tisch durch durchschnittliche große Personen bedienbar zu halten (Höhe nicht mehr als etwa 100 cm) -- ist hier der Einsatz eines Umlenkspiegels nahezu unabdingbar. Beiden Projektions-Ansätzen gleich ist, dass die abzudeckende Oberfläche variabel durch den Abstand des Projektors gewählt werden kann. Grenzen sind hier nach unten die Fokussierbarkeit des Projektors bei kleinen Abständen und nach oben die Abnahme der Projektionshelligkeit bei großen Abständen. Bei großen Oberflächen ist zudem auf die verfügbare Auflösung des Projektors zu achten, da die Größe eines Bildpunktes mit zunehmendem Abstand so ansteigt, dass eine feinauflösende Projektion der Information auf der Oberfläche nicht mehr möglich ist.

% subsubsection darstellung_auf_der_tischoberfläche (end)

\subsubsection{Aktive Anzeige auf Tokens} % (fold)
\label{ssub:aktive_anzeige_auf_tokens}

Alternativ zur Darstellung auf der Tischoberfläche können bei Tabletop-Interfaces auch die Tokens selbst als Ausgabekanal dienen. Die hier das Eingabemedium gleich dem Ausgabemedium ist, ist diese Form der Ausgabe hinsichlich der Embodiment-Dimension in die Ausprägung "full" einzuordnen. Technologisch können je nach Art der darzustellenden Information Tokens mit Displays ausgestattet werden oder lediglich visuelle Statusanzeigen beinhalten, die Feedback über den aktuellen Zustand des Tokens geben. In beiden Fällen müssen die Tokens generell mit Elektronik und Energieversorgung ausgestattet sein und die Möglichkeit haben, selbst oder über eine Verbindung mit der Infrastruktur ihren Zustand festzustellen.

Um textuelle oder grafische Information auf Tokens darzustellen ist die Verwendung von Displays notwendig. Bei der Verwendung von herkömmlichen LCD-Displays ist durch die notwendige Hintergrundbeleuchtung sowie der notwendigen Stromversorgung zur Aufrechterhaltung der Anzeige der Energieverbrauch verhältnismäßig hoch. Alternativ können neuere Technologien wie OLED- (REF) oder eInk-Displays (REF) verwendet werden. OLEDs bestehen aus organischen Materialien und benötigen keine Hintergrundbeleuchtung, das das Material selbst Licht emmitiert. eInk verwendet eine papierartige Oberfläche zur Anzeige, strahlt selbst kein Licht aus und ist deshalb auf Umgebungshelligkeit zur Verwendung angewiesen. eInk bietet in hellen Umgebungen die besten Kontrastverhältnisse (vergleichbar mit bedrucktem Papier) kann allerdings beim heutigen Stand der Entwickung keine Farben darstellen. Der größte Vorteil von eInk liegt in der Eigenschaft, dass die Anzeige auch ohne Energieversorgung aufrecht bleibt -- Energie ist lediglich zur Änderung des Display-Inhalts notwendig.

Durch das Wegfallen der Hintergrundbeleuchtung sind OLED- und eInk-Displays wesentlich dünner als LCD-Module und können auch auf nicht ebenen Oberflächen angebracht werden. Allen drei Ansätzen gleich ist, dass zur Ansteuerung des Anzeigemoduls Elektronik notwendig ist, die die darzustellende Information auf die zur Verfügung stehen Bildpunkte abbildet und das Display entsprechend ansteuert.

Neben der Verwendung eines Displays kann der Status eines Tokens auch mit Leuchtanzeigen in der Form von LEDs visualisiert werden. Der Nachteil dieses Ansatzes ist die schwierige Realisierbarkeit von komplexen Statusanzeigen - durch die auf zwei Zustände (ein/aus) beschränkte Aussagekraft einer LED sind andere als bipolare Visualisierungen schwer zu realisieren. Möglich ist die Verwendung von mehreren LEDs, wobei diese rasch schwer erfassbar wird, wenn dadurch mehrere voneinander unabhängige Aussagen visualisiert werden. Lediglich die Kopplung mehrere LEDs zur aussagekräftigen Visualisierung von dynamischen Zuständen hat sich als intuitiv erfassbar und verständlich erwiesen (REF Zuckerman Flow Blocks). So können gekopplete LEDs z.B. dazu verwendet werden, Flussrichtungen von Ressourcenströmen anzuzeigen, indem eine LED-Reihe entwender von links nach rechts oder von rechts nach links angesteuert wird. Auch beim Einsatz von LEDs ist einer ständige Energieversorgung notwendig. Zur Reduktion des Energieverbrauchs können wiederum die oben genannten Alternativ-Technologien OLED und eInk zum Einsatz kommen, wobei diese aktuell nicht in den Bauformen herkömmlicher LEDs angeboten werden. eInk-Anzeigen eignen sich aufgrund ihrer langsamen Schaltdauer außerdem nicht für die Realisierung dynamischer Anzeigen.

% subsubsection aktive_anzeige_auf_tokens (end)

\subsubsection{Token mit Aktuatoren} % (fold)
\label{ssub:tokens_mit_aktuatoren}

Neben der Verwendung von Displays zur Realisierung eines "full embodied" Ausgabekanals können Tokens auch mit Aktuatoren ausgestattet werden, die es erlauben, das Token bzw. dessen Verhalten selbst ohne direkte Benutzerinteraktion zu beeinflussen. Beispiele für Aktuatoren sind unter anderem Vibrationsmodule oder mechanische Einheiten zur Veränderung der äußeren Form des Tokens oder dessen Position (auf der Tischoberfläche). Der Einsatz von Aktuatoren ist ob der technologischen und anwendungsspezifischen Vielfalt nicht generisch beschreibbar wie das bei reinen optischen Anzeigeeinheiten der Fall war. Aktuatoren müssen auf den jeweiligen Anwendungsfall abgestimmt sein. Ihr Einsatz ist im Allgemeinen eher disruptiv, unterbricht durch die vom Benutzer nicht selbst ausgelöste Interaktion dessen aktuelle Aktivität und zieht die Aufmerksamkeit auf sich. Dies kann im einzelnen Anwendungsfall erwünscht sein, kann aber zu unerwünschten Effekten bei der Verwendbarkeit des Systems führen.

Wie bei Anzeigemodulen müssen auch hier die Tokens mit Energieversorgung ausgestattet sein und mit Elektronik integriert werden, die für die Ansteuerung der Aktuatoren sorgt.

Im Bereich der Tabletop Interfaces ist die Verwendung von Aktuatoren eher selten anzutreffen. Im Bereich der Ambient Interfaces kann der Einsatz von Aktuatoren aber sinnvoll sein, wenn sich das Interface so in die Umgebung seines Einsatzbereichs integrieren kann (REF Bsp ... Ferscha Blubbersäule?).

% subsubsection tokens_mit_aktuatoren (end)
% subsection kohärente_ausgabe (end)

\subsection{Ansätze zur entkoppelten Ausgabe} % (fold)
\label{sub:entkoppelte_ausgabe}

Ansätze zur enkoppelten Ausgabe sind solche, die von \citet{Fishkin04} in seiner Taxonomie in der Embodiment-Dimension unter den Ausprägungen "environment" oder "distant" eingeordnet werden. 

"Environmental Embodiment" ist dann gegeben, wenn Eingabe- und Ausgabekanäle räumlich nicht kohärent sind, aber Eingaben trotzdem offensichtlich Reaktionen in der unmittelbnaren Umgebung auslösen. Klassische von Fishkin genannte Ansätze sind hier Audiokanäle aber auch die Veränderung von Umgebungslicht oder Temperatur. Auch olfaktorische Interfaces wären in diese Kategorie einzuordnen. In den folgenden Abschnitten werden jedoch ausschließlich audio-basierte Kanäle beschrieben, da diese im Bereich der Tabletop Interfaces Relevanz besitzen (etwa in REF reacTable und REF TEI Paper von Däne)

Die Ausprägung "distant" kennzeichnet Ansätze, in denen die Eingabekanäle räumlich von den Ausgabekanälen vollkommen enkoppelt sind bzw. entkoppelt werden können. Ein Kriterium zur Einordung eines Ansatzes unter "distant" ist, dass Benutzer zur Beobachtung der Ausgabe nicht mehr die Eingabe im Blickfeld haben können (was bei allen anderen Ausprägungen möglich ist). Klassische Vertreter dieses Ansatzes sind alle Ansätze die auf der Darstellung von Information auf herkömmlichen Bildschirmen oder Projektionsflächen basieren.

\subsubsection{Darstellung auf Monitoren} % (fold)
\label{ssub:tokens_mit_aktuatoren}

Bei der Darstellung von Information auf Bildschirmen kommen die auch in der herkömmlichen Desktop-basierten Mensch-Maschine-Interaktion gängigen Anzeigetechnologien zur Anwendung. Im Kontext von Tabletop Interfaces ist bei Monitoren auf die Sichtbarkeit der Information für alle an der Interaktion beteiligten Personen zu achten -- diese ist nicht nur von der Entfernung zum Monitor abhängig sondern bei den heute gängigen LCD-Displays auch vom Blickwinkel. Gegebenenfalls müssen mehrere Monitore verwendet werden, auf denen entweder simultan die gleiche  Information dargestellt wird oder -- abhängig von der Anwendung -- lediglich für die jeweils eingenommene Perspektive relevante Information angezeigt wird. Mit Monitoren kann auch eine vollständig räumlich entkoppelte Darstellung realisiert werden, indem die darzustellende Information auf entfernte Displays übertragen wird. So kann zum Beispiel die Interaktion auf einem Tabletop Interface bzw. deren Auswirkungen auch räumlich entfernt verfolgt werden.

Generell muss bei entkoppelten Ausgabekanälen darauf geachtet werden, dass bei Interaktionen mit den tangiblen Eingabekanälen entsprechend eindeutig zuzuordnendes Feedback über die Ausgabekanäle rückgespiegelt wird. Bei Tabletop Interfaces ist hierbei (wiederum abgängig von der Anwendung) eine schematische Darstellung der Tischoberfläche mit einer Kenntlichmachung des Bereichs, in dem eine Interaktion erkannt wurde, sinnvoll.

% subsubsection tokens_mit_aktuatoren (end)

\subsubsection{Projektion auf entfernte Oberflächen} % (fold)
\label{ssub:tokens_mit_aktuatoren}

Bei Ausgabe mittels Projektion auf entfernte Oberfläche (im Sinne von Oberflächen, die nicht dem eigentlichen Interface zuzuordnen sind) sind Aspekte wie Größe des Bildes oder Blickwinkelabhängigkeit der Darstellung meist keine Herausforderung. Mit einem Projektor können zumeist alle Benutzer ausreichend mit Information bedient werden. Ansonsten gelten die obigen Ausführungen hinsichtlich eindeutig zuzuordnendem Feedbacks analog.

Bei individueller Nutzung eines Tangible Interfaces ist in der Verwendung von Bildschirm oder Projektor noch ein unterschiedlich hoher Grad an Privatheit der Ausgabekanäle festzustellen. Während Bildschirme eher dem mit dem System interagierenden Individuum als Ausgabekanal vorbehalten bleiben, sind projezierte Informationen quasi öffentlich verfügbar. Abhängig vom Anwendungsszenario des Tangible Interfaces kann dies erwünscht sein oder nicht.

% subsubsection tokens_mit_aktuatoren (end)

\subsubsection{Audio-basierte Ausgabekanäle} % (fold)
\label{ssub:tokens_mit_aktuatoren}

% subsubsection tokens_mit_aktuatoren (end)

% subsection entkoppelte_ausgabe (end)

\subsection{Technologie-Entscheidung} % (fold)
\label{sub:output_ansatz_entscheidung}

% subsection output_ansatz_entscheidung (end)

\subsection{Frameworks zur Ausgabe} % (fold)
\label{sub:frameworks_zur_ausgabe}

\subsubsection{In Frage kommende Frameworks} % (fold)
\label{ssub:in_frage_kommende_frameworks}

--> JHotDraw + evtl. andere Frameworks aus KnowIT-Diplomarbeit

\paragraph{JHotDraw} % (fold)
\label{par:jhotdraw}

% paragraph jhotdraw (end)

\subsubsection{Framework-Entscheidung} % (fold)
\label{ssub:output:framework_entscheidung}

% subsubsection output_framework_entscheidung (end)
% subsubsetion in_frage_kommende_frameworks (end)

% subsection subsection_name (end)
% subsubsection jhotdraw (end)

% subsection frameworks_zur_ausgabe (end)

% section technologische_grundlage_der_visualisierung (end)

\section{Ausgabe von Information} % (fold)
\label{sec:ausgabe_von_information}

\subsection{Konzept} % (fold)
\label{sub:ausgabe_konzept}

Dual View
% subsection ausgabe_konzept (end)

\subsection{Architekur} % (fold)
\label{sub:architekur}

Modularchitektur beeinflusst von den Vorgaben der Entwurfsmuster in JHotDraw
% subsection architekur (end)

\subsection{Ausgabe von Information zum Modell} % (fold)
\label{sub:ausgabe_von_information_zum_modell}

\subsubsection{Information zu Modellelementen} % (fold)
\label{ssub:information_zu_modellelemeenten}

% subsubsection information_zu_modellelementen (end)

\subsubsection{Information zu Verbindern} % (fold)
\label{ssub:information_zu_verbindern}

% subsubsection information_zu_verbindern (end)
% subsection ausgabe_von_information_zum_modell (end)

\subsection{Ausgabe zur Kontrolle des Systems} % (fold)
\label{sub:ausgabe_zur_kontrolle_des_systems}

\subsubsection{Zustandsmeldungen} % (fold)
\label{ssub:zustandsmeldungen}
Löschmodus etc.

% subsubsection zustandsmeldungen (end)

\subsubsection{Wiederherstellungsunterstützung} % (fold)
\label{ssub:wiederherstellungsunterstützung}

% subsubsection wiederherstellungsunterstützung (end)
% subsection ausgabe_zur_kontrolle_des_systems (end)

% section ausgabe_von_information (end)

\section{Umsetzung der Ausgabe mit Software} % (fold)
\label{sec:umsetzung_der_ausgabe_mit_software}

\subsection{Ausgabe des Modellzustands} % (fold)
\label{sub:einsatz_von_jhotdraw}

Bildschirm und Projektor

\subsubsection{Kalibrierung der Ausgabe} % (fold)
\label{ssub:kalibrierung_der_ausgabe}

% subsubsection kalibrierung_der_ausgabe (end)

\subsection{Weitere Ausgabekanäle} % (fold)
\label{sub:weitere_ausgabekanäle}

Verteilter Viewer

% subsection weitere_ausgabekanäle (end)
% section umsetzung_der_ausgabe_mit_software (end)

% chapter visualisierung (end)