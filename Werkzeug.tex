\part{Unterstützung} % (fold)
\label{prt:umsetzung}

\section*{Einleitung} % (fold)
\label{sec:umsetzung_einleitung}
\thispagestyle{empty}

Basierend auf den diese Arbeit motivierenden Grundlagen, die in den letzten Kapiteln beschrieben wurden, wird in diesem Teil auf die konkreten Maßnahmen zur Unterstützung von Articulation Work eingegangen. Ziel dieses Teils ist es, sowohl das hier vorgeschlagene Vorgehen bei der Unterstützung expliziter Articulation Work als auch die Unterstützung, die ein Werkzeug dabei leisten kann, umfassend dazustellen.

Auf Grundlage der Methodik, die im Kontext von Concept Mapping und Strukturlegetechniken vorgeschlagen wird und unter Berücksichtigung der Anforderungen, die aus dem inhärent kollaborativen Anwendungsszenario abgeleitet werden können, muss das Vorgehen zur Durchführung von expliziter Articulation Work festgelegt werden. 

In Rahmen der Festlegung des Vorgehens werden auch jene Aspekte identifiziert, in denen Unterstützung durch technische Werkzeuge sinnvoll und notwendig ist. Die Anforderungen, die sich aus diesen Aspekten ableiten lassen, bilden die Grundlage für die Konzeption und Umsetzung eines Werkzeugs, das diese Unterstützung bietet. Die technischen Details der Implementierung dieses Werkzeugs und die zugrunde liegenden konzeptionellen und technologischen Grundlagen bilden den Kern dieser Arbeit.

Der Aufbau dieses Teils folgt dem eben umrissenen inhaltlichen Vorgehen. In Kapitel \ref{cha:methodik} wird die Methodik zur Unterstützung expliziter Artikulation Work beschrieben. Aus diesem werden im darauf folgenden Kapitel \ref{cha:anforderungen} jene Bereiche identifiziert, in denen eine technologische Unterstützung notwendig ist und die Anforderungen an ein Werkzeug abgeleitet, das diese Unterstützung bietet. Die vier folgenden Kapitel beschäftigen sich mit der Umsetzung des Werkzeugs. Kapitel \ref{cha:implementierung_Überblick} beschäftigt sich dabei mit den konzeptionellen Grundlagen des Forschungsgebiets "Tangible Interfaces", das die Basis für die technische Umsetzung bildet. Kapitel \ref{cha:input_&_interpretation} beschäftigt sich mit jenen Technologien und Softwarekomponenten, die für die Informationseingabe in das technische System verwendet werden. Dabei wird auch auf die konkrete Interaktion der Benutzer mit dem System eingegangen. Kapitel \ref{cha:visualisierung} beschreibt die Ausgabeseite des technischen Systems und behandelt die Umsetzung des Informationsflusses vom System zu den Benutzern. Letztendlich wird in Kapitel \ref{cha:persistierung} beschrieben, welche Maßnahmen zu Sicherung der Ergebnisse der expliziten Articulation Work getroffen werden müssen und welche Möglichkeiten der technischen Umsetzung bestehen bzw. gewählt wurden.

% section umsetzung_einleitung (end)

\chapter{Methodik} % (fold)
\label{cha:methodik}

% chapter methodik (end)
\chapter{Anforderungen an ein Werkzeug} % (fold)
\label{cha:anforderungen}

\paragraph{Physische Abbildung beliebiger diagrammatischer Modelle} % (fold)
\label{par:physische_abbildung_legen_beliebiger_diagrammatischer_modelle}

Ein Werkzeug zur Unterstützung von Strukturlegetechniken muss das grundlegende Konzept der Methodik vollständig unterstützten. Es muss möglich sein, Konzepte auf einer Modellierungs-Oberfläche zu platzieren und zueinander in Beziehung zu setzen. Der gesamte Modellstatus muss visuell auf der Oberfläche erkennbar sein.

% paragraph physische_abbildung_legen_beliebiger_diagrammatischer_modelle (end)

\paragraph{Kollaborative und unmittelbare Manipulierbarkeit des Modells} % (fold)
\label{par:kollaborative_und_unmittelbare_manipulierbarkeit_des_modells}

Zur Unterstützung von expliziter „Articulation Work“ muss das Werkzeug kollaborative Strukturlege-Prozesse erlauben. Es muss möglich sein, das gelegte Modell simultan zu erweitern oder zu verändern.

% paragraph kollaborative_und_unmittelbare_manipulierbarkeit_des_modells (end)

\paragraph{Nicht vorgegebene Semantik der Modellierungselemente} % (fold)
\label{par:nicht_vorgegebene_semantik_der_modellierungselemente}

Wie oben bereits argumentiert, sind zur Unterstützung von expliziter „Articulation Work“ vor allem Varianten von Strukturlegetechniken geeignet die keine Vorgaben hinsichtlich der zu verwendenden Konzepte und Verknüpfungen machen. Das Werkzeug muss dementsprechend die Offenheit bieten, beliebige Klassen von Konzepten und Verknüpfungen zu definieren (z.B. Klasse „organisationale Rolle“) und von diesen beliebige Instanzen zu bilden und zu benennen (z.B. Instanz „Geschäftsführer“). Gleichzeitig muss sichergestellt werden, dass die festgelegte Semantik im Modell mit abgebildet wird und nicht verloren geht.

% paragraph nicht_vorgegebene_semantik_der_modellierungselemente (end)

\paragraph{Unterstützung der iterativen Aushandlung des Modells} % (fold)
\label{par:unterstützung_der_iterativen_aushandlung_des_modells}

Im Sinne der Unterstützung der Dialog-Konsens-Methodik sind ist der Austausch über das Modell durch das Werkzeug zu unterstützen. Vor allem muss es möglich sein, Anmerkungen über Konsens oder Dissens über einzelnen Modellteile oder das gesamte Modell explizit mit in die Repräsentation aufzunehmen. 

% paragraph unterstützung_der_iterativen_aushandlung_des_modells (end)

\paragraph{Persistente Ablage des Modells und Möglichkeit zur Rekonstruktion} % (fold)
\label{par:persistente_ablage_des_modells_möglichkeit_zur_rekonstruktion}

Die persistente Ablage eines Modells (z.B. als digitale Repräsentation) und Werkzeugunterstützung zur Rekonstruktion eines abgelegten Modells erlaubt die Wiederaufnahme eines unterbrochenen Strukturlegeprozesses bzw. die Reflexion und Anpassung bereits erstellter Modelle zu einem späteren Zeitpunkt.

% paragraph persistente_ablage_des_modells_möglichkeit_zur_rekonstruktion (end)

\paragraph{Ermöglichung experimenteller Veränderungen am Modell} % (fold)
\label{par:ermöglichung_experimenteller_veränderungen_am_modell}

Es muss möglich sein, das Modell experimentell zu verändern und ggf. zu einem früheren stabilen Modellzustand zurückzukehren. Dies erlaubt eine konsequenzlose Erkundung von Lösungsräumen und unterstützt damit den Dialog-Konsens-Prozess. Das Werkzeug muss also stabile Modellzustände erfassen und deren Rekonstruktion unterstützen.
% paragraph ermöglichung_experimenteller_veränderungen_am_modell (end)

\paragraph{Verknüpfung mit digitalen Ressourcen} % (fold)
\label{par:verknüpfung_mit_digitalen_ressourcen}

Die Einbindung von digitalen Ressourcen (Dateien, Hyperlinks,...) ermöglicht die Einbindung des Modells in den organisationalen Kontext und erleichtert so einerseits die Verständnisbildung und ermöglicht andererseits die Verwendung der Repräsentation als unmittelbare Handlungsanleitung mit Verknüpfungen zu den betroffenen Arbeitsgegenständen.

% paragraph verknüpfung_mit_digitalen_ressourcen (end)

\paragraph{Bearbeitung von beliebig komplexen Modellen} % (fold)
\label{par:bearbeitung_von_beliebig_komplexen_modellen}

Komplexe Modelle enthalten oft eine große Anzahl von Konzepten und viele Verknüpfungen. Das Werkzeug muss das Modell in einer Form darstellen, die dessen Erfassung und Manipulation ermöglicht, ohne die Repräsentierenden kognitiv zu sehr zu belasten.
% paragraph bearbeitung_von_beliebig_komplexen_modellen (end)






\footnote{By recording and replaying the authoring process, navigable history can re-situate an author after a gap in the authoring process. Similarly, in a collaborative authoring process, an author can play through the events since his/her last authoring session to quickly determine the activity of the other authors. Finally, in many situations, information becomes harder to interpret as its context changes over time. By returning to the state of the information space at the time of authoring, disambiguation of the information may become possible. For the reader who is not also the writer of the hypertext there are additional uses of navigable history. A reader replaying the author’s writing process can gain insight into the motivation of the author and have a greater understanding of the author’s writing style. Such an understanding is important in collaborative work and in other contexts, like education and literary analysis. \citep{Shipman00}}

% chapter anforderungen (end)

\chapter{Grundlagen der Implementierung} % (fold)
\label{cha:implementierung_Überblick}

Wie im Kapitel "Design" gefordert, wurde zur Umsetzung des Werkzeugs ein "Tangible Tabletop Interface" verwendet. Tabletop Interface zeichnen sich im Generellen dadurch aus, dass im Gegensatz zu handelsüblichen Rechnern nicht nur die Software sondern auch die Hardware applikationsspezifisch ist und nicht generisch eingesetzt werden kann. Die Hardware bildet dabei einen Teil oder die gesamte Benutzungsschnittstelle ab. Im speziellen Fall eines "Tangible Tabletop Interfaces" basiert der Benutzerinteraktion auf der Verwendung physischer Bausteine ("Tokens"), die auf der physischen Oberfläche des Interfaces manipuliert werden. Dieses Paradigma wird ergänzt von Tabletop Interfaces, die die Benutzerinteraktion ausschließlich auf Gesten bzw. Berührungen der Oberfläche abbilden (horizontal verbaute "Touch-" bzw. "Multi-Touch-Displays").

REFs!!! Die Entwicklung von Tabletop Interfaces begann Mitte der 1990er-Jahren mit den Arbeiten von Ishii \& Ullmer. Auch die erste Anwendung, die sich mit Modellierungs-Ansätzen mit Hilfe von Tabletop Interfaces konzentriert, stammt aus dieser Zeit. Mit dem fortschreiten der technologischen Entwicklung ist heute ein Status erreicht, in dem mit Hilfe generischer Identifikations-Frameworks schnell und ohne großen Aufwand Applikationen mit "tangiblen" Inputkanälen erstellt werden können. Zur Zeit noch im Prototypenstatus befinden sich Ansätze, die sich mit generischen Möglichkeiten des tangiblen Informationsoutputs beschäftigt. Der Rückkanal vom Rechner zum Benutzer wird heute zumeist mit der Projektion von Inhalten auf die Arbeitsoberfläche umgesetzt.

In den folgenden Abschnitten wird die historische Entwicklung von Tabletop Interfaces sowie der aktuelle Stand der Entwicklung im Anwendungsbereich dieser Arbeit betrachtet. Es werden dabei die grundlegenden Konzepte und Eigenschaften der jeweiligen Arbeiten betrachtet und das Potential hinsichtlich der Umsetzung von in Kapitel XY identifizierten Anforderungen an das hier entwickelte Werkzeug betrachtet. 

\section{Entwicklung von Tangible Interfaces} % (fold)
\label{sec:tangible_interfaces}

Der Begriff der Tangible bzw. Graspable Interfaces – also der "berührbaren" oder "begreifbaren" Benutzungsschnittstellen — stammt aus der Mitte der neunziger Jahre des zwanzigsten Jahrhunderts. \citet{Fitzmaurice95} werden im Allgemeinen als die ersten betrachtet, die den Begriff des "Graspable User Interfaces" prägen und damit die Manipulierbarkeit digitaler Information durch physische Mittel beschreiben. \citet{Fitzmaurice96} präzisiert später den Begriff durch die Abgrenzung zwischen (herkömmlichen, maus-, tastatur- und bildschirmbasierenden) zeitlich gemultiplexten Schnittstellen, bei denen der Informationsaustausch zwischen Benutzer und System über einen Kanal zeitlich hintereinander erfolgt und den (neuartigen, berührbaren) räumlich gemultiplexten Schnittstellen, bei denen mehrere Kanäle gleichzeitig zur Interaktion zwischen Benutzer und System verwendet werden können. 

Der Begriff des "Tangible User Interfaces" wurde kurz danach bzw. parallel dazu von \citet{Ishii97} eingeführt. \citeauthor{Ishii97} verfolgen dabei bei der Definition den umgekehrten Weg und sprechen von einer "Augmentation der realen Welt durch eine Kopplung von digitaler Information and physische Objekte"\footnote{\emph{“augment the real physical world by coupling digital information to everyday physical objects and environments”}\citep{Ishii97}}. 

\subsection{Ubiquitous Computing}

\subsection{Augmented Reality} 

% subsection tangibles_historischer_hintergrund (end)

\section{Konzeptualisierung und Einteilung von Tangible Interfaces} % (fold)
\label{sec:konzeptualisierungen_von_tangible_interfaces}

Die Entwicklung des Forschungsgebiets der "Tangible Interfaces" wurde von mehreren konzeptuellen Arbeiten maßgeblich beeinflusst. Die dort vorschlagenen Erklärungsmodelle definieren das Gebiet und grenzen es gegenüber anderen Forschungsbereichen ab. Sie dienen außerdem als Grundlage für Erklärung und Konzeption konkreter Tangible Interfaces. Im Folgenden wird die historische Entwicklung dieser konzeptuellen Modelle beschrieben und auf deren Spezifika eingegangen.

Zur struktrierten Betrachtung von Tangible Interfaces ist es außerdem notwendig, jene Dimensionen zu identifizieren, an denen sich einzelne Tangible Interfaces einordnen und unterscheiden lassen. Die Ausprägungen dieser Dimensionen liefern kombiniert ein Begriffssystem, dass bei der Aufbereitung von unterschiedlichen Ansätzen im Bereich der Tangible Interface sowie deren Vergleich helfen kann. Die hier vorgestellten Ansätze tragen unterschiedlich detailliert und aus unterschiedlichen Gesichtspunkten zu dieser Thematik bei. Die einzelnen Ansätze werden hier dargestellt und in Kapitel XY auf das in dieser Arbeit entwickelte System angewandt um so das System-Design aus konzeptueller Sicht zu reflektieren und potentielle Verbesserungs- und Erweiterungsmöglichkeiten zu identifizieren.

\subsection{Graspable User Interfaces}

\citeauthor{Fitzmaurice96} legt in jener Arbeit, in der es den Begriff des "Graspable User Inferfaces" prägt \citep{Fitzmaurice96}, auch Eigenschaften fest, anhand deren sich die "Graspability" einer Benutzungsschnittstelle zeigt und beurteilen lässt. Diese Beurteilung erfolgt auf einer generischen Skala mit Ausprägungen von "niedrig" bis "hoch", wobei "hohe" Werte in mehreren Eigenschaften auf eher hohe "Graspability" hinweist.

\subsubsection{Space Mulitplexing}
\subsubsection{Concurrency}
\subsubsection{Physical Form}
\subsubsection{Spartially aware}
\subsubsection{Spatial recofigurability}

\subsection{Tangible Bits}

\citep{Ishii97}

\subsection{Containers, Tokens und Tools}

\citep{Holmquist99} legen ihre Arbeit als konzeptuelle Betrachtung von interaktiven Systemen an, in denen physische Objekte verwendet werden, um auf digitale Information zuzugreifen bzw. diese zu manipulieren. Das Einteilungsschema, das die Autoren vorschlagen, basiert auf der Art und Weise, in der Information an diese physischen Objekte gebunden ist. 

\subsubsection{Containers}

\subsubsection{Tokens}

\subsubsection{Tools}

\subsection{Das MCRpd Interaktions-Modell}
\citep{Ullmer00}

\subsection{Degree of Coherence}
\citep{Koleva03}

\subsection{Tokens und Constraints nach Shaer et al.}
\citep{Shaer04}

\subsection{Einteilung nach Klemmer, Li, Lin und Landay}
\citep{Klemmer04}

\subsection{Taxonomie nach Fishkin}
\citep{Fishkin04}

\subsection{Tokens und Constraints nach Ullmer et al.}
\citep{Ullmer05}

\subsection{Tangible Bits: Beyond Pixels}
\citep{Ishii08}

% section konzeptualisierungen_von_tangible_interfaces (end)

\section{Tangible Interfaces in kooperativer Verwendung} % (fold)
\label{sub:tangible_interfaces_in_kooperativer_verwendung}
\citep{Hornecker04}
% subsection tangible_interfaces_in_kooperativer_verwendung (end)

% section tangible_interfaces (end)

\section{Tabletop Interfaces} % (fold)
\label{sec:tabletop_interfaces}

Grundlagen

\subsection{Historische Entwicklung} % (fold)
\label{sub:historische_entwicklung_von_tabletop_interfaces}

\subsubsection{Sensetable} % (fold)
\label{subs:sensetable}
Der Sensetable \citep{Patten01}
% subsubsection sensetable (end)

\subsubsection{BUILD-IT} % (fold)
\label{par:build_it}
\citep{Fjeld01}
% subsubsection build_it (end)
% subsection historische_entwicklung_von_tabletop_interfaces (end)
% section tabletop_interface (end)

\section{Tangible Interfaces zur Modellbildung} % (fold)
\label{sub:tangible_interfaces_zur_modellbildung}

% subsection tangible_interfaces_zur_modellbildung (end)

\subsection{Aktuelle verwandte Ansätze} % (fold)
\label{sub:aktuelle_verwandte_ansätze}

% subsection aktuelle_verwandte_ansätze (end)
\begin{itemize}
	\item Historische Entwicklung von Tabletop Interfaces
	\begin{itemize}
		\item Sensetable
		\item Morten Fjeld
		\item ReacTable
		\item Eva Hornecker
	\end{itemize}
	\item Historische Entwicklung von Tangible Interfaces zur Modellbildung
	\begin{itemize}
		\item Sensetable Modeling Application
		\item Designer's Outpost (Klemmer)
	\end{itemize}
	\item Aktuelle verwandte Ansätze
	\begin{itemize}
		\item Antle (TEI Mail-Pointer)
		\item Sun (TEI Demo)
	\end{itemize}
\end{itemize}


% section grundlegende_&_verwandte_arbeiten (end)

% chapter implementierung_Überblick (end)
\chapter{Input \& Interpretation} % (fold)
\label{cha:input_&_interpretation}

\section{Möglichkeiten zur Erfassung von Benutzerinteraktion} % (fold)
\label{sec:möglichkeiten_zur_erfassung_von_benutzerinteraktion}

\subsection{Potentielle technologische Ansätze} % (fold)
\label{sub:potentielle_technologische_ansätze}
\begin{itemize}
	\item optisch
	\item kapazitiv
	\item elektromagnetisch (RFID)
	\item akustisch (Ultraschall)
\end{itemize}

% subsection potentielle_technologische_ansätze (end)
\subsection{Verfügbare Frameworks} % (fold)
\label{sub:verfügbare_frameworks}
\begin{itemize}
	\item ReacTIVision
	\item ARToolkit
	\item Papiermache (Klemmer)
	\item Visual Codes (ETH)
	\item Frameworks aus der TU LVA
\end{itemize}

% subsection verfügbare_frameworks (end)

\subsection{Technologieentscheidung} % (fold)
\label{sub:technologieentscheidung}
-> ReacTIVision
% subsection technologieentscheidung (end)
% section möglichkeiten_zur_erfassung_von_benutzerinteraktion (end)

\section{Konzeption und Umsetzung der Hardwarekomponenten} % (fold)
\label{sec:konzeption_und_umsetzung_der_hardwarekomponenten}

\subsection{Überblick} % (fold)
\label{sub:Überblick}
Grafik aus dem TEI-Paper
Zerlegbarkeit

% subsection Überblick (end)

\subsection{Tokens \& Input-Werkzeuge} % (fold)
\label{sub:tokens_&_input_werkzeuge}

% subsection tokens_&_input_werkzeuge (end)

\subsection{Input auf der Tischoberfläche} % (fold)
\label{sub:input_auf_der_tischoberfläche}

Semitransparente Oberfläche, Projektion von unten (Umlenkspiegel), Kamera von untern - Überleitung zur 
Illumination via Interferenz zwischen Beamer und Kamera

% subsection input_auf_der_tischoberfläche (end)

\subsection{Illumination und Umgebungslichtabhängigkeit} % (fold)
\label{sub:illumination_und_umgebungslichtabhängigkeit}

% subsection illumination_und_umgebungslichtabhängigkeit (end)
% section konzeption_und_umsetzung_der_hardwarekomponenten (end)

\section{Erfassung der Benutzerinteraktion durch Software} % (fold)
\label{sec:erfassung_der_benutzerinteraktion_durch_software}

% section erfassung_der_benutzerinteraktion_durch_software (end)

\section{Interpretation der Rohdaten und Stabilisierung der Erkennungsleistung} % (fold)
\label{sec:interpretation_der_rohdaten_und_stabilisierung_der_erkennungsleistung}

% section interpretation_der_rohdaten_und_stabilisierung_der_erkennungsleistung (end)
% chapter input_&_interpretation (end)
\chapter{Visualisierung und Modellierungsunterstützung} % (fold)
\label{cha:visualisierung_und_modellierungsunterstützung}

\section{Technologische Grundlage der Visualisierung} % (fold)
\label{sec:technologische_grundlage_der_visualisierung}

\subsection{JHotDraw – Überblick} % (fold)
\label{sub:jhotdraw_Überblick}

% subsection jhotdraw_Überblick (end)
\subsection{Einsatz von JHotDraw} % (fold)
\label{sub:einsatz_von_jhotdraw}

% subsection einsatz_von_jhotdraw (end)
\subsection{Projektion von Information auf die Tischoberfläche} % (fold)
\label{sub:projektion_von_information_auf_die_tischoberfläche}

% subsection projektion_von_information_auf_die_tischoberfläche (end)
% section technologische_grundlage_der_visualisierung (end)

\section{Umsetzung der Anforderungen zur Modellierungsunterstützung} % (fold)
\label{sec:umsetzung_der_anforderungen_zur_modellierungsunterstützung}

\subsection{Benennung von Blöcken und Verbindungen} % (fold)
\label{sub:benennung_von_blöcken_und_verbindungen}
inkl. Löschen

% subsection benennung_von_blöcken_und_verbindungen (end)

\subsection{Abstraktion} % (fold)
\label{sub:abstraktion}
Container

% subsection abstraktion (end)

\subsection{Modellierungshistorie} % (fold)
\label{sub:modellierungshistorie}
automatisches Tracking vs. Snapshots

% subsection modellierungshistorie (end)

\subsection{Wiederherstellungsunterstützung} % (fold)
\label{sub:wiederherstellungsunterstützung}

% subsection wiederherstellungsunterstützung (end)
% section umsetzung_der_anforderungen_zur_modellierungsunterstützung (end)
% chapter visualisierung_und_modellierungsunterstützung (end)
\chapter{Persistierung} % (fold)
\label{cha:persistierung}

In den vorangegangen drei Kapiteln wurde die Umsetzung des eigentlichen Werkzeugs beschrieben. Neben der Unterstützung des Modellierungsvorgangs ist aber auch die persistente Speicherung der erstellten Modelle zum Zwecke der Weiterverarbeitung ein hier zu beleuchtender Aspekt. Auf die Persistierung wirken vor allem zwei der in Kapitel XY identifzierten Anforderungen ein. Zum ersten ist die Nachvollziehbarkeit des Modellierungsvorganges sicherzustellen -- dies gilt nicht nur während des Vorgangs selbst, sondern auch danach. Dementsprechend ist sämtliche Information zu persistieren, die zur Wiederherstellung nicht nur des Modells selbst sondern auch der gesamten Modellierungshistorie notwendig ist. Zum zweiten hat die Forderung nach semantischer Offenheit bei der Modellierung auch unmittelbare Auswirkungen auf die Persistierung. Neben dem Modell selbst muss aufgrund dieser Anforderung auch die Bedeutung der verwendeten Modellierungselemente miterfasst und persistiert werden, so dass diese bei der Weiterverarbeitung der Modelle verwendet werden kann.

In diesem Kapitel werden nun aufgrund der eben genannten Forderungen technologische Ansätze identifiziert, beschrieben und schließlich hinsichtlich ihrer Eignung für den konkreten Einsatz beurteilt. Der ausgewählte Ansatz wird im darauf folgenden Abschnitt konzeptuell beschrieben. Die Abbildung der Modelle und der ebenfalls zu persistierenden zusätzlichen Information in ein geeignetes Datenmodell ist Gegenstand des darauf folgenden Abschnitts. Schließlich wird die konkrete technische Umsetzung der Persistierung dargelegt und die dazu notwendigen Software-Module im Detail beschrieben.
 
\section{Möglichkeiten der Persistenzsicherung} % (fold)
\label{sec:möglichkeiten_der_persistenzsicherung}

\begin{itemize}
	\item Serialisierung von Java-Objekten
	\item Relationale Datenbanken
	\item XML Topic Maps
\end{itemize}

% section möglichkeiten_der_persistenzsicherung (end)

\section{Topic Maps} % (fold)
\label{sec:topic_maps}

Topic Maps \citep{TMDM08} sind wie bereits in Abschnitt XY beschrieben ein Mittel zur Abbildung von semantischen Netzen. In Topic Maps können beliebige Daten strukutriert aufbereitet und zueinander in Beziehung gesetzt werden. Die Art der zu repräsentierenden Daten ist dabei irrelvant, eine Topic Map trifft keine Aussage über ein den repräsentierten Daten zugrundeliegendes Begriffsystem (sie ist „ontology-agnostic“ \citep{Vatant04}).

Historisch stammen Topic Maps aus dem Bereich der technischen Repräsentation von Thesauri und Indizes \citep{Pepper00} \citep{Rath03}. Aus diesen Bereichen motivieren sich auch die Bausteine einer Topic Map, wenngleich der Verwendung durch diesen Ursprung nicht eingeschränkt wird. Die grundlegenden Elemente einer Topic Map sind „Topics“, „Associations“ und „Occurrences“ (siehe Abbildung \ref{fig:img_Persistenz_TMBasic}). 

\begin{figure}[htbp]
	\centering
		\includegraphics[width=10cm]{img/Persistenz/TMBasic.png}
	\caption{Grundlegende Elemente einer Topic Map}
	\label{fig:img_Persistenz_TMBasic}
\end{figure}

„Topics“ sind stellen Begriffe dar und bilden die Knoten des semantischen Netzes. Ein Topic kann beliebige Information darstellen, repräsentiert aber immer genau ein Phänomen der realen Welt (d.h. zu einem Topic muss es eine Entsprechung außerhalb der Topic Maps geben, die beobachtbar oder beschreibbar ist und auf die die modellierende Person Bezug nehmen will \footnote{„A subject can be anything whatsoever, regardless of whether it exists or has any other specific characteristics, about which anything whatsoever may be asserted by any means whatsoever. In particular, it is anything about which the creator of a topic map chooses to discourse.“ \citep[][S.8]{TMDM08}}). Eine Topic Map ist damit im Sinne von \citet{Stachowiak73} ein diagrammatisches Modell, das einen bestimmten, für den Modellersteller relevanten Ausschnitt der Realität abbildet.

"Associations" bilden die Beziehungen zwischen Topics ab und stellen damit die Kanten des semantischen Netzes dar. Eine Association verknüpft Topics semantisch miteinander und kann frei mit Bedeutung belegt werden. Die Art der Beziehungen ist also nicht festgelegt und wird wie die Bedeutung der Topics frei gewählt werden. Topics und Associations decken historisch den Bereich der Darstellung von Thesauri ab, in denen Begriffe definiert und zueinanden in Beziehung gesetzt werden. 

Der zweite historische Ursprung von Topic Maps, die Indizes, werden durch das Konstrukt der "Occurences" abgedeckt. Occurences ("Auftreten") sind Referenzen aus der Topic Map in die reale Welt. Sie setzen die Topics einer Topic Map in Bezug zu beliebiger referenzierbarer Information (z.B. Dokumente). Im Kontext der eben genannten Indizes, kann eine Topic Map als der mit Querverweisen versehene Index eines Buches verstanden werden, in dem durch die Angabe von Seitenzahlen auf den Text des Buches verwiesen wird. Diese Verweise durch Angabe der Seitenzahlen sind in diesem Zusammenhang die Occurrences.

Die Ansammlung von durch Associations verknüpften und mit Occurrences versehenen Topics bilden eine Topic Map. Darüber hinaus kann in Topic Maps jedoch noch weiterführende Information repräsentiert werden (siehe Abbildung \ref{fig:img_Persistenz_TMFull}), die Gegenstand der folgenden Abschnitte sein werden.

\begin{figure}[htbp]
	\centering
		\includegraphics[width=10cm]{img/Persistenz/TMFull.png}
	\caption{Umfassende Darstellung der Elemente einer Topic Map}
	\label{fig:img_Persistenz_TMFull}
\end{figure}

\subsection{Topics, Subjects, Topic Names und Variants} % (fold)
\label{sub:topics_subjects_topic_names_und_variants}

Dieser

\begin{figure}[htbp]
	\centering
		\includegraphics[width=10cm]{img/Persistenz/TopicNaming.png}
	\caption{Benennung von Topics}
	\label{fig:img_Persistenz_TopicNaming}
\end{figure}


% subsection topics_subjects_topic_names_und_variants (end)

\subsection{Associations und Roles} % (fold)
\label{sub:associations_und_roles}

% subsection associations_und_roles (end)

\subsection{Occurrences und Datatypes} % (fold)
\label{sub:occurrences_und_datatypes}

% subsection occurrences_und_datatypes (end)

\subsection{Metamodellierung in Topic Maps} % (fold)
\label{sub:metamodellierung_in_topic_maps}

Topic Types, Association Types und Occurrence Types

\begin{figure}[htbp]
	\centering
		\includegraphics[width=10cm]{img/Persistenz/MetaModelExample.png}
	\caption{Beziehungen in der Metamodellbildung in Topic Maps}
	\label{fig:img_Persistenz_MetaModelExample}
\end{figure}

% subsection metamodellierung_in_topic_maps (end)

\subsection{Scopes} % (fold)
\label{sub:scopes}

% subsection scopes (end)

\subsection{Weiterführende Konzepte} % (fold)
\label{sub:tm_weiterführend}

\subsubsection{Reification} % (fold)
\label{ssub:reification}

% subsubsection reification (end)

\subsubsection{Merging} % (fold)
\label{ssub:merging}

% subsubsection merging (end)
% subsection tm_weiterführend (end)

\subsection{Einschränkungen} % (fold)
\label{sub:einschränkungen}

Regeln und verbindliche Strukturvorgaben
Datenhaltung und Abfrage
% subsection einschränkungen (end)
% section topic_maps (end)

\section{Abbildung von Modellen auf Topic Maps} % (fold)
\label{sec:abbildung_von_modellen_auf_topic_maps}
-> DA Matthias
% section abbildung_von_modellen_auf_topic_maps (end)

\section{Technische Umsetzung der Persistierung von Modellen} % (fold)
\label{sec:technische_umsetzung_der_persistierung_von_modellen}
Topic Map Engine Persistence Layer
% section technische_umsetzung_der_persistierung_von_modellen (end)

\section{Zusammenfassung} % (fold)
\label{sec:persistierung_zusammenfassung}

% section persisitierung_zusammenfassung (end)
% chapter persistierung (end)


% part umsetzung (end)

