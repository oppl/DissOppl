\chapter{Input \& Interpretation} % (fold)
\label{cha:input_&_interpretation}

In diesem Kapitel wird jener Teil des Werkzeuges beschrieben, in dem die Interaktion der der Benutzer mit dem Werkzeug erfasst und interpretiert wird. Der erste Abschnitt behandelt grundlegende Möglichkeiten zur Erfassung der Benutzerinteraktion auf tisch-basierten Benutzungsschnittstellen und endet mit der Identifikation der im konkreten Anwendungsfalls geeignetsten Technologie. Diese wird durch die Beschreibung von dafür verfügbaren Frameworks konkretisiert, was letzendlich in der Entscheidung für ein konkretes Produkt mündet.

Basierend auf dieser Entscheidung wird in den folgenden beiden Abschnitten auf das Design der Hard- und Softwarekomponenten eingegangen, die unmittelbar der Eingabe von Information durch Benutzer dienen. Der Ausgabeaspekt wird hier bewusst ausgeklammert und im nächsten Kapitel beschrieben. Dieses Kapitel endet mit eine Beschreibung der Interpretationsroutinen, die aus den durch das Framework gelieferten Rohdaten höherwertige, anwendungsspezifische Information extrahieren und diese den nachgeordneten Software-Modulen zur Verfügung stellen.

\section{Möglichkeiten zur Erfassung von Benutzerinteraktion} % (fold)
\label{sec:möglichkeiten_zur_erfassung_von_benutzerinteraktion}

Im Gegensatz zu Systemen mit dezidierten Eingabegeräten (wie Tastatur oder Maus) ist die Informationseingabe bei Tangible Interfaces unmittelbar an physische Tokens gebunden, die unabhängig voneinander und gegebenenfalls auch simultan manipuliert werden können. Diese Manipulation wird von einer vorhandenen Infrastruktur erfasst und im Sinne von Eingabedaten interpretiert. Der wesentliche Unterschied zu dezidierten Eingabegeräten besteht darin, dass die Manipulation des physischen Artefakts selbst für den Benutzer bedeutungstragend ist und nicht nur dem Zweck einer Zustandsänderung des digitalen Informationsraums dient. Dies impliziert, dass der Zustand der verwendeten Tokens bzw. der aktuelle Wert deren relevanten Parameter (z.B. Position, Rotation, Form, ...) erfasst werden kann, ohne die Bedeutung der Tokens noch deren Manipulierbarkeit in der realen Welt zu beeinflussen. Je nach Anwendungsfall kommen dafür mehrere Unterschiedliche technologische Ansätze in Frage. Die Beurteilungskriterien die dabei zu berücksichtigen sind, liegen nicht nur in den zu erhebenden Parametern begründet sondern umfassen auch die notwendige Erfassungsrate des Zustandes der Tokens sowie die Anzahl der simultan zu erfassenden Tokens bzw. Eigenschaften.

Im konkreten System muss - wie in Kapitel XY beschrieben - die planare Position von mehreren Tokens in Echtzeit (d.h. mehrmals pro Sekunde mit für den Benutzer nicht wahrnehmbaren Verzögerungen) erfasst werden. Neben der Position ist noch die Rotation eines Tokens als Raumparameter von Interesse. Bezüglich des Zustands eines Tokens muss erfasst werden können, ob es geöffnet oder geschlossen ist und ob es eingebettete Objekte enthält oder nicht. Mit diesen Anforderungen wird in den folgenden Unterabschnitten ein technologischer Ansatz zur Umsetzung der Interaktionserkennung ausgewählt.

\subsection{In Frage kommende technologische Ansätze} % (fold)
\label{sub:potentielle_technologische_ansätze}
Bei der Auswahl möglicher technologischer Ansätze zur Erfassung der Benutzerinteraktion müssen die zu erfassenden Parameter unterschiedlich behandelt werden. Konkret werden hier Ansätze zur Erfassung der Raumparameter (Position, Rotation) und Ansätze die Zustandsänderungen des Tokens erfassbar machen unterschieden. 

\subsubsection{Raumparameter} % (fold)
\label{ssub:raumparameter}
\index{Positionsbestimmung} 
Zur Erfassung von Raumparametern von Tokens bieten sich mehrere technologische Ansätze an. In Frage kommen für das konkrete - tisch-basierte System - nur Technologien, die eine Erfassung dieser Parameter mit einer Genauigkeit im Zentimeter- bis Millimeter-Bereich ermöglichen, da eine niedrigere Raumauflösung zu zu großen Ungenauigkeiten in der Positionsbestimmung führen würden, die einen Einsatz für das hier vorgestellte Werkzeug nicht erlauben würden. Im Folgenden werden die in Frage kommenden Technologien in ihren Grundzügen beschrieben und hinsichtlich ihrer Eignung für das konkrete System bewertet.

\paragraph{Optisch} % (fold)
\label{par:optisch}
\index{Positionsbestimmung!optisch} 
\index{Barcode} 

Optische Positionsbestimmung erfolgt immer mit Hilfe von Kamera-Systemen und Methoden der digitalen Bildverarbeitung. Die Kamera erfasst dabei die zu identifizierenden Tokens, dass resultierende Bild wird mit in Software umgesetzten Algorithmen ausgewertet, wodurch zumindest Identität und Position, zumeist aber auch weitere Raumparameter (wie Rotation) aller im Kamerabild befindlichen Tokens ermittelt werden können. Die Erkennung ist dabei auf den Erfassungsbereich der Kamera beschränkt. Neben diesem Einflussfaktor bestimmen zudem die Auflösung der Kamera sowie die Größe der Tokens die letztendlich erfassbare Fläche. Optische Systeme sind generell bei schlechten oder wechselnden Lichtverhältnissen eher fehleranfällig und nicht robust gegen Verdeckungen von Tokens (etwa durch Gliedmaßen oder andere Tokens).

Hinsichtlich des Identifikationsansatzes können zwei Arten von Systemen unterschieden werden. \emph{Codebasierte} Systeme verwenden zur Identifikation eines Tokens einen von der Kamera erfassbaren Code (etwa einen "Barcode"), der eindeutig einem Token zugeordnet werden kann. \emph{Featurebasierte} Systeme identifizieren ein Token aufgrund seiner äußeren Eigenschaften, zumeist über dessen Form (Schattenriss). Letztere bieten den Vorteil, dass ein Token nicht durch das Anbringen eines zusätzlichen Codes optisch verändert werden muss. Der größte Nachteil besteht in der Eigenschaft, dass nur Token mit unterschiedlichen Formen eindeutig identifiziert werden können. Die eindeutige Identifikation von mehreren Tokens einer Bauart ist bei featurebasierten Systemen nicht möglich. Codebasierte Systeme verwenden zumeist nicht herkömmliche Barcodes sondern robustere Systeme, bei denen eine Erkennung auch unter widrigen Beleuchtungsbedingungen oder niedrigen Bildauflösungen möglich ist und die zum Teil auch die Extraktion zusätzliche Information über weitere Raumparameter (wie Rotation, teilweise auch Parameter der dritten Dimension wie Neigung oder Entfernung) ermöglichen. 

Codebasierte Systeme können hinsichtlich der Art der Codierung der Indentitätsinformation wiederum in zwei Klassen unterschieden werden. Eine Gruppe von Ansätzen integriert die eigentliche Nutzinformation, also im Wesentlichen die tokenspezifische Identifikationsnummer, direkt in den Code und ermöglicht so ein direktes Auslesen der Information (z.B. bei QRCode (REF)). Die zweite Gruppe verwendet eine indirekte Zuordnung zwischen Token-ID und Code. Bei derartigen Ansätzen muss die Identität eines Tokens in einem Zwischenschritt über eine Mapping-Tabelle abgebildet werden, im Gegenzug ist die Ausgestaltung des Codes flexibler, im Allgemeinen kann dabei eine höhere Robustheit bei der Erkennung erreicht werden (z.B. ARToolkit (REF)).

% paragraph optisch (end)

\paragraph{Kapazitiv} % (fold)
\label{par:kapazitiv}
\index{Positionsbestimmung!kapazitiv}  

Kapazitive Ansätze basieren auf der Änderung der Kapazität von Leiterbahnen, die durch deren Berührung mit leitfähigem Material verursacht wird. Ursprünglich wurde die Technologie zur Umsetzung von berührungssensitiven Oberflächen entwickelt, kann jedoch auch zum Tracking von Tokens verwendet werden. Im Gegensatz zu druckempfindlichen Oberflächen (klassischen "Touchscreens") ist keine Druckausübung zur Erkennung notwendig, es können außerdem auch mehrere Tokens (bzw. Finger) gleichzeitig erkannt werden.

Technologisch bedingt müssen bei kapazitiven Ansätzen alle zu identifizierenden Objekte die Oberfläche des Systems berühren. In dieser Oberfläche ist ein Metallgitter eingebettet, zwischen dessen Adern eine elektrische Kapazität gemessen werden kann. Diese Kapazität verändert sich, sobald diese Adern berührt werden (wobei die Token in einen entsprechend geeigneten Material ausgeführt sein müssen). Durch die lokale Änderung der Kapazität kann die Position einer Berührung festgestellt werden. Die Genauigkeit ist dabei durch die Rasterweite des Metallgitters eingeschränkt. Der größte Nachteil eines kapazitiven Ansatzes ist in diesem Kontext aber, dass die Identität eine Tokens nicht direkt festgestellt werden kann (die Kapazitätsänderung ist für alle Token identisch). Zudem ist die Extraktion weiterer Raumparameter (wie Rotation) nicht bzw. nur mit zusätzlichen Aufwand möglich. Die Vorteile von kapazitiven Systemen liegen in der hohen Robustheit der Erkennung auch bei widrigen Umgebungseinflüssen (Lichtverhältnisse, Schmutz) sowie der prinzipiell beliebig großen und beliebig geformten Oberfläche, die zur Erkennung verwendet werden kann.

Kapazitive Systeme eignen sich also zur Positionsbestimmung, nicht aber zur Identifikation von Tokens. Dies macht sie für den konkreten Anwendungsfall nur in Kombination mit einer anderen Technologie geeignet. 

% paragraph kapazitiv (end)

\paragraph{Elektromagnetisch} % (fold)
\label{par:elektromagnetisch}
\index{Positionsbestimmung!elektromagnetisch} 
\index{RFID}
 
Die Ausstattung von Tokens mit elektromagnetisch erfassbaren Einheiten (z.B. RFID-Chips) ermöglicht ebenfalls die Erfassung von Raumparametern. Vorrangig eignet sich diese Technologie jedoch zur Identifikation von Tokens, die Positionsbestimmung kann nur mit erheblichem technischen Aufwand durchgeführt werden.

RFID-Chips (als Beispiel für einen elektromagnetischen Ansatz) sind passive Bauteile, die bei Energieversorgung durch ein elektrisches Feld aktiv werden und ihrerseits eine eindeutige Identifikationsnummer senden (im einfachsten Fall, komplexere Varianten sind möglich, werden hier aber nicht betrachtet). Historisch stammt die Technologie aus der Logistik und Warenwirtschaft und dient der Identifikation von Gütern und nicht der exakten Positionsbestimmung. Diese ist somit auch nur mittels erweiterter Infrastruktur möglich. Zum Auslesen eines RFID-Chips wird ein Lesegerät mit Antenne benötigt. Aus der Feldstärke, mit der diese Antenne die Antwort des Chips empfängt, kann auf die Entfernung des Chips von der Antenne geschlossen werden. Durch Kreuzpeilung mit mindestens zwei Antennen, deren Position bekannt ist, kann somit auf die ungefähre Position des Chips (und damit des Tokens, in das dieser eingebaut ist) geschlossen werden. Durch den Einsatz von "Antennenarrays" (matrixförmig angeordneten Antennen) mit geringer Reichweite ist so eine verhältnismäßig exakte (Größenordnung einige cm) Positionsbestimmung möglich. Die Feststellung der Ausrichtung eines Tokens (Rotation) ist auf diesem Wege allerdings nicht möglich. Die Identifikation eines Tokens ist jedoch unabhängig von Sichtkontakt und unmittelbarer Berührung und somit äußerst robust gegen Umgebungseinflüsse.

Elektromagnetische Systeme eignen sich wegen des hohen technischen Aufwandes bei gleichzeitig beschränkter Genauigkeit nur bedingt zur Feststellung von Raumparametern. Durch die Ausrichtung auf Extraktion der Identitätsinformation ist der Ansatz jedoch gut zur Kombination mit anderen Technologien wie kapazitiven Ansätzen geeignet, die ihre Stärken in der Bestimmung der Raumparameter haben.
 
% paragraph elektromagnetisch (end)

\paragraph{Akustisch} % (fold)
\label{par:akustisch}
\index{Positionsbestimmung!akustisch} 
\index{Ultraschall} 

Akustische Ansätze zur Positionsbestimmung basieren im Generellen auf der Laufzeitmessung von Ultraschallwellen im Raum. Mit entsprechender Infrastruktur ist damit in einem begrenzten Bereich ein hochexakte Feststellung der Raumparameter in drei Dimensionen (Genauigkeit im mm-Berich) sowie die Identifikation von Tokens möglich.

Ultraschallbasierte Techniken zur Positionsbestimmung basieren auf dem Einsatz von Bakensendern an bekannten Positionen. Diese Sender werden zumeist an der Zimmerdecke montiert und senden periodisch einen Ultraschallimpuls aus. Dieser Impuls wird von den Tokens (die in diesem Fall aktive Bauteile mit Stromversorgung sind) empfangen, die daraufhin einen sie identifizierenden Impuls zurücksenden. Aus der Laufzeit zwischen Absetzen des Sendeimpuls und Empfangen des Antwortimpulses bei verschiedenen Baken lässt sich so die Position des Tokens im Raum feststellen. Problematisch ist hierbei jedoch die durch den auf sequentieller Zeitmessung basierenden Ansatz beschränkte Anzahl von verfolgbaren Tokens, wenn Echtzeit-Ansprüche gestellt werden. Zudem ist der Ansatz nicht robust gegen (akustisch) verdeckte Tokens. Eine Anfälligkeit gegenüber anderen Störeinflüssen besteht nicht. 

Für die Feststellung von Raumparametern sind ultraschall-basierte Systeme generell ausgezeichnet geeignet. Auch die Identifikation von Tokens ist prinzipiell möglich. Bei der Bewertung hinsichtlich des Einsatzes für tisch-basierte Systeme ist jedoch zu bedenken, dass eine drei-dimensionale Positionierung nicht zu den allgemeinen Anforderungen zählt und nur in speziellen Anwendungsfällen sinnvoll sein kann. Zudem kann die Notwendigkeit von stromversorgten Tokens einen Nachteil bzw. ein Hindernis beim Einsatz darstellen.

% paragraph akustisch (end)

\paragraph{Bewertung} % (fold)
\label{par:bewertung}

Im konkreten Anwendungsfall ist die Feststellung der Identität sowie der planaren Position und Rotation von mehreren Tokens in hoher Genauigkeit sowie in Echtzeit gefordert. Aus oben genannten Gründen sind kapazitive und elektromagnetische Systeme im Einzeleinsatz nur bedingt geeignet. Akustische Systeme erscheinen für den Anwendungsfall als zu aufwändig und unflexibel und stoßen außerdem bei der Anzahl der simultan zu verfolgenden Tokens an ihre Grenzen.

Die Kombination von kapazitiven und elektromagnetischen Systemen ist grundsätzlich eine Möglichkeit, die in Betracht gezogen werden könnte. Auch optische Systeme genügen den Anforderungen und kommen damit in Frage. Der kombinierte Ansatz ist im Vergleich mit optischen Systemen als robuster gegen Störeinflüsse aus der Umgebung zu betrachten. Für optische Systeme sprechen hingegen die weitaus geringeren Aufwände für Infrastruktur und Tokens sowohl bei Anschaffung als auch bei Wartung und Betrieb. Durch die geringere Komplexität des Systems sind auch weniger potentielle Fehlerquellen vorhanden, was bei der Erstellung des Werkzeug-Prototypen hilfreich ist. Aufgrund dieser Aspekte und einer vergleichbaren zur erwartenden Erkennungsleistung wurde für die hier vorgestellten Anwendungfalls die Entscheidung getroffen, ein optisches System zur Bestimmung der Positionsparameter sowie der Identität der Tokens einzusetzen.

% paragraph bewertung (end)

% subsubsection raumparameter (end)

\subsubsection{Tokenzustand} % (fold)
\label{ssub:tokenzustand}

Hinsichtlich des Tokenzustands sind im Kontext des hier vorgestellten Anwendungsfall Informationen zu erheben, die den Inhalt des Tokens betreffen. Wie in Kapitel XY beschrieben sind die Modellierungs-Tokens als Container ausgeführt, die geöffnet und geschlossen werden können und in die kleiner Tokens als Trägen von Zusatzinformation hineingelegt werden können. Die Auswahl eines Ansatzes, der die Identifikation des Öffnungs-Zustandes eines Tokens sowie dessen Inhalt erlaubt, ist Gegenstand dieses Abschnitts. Dazu wird grundlegend zwischen dem Einsatz von passiven Tokens und aktiven Tokens unterschieden. Passive Tokens besitzen keine zusätzliche Elektronik, die geforderten Informationen können lediglich durch die bereits vorhandene (optische) Infrastruktur festgestellt werden. Aktive Tokens werden hingegen mit zusätzlicher Elektronik zur Zustandsbestimmung ausgestattet, was allerdings eine Energieversorgung jedes Tokens bedingt.

\paragraph{Passive Token} % (fold)
\label{par:passive_token}
\index{Token!passive}
 
Bei passiven Tokens muss sichergestellt werden, dass die bereits vorhandene Infrastruktur die Zustandsänderungen eines Tokens erfassen kann. Das die vorhandene Infrastruktur auf optischen Technologien basiert, müssen sich alle Zustandsänderungen im äußeren - durch die Kamera erfassbaren - Erscheinungsbild eines Tokens wieder spiegeln.

Der Öffnungszustand eines Tokens kann durch Kameras einfach erfasst werden, wenn sich - je nach eigesetzter Technologie - durch das Öffnen der Umriss des Tokens verändert oder ein weiterer Code sichtbar wird bzw. der bestehende Code modifiziert wird. Diese Anforderung kann also durch passive Tokens erfüllt werden.

Zur Erfassung des Inhalts eines Container-Tokens sind zwei Ansätze denkbar. Einerseits kann der Inhalt eines Tokens zu einem bestimmten Zeitpunkt erfasst werden, andererseits ist auch eine Erfassung der Änderung des Tokeninhalts möglich (Erfassung des Vorgangs von Hineinlegen und Herausnehmen). Diese beiden Möglichkeiten sind hinsichtlich der Umsetzbarkeit mit passiven Token unterschiedlich zu beurteilen. Eine Erfassung das aktuellen Tokeninhalts ist mit optischen Systemen nur schwer möglich. Die einzige sich bietende Möglichkeit ist die von transparenten Teilbereichen der Außenfläche eines Tokens. Damit ist es grundsätzlich möglich, den Inhalt eines Tokens mit einer externen Kamera zu erfassen, sowohl bei feature- als auch code-basierten Ansätzen sind jedoch Verdeckungen, Verzerrungen oder zu geringe Kameraauflösung potentiell problematisch und lassen diesen Ansatz für den praktischen Einsatz als ungeeignet erscheinen.

Die Erfassung der Änderung des Tokeninhalts lässt sich mit optischen Systemen einfach implementieren. So kann der Vorgang des Hineinlegens als auch des Herausnehmens von einer Kamera erfasst werden. Die größte Herausforderung hierbei ist die Identifikation des Tokens, das eingebettet wird. Hier kann es wiederum durch Verdeckungen zu Erkennungsschwierigkeiten führen, was in diesem Fall einen permanent fehlerhaften Modellzustand zur Folge hat, der sich im Falle wiederholter Fehlerkennungen sogar inkrementell verschlimmern kann. Diesem Umstand kann lediglich durch eine explizite Aktion des Benutzers Rechnung getragen werden, der das betreffende einzubettende Token ins Sichtfeld der Kamera halten muss, bis das System Feedback über eine erfolgreiche Erkennung gibt. Diese Lösung erscheint allerdings hinsichtlich der Anforderung, die Technologie für den Benutzer vollkommen in den Hintergrund treten zu lassen, als eher suboptimal.

% paragraph passive_token (end)

\paragraph{Aktive Token} % (fold)
\label{par:aktive_token}
\index{Token!aktive}

Aktive Tokens beinhalten zusätzliche Sensorik, die die Erfassung des Tokenzustands ermöglicht. Derartige Tokens benötigen allerdings eine Energieversorgung und müssen über eine Möglichkeit zur Datenübertragung verfügen, um den Tokenzustand an das System zu übermitteln. Weiters ist im Allgemeinen eine Steuereinheit notwendig, um die Sensoren zu kontrollieren, die Daten zu aggregieren und letzendlich zu übertagen.

Im konkreten Fall einer optisch arbeitenden Infrastruktur bietet sich eine (ggf. aufladbare) Batterie als Energiequelle an, um im Kamerabild Verdeckungen durch ansonsten eventuell zu verwendende Kabel zu vermeiden. Eine Stromversorung über die Oberfläche (wie z.B. im Smart PINS (Gellersen REF) Ansatz vergestellt) scheidet hier aus, da die Blöcke dann mit Krafteinsatz auf die Oberfläche gesetzt werden müssten und nicht verschoben werden können. 

\index{SmartIT} 
Als Steuerungseinheit bietet sich neben selbst auf der Basis von Mikrocontrollern wie dem PIC oder 8051 (REFs) konzipierten Systemen auch Plattformen an, die explizit für den Anwendungszweck der Ansteuerung von Sensoren oder Aktuatoren und der Kommunikation mit einem Basissystem gefertigt werden. Exemplarisch kann hier die Smart-ITs-Plattform (REF) angeführt werden, die neben der flexiblen Ansteuerbarkeiten von unterschiedlichen Sensoren bereits Module zur Vernetzung untereinander und mit zentralen Diensten in der Infrastruktur anbietet.

\index{ZigBee} 
\index{Bluetooth} 
Aus den eben angeführten Gründen erscheint zur Datenübertragung eine drahtlos arbeitende Technologie am geeignetsten. Aufgrund der geringen benötigten Reichweite und der Anforderung, möglichst energieeffizient zu arbeiten, bieten sich die Technologien "Bluetooth" und "ZigBee" an. Bluetooth erreicht höhere Übertragungsraten, ist aber in der Anzahl der gleichzeitig verwendbaren Geräte (max. 7) für den hier vorgestellten Anwendungsfall zu beschränkt. Ein ZigBee-Netz kann mit bis zu 255 Geräten gleichzeitig arbeiten und ist außerdem im Einsatz energiesparender. Für den gegebenen Anwendungsfall erschiene also ZigBee als geeignete Technologie (und wurde auch bereits in (AON Cube, Simon Vogl REF) in einem ähnlichen Anwendungsfall erfolgreich eingesetzt).

Zur Feststellung des Öffnungsstatus eines Container-Tokens bieten sich bei aktiven Sensortechnologien mehrere Möglichkeiten an. Der Einsatz eines Schaltelements, das beim Öffnen den Kontakt herstellt oder unterbricht, erscheint als eine nahe liegende Lösung. Auch der Einsatz eines Drehelements am Angelpunkt des Öffnungsschaniers, dessen elektrische Eigenschaften (z.B. Widerstand oder Kapazität) mit dem Öffnungswinkel ändern, kann angedacht werden. Damit ist nicht nur eine Unterscheidung zwischen "offen" oder "geschlossen" sondern auch die Identifikation von Zwischenzuständen möglich.

Der Inhalt eines Container-Tokens kann ebenfalls mit unterschiedlichen Technologien erfasst werden. Die Zielsetzung ist hier nicht der Positionsbestimmung der eingebetteten Tokens sondern lediglich die Feststellung derer Identität. Naheliegend ist hierzu der Einsatz von elektromagnetischen Ansätzen wie oben beschrieben. Durch das Anbringen von z.B. RFID-Chips an den einzubettenden Tokens sowie eines Lesegeräts im Container-Token kann die Identifikation robust durchgeführt werden. Alternativ bieten sich Systeme an, die auf Gewichtsmessung basieren. Über einen in das Containertoken eingebauten Sensor wird dabei das Gesamtgewicht der eingebetteten Tokens bestimmt. Bei entsprechender Konzeption der einzubettenden Tokens (unterschiedliche Gewichte) kann aus dem Gesamtgewicht auf die tatsächlich enthaltenen Tokens geschlossen werden. Ein Nachteil dieses Ansatzes ist die beschränkte Anzahl von Tokens und die notwendige exakte Fertigung jedes einzelnen Tokens, da es bei Gewichtsabweichungen zu Fehlerkennungen kommt.

% paragraph aktive_token (end)

\paragraph{Bewertung} % (fold)
\label{par:zustand_bewertung}

Hinsichtlich der erreichbaren Flexibilität und zu erwartenden Servicequalität wäre in diesem Abschnitt eine Entscheidung zugunsten aktiver Tokens zu treffen. Im Gesamtkontext betrachtet und unter Berücksichtigung der Entscheidung für optische Systeme zur Bestimmung der Positionsparameter ist diese Wahl jedoch zu relativieren. Wie oben beschrieben, erlaubt eine auf optischen Systemen beruhende Infrastruktur grundlegend die Umsetzung der geforderten Funktionalität. Gleichzeitig wird die Komplexität des Systems massiv reduziert und die Erstellung zusätzlicher Tokens vereinfacht (da keine zusätzliche Elektronik notwendig ist). Der Wegfall von Energieversorgung und Sensorlogik in den Token reduziert deren Gewicht und ermöglicht gleichzeitig mehr Platz für einzubettende Tokens.

Für den hier beschriebenen Anwendungsfalls bzw. die prototypische Umsetzung des Werkzeugs wird deshalb auf aktive Tokens verzichtet und der Einsatz von passiven Tokens bevorzugt. Der Mehraufwand in Erstellung und Wartung des Systems beim Einsatz aktiver Tokens wiegt in der Gesamtheit betrachtet die zu erwartende höhere Erkennungsqualität nicht auf.
% paragraph zustand_bewertung (end)
% subsubsection tokenzustand (end)

% subsection potentielle_technologische_ansätze (end)
\subsection{In Frage kommende Frameworks} % (fold)
\label{sub:verfügbare_frameworks}

Unter Anbetracht der im vorherigen Abschnitt getroffenen grundlegender Technologieentscheidung zugunsten optischer Erkennungstechnologie mit passiven Tokens werden nun unterschiedliche Frameworks betrachtet, die die Umsetzung dieses Ansatzes erlauben. Es sind dabei zwei Klassen von Frameworks zu unterscheiden. \emph{Generische Frameworks für Tangible Interfaces} beschäftigen sich generell mit dem zur Verfügung stellen von Services, die Kopplung von Sensoren und Aktuatoren mit Interpretations-Logik und letzendlich konkreten Applikationen erlauben. Sie gehen dabei nicht auf konkrete  Sensortechnologie (wie die hier verwendeten optischen Ansätze) ein sondern versuchen eine Abstraktionsebene einzuführen, die die Applikationen von der konkreten Technologie entkoppelt und damit flexibler macht. \emph{Frameworks für video-basierten Input für Tangible Interfaces} sind hochspezialisierte Produkte, die konkret für die Umsetzung von optischen Ansätzen zur Eingabe von Daten bei Tangible Interfaces entwickelt werden. Ihr Vorteil liegt im durch die Spezialisierung im Allgemeinen geringeren Aufwand zur Einrichtung und auch während des Betriebs. Echtzeit-Anforderungen sind oft nur mit spezialisierten Frameworks zu erreichen. Eine Kombinationsmöglichkeit zwischen Produkten der beiden Kategorien ergibt sich beim Einsatz eines spezialisierten Frameworks als Eingabe-Modul für eine generisches Framework. Durch diesen Ansatz kann die einfache Inbetriebnahme spezialisierter Frameworks mit der Flexibiltiät generischer Frameworks zusammengeführt werden. 

\subsubsection{Generische Frameworks} % (fold)
\label{ssub:generische_frameworks}

Generische Frameworks zur Behandlung von Input und Output bei Tangible Interfaces sind historisch nicht exakt von anderen Frameworks abzugrenzen, die im Umfeld des Ubiquitous bzw. Pervasive Computing entwickelt wurden. Derartige Ansätze wurden erstmals im Zusammenhang mit "Context Computing" erwähnt, um Applikationen eine generische Möglichkeit zu bieten, Information aus der Umgebung über beliebige Sensoren zu erfassen und diese zu aggregieren und zu interpretieren. Aufbauend auf dieser Interpretation sollen Aussagen über den aktuellen Zustand der Umgebung (den "Kontext") getroffen werden können, die diese Applikationen zur Adaption benutzen können. Der Rückkanal, also die Ansteuerung von Aktuatoren, wurde erst in späteren Entwicklungen berücksichtigt. Die meisten Systeme dienen explizit nicht der Erstellung von marktreifen Applikationen sonderen widmen sich eher der Umsetzung von "Rapid Prototyping"-Ansätzen im Bereich der Tangible Interfaces. Begründet wird dies mit der oft sub-optimalen Ressourcen-Ausnutzung, die mit der Gerneralisierung und Flexiblisierung des Frameworks einhergeht. 

Die Aufzählung der hier beschriebenen Frameworks erhebt keinen Anspruch auf Vollständigkeit. Es wurde eher darauf geachtet, historische bzw. für den konkreten Anwendungsfall geeignete Ansätze aufzunehmen und in ihren wesentlichen Eigenschaften zu beschreiben.

\paragraph{Context Toolkit} % (fold)
\label{par:context_toolkit}
\index{Context Toolkit}
 
Das Context Toolkit \citep{Dey01} ist das historisch erste Framework, das versucht, die starre Verbindung zwischen Sensoren und Applikationslogik aufzubrechen und eine konfigurierbare Schicht einzuziehen, die eine schnellere, generischere Applikationsentwicklung ermöglicht und die Wiederverwendbarkeit einmal entwickelter Komponenten erhöht.

Konzeptuell existieren im Framework drei Arten von Komponenten: Context Widgets, Context Interpreters und Context Aggregators. Context Widgets implementieren die Ansteuerung beliebiger Software- und Hardwaresensoren und sind für das Sammeln von Information über die Umgebung zuständig. Sie vermitteln zwischen der physischen Umgebung und den konzeptuell höher liegenden Komponenten indem sie die unverarbeiteten Kontextdaten mittels einer geeigneten Schnittstelle kapseln und bestimmte Funktionen zur ersten Auswertung der Rohdaten ausführen. Eine Anwendung kann diese Daten verwenden, ohne dass sie Detailkenntnisse über die zugrunde liegenden Sensortechnologien haben muss. Context Interpreter aggregieren die Sensordaten zu komplexeren Kontextinformationen d.h. sie konvertieren und interpretieren die Daten mehrerer Context Widgets und versuchen diese zu einheitlichen Clustern zusammenzufassen. Context Aggregators dient der Zusammenführung verschiedener Kontextinformationen die für bestimmte Anwendungen relevant sind. Context Aggregators bilden damit die Schnittstelle zu den eigentlichen Applikationen.

Das Context Toolkit bietet nicht nur eine Softwareschnittstelle zu physischen Sensoren, es trennt auch die Akquisition und Repräsentation von der Auslieferung der Daten an kontextsensitive Applikationen.
% paragraph context_toolkit (end)

\paragraph{SiLiCon Context Framework} % (fold)
\label{par:silicon_context_framework}

Das SiLiCon Context Framework \citep{Beer03} ist ein Vertreter jener Klasse von Frameworks, deren Verhalten zur Laufzeit dynamisch konfigurierbar sind. Dies bedeutet im konkreten Fall, dass Applikationen die auf Basis des SiLiCon Context Framework erstellt wurden ihr Verhalten und ihren Aufbau aufgrund eintretender Ereignisse verändern können. Dies betrifft sowohl das Aktivieren und Deaktivieren von Input- und Output-Kanälen also auch die Interpretation der eingehenden Information und die Reaktion darauf. Das Framework wurde entworfen, um Szenarien zu beschreiben, in denen interaktive Systeme kontextsensitiv - d.h. abhängig vom aktuellen Zustand ihrer Umwelt - reagieren müssen.

Die grundlegenden Bausteine des SiLiCon Context Framework sind "Entiäten", die Objekte der realen Welt konzeptuell abbilden. Diese "Entitäten" besitzen "Attribute", also Eigenschaften, mit Hilfe derer die Entität näher beschrieben wird. Über "Attribute" kann die Wahrnehmung einer Entität von deren Umwelt sowie deren Interaktionsmöglichkeiten mit derselben beschrieben werden. Mit Hilfe von "ECA"-Regeln (\emph{E}vent-\emph{C}ondition-\emph{A}ction) wird beschrieben, auf welche Wahrnehmung der Umwelt (Event) eine Entität unter welchen Bedingungnen (Condition, formuliert auf Basis des internen Zustands der Entität) mit welchen Aktivitäten (Action) reagiert. Diese Regeln können zur Laufzeit dynamisch verändert und nachgeladen werden. Außerdem ist es möglich, in "Actions" das Nachladen von Entitäten oder das Hinzufügen oder Entfernen einzelner Attribute durchzuführen \citep[][S. 90]{Oppl04}.



% paragraph silicon_context_framework (end)

\paragraph{Papiermaché} % (fold)
\label{par:papiermaché}
Eines der ersten Rapid-Prototyping Frameworks
% paragraph papiermaché (end)

\paragraph{TUIpist} % (fold)
\label{par:tuipist}

Das TUIpist-Framework \citep{Furtmuller07} \citep{Furtmuller07a}

% paragraph tuipist (end)

% subsubsection generische_frameworks (end)

\subsubsection{Frameworks für video-basierten Input} % (fold)
\label{ssub:frameworks_für_video_basierten_input}
\begin{itemize}
	\item ReacTIVision
	\item ARToolkit
	\item Visual Codes (ETH)
\end{itemize}

% subsubsection frameworks_für_video_basierten_input (end)

% subsection verfügbare_frameworks (end)

\subsection{Technologieentscheidung} % (fold)
\label{sub:technologieentscheidung}
-> ReacTIVision
% subsection technologieentscheidung (end)
% section möglichkeiten_zur_erfassung_von_benutzerinteraktion (end)

\section{Konzeption und Umsetzung der Hardwarekomponenten} % (fold)
\label{sec:konzeption_und_umsetzung_der_hardwarekomponenten}

\subsection{Überblick} % (fold)
\label{sub:Überblick}
Grafik aus dem TEI-Paper
Zerlegbarkeit

% subsection Überblick (end)

\subsection{Tokens \& Input-Werkzeuge} % (fold)
\label{sub:tokens_&_input_werkzeuge}

% subsection tokens_&_input_werkzeuge (end)

\subsection{Input auf der Tischoberfläche} % (fold)
\label{sub:input_auf_der_tischoberfläche}

Semitransparente Oberfläche, Projektion von unten (Umlenkspiegel), Kamera von unten - Überleitung zur 
Illumination via Interferenz zwischen Beamer und Kamera

% subsection input_auf_der_tischoberfläche (end)

\subsection{Illumination und Umgebungslichtabhängigkeit} % (fold)
\label{sub:illumination_und_umgebungslichtabhängigkeit}

% subsection illumination_und_umgebungslichtabhängigkeit (end)
% section konzeption_und_umsetzung_der_hardwarekomponenten (end)

\section{Erfassung der Benutzerinteraktion durch Software} % (fold)
\label{sec:erfassung_der_benutzerinteraktion_durch_software}

% section erfassung_der_benutzerinteraktion_durch_software (end)

\section{Interpretation der Rohdaten und Stabilisierung der Erkennungsleistung} % (fold)
\label{sec:interpretation_der_rohdaten_und_stabilisierung_der_erkennungsleistung}

% section interpretation_der_rohdaten_und_stabilisierung_der_erkennungsleistung (end)
% chapter input_&_interpretation (end)