\part*{}

\chapter{Schlussbetrachtungen} % (fold)
\label{cha:schlussbetrachtungen}

\section{Anwendungsszenarien} % (fold)
\label{sec:anwendungsszenarien}

\subsection{Problembeschreibung und Arbeitsabstimmung}

Einzel- oder Gruppensessions. 

Aufgabenstellung: meist aus Arbeitsabläufen der beteiligten Pesonen

Merkmal: Tisch ist Mittel zum Zweck, gelegtes Modell fungiert als Diskussionsgrundlage. Modell ist statisch, wird einmal gelegt und nicht mehr verändert. Eher kompakte Modelle, die den Kontext eines Problems beschreiben. Die eigentliche Problematik ist selten explizit im Modell dargestellt.

Anwendungsbeispiele: Block 1, Block 2, Block 4 (Session 1-3).

Vorteile:

\subsection{Concept Mapping}

Einzel- oder Gruppensessions. 

Aufgabenstellung: zur Erhebung bzw. Überprüfung von domänenspezifischen (Struktur-)Wissen

Merkmal: Tisch 

Anwendungsbeispiele: Block 3, Block 5

\subsection{Strukturaufstellung und Manipulation}

Anwendungsbeispiele: Block 4 (Session 4).

AUfgabenstellung: Erhebung und Reflexion der Strukturen in denen Arbeitsabläufe situiert sind (Abteilungen, Personen, Kommunikationskanäle).

Merkmal: Modelle sind nicht statisch - werden nach der Erstellung zwar meist nicht erweitert aber in ihrer Struktur verändert (räumliche Relation der Knoten zueinander). 

% chapter schlussbetrachtungen (end)