\part*{}

\chapter{Schlussbetrachtungen} % (fold)
\label{cha:schlussbetrachtungen}

In diesem Kapitel werden die Inhalte der gesamten Arbeit zusammengefasst, die Ergebnisse zusammengeführt und einander gegenübergestellt. Ziel dieses Kapitels ist es, letztendlich zu einer Beurteilung der Arbeit hinsichtlich Erfüllung der globalen Zielsetzung zu gelangen. Die Darstellung von weiterem Entwicklungspotential des Werkzeugs schließt die Arbeit ab.

\begin{figure}[htbp]
	\centering
		\includegraphics[scale=0.6]{img/Kontextgrafiken/k15.png}
	\caption{Kapitel „Schussbetrachtungen“ im Gesamtzusammenhang}
	\label{fig:img_Kontextgrafiken_k15}
\end{figure}

Abbildung \ref{fig:img_Kontextgrafiken_k15} zeigt die Kapitel, deren Ergebnisse in dieses Kapitel einfließen. Die einzelnen Inhalte werden schrittweise einander strukturiert gegenüber gestellt, was eine Beurteilung der Zielerreichung auf jeder Betrachtungsebene der Arbeit (Fragestellungen der empiririschen Untersuchung, Anforderungen an das Werkzeug, Globale Zielsetzung) ermöglicht. Nach einer Darstellung der gesamten dieser Arbeit zugrundeliegenden Konzepte und deren Zusammenhang in der Arbeit in Abschnitt \ref{sec:überblick_über_den_gesamtzusammenhang} werden die einzelnen oben genannten Betrachtungsebenen in umgekehrter Reihenfolge als in der Arbeit dargestellt (also vom Konkreten zum Allgemeinen) einzeln beschrieben. Nach einer Zusammenfassung der Evaluierungsergebnisse und einer Gegenüberstellung mit den Ergebnissen der konzeptuellen Einordnung in Abschnitt \ref{sec:zusammenfassung_der_evaluierung} werden diese in Abschnitt \ref{sec:erfüllung_der_anforderungen_an_das_werkzeug} den in Kapitel \ref{cha:anforderungen} angeführten Anforderungen an das Werkzeug gegenübergestellt. Im letzten Schritt werden die erreichten Ergebnisse hinsichtlich der globalen Zielsetzung bewertet (siehe Abschnitt \ref{sec:bewertung_hinsichtlich_der_globalen_zielsetzung}). Auf Basis dieser Ausführungen ist es in Abschnitt \ref{sec:offene_aspekte_und_entwicklungspotential} möglich, die noch offenen Aspekte dieser Arbeit und weiteres Entwicklungspotential zu identifizieren.

\section{Überblick über den Gesamtzusammenhang} % (fold)
\label{sec:überblick_über_den_gesamtzusammenhang}

In Kapitel \ref{cha:einführung} wurde in Abbildung \ref{fig:img_ArticulationWork_ArbeitInteraktion} die dieser Arbeit zugrunde liegende Konzeptualisierung von Arbeit in „Production Work“ und „Articulation Work“ dargestellt. Auf Basis der Ausführungen in den dazwischen liegenden Kapiteln kann diese Abbildung wie in Abbildung \ref{fig:img_Schlussbetrachtungen_ArbeitInteraktionMentaleModelleTabletop} dargestellt erweitert werden und stellt so die gesamten konzeptionellen Grundbegriffe im Zusammenhang dieser Arbeit dar. 

\begin{figure}[htbp]
	\centering
	\includegraphics[width=0.9\textwidth]{img/Schlussbetrachtungen/ArbeitInteraktionMentaleModelleTabletop.png}
	\caption{Gesamtzusammenhang der verwendeten Konzepte}
	\label{fig:img_Schlussbetrachtungen_ArbeitInteraktionMentaleModelleTabletop}
\end{figure}

Das Ziel dieser Arbeit ist die Unterstützung der Auflösung von Situationen, in denen produktive Arbeit nicht (mehr) möglich ist. Dies ist dann der Fall, wenn unklar ist, wie die Zielerreichung gewährleistet werden kann. Diese Arbeit basiert auf der Annahme, dass Interaktion zwischen Individuen immer ein wesentlicher Aspekt bei der Durchführung von Arbeit ist. Arbeit enthält deshalb auch immer einen Anteil, der mit der Herstellung und Aufrechterhaltung der Interaktion der beteiligten bzw. betroffenen Individuen beschäftigt ist. Dieser Arbeitsanteil wird als „Articulation Work“ bezeichnet. Jedes beteiligte Individuum entwickelt auf Basis früherer Erfahrungen oder Lernprozessen Erklärungsmodelle (oder „Mentale Modelle“) für adäquate Aktivitäten bzw. Reaktionen im betreffenden Arbeitsablauf. Wesentlich für die erfolgreiche Interaktion mehrerer Individuen in einem Arbeitsablauf ist die Abstimmung dieser Erklärungsmodelle und die Entwicklung einer gemeinsamen Sichtweise auf jenen Teil des Arbeitsablaufs, in dem zusammengearbeitet werden muss. Diese Abstimmungsprozesse laufen implizit immer ab, wenn Individuen interagieren. In bestimmten, als komplex oder problematisch wahrgenommenen Arbeitssituationen reicht die implizite Abstimmung nicht mehr aus -- es ist notwendig, „Articulation Work“ explizit anzustoßen und den Abstimmungsvorgang bewusst zu unterstützen. Ein wesentlicher Schritt zur Abstimmung mentaler Modelle ist deren Externalisierung, also deren Abbildung in einer kommunizierbaren Form. Dazu werden unterschiedliche Methoden vorgeschlagen, denen gemein ist, dass die Abbildung in Form diagrammatischer Modelle erfolgt. In der konkreten Umsetzung unterscheiden sich die beiden Methoden „Strukturlegetechnik“ und „Concept Mapping“ voneinander, bieten aber beide Vorteile bei der Unterstützung der für „Articulation Work“ wichtigen Kommunizierbarkeit mentaler Modelle. In dieser Arbeit werden deshalb beide Ansätze berücksichtigt und in einer Methodik zusammengeführt, die an die im Rahmen von Strukturlegetechniken vorgeschlagene „Dialog-Konsens Methodik“ angelehnt ist. Generell hat bei der Konzeption der Werkzeugunterstützung der bei Strukturlegetechniken vorgeschlagene Ansatz das Primat, weil in ihm die kooperative Durchführung der Modellbildung explizit vorgesehen ist, was der Abstimmung mentaler Modelle eher entgegenkommt als der grundsätzlich eher individuell orientierte Concept-Mapping-Ansatz. Letztendlich wird also in dieser Arbeit der durch Strukturlegetechniken vorgeschlagene Ansatz des physischen, kooperativen Abbildung von Modellen verfolgt, weshalb die Abbildung der Modelle im physischen Raum, konkret auf eine kooperativ bearbeitbaren Tischoberfläche erfolgt. Die Berücksichtigung der Anforderungen des Concept Mapping begründet letztendlich die Notwendigkeit der Hinterlegung der physischen Modellierungsmöglichkeit mit computergestützten Werkzeugen. Um den Anspruch der physischen, kooperativen Modellbildung mit den computergestützten Unterstützungswerkzeugen zu vereinen, wird zur technologischen Umsetzung des Werkzeugs ein „Tabletop Interface“ verwendet, in dem beide Aspekte berücksichtigt und verknüpft werden können.

% section überblick_über_den_gesamtzusammenhang (end)

\section{Zusammenfassung der Evaluierung}
\label{sec:zusammenfassung_der_evaluierung}

In diesem Abschnitt wird die in den Kapiteln \ref{cha:eval_ueberblick}, \ref{cha:eval_werkzeug}, \ref{cha:eval_modell} und \ref{cha:eval_aw} beschriebene empirische Untersuchung zusammengefasst und den Ergebnissen der konzeptuellen Einordnung in Kapitel \ref{cha:konzeptionelle_evaluierung} gegenübergestellt.

\subsection{Empirische Untersuchung}

In der empirischen Untersuchung waren folgende in Kapitel \ref{cha:eval_ueberblick} formulierte Untersuchungsfragen zu beantworten:

\begin{itemize}
 \item Sind das Werkzeug und dessen Komponenten verständlich und wie intendiert einsetzbar? (Aspekt: Werkzeug)
 \item Erlauben Werkzeug und Methode die Abbildung semantisch offener diagrammatischer Modelle? (Aspekt: Modell)
 \item Unterstützen Werkzeug und Methode Articulation Work? (Aspekt: Articulation Work)
\end{itemize}

Jede dieser Fragen wurde in einem separaten Kapitel bearbeitet. Die Untersuchungsfragen wurden in Hypothese konkretisiert, die den jeweiligen Untersuchungsgegenstand in Bezug zu den aus den konzeptuellen Grundlagen abgeleiteten Anforderungen an die Unterstützung von Articulation Work stellen.

Bei der Betrachtung der ersten Untersuchungsfrage wurden 8 Hypothesen geprüft, die die grundlegenden Anforderungen an das Werkzeug abdecken bzw. die Verständlichkeit und Verwendbarkeit der implementierten Funktionen testen. Insgesamt scheint das Werkzeug für den intendierten Verwendungszweck -- der kooperativen Erstellung von diagrammatischen Modellen -- in unterschiedlichen Anwendungsgebieten einsetzbar zu sein. Stabilitätsprobleme in der technischen Umsetzung führten jedoch in den ersten Phasen der Evaluierung zu Behinderungen bei der Modellbildung, was jedoch durch nachträglich durchgeführte Verbesserungen weitgehend kompensiert werden konnte. Herausforderungen zeigten sich im Interaktionsdesign der über die Kernfunktionalität hinausgehenden Funktionen zur Unterstützung des Modellbildungsprozesses. Die aus der Literatur begründbare Funktion zur Verfolgung der Modellierungshistorie und der Wiederherstellung vergangener Modellzustände wurde kaum genutzt. Auch die Verwendung der Funktion zur Entfernung von unerwünschten Verbindern im Modell war den Benutzern in der ersten Version unverständlich. Dieses Problem konnte durch ein Redesign des entsprechenden Teilwerkzeugs und dessen Bedienung beseitigt werden. Generell scheint das Werkzeug schnell erlernbar zu sein, so dass die Anzahl der Fehlbedienungen durch Missverständnisse bereits bei der zweiten Anwendung des Werkzeugs durch die Benutzer massiv reduziert bzw. nicht mehr vorhanden war. Die oben formulierte Frage kann also mit Vorbehalten positiv beantwortet werden. Generell scheint das Werkzeug wie intendiert einsetzbar und zum Großteil verständlich zu sein, einige Komponenten weisen jedoch Defizite in der Verständlichkeit auf.

Für die zweite Untersuchungsfrage wurden 5 Hypothesen geprüft, die sich auf die Verwendung der Werkzeugs zur Modellbildung im Sinne der vorgeschlagenen Methodik zur Unterstützung von „Articulation Work“ beziehen. Insgesamt scheint das Werkzeug zwar nur eingeschränkt für die allgemeine Abbildung von Modellen beliebigen Inhalts und beliebiger Semantik geeignet zu sein, seinen Verwendungszweck für die auf Modellen basierende Kommunikation und Abstimmung von individuellen Sichtweise scheint das Werkzeug aber geeignet zu sein. Die eingeschränkte Eignung für die Abbildung beliebiger Modelle scheint vor allem in der beschränkten Größe der Modellierungsoberfläche begründet zu liegen, für die die Möglichkeit zur Einbettung von Teilmodellen keinen adäquaten Ersatz darzustellen scheint. Zudem scheint in Einzelfällen die in der aktuellen Hardware-Implementierung vorhandene Beschränkung auf drei semantisch unterschiedliche Modellelementtypen einschränkend wahrgenommen zu werden, was ebenfalls dem Anspruch eines semantisch vollständig offen Modellierungswerkzeugs widerspricht. Dies ist insofern zu relativieren, als dass die Anzahl der Elementtypen in den meisten Fällen zur Kommunikation der individuellen Sichtweisen auszureichen scheint und lediglich in Fällen zu gering wahrgenommen wird, wo eine „vollständige“ Abbildung eines Sachverhaltes angestrebt wurde. Generell scheint das Werkzeug deshalb bei der Modellierung für den intendierten Verwendungszweck in einem Großteil der Anwendungsfälle geeignet zu sein.

In der dritten Untersuchungsfrage wurde letztendlich geprüft, ob das Werkzeug im Sinne der globalen Zielsetzung tatsächlich „Articulation Work“ unterstützt. Dazu wurden 2 Hypothesen gebildet, die die zu erwartenden Wirkungen erfolgreicher „Articulation Work“ abdecken. Die unmittelbare Wirkung der Durchführung von „Articulation Work“ ist die Bildung eines gemeinsamen Verständnisses über den betrachteten Arbeitsgegenstand. Diese Wirkung konnte in der Untersuchung nachgewiesen werden, das Werkzeug erfüllt also diesen Teil der Anforderungen. Mittelbar sollte die Durchführung von „Articulation Work“ auch Auswirkungen auf die betrachteten Arbeitsabläufe selbst haben und die Interaktion im Idealfall verbessern. Derartige Wirkungen konnten jedoch in der Untersuchung nicht nachgewiesen werden. Einige Beobachtungen in Einzelfällen weisen auf eine entsprechende Wirkung hin, insgesamt kann diese aber nicht generalisiert werden. Ob dies am Werkzeug selbst festzumachen ist oder die in der Untersuchung betrachteten Arbeitsabläufe nicht optimal gewählt wurden, wird in weiteren Untersuchung zu prüfen sein.

\subsection{Gegenüberstellung der empirischen und konzeptuellen Untersuchung}

In diesem Abschnitt werden die Ergebnisse der empirischen Untersuchung den in in Abschnitt \ref{sub:verbesserungspotential_für_das_werkzeug} angeführten Konsequenzen für die Werkzeuggestaltung aus dessen konzeptueller Untersuchung gegenübergestellt.

In Kapitel \ref{cha:konzeptionelle_evaluierung} wurde das Werkzeug im Vorfeld der praktischen Untersuchung in die in Abschnitt \ref{sec:konzeptualisierungen_von_tangible_interfaces} beschriebenen Frameworks zur Beschreibung von Tangible Interfaces eingeordnet. Daraus konnte in Einzelaspekten Verbesserungpotetial für das Werkzeug abgeleitet werden, falls die tatsächliche Umsetzung einer Funktionalität des Werkzeugs nicht mit den konzeptionell begründbaren Designentscheidung übereinstimmte. In diesem Abschnitt wird nun angeführt, in welchen Fällen das Verbesserungpotential, d.h. im Umkehrschluss eine Schwäche der aktuellen Implementierung, in der empirischen Untersuchung bestätigt werden konnte. Die Ergebnisse erlauben eine qualitative Aussage über die Eignung der betrachteten Frameworks für die Konzeption von Tangible Interfaces.

\subsubsection{Missverständlichkeit des Löschtokens}

Die ursprüngliche Konzeption des Löschtokens zeigte eine Diskrepanz zwischen dessen wahrgenommener Verwendung und dem tatsächlichen Vorgehen bei dessen Einsatz. Diese Schwäche konnte aus mehreren Frameworks abgeleitet werden, da bei konsistenter Anwendung derselben das Interaktionsdesign anders als tatsächlich umgesetzt hätte ausfallen müssen.

Tatsächlich konnte die Mißverständlichkeit der Löschtokens in der empirischen Untersuchung bestätigt werden (siehe Abschnitt \ref{sub:verwendung_des_löschtokens}). Nach einem Redesign der Verwendung des Löschtokens unter Berücksichtigung der konzeptuell indizierten Gestaltungs-Kriterien traten keine Mißverständnisse mehr auf, das Ausmaß des Einsatzes stieg signifikant an.


\subsubsection{Schwache Ein-Ausgabe-Kopplung bei der Modellierungshistorie}

Nach der Aktivierung der Modellierungshistorie wird diese ausschließlich durch ein Werkzeug auf der Modellierungsoberfläche kontrolliert. Die Ausgabe erfolgt jedoch ausschließlich auf dem sekundären Ausgabekanal, auf der Tischoberfläche erfolgt kein visuelles Feedback weder über die Aktivierung des Historienmodus noch über deren aktuellen Zustand (konkret die aktuelle Position in der Zeitlinie). Nach \citep{Ullmer00} wäre diese schwache Kopplung zu hinterfragen.

In der empirischen Untersuchung konnte dieser Aspekt nicht als Schwachstelle bestätigt werden. Die Entkopplung von Kontrolle und Ausgabe in diesem Anwendungsfall wurde von keinem Teilnehmer als missverständlich wahrgenommen und führte auch nie zu Fehlbedienungen. Dies mag daran liegen, dass der sekundäre Ausgabekanal räumlich nahe an der Tischoberfläche angeordnet war und von vielen Teilnehmern bei der Modellerstellung ohnehin als primäre Informationsquelle -- noch vor der Tischoberfläche selbst -- genutzt wurde. Erst bei der Diskussion verlagerte sich der Fokus der Aufmerksamkeit hin zu dem auf der Tischoberfläche gelegten Modell. Die von der herkömlichen Interaktionsform mit Maus und Tastatur vertraute entkoppelte Ein- und Ausgabe scheint in diesem Fall Bedienungsprobleme vermieden zu haben. Dass keine Desorientierung der Benutzer beim Wechsel in den Historienmodus auftrat, scheint damit zu begründen sein, dass der sekundäre Ausgabekanal permanent aktiv war, den Modellzustand immer synchron darstellte und deshalb wie oben erwähnt oft ohnehin als primärer Informationskanal genutzt wurde.  

\subsubsection{Lange Antwortzeiten auf Interaktionen}

In der ersten Implementierung des Systems kam es vor allem bei der Interaktion zur Herstellung von Verbindern zu Verzögerungen in der Reaktion des Systems. Diese Verzögerungen sollten nach \citep{Bellotti02} vermieden werden, um den Benutzern adäquat Rückmeldung über deren Interaktionswunsch geben zu können.

In der empirischen Untersuchung konnte diese Schwachstelle bestätigt werden. Die Möglichkeit zur Herstellung von Verbindern wurde in der ursprünglichen Implementierung kaum genutzt. Erst als eine weitere Möglichkeit zur Herstellung von Verbindern implementiert wurde, deren wahrgenommer Reaktionszeit geringer war, wurde diese Möglichkeit in signifikant höherem Ausmaß genutzt (siehe Abschnitt \ref{sub:herstellung_von_verbindern}).

\subsubsection{Fehlende einfache Undo-Funktion}

Die Möglichkeit zur Wiederherstellung von vergangenen Modellzuständen ("Undo") ist in der aktuellen Implementierung zwar gegeben, bedingt aber eine zeitaufwändige und in mehreren Schritten durchzuführende Interaktion mit dem System. Dies sollte nach \citep{Bellotti02} vermieden werden, auch das Schema von \citep{Shaer04} lässt hier Inkonsistenzen mit den den ansonsten ausschließlich einschrittigen Interaktionen mit dem Werkzeug erkennen.

In der empirischen Untersuchung wurde die Möglichkeit zur Wiederherstellung zwar verwendet, das Ausmaß des Einsatzes sank aber signifikant, als das Löschtoken neu gestaltet wurde und so eine einschrittige Interaktion zur Korrektur von fehlerhaften Verbindungen (dem häufigsten Grund des Einsatz der Undo-Funktion) geschaffen wurde. Die Gestaltung der Wiederherstellungsfunktion kann damit als Schwachstelle bestätigt werden.

\subsubsection{Mangelnde Verständlichkeit des Snapshot-Tokens}

Das Design des Snapshot-Tokens weist in der derzeitigen Umsetzung nicht auf dessen Funktion hin, was nach bei einer Einordnung in die Taxonomie nach \citep{Fishkin04} gegenüber den anderen Werkzeugen inkonsistent ist. Das Token suggeriert zwar nicht wie im Fall des Löschtoken eine andere Funktionalität, gibt aber den Benutzern durch sein Erscheinungsbild keinen Hinweis auf die Verwendung (mangelhafte Affordances \citet{Norman90}).

Tatsächlich zeigt sich in der empirischen Untersuchung, dass die Funktion zur expliziten Erfassung von Snapshots kaum bzw. in den meisten Anwendungen nicht verwendet wird. Trotz einer erklärenden Einführung in das Werkzeug vor der Modellbildung bestehen häufig Unklarheiten zur Verwendung des Tokens, sobald Benutzer während der Anwendung damit konfrontiert werden. Auch diese Designentscheidung ist damit als Schwachstelle zu bestätigen. 

\subsubsection{Funktional belegte Bereich der Modellierungsoberfläche}

Der hier beschriebene Aspekt ist keine Schwachstelle des Werkzeugs sondern eine Erweiterungsmöglichkeit, die auf Basis der Ausführungen von \citep{Ishii97} identifiziert wurde. Die Autoren schlagen vor, Teile der Interaktionsoberfläche („Trays“) mit Funktionen zu belegen, die eine bestimmte Operation auf einem auf ihm platzierten Token ausführen (etwa: „Details anzeigen“). Bei der Umsetzung des Werkzeugs wurde diese Möglichkeit nicht berücksichtigt, es sind jedoch sinnvolle Anwendungsmöglichkeiten denkbar.

In der empirischen Untersuchung wurden derartige Funktionen von keinen Benutzern erwartet oder gefordert. Da diese Form der Interaktion aber nicht der gängigen Desktop-Metapher entspricht, ist dies auch nur eingeschränkt zu erwarten. Es kann damit auf Basis der Untersuchung keine Aussage über die Notwendigkeit oder Sinnhaftigkeit einer derartigen Erweiterung getroffen werden.

\subsubsection{Verwendbarkeit frei wählbarer Tokens}

Zur Anpassung des Modellierungswerkzeugs an die jeweilige Anwendungsdomäne führt \citep{Holmquist99} als Möglichkeit die Verwendung von domänenspezifischen Token an. Diese Forderung ist grundsätzlich mit den Ansprüchen der semantischen Offenheit des Werkzeugs und der domänenübergreifenden Anwendbarkeit des Ansatzes vereinbar und stützt diese. In der derzeitigen Implementierung ist die Zahl der unterschiedlichen Modellelementtypen auf drei begrenzt. Diese Elementtypen sind in generischen Formen ausgeführt und per se nicht intuitiv domänenspezifisch interpretierbar.

In der empirischen Untersuchung wurde mehrfach die Forderung nach einer höheren Anzahl von unterschiedlichen Elementtypen geäußert, was aber die hier betrachtete Erweiterungsmöglichkeit nur am Rande betrifft (die Verwendung beliebiger Elementtypen würde deren Anzahl zwar erhöhen, umgekehrt implizit die Forderung nach mehr Elementtypen nicht jene nach beliebigen Ausprägungen derselben). Tatsächlich wurde aber in einigen Anwendungen explizit der Wunsch geäußert, Objekt aus dem spezifischen Anwendungsfall (etwa Dokumente oder Werkzeuge) als Tokens verwenden zu können. Dies entspricht im Wesentlichen der oben beschriebenen Erweiterungsmöglichkeit, weshalb die Erweiterungsmöglichkeit als sinnvoll erachtet werden kann. 

\section{Erfüllung der Anforderungen an das Werkzeug}
\label{sec:erfüllung_der_anforderungen_an_das_werkzeug}

In diesem Abschnitt werden die in Kapitel \ref{cha:anforderungen} formulierten Anforderungen an das Werkzeug hinsichtlich ihrer Erfüllung betrachtet. Die Beurteilung der Erfüllung erfolgt anhand der empirischen Ergebnisse, die in den Kapiteln \ref{cha:eval_werkzeug}, \ref{cha:eval_modell} und \ref{cha:eval_aw} beschrieben wurden.

Anforderung \ref{anf:physische_abbildung_legen_beliebiger_diagrammatischer_modelle} (Physische Abbildung beliebiger diagrammatischer Modelle) wurde in den Hypothesen \ref{hyp:diagmodelle}, \ref{hyp:behinderung} und \ref{hyp:gewöhnung} untersucht. Die grundlegende Abbildung diagrammatischer Modelle mit dem Werkzeug ist ohne Einschränkung der Anwendungsdomäne möglich. Bei der Untersuchung des Prozesses der physischen Abbildung der Modelle konnten in den ersten durchgeführten Evaluierungsblöcken behindernde Aspekte identifiziert werden. Diese hatten Einfluss auf die Abbildung der Modelle, insbesondere die Möglichkeit des Einsatzes von Verbindern im Modell wurde nur eingeschränkt wahrgenommen. Nach Überarbeitungen des Interaktionsdesigns und der Implementierung mehrerer Maßnahmen zur Steigerung der Robustheit gegenüber Fehlerkennungen von Benutzereingaben konnten die als behindernd wahrgenommenen Faktoren reduziert werden. Bei mehrmaliger Verwendung des Werkzeugs zeigt sich eine Verbesserung des Umgangs mit den zur Verfügung gestellten Ausdrucksmöglichkeiten und eine stärke Fokussierung auf den zu repräsentierenden Sachverhalt. Insgesamt kann diese Anforderung als erfüllt angesehen werden.

Anforderung \ref{anf:unterstützung_der_iterativen_aushandlung_des_modells} (Unterstützung der iterativen Aushandlung des Modells) wurde in der Hypothese \ref{hyp:abstimmung} untersucht. Ziel der Unterstützung von Aushandlungsprozessen ist die Bildung einer gemeinsamen Sichtweise auf das betrachtete Problem. Die kooperative Modellierung ist durch das Design von Hard- und Software grundsätzlich möglich, durch die Verwendung eines Tabletop Interface wird die kooperative Modellbildung und der Austausch über den Modellierungsgegenstand (also das betrachtete Problem) ermöglicht. Die beabsichtigte Wirkung der Abstimmung der individuellen Sichtweisen konnte in der empirischen Untersuchung bestätigt werden. Insgesamt kann diese Anforderung also bestätigt werden.

Anforderung \ref{anf:ermöglichung_experimenteller_veränderungen_am_modell} (Ermöglichung experimenteller Veränderungen am Modell) wurde in der Hypothese \ref{hyp:wiederherstellung} untersucht. Experimentelle Veränderungen am Modell werden durch die Unterstützung der Wiederherstellung von vergangenen Modellzuständen realisiert. Diese Funktion wurde technisch implementiert und hinsichtlich ihrer Funktionsfähigkeit getestet. Diese ist vollständig gegeben. In der empirischen Untersuchung wurde die Funktionalität von den Teilnehmern jedoch nicht genutzt, so dass die Erfüllung der Anforderung empirisch nicht bestätigt werden kann. Aus den Ergebnissen der Untersuchung ist vielmehr zu hinterfragen, ob diese aus der Theorie der Externalisierung mentaler Modelle abgeleitete Anforderung in der Praxis tatsächlich relevant ist.

Anforderung \ref{anf:nicht_vorgegebene_semantik_der_modellierungselemente} (Nicht vorgegebene Semantik der Modellierungselemente) wurde in den Hypothesen \ref{hyp:kontexte} und \ref{hyp:keine_einschränkung} untersucht. Durch die nicht vorgegebene Semantik der Modellierungselemente wird einerseits der Einsatz des Werkzeugs in unterschiedlichen Anwendungskontexten ermöglicht, andererseits werden Benutzer nicht gezwungen, ein vorgegebenes Abbildungsschema für die von ihnen auzudrückende Information zu verwenden. Dies ermöglicht ein Fokussierung auf die Modellierungsinhalte und die Kommunikation derselben und vermeidet einen zusätzlichen kognitiv belastenden Übersetzungsschritt. Technisch wurde diese Anforderung durch die Verwendung generischer (also semantisch nicht vorbelegter) Modellierungsbausteine in Kombination mit dem Einsatz von Topic Maps zur Repräsentation der Modelle umgesetzt. Das Werkzeug erlaubt grundsätzlich die Verwendung von beliebigen und beliebig vielen unterschiedlichen Modellierungsbausteinen, diese Möglichkeit wurde jedoch im vorliegenden Prototypen nicht eingesetzt. Die empirische Untersuchung konnte die Einsetzbarkeit in unterschiedlichen Anwendungskontexten ohne Einschränkung bestätigen. Die Modellierung selbst wurde jedoch durch die im Prototypen auf drei unterschiedliche Modellierungselemente eingeschränkte Ausdrucksstärke des Werkzeugs teilweise als einschränkend empfunden. Da dies jedoch keine konzeptionelle Einschränkung ist, sondern ausschließlich dem aktuellen Entwicklungsstand der Werkzeug-Hardware geschuldet ist, kann diese Anforderung insgesamt als erfüllt angesehen werden. 

Anforderung \ref{anf:verknüpfung_mit_digitalen_ressourcen} (Verknüpfung mit digitalen Ressourcen) wurde im Rahmen der empirischen Untersuchung nicht berücksichtigt. Die Verknüpfung mit digitalen Ressourcen wurde analog zur Funktion zur Einbettung von Teilmodellen implementiert. Digitale Dokumente können an einbettbare Tokens gebunden werden und in einem Container durch Hineinlegen zugewiesen werden. Diese Funktionalität wurde technisch implementiert und hinsichtlich ihrer Funktionsfähigkeit getestet. Im Rahmen der Evaluierung wurde das Werkzeug ausschließlich mit einem Rechner betrieben, auf dem keine fallspezifischen bzw. anwendungsrelevanten digitalen Dokumente vorhanden waren. Die Einbindung von Dokumenten in Modelle konnte deswegen nicht vorgenommen werden. Die Anforderung ist also technisch erfüllt, konnte jedoch empirisch nicht überprüft werden. 

Anforderung \ref{anf:bearbeitung_von_beliebig_komplexen_modellen} (Bearbeitung von beliebig umfangreicher Modellen) wurde in der Hypothese \ref{hyp:beliebige_komplexität} untersucht. Durch die phyisch beschränkte Größe der Modellierungsfläche konnte die Erweiterung der Modellgröße lediglich durch die Erstellung von Teilmodellen erreicht werden, wobei deren Zusammenhang durch die Einbettung der Teilmodelle in Überblicksmodelle ausgedrückt wird. Die Einbettbarkeit von Teilmodellen wurde mittels der Verwendung von Modellierungsblöcken als Container realisiert, in die Tokens als Repräsentaten der Teilmodelle gelegt werden können. Die dazu notwendigen Funktionen wurden technisch implementiert und hinsichtlich ihrer Funktionsfähigkeit erfolgreich getestet. Auch in der empirischen Untersuchen wurde die Erweiterung der Modellgröße durch Einbettung von Teilmodellen erfolgreich eingesetzt. Im Vergleich der erstellten Modelle mit Modellen, die mit einem Werkzeug mit nicht beschränkter Modellierungsfläche erstellt wurden, zeigte sich jedoch eine signifikant geringere Modellgröße bei der Erstellung mit dem hier vorgestellten System. Die Erstellung beliebig großer Modelle ist somit technisch grundsätzlich möglich und auch verwendbar, empirisch konnte die Erfüllung der Anforderung dennoch nicht nachgewiesen werden.

Anforderung \ref{anf:kollaborative_und_unmittelbare_manipulierbarkeit_des_modells} (Kooperative und unmittelbare Manipulierbarkeit des Modells) wurde in den Hypothesen \ref{hyp:kollaborativ} und \ref{hyp:stärkere_kooperation} untersucht. Die Möglichkeit, Modelle kooperativ zu erstellen und zu manipulieren, ist grundsätzlich gegeben, die Möglichkeit der Durchführung eines kooperativen Modellierungsprozesses konnte empirisch belegt werden. Im Vergleich zu einem bildschirmbasierten System zeigt sich außerdem ein signifikant höherer Anteil an Interaktion zwischen den Teilnehmern bei der Durchführung der Modellierung mit dem hier vorgestellten System. Die Anforderung kann somit als erfüllt angesehen werden.

Anforderung \ref{anf:persistente_ablage_des_modells_möglichkeit_zur_rekonstruktion} (Persistente Ablage des Modells und Möglichkeit zur Rekonstruktion) wurde in der Hypothese \ref{hyp:historie} untersucht. Die Persistente Ablage der erstellten Modelle wurde mittels XML-Topic Maps realisiert. Die Funktionsfähigkeit der Persistierung wurde insofern technisch nachgewiesen, als dass die exportierten Modellrepräsentationen valide \gls{XTM}-Dateien waren und sämtliche auf der Modellierungsoberfläche repräsentierte Information in der Datei abgebildet war. Hinsichtlich der Ermöglichung der Rekonstruktion und der Nachvollziehbarkeit des Modellierungsprozesses konnte gezeigt werden, dass die bei der Persistierung inkludierte Entstehungshistorie des Modells die Nachvollziehbarkeit der im Modell repräsentierten Information zwar tendenziell erleichterte, aber nicht in allen Fällen ermöglichte. Die Anforderung kann damit technisch als erfüllt angesehen werden, die positive Wirkung der eingebetteten Entstehungshistorie des Modells konnte ebenfalls bestätigt werden. Insgesamt ist die Nachvollziehbarkeit der Modelle ausschließlich auf Basis der gespeicherten Repräsentationen aber in Frage zu stellen.

Zusammenfassend ergibt sich hinsichtlich der Erfüllung der Anforderungen die in Tabelle \ref{tab:erfuellung_der_anforderungen} dargestellte Übersicht. In ihr sind die Anforderungen jenen Kapiteln und Abschnitten des Implementierungsteils (\emph{Impl.}) zugewiesen, in denen ihre technische Umsetzung beschrieben wird, sowie den Hypothesen (\emph{Hyp.}) zugeordnet, in denen die tatsächliche Überprüfung der Erfüllung durchgeführt wird.

\begin{table}[htbp]
	\centering
	\caption{Erfüllung der Anforderungen}
\begin{tabular}{| c | p{5cm} | p{1cm} | c | p{4cm} |} 
  \hline
  & Anforderung & Impl. & Hyp. & Beurteilung \\ \hline \hline
  \ref{anf:physische_abbildung_legen_beliebiger_diagrammatischer_modelle} & Physische Abbildung beliebiger diagrammatischer Modelle &  \ref{sub:erkennen_von_verbindungen}, \ref{sub:benennung_von_modellelementen}, \ref{sub:ausgabe_von_information_zum_modell} & \ref{hyp:diagmodelle}, \ref{hyp:behinderung}, \ref{hyp:gewöhnung} & technisch möglich, empirisch teilweise bestätigt \\ \hline
  \ref{anf:unterstützung_der_iterativen_aushandlung_des_modells} & Unterstützung der iterativen Aushandlung des Modells & \ref{ssub:zustands_und_ereignismeldungen} & \ref{hyp:abstimmung} & technisch möglich, empirisch bestätigt \\ \hline
  \ref{anf:ermöglichung_experimenteller_veränderungen_am_modell} & Ermöglichung experimenteller Veränderungen am Modell & \ref{sub:tracking_des_modellzustandes}, \ref{ssub:wiederherstellungsunterstützung} & \ref{hyp:wiederherstellung} & technisch möglich, empirisch nicht bestätigt \\ \hline \hline
  \ref{anf:nicht_vorgegebene_semantik_der_modellierungselemente} & Nicht vorgegebene Semantik der Modellierungselemente & \ref{sub:festlegung_der_bedeutung_von_modellelementen}, \ref{sub:abbildung_des_metamodells} & \ref{hyp:kontexte}, \ref{hyp:keine_einschränkung} & technisch möglich, empirisch teilweise bestätigt \\ \hline
  \ref{anf:verknüpfung_mit_digitalen_ressourcen} & Verknüpfung mit digitalen Ressourcen & \ref{sub:erkennung_von_geöffneten_tokens}, \ref{sub:ausgabe_von_information_zum_modell} & --- & technisch möglich, empirisch nicht geprüft \\ \hline
  \ref{anf:bearbeitung_von_beliebig_komplexen_modellen} & Bearbeitung von beliebig umfangreicher Modellen & \ref{sub:erkennung_von_geöffneten_tokens} & \ref{hyp:beliebige_komplexität} & technisch möglich, empirisch nicht bestätigt \\ \hline \hline
  \ref{anf:kollaborative_und_unmittelbare_manipulierbarkeit_des_modells} & Kooperative und unmittelbare Manipulierbarkeit des Modells & \ref{sub:verteilung_des_modellzustandes}, \ref{sub:einsatz_von_jhotdraw} & \ref{hyp:kollaborativ}, \ref{hyp:stärkere_kooperation} & technisch möglich, empirisch bestätigt \\ \hline
  \ref{anf:persistente_ablage_des_modells_möglichkeit_zur_rekonstruktion} & Persistente Ablage des Modells und Möglichkeit zur Rekonstruktion & \ref{sub:tracking_des_modellzustandes}, \ref{ssub:abruf_der_modellierungshistorie}, \ref{sub:grundlegende_abbildung} & \ref{hyp:historie} & technisch möglich, empirisch bestätigt \\ \hline 
\end{tabular}
	\label{tab:erfuellung_der_anforderungen}
\end{table}


\section{Bewertung hinsichtlich der globalen Zielsetzung}
\label{sec:bewertung_hinsichtlich_der_globalen_zielsetzung}

In Kapitel \ref{cha:einführung} wurde die globale Zielsetzung wie folgt formuliert:

\fbox{\parbox{13cm}{\textbf{In der vorliegenden Arbeit sind die methodischen und technischen Möglichkeiten zur Ermöglichung und Unterstützung von expliziter Articulation Work zu ergründen, die gewonnenen Erkenntnissen in einem Werkzeug umzusetzen und dessen Auswirkungen auf die Interaktion zur Verbesserung der Production Work zu bewerten.}}}

Diese Zielsetzung wurde in drei Forschungsfragen detailliert:
\begin{enumerate}
	\item Wie kann explizite Articulation Work ermöglicht und unterstützt werden?
	\item Was muss ein Werkzeug zur Unterstützung von expliziter Articulation Work leisten?
	\item Inwiefern unterstützt das entwickelte Werkzeug die Durchführung von Articulation Work?
\end{enumerate}

\section{Offene Aspekte und Entwicklungspotential}
\label{sec:offene_aspekte_und_entwicklungspotential}

noch offen

\section{Schluss}
\label{sec:schluss}

hier rein: Schlussresümee

Hypothesen zeigen: Geeignet für Articulation Work, eher nicht geeignet für detaillierte Modellbildungen, technisch keine Probleme mehr.

Schlussfolgerung: Werkzeug unterstützt den Prozess der kommunikativen Abstimmung von mentalen Modellen, nicht aber die vollständige Externalisierung derselben.

% \section{Anwendungsszenarien} % (fold)
% \label{sec:anwendungsszenarien}
% 
% \subsection{Problembeschreibung und Arbeitsabstimmung}
% 
% Einzel- oder Gruppensessions. 
% 
% Aufgabenstellung: meist aus Arbeitsabläufen der beteiligten Personen
% 
% Merkmal: Tisch ist Mittel zum Zweck, gelegtes Modell fungiert als Diskussionsgrundlage. Modell ist statisch, wird einmal gelegt und nicht mehr verändert. Eher kompakte Modelle, die den Kontext eines Problems beschreiben. Die eigentliche Problematik ist selten explizit im Modell dargestellt.
% 
% Anwendungsbeispiele: Block 1, Block 2, Block 4 (Session 1-3).
% 
% Vorteile:
% 
% \subsection{Concept Mapping}
% 
% Einzel- oder Gruppensessions. 
% 
% Aufgabenstellung: zur Erhebung bzw. Überprüfung von domänenspezifischen (Struktur-)Wissen
% 
% Merkmal: Tisch 
% 
% Anwendungsbeispiele: Block 3, Block 5
% 
% \subsection{Strukturaufstellung und Manipulation}
% 
% Anwendungsbeispiele: Block 4 (Session 4).
% 
% AUfgabenstellung: Erhebung und Reflexion der Strukturen in denen Arbeitsabläufe situiert sind (Abteilungen, Personen, Kommunikationskanäle).
% 
% Merkmal: Modelle sind nicht statisch - werden nach der Erstellung zwar meist nicht erweitert aber in ihrer Struktur verändert (räumliche Relation der Knoten zueinander). 
% 
% % chapter schlussbetrachtungen (end)