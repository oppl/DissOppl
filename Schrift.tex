%!TEX TS-program = xelatex
%!TEX encoding = UTF-8 Unicode

\documentclass[12pt,a4paper,twoside, openright, cleardoubleempty, headsepline, footnosepline, nochapterprefix, appendixprefix, bibtotoc, abstracton]{scrreprt}

\usepackage{pdfsync}
\usepackage{ngerman}
\usepackage[ngerman]{babel}

\usepackage{multicol}
\usepackage{makeidx}
\usepackage{scrpage2}
\usepackage{natbib}
\usepackage{enumerate}
\usepackage{multirow}
\usepackage{longtable}
\usepackage{lscape}

% Surround parts of graphics with box
\usepackage{boxedminipage}

% Package for including code in the document
\usepackage{listings}

% If you want to generate a toc for each chapter (use with book)
\usepackage{minitoc}

\usepackage{fontspec}
\defaultfontfeatures{Mapping=tex-text}
%%\setromanfont{Hoefler Text}
%%\setromanfont{Gentium}
%%\setromanfont{LucidaGrande}
%%\setsansfont{Gill Sans}
%%\setsansfont{Helvetica Neue}
\setmonofont{Courier}
\makeindex

% This is now the recommended way for checking for PDFLaTeX:
\usepackage{ifpdf}

\ifpdf
\usepackage[pdftex]{graphicx}
\else
\usepackage{graphicx}
\fi

\usepackage{geometry}

\usepackage[final, activate, verbose=true]{microtype}
\usepackage{color}

\usepackage[bookmarks, backref=page, pdfborder={0 0 0}]{hyperref}
\renewcommand*{\backrefpagesname}{\footnotesize Referenziert auf S.}
\renewcommand*{\backref}{\backrefpagesname\ }

\usepackage[
nonumberlist, %keine Seitenzahlen anzeigen
acronym,      %ein Abkürzungsverzeichnis erstellen
footnote]      %im Inhaltsverzeichnis auf section-Ebene erscheinen
{glossaries}
\renewcommand*{\glspostdescription}{}
\renewcommand*{\glspluralsuffix}{s}
\makeglossaries
\loadglsentries[\acronymtype]{Acronyme}

\definecolor{darkgray}{gray}{0.2} 
\definecolor{lightgray}{gray}{0.8} 
\fboxrule2pt
\fboxsep10pt
\parskip1ex

% Keine "Schusterjungen"
\clubpenalty = 10000
% Keine "Hurenkinder"
\widowpenalty = 10000 \displaywidowpenalty = 10000

\setcounter{tocdepth}{2}

\newcommand*\uebersicht{%
\addchap*{Inhaltsübersicht}
\markboth{Inhaltsübersicht}{}
\begingroup
\value{tocdepth}\shorttocdepth\relax % uebler Hack!
\makeatletter
\input{\jobname.toc}
\makeatother
\endgroup
}
\newcommand*{\shorttocdepth}{0}

\renewcommand*{\dictumwidth}{.5\textwidth} 
\renewcommand{\baselinestretch}{1.0}

\usepackage[framed, hyperref]{ntheorem}
\newtheorem{hyp}{Hypothese}
\newtheorem{anf}{Anforderung}

\newenvironment{transkript}{\small \tt \begin{quote}}{\end{quote} \rm \normalsize}

%\usepackage[numbers]{natbib}
%\bibliographystyle{dinat}
\bibliographystyle{apalike}


\newcommand{\wichtig}{
%\marginpar[\raggedleft$\Rightarrow$]{\raggedleft$\Leftarrow$}
}
\newcommand{\todo}{\marginpar{\rule[-25mm]{3mm}{20mm}}}

\title{Unterstützung expliziter Articulation Work}
\subtitle{Ein Werkzeug für Externalisierung und Abgleich mentaler Modelle}
\author{Stefan Oppl}
\publishers{Dissertation\\\normalsize zur Erlangung des Grades eines Doktors der Technischen Wissenschaften\\vorgelegt an der Fakultät für Informatik an der Technischen Universität Wien

\vspace{4cm}

Erstgutachter: o. Univ.-Prof. DI Dr. Christian Stary\\Zweitgutachter: o. Univ.-Prof. DI Dr. Christian Breiteneder}
%\dedication{Für Felix und Sabrina}

\begin{document}

\pagestyle{empty}

\maketitle

%\begin{abstract}

\section*{Kurzfassung}
Der Erfolg kooperativer Arbeit beruht auf auf einem gemeinsamen Verständnis der betroffenen Abläufe durch die beteiligten Personen. Dieses gemeinsame Verständnis wird durch die ständige und unbewusste Durchführung von „Articulation Work“ sichergestellt. In Situationen, die von den beteiligten als komplex und problematisch wahrgenommen werden, ist implizite, arbeitsbegleitende Durchführung von „Articulation Work“ unter Umständnen nicht mehr ausreichend. Es ist dann notwendig, sich explizit mit der Abstimmung der individuellen Sichtweisen und der Bildung eines gemeinsamen Verständnisses zu beschäftigen.

Um explizite „Articulation Work“ zu unterstützen, wird in dieser Arbeit versucht, die Interaktion der Beteiligten durch die koopertive Bildung und Diskussion diagrammatischer Modelle zu ermöglichen bzw. zu erleichtern. Dieser Zugang ist aus der Theorie der Bildung und Veränderung mentaler Modelle abgeleitet. Die Externalisierung der mentalen Modelle in Form von diagrammatischen Modellen wird dort als adäquates Mittel zur Refelexion und Kommunikation derselben identifiziert. Methodisch baut die Arbeit dabei auf Sturkturlegetechniken und Concept Mapping auf. Die dort vorgeschlagenen Methoden und Anforderungen an einer Werkzeugunterstützung werden unter Bezugnahme auf „Articulation Work“ zusammengeführt. Die resultierende Methodik wird durch ein Tabletop Interface -- eine horizontale Interaktionsoberfläche mit rechnerbasierten Unterstützungsfunktionen -- unterstützt.

Das Werkzeug selbst wird hinsichtlich seiner Umsetzung in Hard- und Software beschrieben und einer empirischen Untersuchung unterzogen. Dabei wird die Verwendbarkeit des Werkzeugs selbst, dessen Nutzen bei der Abstimmung mentaler Modelle sowie letztendlich die Auswirkungen bei der Durchführung von „Articulation Work“ untersucht. Die Ergebnisse deuten darauf hin, dass das Werkzeug den Anforderungen genügt und sowohl bei der Abstimmung mentaler Modelle als auch zum Teil im Kontext der durchgeführten „Articulation Work“ zu den intendierten Wirkungen führt.
%\end{abstract}

\cleardoublepage

%\vspace*{\fill}

\begin{center}
für \\
Felix und Sabrina
\end{center}

\vspace*{\fill}

\cleardoublepage

\section*{Vorwort} % (fold)
\label{sec:vorwort}

Arbeiten wie diese entstehen nie ohne Einfluss und Unterstützung von außen. Zahlreiche Menschen haben mich auf meinem Weg zum Abschluss dieser Arbeit begleitet. Diesen Menschen die ihnen zustehende Würdigung zukommen zu lassen ist das Ziel dieser einleitenden Sätze.

Prof. Alois Ferscha habe ich zu verdanken, dass sich das weite Feld der Forschung als Perspektive für mich auftat. Er nahm mich noch während meines Informatik-Studiums am Institut für Pervasive Computing an der Johannes Kepler Universität Linz auf und ermöglichte mir erste zaghafte Schritte in der Welt der Forschung. Für seine visionäre Anleitung und Unterstützung möchte ich mich bedanken.

uni

Peter, Jeannette, Matthias, Simon Vogl

übrige Arbeitskollegen

Christian

Breiteneder

evtl. Tom Gross

Diplomanden: Florian Furtmüller, Thomas Feiner, Matthias Neubauer, Josef Bohninger, Daniel Bindreiter, Patrick Wahlmüller und zahllose andere Studierende in Praktika und Projektstudien. Alle Studienteilnehmer

Eltern

Sabrina & Felix

% section vorwort (end)

\pagestyle{scrheadings}
\automark[section]{chapter} 

\pagenumbering{roman}

\uebersicht
\tableofcontents

\newpage
\cleardoublepage

\pagenumbering{arabic}

\chapter{Einführung} % (fold)
\label{cha:einführung}

\emph{„How people work is one of the best kept secrets in America.“} (Wellman, D. zitiert nach \cite{Suchman95}).

Diese Aussage David Wellmans diente Lucy Suchman in den 90er-Jahren des 20. Jahrhunderts als Motivator für ihre Forschung über die Natur menschlicher Arbeit im Computerzeitalter und deren Unterstützung durch neue Technologien. Ihre Arbeiten und viele andere (etwa \cite{Schmidt92} oder \cite{Sachs95}) argumentieren für eine stärkere Berücksichtigung der Rolle des Menschen im Arbeitsprozess.

Wellman spielt mit diesem Zitat auf die oft auftretende Diskrepanz zwischen der (niedergeschriebenen) Definition eines Arbeitsablaufs und dessen tatsächlicher Umsetzung in der konkreten betrieblichen Umgebung an.

Der Druck in der heutigen Geschäftswelt, bestimmte Qualitätskriterien garantiert erfüllen zu können, hat zu einer beinahe flächendeckenden Verbreitung von Qualitätszertifizierungen geführt. Die bekanntesten Vertreter dieser Zertifizierungen sind wohl die Standards aus der Familie der ISO 9000 Normen \cite{ISO05}. In der ISO 9001-Norm \cite{ISO00}, in der die Anforderungen an Qualitätsmanagement-Systeme definiert sind, ist festgeschrieben, dass eine prozessorientierte Organisation eine der Voraussetzungen für erfolgreiches Qualitätsmanagement ist. Ein wesentliches Merkmal einer prozessorientierten Organisation ist, dass ihre organisationalen Prozesse — also ihre Arbeitsorganisation und -abläufe — bekannt, benannt und definiert sind. 

Die Unterschiede zwischen festgeschriebenen und gelebten Arbeitsabläufen wurde schon 1978 von Argyris und Schön \cite{Argyris78} beschrieben. Mit der Unterscheidung zwischen „\emph{espoused theories}“ (= \emph{die offiziell veröffentlichten Theorien über Arbeit}) und den „\emph{theories-in-use}“ (= \emph{die tatsächlich handlungsleitenden Theorien}) wurden dieser Gegensatz auch explizit benannt. Sachs \cite{Sachs95} beschreibt das gleiche Phänomen und unterscheidet zwischen dem „\emph{explicit organisational view}“ und dem „\emph{tacit organisational view}“ auf Arbeit. Sie beschreibt damit einerseits eine explizit formulierte und statische Sicht auf Arbeit und andererseits eine informelle, im Fluss befindliche und zum Zeitpunkt der Betrachtung nirgendwo niedergeschriebene Sicht auf Arbeit. Letztere kann lediglich aus der Analyse der tatsächlichen Arbeitsabläufe gewonnen werden kann, die den Tätigkeiten zugrunde liegenden Annahmen („theories-in-use“ \cite{Argyris78}) sowie deren vom Arbeitenden konkret wahrgenommene organisationale Rahmenbedingungen bleiben allerdings verborgen — die Frage nach dem „Warum?“, die die Form des konkreten Arbeitsablaufs motiviert, kann nicht unmittelbar beantwortet werden.

Wie Sachs verdeutlichen auch Wellman sowie Argyris und Schön, dass das zweitgenannte Verständnis von Arbeit nicht explizit niedergeschrieben und formal definiert ist — in seinem Wesen also „unbekannt“ ist. Wenn „Arbeit“ oder die ihr zugrunde liegenden Annahmen unbekannt sind, kann eine Veränderung ihrer selbst oder der Umgebung, in der sie durchgeführt wird, zu schwerwiegenden Problemen führen (wie z.B. von \cite{Nonaka95}, \cite{Krogh00} oder \cite{Gerson86} beschrieben). Der Begriff „Veränderung“ deckt dabei nicht nur tiefgreifende organisatorische Änderungen im Unternehmen ab, sondern durchaus auch „marginale“ Änderungen wie z.B. die Einstellung neuer Mitarbeiter oder dem Einsatz eines neuen Werkzeugs \cite{Olesen03}. Daraus kann man schließen, dass potentiell „problematische“ Situationen häufig auftreten können. 

„Problematische“ Situationen sind dabei all jene Situationen, in denen die Zielerreichung erschwert wird, weil die dazu notwendigen Schritte entweder unklar sind oder nicht operationalisiert werden können. Im Kontext der obigen Aussage bedeutet dies, dass durch eine organisationale Veränderung neue Arbeitsschritte notwendig werden bzw. die bisherigen nicht mehr funktionieren oder angemessen sind. Beim Auftreten einer derartigen Veränderung ist daher eine erneute Planung bzw. Abstimmung der zur Zielerreichung notwendigen Arbeitsschritte notwendig. Fujimura \cite{Fujimura87} unterscheidet in diesem Sinne zwischen zwei Formen von Arbeit — der „Produktion“ („\emph{production}“) und der „Artikulation“ („\emph{articulation}“), wobei letztere alle Tätigkeiten umfasst, die die Umsetzung bzw. Aufrechterhaltung der „Produktion“ ermöglichen.

„Artikulation“ ist ein integraler Bestandteil von Arbeit \cite{Strauss85}. Mit der Komplexität der „Produktion“ steigt auch der Aufwand der dazu notwendigen „Artikulation“ an \cite{Strauss88}. Die Komplexität steigt hier mit der Anzahl der benötigten Arbeitsschritte, den dazu benötigten Kompetenzen und der Anzahl der involvierten Personen. Nicht bekannte, falsche oder zurückgehaltene Information über die „Produktion“ erschweren die „Artikulation“ oder machen sie unmöglich \cite{Fujimura87}. Dies hat jedoch nicht nur negative Auswirkungen auf die „Produktion“ sondern verhindert auch eine tiefergehende Beschäftigung mit der aktuellen Arbeitspraxis und eine potentielle Verbesserung derselben \cite{Argyris78}. Erfolgreiche Artikulation ist damit nicht nur eine Voraussetzung für eine funktionierende Produktion sondern auch ein Enabler für organisationale Veränderungen im Sinne eines organisationalen Lernprozesses (z.B. \cite{Kim93} oder \cite{Firestone03a}).

Eine methodische und technische Unterstützung kann Artikulation ermöglichen oder deren erfolgreichen Ablauf erleichtern (\cite{Schmidt92}, \cite{Simone99}, \cite{Jorgensen04}, \cite{Baker07}).

\fbox{\parbox{13cm}{\textbf{In der Arbeit sind die methodischen und technischen Möglichkeiten zur Ermöglichung und Unterstützung von Articulation Work zu ergründen, die gewonnenen Erkenntnissen in einem Werkzeug umzusetzen und dessen Auswirkungen auf die Interaktion zur Verbesserung der Production Work zu bewerten.}}}

\section{Forschungsfragen} % (fold)
\label{sec:forschungsfragen}

\begin{enumerate}
	\item Wie kann Articulation Work ermöglicht und unterstützt werden?
		\begin{enumerate}
			\item Durch welche Aktivitäten zeigt sich Articulation Work im Arbeitsprozess?
			\item Welche Rahmenbedingungen ermöglichen bzw. begünstigen Articulation Work?
			\item Wie können die in 1.1. identifizierten Aktivitäten unterstützt werden?
			\item Wie können die in 1.2. identifizierten Rahmenbedingungen und die in 1.3. identifizierten Anforderungen in einer Methodik umgesetzt werden?
		\end{enumerate}
	\item Was muss ein Werkzeug zur Unterstützung von Articulation Work leisten?
		\begin{enumerate}
			\item Welche Anforderungen an ein Werkzeug ergeben sich aus der in 1.4. entwickelten Methodik?
			\item Wie können diese Anforderungen technologisch umgesetzt werden?
		\end{enumerate}
	\item Inwiefern unterstützt das entwickelte Werkzeug die Durchführung von Articulation Work?
		\begin{enumerate}
			\item Wie kann die Unterstützungsleistung bewertet werden?
			\item Welche Auswirkungen auf die Interaktion hat die Anwendung von Methodik und Werkzeug?
		\end{enumerate}
\end{enumerate}

% section forschungsfragen (end)

\section{Aufbau der Arbeit} % (fold)
\label{sec:aufbau_der_arbeit}

% section aufbau_der_arbeit (end)

% chapter einführung (end)

\part{Grundlagen} % (fold)
\label{prt:grundlagen}

%\input{Design}

\chapter{Articulation Work} % (fold)
\label{cha:articulation_work}

In diesem Kapitel wird das Konzept „Articulation Work“ dargestellt und in den Kontext von menschlicher Arbeit gestellt. Im ersten Teil des Kapitels wird auf die historische Entwicklung des Begriffs „Articulation Work“ und die unterschiedlichen Herangehensweise zu dessen Verständnis eingegangen. Der zweite Teil des Kapitels widmet sich den Aktivitäten, die „Articulation Work“ ausmachen, den Merkmalen, an denen sich gute „Articulation Work“ zeigt, sowie den Möglichkeiten der Unterstützung von „Articulation Work“ durch organisationale und technische Maßnahmen.

\section{Begriffsbestimmung} % (fold)
\label{sec:aw_begriffsbestimmung}

Das Konzept der "Articulation Work" wurde als Erklärungsmodell für eine bestimmte Art von menschlicher Arbeit Mitte der 1980er Jahre von \citet{Strauss85} eingeführt. Neben \citet{Strauss85} tragen auch die Arbeiten von \citet{Gerson86} und \citet{Fujimura87} wesentlich zur Begriffsbestimmung und Konzeptbildung bei. Die vorhandene Literatur, die Bezug auf „Articulation Work“ nimmt, referenziert im Wesentlichen auf diese drei Arbeiten bzw. eine dieser drei Arbeiten. Der Kontext, in dem die Entwicklung der im folgenden vorgestellten Konzepte erfolgte, war die komplexe, von viel Interaktion an zahlreichen Schnittstellen geprägte Arbeit in Krankenhäusern \citep{Strauss85}, in der Wissenschaft \citep{Fujimura87} und in Versicherungsunternehmen \citep{Gerson86}, die die jeweiligen Autoren in mehreren Fallstudien untersuchten. 

Um in der Folge einen einheitlichen Begriffsraum aufspannen zu können, ist vorab der Begriff „Arbeit“ zu klären. Die eben genannten Autoren führen keine explizite Definition an, weshalb hier auf eine Definition zurückgegriffen wird, die im Kontext der folgenden Ausführungen zur „inneren“ Struktur von Arbeit nach „außen“ hinreichend umfassend ist\footnote{Auf eine umfassende Literaturstudie und die Entwicklung eines darauf aufbauenden „Arbeits“-Begriffs wurde hier verzichtet, da dies über den Betrachtungsbereich und Anspruch dieser Arbeit hinausgeht}. \citep{Semmer04} definieren vor dem Hintergrund der Organisationspsychologie „Arbeit“ wie folgt: \emph{„Arbeit ist zielgerichtete menschliche Tätigkeit zum Zwecke der Transformation und Aneignung der Umwelt aufgrund selbst- oder fremddefinierter Aufgaben, mit gesellschaftlicher, materieller oder ideeller Bewertung, zur Realisierung oder Weiterentwicklung individueller oder kollektiver Bedürfnisse, Ansprüche und Kompetenzen.“}. Arbeit ist also ein menschliches Phänomen, Träger von Arbeit sind immer Menschen. Arbeit definiert sich außerdem durch ihre Zielgerichtetheit und findet immer in Interaktion mit der Umwelt statt. Die Ziele, auf die Arbeit ausgerichtet ist, leiten sich aus Aufgaben ab, die sich Menschen selbst setzten können oder die ihnen vorgegeben werden. Diese Aufgaben dienen der Erreichung von individuellen oder kollektiven Bedürfnissen und Ansprüchen bzw. der (Weiter-)Entwicklung von Kompetenzen. Die Bewertung der Zielerreichung muss nicht unbedingt aus materieller Perspektive erfolgen sondern kann auch ideell oder gesellschaftlich begründet sein. In dieser Arbeit wird der Begriff „Arbeit“ vor allem auch im organisationalen Kontext gesehen. Ein wesentlicher Aspekt, der hierbei zu berücksichtigen ist, ist die Arbeitsteilung, also die koordinierte Tätigkeit mehrerer Individuen um ein gemeinsames Ziel zu erreichen (siehe dazu z.B. ). Dies stellt die obige Definition nicht in Frage, erweitert jedoch den Betrachtungsbereich explizit auch auf Arbeit, die gemeinschaftlich durchgeführt wird. 

„Articulation Work“ ist jener Anteil der gesamten durchgeführten Arbeit, der der Abstimmung mit anderen Individuen dient. Diese Abstimmung ist notwendig, um das eigentliche Arbeitsziel erreichen zu können. Arbeit wird von den oben angeführten Autoren als inhärent kooperativer Prozess gesehen, der immer auf Interaktion mit anderen Menschen basiert bzw. diese bedingt (Strauss formuliert diese Annahme in Bezugnahme auf \citet{Hughes71} prägnant mit der Aussage \emph{„work rests ultimately on interaction“}). Diese Annahme erscheint insofern als zulässig, als dass selbst Arbeitsabläufe, die selbst keine Kooperation mit anderen Menschen mit sich bringen, zumindest auf den Ergebnissen anderer Arbeitsabläufe aufbauen oder als Grundlage weiterer Arbeitsabläufe dienen. Interaktion tritt also in jedem Arbeitsprozess zumindest zu Beginn und am Ende in unmittelbarer oder mittelbarer\footnote{Unter "mittelbar" ist hier Interaktion zu verstehen, die nicht im direkten Kontakt zwischen Individuen abläuft sondern lediglich indirekt durch die Ergebnisse eines Arbeitsprozesses (Materialien, Dokumente, \ldots) vermittelt wird.} Form auf.

\textbf{Abbildung, in der kooperative Arbeitsprozesse und solche mit mittelbarer und unmittelbarer Interaktion zu Beginn oder am Ende dargestellt werden}

Jener Teil von Arbeit, der der eigentlichen Zielerreichung dient, wird im hier vorgestellten Erklärungsmodell als „Production Work“ bezeichnet \citep{Fujimura87}. „Production Work“ ist komplementär zu „Articulation Work“ zu sehen und umfasst alle Aktivitäten, die der „Wertschöpfung“ im wörtlichen Sinn dienen. „Production Work“ sind also alle Tätigkeiten, die mit der Schaffung jener Werte (oder Ergebnisse) befasst sind, die durch den Arbeitsablauf erreicht werden sollen.  

\begin{figure}[htbp]
	\centering
		\includegraphics[height=3in]{img/ArticulationWork/ArbeitInteraktion.png}
	\caption{Struktur von Arbeitsabläufen}
	\label{fig:img_ArticulationWork_ArbeitInteraktion}
\end{figure}

Teile eines Arbeitsablaufs dienen also der Zielerreichung an sich („Production Work“). Andere Teile dienen der Abstimmung zwischen den involvierten Akteuren, um ein gemeinsames Verständnis über die jeweiligen Schnittstellen – also die Berührungspunkte zwischen den Tätigkeiten – zu entwickeln (siehe Abbildung \ref{fig:img_ArticulationWork_ArbeitInteraktion}). Diese Entwicklung eines gemeinsamen Verständnisses oder „Koordination“ ist kritisch für den Erfolg von kooperativer Arbeit \citep{Strauss93} und wird als „Articulation Work“ bezeichnet.\footnote{\emph{„Without an understanding of articulation, the gap between requirements and the actual work process in the office will remain inaccessible to analysis. That is, it will be possible to describe tasks in an idealized form but not to describe actual situations.“}\citep{Gerson86}} 

„Articulation Work“ ist also ein Enabler für funktionierende Kommunikation und Zusammenarbeit im eigentlichen Arbeitsablauf. Zentral ist dabei vor allem die gegenseitigen Offenlegung der Annahmen aller beteiligten Personen, die den individuellen Arbeitsbeiträgen zugrunde liegen\footnote{\emph{"Reconciling incommensurate assumptions and procedures in the absence of enforceable standards is the essence of articulation.}\citep[][S. 266]{Gerson86}}. Konkret kann „Articulation Work“ unterschiedliche Ausprägungen annehmen \citep{Gasser86}\citep{Bendifallah87}:
\begin{description}
	\item[Fitting] (bzw. „Accomodation“ \citep{Bendifallah87}) Tätigkeiten zur Planung bzw. Anpassung der Arbeitspraxis an gegebene bzw. veränderte Umweltbedingungen.
	\item[Augmenting] (bzw. „Negotiation of additional [maintainance] activities“ \citep{Bendifallah87}) Planung von zusätzlichen kurz- oder mittelfristigen Tätigkeiten, um das Auftreten von erkannten Problemen zu verhindern. 
	\item[Working around] Entwicklung von Strategien zur Vermeidung des Auftretens von Situationen, in denen Probleme auftreten, ohne deren Ursache zu beseitigen.
\end{description}

 „Articulation Work“ ist keine Tätigkeit, die zu einem bestimmten Zeitpunkt im Arbeitsprozess durchgeführt wird und dann als abgeschlossen betrachtet werden kann. Vielmehr wird „Articulation Work“ immer auch begleitend zur eigentlichen produktiven Arbeit durchgeführt und umfasst neben planenden und koordinierenden Tätigkeiten auch das Erkennen von Fehlentwicklungen bzw. von Situationen, in denen eine erneute  Koordination notwendig ist\footnote{\emph{Articulation consists of all the tasks involved in assembling, scheduling, monitoring, and coordinating all of the steps necessary to complete a production task."}\citep[][S. 266]{Gerson86}}. 

\begin{figure}[htbp]
	\centering
		\includegraphics[height=3in]{img/ArticulationWork/ArtikulationProduktion.png}
	\caption{Konzeptualisierung von „Arbeit“ nach \citep{Strauss85} und \citep{Fujimura87}}
	\label{fig:img_ArticulationWork_ArtikulationProduktion}
\end{figure}

Der Begriff „Articulation Work“ ist im Englischen zweideutig und von \citeauthor{Strauss85} auch bewusst so gewählt. Einerseits wird damit ausgedrückt, dass \emph{Arbeit} ("Work") artikuliert wird, andererseits zeigt der Begriff, das die \emph{Artikulation} selbst ebenfalls Arbeit ist (also Zeit und Ressourcen in Anspruch nimmt) und auch also solche wertgeschätzt werden muss \citep{Fujimura87}. „Articulation Work“ ist kein klar abgegrenztes und strukturiertes Konzept – sie tritt je nach Arbeitssituation in unterschiedlichen Spielarten auf. Die Unterscheidung dieser Arten von „Articulation Work“ ist für die Unterstützung derselben relevant und wird daher im folgenden Abschnitt genauer betrachtet.
% section begriffsbestimmung (end)

\section{Ausprägungen von Articulation Work} % (fold)
\label{sec:arten_von_articulation_work}

Wie bereits von \citet{Gerson86} angeführt (siehe oben), argumentiert auch \citeauthor{Strauss88}, dass Artikulation immer passieren muss (und passiert), wo Menschen zusammenarbeiten, um zu vermeiden, dass unbekannte Aspekte Probleme bei der Durchführung der Arbeit verursachen \citep{Strauss88}. „Articulation Work“ ist kein revolutionäres Konzept, sondern fasst Tätigkeiten unter einem Begriff zusammen, die seit jeher Teil jeder Zusammenarbeit zwischen Menschen sind. Grundsätzlich geht Strauss davon aus, dass Artikulation immer abläuft, egal wie einfach oder kompliziert, wie eingespielt oder neuartig eine (Zusammen-)Arbeit ist \citep{Strauss88}. Sehr wohl existieren jedoch Unterschiede in der Qualität der Arbeit, die sich auf die Form der Artikulation auswirken, die zu deren Abstimmung notwendig ist: \emph{„A useful fundamental distinction between classes of interaction is between the routine and the problematic. Problematic interactions involve 'thought', or when more than one interactant is involved then also 'discussion'.“} \citep{Strauss93}. Dieses Zitat zeigt im Übrigen auch, dass „Interaction“ im Sinne von Strauss nicht unbedingt ein kollektives Phänomen ist, sondern auch individuell (im Bezug auf die (unbelebte) Umgebung) auftreten kann.

Je komplexer („problematic“) eine Interaktion ist, desto notwendiger wird laut Strauss eine explizite Beschäftigung mit dem Vorgang der Artikulation. Bei einfachen, eingespielten („routine“) Interaktionen bleibt die Artikulation zumeist implizit, verborgen und informell \citep{Hampson05} (entsprechend der „Sozialisation“ im aus der Domäne der Wissensgenerierung und -teilung stammenden SECI-Zyklus \citep{Nonaka95}). Ein grundlegendes Problem, dass Artikulation für jeden noch so einfach erscheinend Arbeitsvorgang potentiell relevant macht, spricht Strauss mit den Worten von Hughes unmittelbar nach der Definition von „problematic interaction“ an: \emph{„[O]ne man's routine of work is made up of the emergencies of other people“} \citep{Hughes71} zitiert nach \citep{Strauss93}.

„Articulation Work“ tritt also in zwei Qualitäten auf. Ist der Bedarf zur Abstimmung bekannt und werden Tätigkeiten zur Abdeckung dieses Bedarf bewusst durchgeführt, so spricht man von \emph{expliziter} „Articulation Work“ \citep{Strauss88} \citep{Fjuk97}. Die Abstimmung von Tätigkeiten, die ständig während der Zusammenarbeit unbewusst ausgeführt wird, bezeichnet man als \emph{implizite} „Articulation Work“\footnote{\emph{The explicit articulation is thus connected to the planning and decisions regarding the salient dimensions of work -- who, what, when, how -- while implicit articulation is invaluable when carrying out activities in situated circumstances, in order to handle contingencies.}\citep[][S.5]{Fjuk97}}. Letztgenannte Art ist es auch, die von den Arbeitenden „automatisch“ zur Anwendung gebracht wird, sobald Änderungen in der Arbeitsumgebung oder Probleme auftreten \citep{Strauss88}. Implizite „Articulation Work“ stößt aber an ihre Grenzen, wenn die Arbeitssituation als „problematisch“ \citep{Strauss88} oder „komplex“ \citep[][S. 23f]{Schmidt90} wahrgenommen wird. Es wird dann notwendig, dezidierte Abstimmungs-Aktivitäten anzustoßen, also explizite „Articulation Work“ durchzuführen.

Neben der Unterscheidung zwischen impliziter und expliziter Articulation Work anhand der Komplexität der zugrunde liegenden Interaktion führt Strauss keine systematische Betrachtung von Articulation Work hinsichtlich deren Ausprägungen durch. Offensichtlich wird in seinen Texten jedoch, dass es Articulation Work als beobachtbares und eindeutig also solche identifizierbares Phänomen nicht gibt. Abhängig vom betrachteten Arbeitsablauf, der Arbeitsumgebung und den beteiligten Personen zeigt sich Articulation Work in unterschiedlichen Formen.

In der Literatur existieren zwei Ansätze zur Differenzierung zwischen unterschiedlichen Arten von Articulation Work. \citet{Fjuk97} stellen Articulation Work der Activity Theory \citep{Leontev78} gegenüber und unterscheiden so verschiedene Ebenen. \citet{Hampson05} führen ein Raster ein, das Articulation Work hinsichtlich der Art des Arbeitsprozesses unterschiedet, in dem sie zur Anwendung kommt. Beide Ansätze werden in der Folge im Detail beschrieben und bezüglich ihrer Implikationen für diese Arbeit betrachtet.

\subsection{Unterscheidung nach Fjuk, Smørdal und Nurminen}
\label{sub:arten_fjuk}

\citet{Fjuk97} betrachten Articulation Work im Kontext von \gls{CSCW} und versuchen ein konzeptionelles Framework zu entwickeln, das die Rolle von Computersystemen im Kontext indvidueller und kollektiver Tätigkeiten erklärt -- sie entwickeln also ein Erklärungsmodell für die Funktionsweise sozio-technischer Systeme \citep{Emery60}. Während die Implikationen von „Articulation Work“ für \gls{CSCW} an dieser Stelle nicht näher von Belang sind ist aber das theoretische Framework, das die Autoren ihren Ausführungen zu Grunde legen von Interesse. 

\citet{Fjuk97} beziehen sich bei ihren Überlegungen auf die „Activity Theorie“ (Tätigkeits-Theorie), die maßgeblich von \citep{Leontev72} geprägt wurde. Die Autoren argumentieren, dass diese einen Ansatzpunkt bietet, die von Strauss als relevant erkannten aber nicht näher behandelten „externen Faktoren“, die Arbeit beeinflussen, zu berücksichtigen. Der Begriff der „externen Faktoren“ wird mit allen Einflussfaktoren beschrieben, die nicht unmittelbar Teil des Arbeitsablaufs sind sondern technologischer, organisationaler, kultureller, wirtschaftlicher oder physiologischer Natur sind. 

Ohne an dieser Stelle näher auf die „Activity Theory“ \footnote{für eine allgemein verständliche Einführung unter Berücksichtigung der praktischen Implikationen siehe \citet{Dahme97} oder \citet{Nardi06}} einzugehen, seien hier die drei Kernkonzepte der Theorie erwähnt:
\begin{itemize}
	\item Activity (Tätigkeit)
	\item Action (Aktion)
	\item Operation (Operation)
\end{itemize}

Diese drei Konzepte bilden eine Hierarchie, in denen eine „Activity“ an oberster Stelle steht. Eine „Activity“ ist eine menschliche Tätigkeit, die durch ein Motiv getrieben ist und der (vorerst) individuellen Bedürfnisbefriedigung dient. Eine „Activity“ setzt sich aus mehreren „Actions“ zusammen, die jede für sich ein aus dem Motiv heraus begründbares Ziel haben und zur Bedürfnisbefriedigung direkt oder indirekt betragen. „Actions“ setzen sich wiederum aus „Operations“ zusammen, also einzelnen, nicht mehr bewusst ausgeführten Handlungen, die durch die Bedingungen des jeweiligen Umgebungskontexts bestimmt werden. Während sie lernen, transformieren Individuen laufend „Actions“ zu „Operations“, automatisieren also deren Ausführung, sodass sich die kognitive Belastung verringert (als klassisches Beispiel kann hier das Erlernen des Autofahrens dienen).

Die „Activity Theory“ beschreibt als psychologisches Modell vorerst das Individuum und dessen Verhalten. In sozialen Systemen, die auf Interaktion basieren, stößt das Modell jedoch an die Grenzen der erklärbaren Phänomene. \citet{Engestrom87} baut auf der klassischen „Activity Theory“ auf und erweitert diese um den Aspekt der Gemeinschaft sowie der Interaktion in dieser bzw. der Rolle von Artefakten („Objects“) in derartigen Settings. \citet{Fjuk97} bemängeln aber in ihren Ausführungen, dass \citeauthor{Engestrom87} in seinen Ausführungen abstrakt bleibt und nicht den Konkretisierungsgrad der originären „Activity Theory“ erreicht, was das Zusammenspiel der unterschiedlichen Ebenen („Activity“, „Action“ und „Operation“) betrifft.

Hinsichtlich der näheren Betrachtung von Articulation Work unterscheiden \citet{Fjuk97} in Bezugnahme auf \citet{Strauss93} vorerst zwei Ebenen („levels“) von „Articulation Work“, namentlich „planned“ und „situated Articulation Work“. Diese Unterscheidung korrespondiert den Autoren nach im Wesentlichen mit der Unterscheidung zwischen „expliziter“ und „impliziter Articulation Work“, da erstere geplant und zu einem zuvor bestimmten Zeitpunkt ausgeführt wird und zweitere ad-hoc, bei Bedarf, und eher informell abläuft. Diese Entsprechung steht jedoch teilweise im Konflikt mit späteren Aussagen in der Arbeit, in der auf „situated Articulation Work“ Bezug genommen wird, die aber ob der herausfordernden Natur des Arbeitsablaufs „expliziter“ abzulaufen habe \citep[][S. 15]{Fjuk97}

Unter Einbeziehung der „Activity Theory“ und basierend auf der Unterscheidung zwischen „Activity“, „Action“ und „Operation“ führen \citet{Fjuk97} außerdem zwei unterschiedliche Arten von „Articulation Work“ ein, die sich in ihren Bezugspunkten unterschieden und jeweils für den Fall individueller und kollektiver Tätigkeiten bzw. Aktionen betrachtet werden.  

\begin{description}
	\item[Articulation of action within individual activity] Die Artikulation von Aktionen innerhalb einer Tätigkeit entspricht einer bewussten Planung eines Vorgehens zur Erreichung von definierten Zielen. Diese Form von „Articulation Work“ ist per Definition geplant („planned“) und damit explizit. Sie umfasst lediglich Planungsaktivitäten eines Individuums und umfasst die Klärung der Fragen „wer“ (in diesem Zusammenhang das Individuum selbst oder andere) „was“ (im Sinne des zu erreichenden Ziels) „wo“ (im Sinne des örtlichen, zeitlichen oder organisationalen Kontexts) „wie“ (im Sinne der Operationalisierung der Aktionen zur Zielerreichung) arbeitet.
	\item[Articulation of operation within action in individual activities] Die Auswahl und Ausführung von Operationen im Kontext einer Aktion erfolgt zumeist nicht bewusst basierend auf Erfahrungswissen. Tatsächlich kann die Auswahl von adäquaten Operationen als ein permanenter Fluss von mit der produktiven Arbeit verwobenen („situated“) „Articulation Work“-Vorgängen gesehen werden, der implizit auch in individuellen Arbeitssituationen abläuft. In diesem Zusammenhang ist es wichtig zu erwähnen, dass Operationen in „problematischen“ Situationen (im Sinne von Strauss) zu Aktionen werden können, die nicht mehr unbewusst und automatisiert ablaufen können. Mit dieser Transformation wird auch die „Articulation Work“ explizit und muss dass individuelle Vorgehen der geänderten Situation anpassen. 
	\item[Articulation of individual action within collective activity] Die Artikulation von Aktionen innerhalb eine kollektiven Tätigkeit geht in den Gegenständen der Artikulation über die im individuellen Fall zu berücksichtigenden Planungsaspekte („wer“, „was“, „wo“, „wie“) hinaus. Zusätzlich müssen um Zuge der Artikulation die Regeln der Kommunikation und Arbeitsteilung zwischen den am Arbeitsprozess Beteiligten artikuliert werden. Die Artikulation umfasst hier auch die die gegenseitige Offenlegung und Kenntnisnahme der individuellen „kognitiven Strukturen“ und existierender Annahmen über den Arbeitsablauf.
	\item[Articulation of individual operation within action in collective activity] Im Gegensatz zur individuellen Artikulation von Operationen im Kontext von Aktionen ist diese im kollektiven Fall seltener implizit abzuwickeln. Unterschiedliche Auffassungen über Herangehensweisen oder Missverständnisse bedürfen zum Teil einer expliziten Klärung  um die Zielerreichung zu gewährleisten. Operationen werden hier damit oft auf die Ebene von Aktionen gehoben und bewusst ausgehandelt.
	\item[Articulation of collective action in collective activity] Die Kategorie der kollektiven Aktion wird von \citep{Fjuk97} nicht im Detail behandelt, da die „Activity Theory“ selbst diese nicht behandelt und auch keinerlei anderen diesbezüglich verwendbaren Forschungsergebnisse verwendbar wären. Jede Tätigkeit involviert auch kollektive Aktionen wie Aushandlungen, Konsensfindung oder gemeinsame Problemlösung. Bei der Artikulation von kollektiven Aktionen müssen alle beteiligten Individuen ihre Perspektive, ihr Wissen und ihre Überlegungen einbringen um die gemeinschaftliche Entwicklung voranzutreiben. \citet{Fjuk97} treffen hier keine Aussagen hinsichtlich der Implikationen für „Articulation Work“.
	\item[Articulation of operations within collective actions in collective activity] Bei Zusammenarbeit auf Aktionsebene kann es durch die per Definition nicht bewusst geplante Durchführung der individuellen Operationen zu Zielerreichung der kollektiven Aktion zu konfliktionären Situationen kommen. Vor allem, wenn die individuellen Vorstellungen des Arbeitsablaufs divergieren („weak common conceptual structures“), kann es notwendig sein, explizite „Articulation Work“ anzustoßen, um diese Vorstellungen offenzulegen und abzugleichen.
\end{description}

Innerhalb eines Arbeitsablaufs können auch mehrere der hier beschriebenen Kategorien auftreten. Teile von Arbeitsabläufen können durch Änderungen im Arbeitskontext die Kategorie wechseln und somit mehr oder weniger explizite „Articulation Work“ notwendig machen. Durch die Unterscheidung zwischen kollektiver Tätigkeit und Aktion wird es möglich „Articulation Work“ je nach Enge der Interaktion und den damit auftretenden unterschiedlichen Artikulationsbedürfnissen entsprechend auszulegen.

\subsection{Unterscheidung nach Hampson und Junor}
\label{sub:arten_hampson}

\citep{Hampson05} verwenden „Articulation Work“ als Framework zur Erklärung von „interactive customer service“, also dem jenen Kundenbeziehungen, bei denen die Interaktion zwischen Anbieter und Kunden im Vordergrund steht. Im Rahmen dieser Arbeit zeigen die Autoren auch die historische Entwicklung des Begriffs „Articulation Work“ auf und entwickeln einen Raster zur Einordnung unterschiedlicher Ausprägungen von Arbeit, die wiederum unterschiedliche Arten von „Articulation Work“ bedingen. Dieses Raster ist hier von Interesse.

Bezugnehmend auf \citep{Strauss93} unterschieden die Autoren einerseits zwischen Arbeitsabläufen, die \emph{routine} sind, und solchen, die \emph{non-routine} sind. Außerdem kann zwischen Arbeitsabläufen unterschieden werden, die \emph{visible} oder \emph{invisible} sind \citep{Star99}. Während \emph{visible work} all jene Arbeitsabläufe umfasst, die als solche wahrgenommen werden, bezieht sich \emph{invisible work} auf alle Arbeitsabläufe, die stattfinden aber nicht „offiziell“ wahrgenommen werden (also etwa nicht in einem Prozessmodell aufscheinen).  Daraus ergeben sich vier zu unterscheidende Settings, in denen „Articulation Work“ stattfindet und die sich sowohl in der konkret als „Articulation Work“ ausgeführten Tätigkeit als auch in der möglichen methodischen und/oder technischen Unterstützung unterscheiden.

\begin{description}
	\item[Visible routine work] beschreibt jene Arbeitsabläufe, die von klassischen Management-Ansätzen erfasst werden, formalisiert werden können und in Unternehmen oft normiert vorgegeben sind (etwa in Form von Prozessmodellen oder durch die Vorgaben eines Workflow-Management-Systems). „Articulation Work“ findet hier zu definierten Zeitpunkten und explizit ausgelöst statt, um die normierten Abläufe zu definieren bzw. diese an veränderte Rahmenbedingungen anzupassen. 
	\item[Visible non-routine work] beschreibt Arbeitsabläufe in Umgebungen, die so dynamisch sind, dass normierte Abläufe aufgrund der raschen, nicht absehbaren Veränderungen der Anforderungen nicht sinnvoll einsetzbar sind. „Articulation Work“ tritt hier regelmäßig implizit und explizit auf, da jede Veränderung eine -- je nach Ausmaß der Veränderung implizite oder explizite -- Neuabstimmung der Zusammenarbeit nach innen und außen benötigt.
	\item[Invisible routine work] umfasst all jene Arbeitsabläufe in Unternehmen, die zwar etabliert sind, von den traditionellen Steuer- und Kontroll-Werkzeugen im Unternehmen jedoch nicht erfasst werden. Sie sind formal nicht normiert, treten jedoch so regelmäßig auf, das sich eine routinemäßige Herangehensweise herausbildet. Articulation Work läuft hier bei Veränderungen der Rahmenbedingungen zumeist implizit ab und sorgt dafür, dass die Interaktion zwischen den Beteiligten weiter funktioniert. Explizite „Articulation Work“ unter Einbeziehung der betroffenen Personen kann hier dafür sorgen, Arbeitsabläufe dieser Kategorie in den Bereich der „visible routine work“ überzuführen.
	\item[Invisible non-routine work] umfasst jene Arbeitsabläufe, die zur Behandlung von unvorhergesehenen Anforderungen durchgeführt werden und die nach außen hin nicht sichtbar wird. Typisch treten derartige Situationen bei Ausnahmefällen in etablierten Arbeitsabläufen auf, bei denen die Tätigkeiten zu Wiederherstellung einer „regelkonformen“ Situation oft nicht durch Steuer- und Kontrollelemente erfasst werden und durch die Einzigartigkeit der Ausnahme oder des Kontexts, in dem diese auftritt, keine etablierten Handlungsmuster existieren. „Articulation Work“ ist hier ad-hoc notwendig, um adäquat auf die Anforderungen der Umwelt reagieren zu können. Sowohl explizite und implizite „Articulation Work“ kann hier zu Anwendung kommen, wobei als Entscheidungskriterien zwischen diesen beiden Ausprägungen die wahrgenommene Komplexität der Situation sowie die zur Lösung zur Verfügung stehende Zeit zu berücksichtigen sind.
\end{description}

In unterschiedlichen Arbeitssituationen können diese vier Kategorien auch kombiniert auftreten. Auch hier können manche Arbeitsabläufe durch erfolgreich durchgeführte „Articulation Work“ in eine andere Kategorie verschoben werden, wo der Bedarf an laufender ad-hoc Abstimmung geringer oder nicht vorhanden ist. Andere Arbeitsabläufe sind ihrer Natur nach nicht strukturierbar und formalisierbar, so dass „Articulation Work“ ein inhärenter Bestandteil des Ablaufs ist und trotz wiederholter Durchführung auch bleibt.

\subsection{Zusammenfassung} % (fold)
\label{sub:aw_zusammenfassung}

In diesem Abschnitt wurden drei Arbeiten näher vorgestellt, die sich der  Strukturierung des Konzepts „Articulation Work“ widmen. Die grundlegende Strukturierung bietet bereits \citet{Strauss85} (bzw. \citet{Strauss88} und \citet{Strauss93}). Die beiden übrigen Arbeiten bauen auf \citeauthor{Strauss85} auf und vertiefen das Verständnis von „Articulation Work“ weiter, in dem sie vor allem den im Zuge von „Articulation Work“ behandelten Gegenstand weiter ausdefinieren und strukturieren. Die beiden Arbeiten gehen hierbei unterschiedliche Wege. \citet{Fjuk97} setzen „Articulation Work“ in Beziehung zur aus der Psychologie stammenden „Activity Theory“ während \citet{Hampson05} im Kontext der Soziologie bleibt und neben den Arbeiten von Strauss z.B. auch auf \citep{Star99} aufbaut.

Fasst man die Konzepte zur Strukturierung von „Articulation Work“ aus allen drei Arbeiten zusammen, so ergib sich folgender Überblick: 

\begin{itemize}
	\item Art der „Articulation Work“
	\begin{itemize}
		\item implizit vs. explizit
		\item situated vs. planned
	\end{itemize}
	\item Gegenstand der „Articulation Work“
	\begin{itemize}
		\item routine vs. non-routine work bzw.
		\item routine vs. problematic interaction (mit der belebten oder unbelebten Umwelt)
		\item visible vs. invisible work
		\item individual activity vs. collective activity vs. collective action
	\end{itemize}
	\item Abstraktionsgrad des Gegenstandes der „Articulation Work“
	\begin{itemize}
		\item activity-action vs. action-operation
	\end{itemize}
\end{itemize}

Bezüglich der unterschiedlichen Arten von „Articulation Work“ sind zwei Gegensatzpaare zu identifizieren, die orthogonal zueinander stehen (auch wenn die Hauptachse „implizit -- situated“ vs. „explizit -- planned“ ist). „Situated Articulation Work“ tritt ungeplant während des Arbeitsablaufs auf und dient der ad-hoc- Abstimmung. Obwohl diese in den meisten Fällen implizit abläuft, sind doch Fälle vorstellbar, in denen eine explizite, d.h. bewusst durchgeführte, „Articulation Work“ sinnvoll bzw. notwendig ist (siehe weiter unten -- Gegenstand der „Articulation Work“). „Planned Articulation Work“ hingegen ist immer explizit, die Kombination von geplanter und implizit, d.h. unbewusst durchgeführter, „Articulation Work“ ist nicht sinnvoll.

Hinsichtlich des Gegenstandes von „Ariculation Work“ sind vier unterschiedliche Unterscheidungskategorien zu identifizieren, die wiederum zum Teil orthogonal sind. Die Ausprägungen der jeweiligen Kategorien weisen zum Teil auf die Art der durchzuführenden Articulation Work hin. 

Die Kategorie „routine -- non-routine work“ bezieht sich darauf, ob der fragliche Arbeitsablauf für die beteiligten Personen alltäglich ist und unter bekannten Rahmenbedingungen stattfindet oder nicht. Je stärker der „non-routine“-Anteil in einem Arbeitsablauf zum Tragen kommt, desto expliziter muss im Allgemeinen die „Articulation Work“ sein -- bei Routine-Arbeit ist der Bedarf an Articulation gering und beschränkt sich auf implizit durchführbare Detailabstimmungen zwischen den Beteiligten. 

Obwohl vordergründig unterschiedlich bezieht sich die nächste Kategorie „routine -- problematic interaction“ auf den gleichen Sachverhalt. \citet{Strauss93}, von dem diese Unterscheidung stammt, bezeichnet Interaktion als die Grundlage von Arbeitsabläufen und als wesentlichen Bestandteil derselben. Der „routine“-Begriff kann deshalb mit jenem der zuvor beschriebenen Unterscheidung geleichgesetzt werden. Der Begriff der „problematic interaction“ beschreibt insofern das gleiche Phänomen wie der der „non-routine work“ als dass er sich ebenfalls auf die erhöhte kognitive Belastung der beteiligten Personen bei der Zielerreichung bezieht. Dementsprechend impliziert „problematic interaction“ eine explizite „Articulation Work“, während „routine interaction“ durch implizite „Articulation Work“ produktiv gehalten werden kann.

Die Unterscheidung zwischen „visible“ und „invisible work“ bezieht sich auf die Kenntnisnahme eines Arbeitsablaufs in seinem Durchführungskontext und dessen durch andere, vor allem auch übergeordnete Instanzen. Während „visible work“ definierten Aufgaben dient, formalisiert werden kann und durch Steuer- und Kontrollinstrumente oder organisationale Unterstützungswerkzeuge erfasst werden kann, bleibt „invisible work“ im organisationalen Kontext verborgen und ist nur für die handelnden Individuen sichtbar (und wird dementsprechend auch organisational nicht unterstützt und wertgeschätzt). Für „Articulation Work“ hat dies per se keine unmittelbaren Auswirkungen, außer dass „visible work“ immer ein Ergebnis expliziter „Articulation Work“ ist. Dies bedeutet gleichzeitig, dass explizite „Articulation Work“ (unter Einbeziehung sowohl der unmittelbar am Arbeitsablauf beteiligten Personen als auch der „übergeordneten“ Instanzen) dazu beitragen kann, „invisible work“ zu „visible work“ zu machen (siehe dazu auch \citep{Fujimura87}). „Articulation Work“ ist somit ein Mittel, einen Abgleich zwischen dem offiziellen (organisational festgeschriebenen) Verständnis eines Arbeitsablaufs und dem tatsächlichen Ablauf, wie er in der Praxis ausgeführt wird, durchzuführen. „Articulation Work“ kann damit eine Realisierung eines organisationalen Lernschritts sein, der im Sinne von \citet{Argyris78} die „espoused theories“ (die offiziell veröffentlichten Theorien über Arbeit) mit den „theories-in-use“ (die tatsächlich handlungsleitenden Theorien) abgleicht bzw. im Sinne von \citet{Sachs95} einen „tacit organisational view“ in einen „explicit organisational view“ überführen (siehe dazu auch die Ausführungen in Kapitel XY (Einführung)).

Die Enge der notwendigen Kooperation bei der Durchführung eines Arbeitsablaufs (festgemacht an den Handlungs-Kategorien der „Activity Theorie“) ist Gegenstand der letzten Kategorie. „Individual activity“ beschreibt Arbeitsabläufe, die im Wesentlichen von einem Individuum ausgeführt werden und lediglich an den Schnittstellen zu Beginn und am Ende Interaktion benötigen. „Collective Activity“ beschreibt Arbeitsabläufe, in denen mehrere Individuen klar abgegrenzte Teile der Arbeit übernehmen und Interaktion an festgelegten Schnittstellen bzw. zu festgelegten Zeitpunkten stattfindet. Dies entspricht im Wesentlichen der klassischen Arbeitsteilung in Unternehmen. „Collective Action“ beschreibt tatsächlich kollaborative Arbeit im engeren Sinn, deren Durchführung nur durch enge Interaktion mehrere Individuen auch in Detailaspekten notwendig ist. Je enger die Kooperation, desto notwendiger wird „Articulation Work“, wobei diese in allen Fällen sowohl in ihrer impliziten als auch expliziten Ausprägung zum Einsatz kommen kann. 

Zur Identifikation der im Einzelfall sinnvollen Variante von „Articulation Work“ (implizit oder explizit) ist die Berücksichtigung der letztgenannten Unterscheidung hinsichtlich des Abstraktionsgrades der „Articulation Work“ notwendig. Beschäftigt sich „Articulation Work“ mit der abstrakteren Ebene zwischen „activity“ und „action“, steht vorrangig die Betrachtungsdimension „Was?“ (also die Ziele und das generelle Vorgehen) im Zentrum. Bei Arbeitsabläufen, die „non-routine“ sind, kommt dabei sowohl „situated“ als auch „planned“ eher explizite Articulation Work zum Einsatz. Bei „routine“-Arbeitsabläufen ist Articulation Work auf dieser Ebene „situated“ nicht notwendig und kommt nur zu Anwendung, wenn der Ablauf selbst („planned“) hinterfragt werden soll (und ist dann seiner Natur nach explizit). Auf der konkreten Ebene zwischen „action“ und „operation“ (also der Frage nach dem „Wie?“) kommt in individuell abgehandelten Arbeitsabläufen vorrangig implizite „Articulation Work“ zum Einsatz. Treten unvorhergesehene Probleme auf oder erscheinen etablierte Operationen nicht mehr adäquat, kommt zur Klärung wieder explizite Articulation Work zum Einsatz. In kollektiven Arbeitsprozessen ist die Enge der Interaktion entscheidend. Bei klar separierbaren Arbeitsanteilen (also bei Interaktion auf „activtiy“-Ebene) bleibt die Entscheidung zur konkreten Umsetzung beim Individuum und die „Articulation Work“ im Normalfall implizit (Ausnahmen siehe Ausführungen zur individuellen Arbeitsabläufen). Bei Arbeitsabläufen mit enger Interaktion auf Aktionsebene muss diese im Normalfall „planned“ im Vorhinein durch explizite „Articulation Work“ ausgehandelt werden. Während des Arbeitsablaufs (also „situated“) kann wiederum implizite „Articulation Work“ zurückgegriffen werden, wobei auf Grund der Anzahl der beteiligten Personen die Wahrscheinlichkeit steigt, das ein beteiligtes Individuum die Interaktion als „problematic“ empfindet und dies wiederum explizite „Articulation Work“ notwendig macht.

Insgesamt können drei Arten von „Articulation Work“ unterschieden werden:
\begin{itemize}
	\item „situated“ implizite „Articulation Work“
	\item „situated“ explizite „Articulation Work“
	\item „planned“ explizite „Articulation Work“
\end{itemize}
Die Kombination „planned“ - implizit ist insofern nicht sinnvoll, als dass geplante „Articulation Work“ deren bewusste Durchführung impliziert und diese deshalb immer explizit ist.

Der am häufigsten auftretende Fall von „Articulation Work“ ist jener, der „situated“ (also im Zuge des Arbeitsprozesses) implizit (also unbewusst) auftritt. Implizite „Articulation Work“ ist integraler Bestandteil jedes Arbeitsprozesses. Explizite „Articulation Work“ ist bei „situated“ Durchführung als Eskalationsstufe zu sehen. Wenn der Arbeitsprozess von zumindest einem Beteiligten als problematisch („problematic“ bzw. „non-routine“) angesehen wird, ist eine bewusste Beschäftigung mit der Arbeitsausführung auf abstrakter („Was?“) und / oder konkreter („Wie?“) Ebene notwendig. Sobald die Probleme beseitigt sind, ist eine Fortsetzung der Arbeit mit impliziter „Articulation Work“ möglich. Geplante Articulation Work dient der initialen Planung und Abstimmung 
von neuen Arbeitsabläufen bzw. der Reflexion und Verbesserung von bereits existierenden Arbeitsabläufen. Sie beschäftigt sich eher mit abstrakten Aspekten des Arbeitsablaufs („Was?“), nur im Falle von enger Kooperationsnotwendigkeit in konkreten Tätigkeiten kann auch die Frage nach dem „Wie?“ relevant sein. Geplante explizite „Articulation Work“ ist auch das Mittel der Wahl „invisible work“ auf organisationaler Ebene sichtbar zu machen und damit Wissen über die tatsächliche Durchführung von Arbeitsabläufen in einer Organisation zu verteilen bzw. deren formale Anerkennung zu ermöglichen.

% subsection aw_zusammenfassung (end)
% section arten_von_articulation_work (end)

\section{Unterstützung von Articulation Work} % (fold)
\label{sec:unterstützung_von_articulation_work}

Nach den ersten Arbeiten von Strauss zum Thema „Articulation Work“ wurde das Konzept rasch als Erklärungsmodell für die Vorgänge im Zuge kooperativer Arbeit aufgenommen. Bereits \citeyear{Gerson86} verweisen \citeauthor{Gerson86} auf die Notwendigkeit einer expliziten Unterstützung von „Articulation Work“\footnote{\emph{„Methods for analyzing 
due process means, in this perspective, explicit procedures for evaluating and reconciling incompatibilities among different bodies of tacit local knowledge.“}\citep[][S. 266]{Gerson86}}. Anhand der historischen Entwicklung von Mitte der 1980er-Jahre bis Ende des ersten Jahrzehntes des neuen Jahrtausends werden im Folgenden Maßnahmen zur Unterstützung beschrieben und in den jeweiligen Anwendungskontext gesetzt. Hierbei werden alle Arbeiten berücksichtigt, die sich direkt auf den von Strauss geprägten „Articulation Work“-Begriff beziehen. In der Literatursuche wurden dazu Datenbanken aus den Bereichen Informatik, Psychologie, Soziologie, den Wirtschaftswissenschaften sowie der Organisationslehre durchsucht. Nach der initialen Suche wurde jeweils auch die in den gefundenen Arbeiten referenzierte Sekundärliteratur aufgearbeitet. Des weiteren wurden mit Hilfe von rückwärts verlinkenden Datenbanken (wo vorhanden) Publikationen erfasst, die die bislang gefundenen Arbeiten referenzieren und diese hinsichtlich ihrer Relevanz überprüft. Insgesamt ergab sich so eine Sammlung von 47 Publikationen (inklusive der hier nicht nochmals behandelten grundlegenden Arbeiten, die bereits oben beschrieben wurden). Von diesen 47 Publikationen trafen XY eine Aussage zu Aspekten, die auf die Unterstützung von „Articulation Work“ abzielen. Die übrigen Arbeiten verwenden „Articulation Work“ als Erklärungs-Framework für Fallstudien und werden weiter unten zusammenfassend angeführt ohne näher auf sie einzugehen.

Zur strukturierten Umsetzung der Betrachtung der Unterstützung von „Articulation Work“ wird ein einheitlicher Raster angewandt, anhand dessen die aus unterschiedlichen Forschungsgebieten stammenden und in unterschiedlichen Anwendungsdomänen angewandten Arbeiten einander gegenüber gestellt werden können. Neben den eigentlichen Unterstützungsmaßnahmen ist zur Bewertung derselben auch Kontextinformation notwendig, die die unterschiedlichen Ansätze offenlegt. Folgende Merkmale bzw. Inhalte einer Arbeit werden dazu betrachtet:

\begin{description}
	\item[Kontext] Forschungsgebiet aus dem das Konstrukt „Articulation Work“ betrachtet wird bzw. in dessen Kontext es zur Anwendung gebracht wird und / oder abstraktes oder konkretes Problemfeld, in dem „Articulation Work“ als Analysedimension oder zur Ableitung von Maßnahmen angewandt wird.
	\item[Unterstützung] Konkrete oder abstrakte Maßnahmen oder Werkzeuge, die zur Unterstützung von „Articulation Work“ vorgeschlagen und/oder umgesetzt werden. Ggf. unterschieden in
	\begin{itemize}
		\item organisationale Unterstützung
		\item methodische Unterstützung
		\item technische Unterstützung
	\end{itemize}
	\item[Auswirkungen] Tatsächliche oder vermutete Auswirkungen der Unterstützung auf die durchgeführte „Articulation Work“.
\end{description}

Die als relevant betrachteten Publikationen sind methodisch unterschiedlich ausgerichtet. Ein großer Anteil beschreibt rein empirisch-deskriptiv ein beobachtetes Phänomen und zieht Schlüsse hinsichtlich möglicher bzw. notwendiger Ausprägungen von "Articulation Work" in bestimmten Anwendungsdomänen. Ein anderer Teil fokussiert auf die organisationale und/oder technische Unterstützung von "Articulation Work", zum Teil ohne auf eigene empirische Ergebnisse aufzubauen oder diese zu erheben. Aus diesem Grund kann das oben angegebene Raster nicht immer vollständig befüllt werden. Wo hinsichtlich einer bestimmten Dimension keine Information vorhanden ist, wird explizit im Text darauf hingewiesen. Wo mehrere Publikation eines Autors oder einer Gruppe zum gleichen Forschungsgegenstand existieren, wurden diese in einem Abschnitt zusammengefasst und in der jeweiligen Einleitung auf die der Beschreibung zugrunde liegenden Publikationen verwiesen.

%TEMPLATE
\subsection{Papertitel / Titel des Forschungsprojekts}

Angabe der zugrundeliegenden Publikation sowie einer kurzen Zusammenfassung des Inhalts

\subsubsection{Kontext}

\subsubsection{Unterstützung}

\subsubsection{Auswirkungen}

\subsubsection{Bewertung}
j
\\[1em]
\begin{tabular}{| p{3cm} | p{10cm} |}
  \hline
  Domäne &  \\ \hline
  Art von AW &  \\ \hline
  Unterstützung &  \\ \hline
  Auswirkungen & \\ \hline
\end{tabular}
%TEMPLATE

\subsection{Modeling Articulation Work in Software Engineering Processes} % (fold)
\label{sub:modeling_articulation_work_in_software_engineering_processes}

Die erste Publikation, die sich mit der expliziten Berücksichtigung von „Articulation Work“ in formalisierten Ablaufmodellen (und damit mit einer organisationalen Unterstützung von „Articulation Work“) beschäftigt, ist die Arbeit von \citet{Mi91}.

\subsubsection{Kontext}

\citet{Mi91} betrachten „Articulation Work“ im Kontext der Softwareentwicklung und argumentieren für deren explizite Berücksichtigung in Software Engineering Prozessen. In einer Literaturstudie zeigen sie, dass (die zum Zeitpunkt der Erstellung) verfügbaren Software-Engineering-Prozess-Modellierungs-Techniken die Einbindung von „Articulation Work“ in die Vorgehensmodelle nicht ermöglich\footnote{\emph{After we compare these findings with the process modeling techniques, it becomes apparent that current software process modeling techniques do not directly address articulation work.}\citep[][S. 192]{Mi91}}. Die Autoren selbst haben ihren Hintergrund ebenfalls in der Domäne des Software Engineering und wählen ihren Zugang zu Thematik dementsprechend, indem sie eine formale Abbildung des „Articulation Work“-Prozesses und eine Unterstützung durch regelbasierte Heuristiken zur Lösungsfindung vorschlagen.

\subsubsection{Unterstützung}

Die Autoren formalisieren im ersten Schritt den Ablauf von „Articulation Work“ im Kontext von Softwareengineering (siehe Abbildung \ref{fig:img_ArticulationWork_mi91-awprocess}). Die einzelnen Schritte leiten sie aus drei empirischen Studien ab, die sowohl hinsichtlich ihres Inhalts als auch ihrer Durchführung nicht näher beschrieben werden.

\begin{figure}[htbp]
	\centering
		\includegraphics[height=3in]{img/ArticulationWork/mi91-awprocess.png}
	\caption[Artikulations-Prozess]{Artikulations-Prozess (entnommen aus \citep{Mi91})}
	\label{fig:img_ArticulationWork_mi91-awprocess}
\end{figure}

Die Autoren beziehen sich also offensichtlich auf explizite „Articulation Work“, die „situated“ -- beim Auftreten eines Problems im Software Engineering Prozess -- ausgelöst wird.

Zur Durchführung dieses Prozesses schlagen die Autoren einen Satz von regelbasierten Heuristiken vor, aus denen die betroffenen Individuen (hier: „agents“) auswählen können. Diese Heuristiken beschreiben mögliche Tätigkeiten im Zuge der „Articulation Work“ (ausdefiniert durch \gls{ECA}-Regelsätze). Dabei geben die Autoren Heuristiken zur Problemlösung („problem-solving heuristics“, die der direkt Behebung der aufgetretenen Probleme dienen) und Heuristiken zur Auswahl geeigneter Lösungen („selection heuristics“) an.

\subsubsection{Auswirkungen}

Das vorgeschlagene System wurde auf konzeptioneller Ebene entwickelt und nicht praktisch umgesetzt. Insofern existieren keinerlei reale Erfahrungen mit dem Ansatz. Anhand eines hypothetischen Beispiels demonstrieren die Autoren jedoch die Anwendung des Modells und der Heuristiken.

Der Vorteil liegt im Wesentlichen darin, dass der „Articulation Work“-Prozess durch seine Formalisierung bekannt ist („visible“ in Sinne der obigen Ausführungen) und dementsprechend auch offiziell auftreten „darf“. Durch den vorgegebenen Satz an Heuristiken sind außerdem die Alternativen zur Problembehandlung und deren Durchführung bekannt. Die Autoren geben diese Heuristiken für den Bereich des Software-Engineering an, betonen aber deren exemplarischen Charakter -- die Heuristiken und vor allem deren konkrete Umsetzung (durch die Spezifikation von \gls{ECA}-Regeln) müssen an die jeweilige Arbeits-Domäne angepasst werden. 

\subsubsection{Bewertung}

Das vorgeschlagene Prozess-Modell von „Articulation Work“ bildet den Ablauf auf so abstrakter Ebene ab, dass es für „situated explicit Articulation Work“ zur Lösung unmittelbar auftretender Probleme allgemein (d.h. unabhängig von der Anwendungsdomäne anwendbar erscheint.

Die Angabe von (exemplarischen) Heuristiken zur Durchführung des Artikulations-Prozesses erscheint insofern sinnvoll, als dass diese domänen- und organisations-spezifische Lösungsstrategien auch für unerfahrene Teilnehmer zugänglich machen können.

In Bezug auf die Generalisierbarkeit des Ansatzes problematisch zu sehen ist jedoch die Verwendung von \gls{ECA}-Regeln zur Spezifikation der Durchführung der einzelnen Heuristiken. Die Angabe derartiger Regeln erscheint nicht in allen Anwendungsbereichen in einem sinnvollen Detaillierungsgrad möglich zu sein. Vor allem soziale Prozesse in kooperativen Umgebungen, auf die die Autoren der ursprünglichen Literatur zum Thema „Articulation Work“ stark Bezug nehmen, können in diesen Regeln nicht sinnvoll (im Sinne einer Durchführungsvorschrift) abgebildet werden.
\\[1em]
\begin{tabular}{| p{3cm} | p{10cm} |}
  \hline
  Domäne & Software Engineering \\ \hline
  Art von AW & situated explizit \\ \hline
  Unterstützung & \emph{organisational} durch Formalisierung und a-priori-Spezifikation des Articulation-Prozesses („Was?“) und dessen Ausgestaltung („Wie?“) \\ \hline
  Auswirkungen & Definiertes Vorgehen, wie „Articulation Work“ durchgeführt werden muss \\ \hline
\end{tabular}

% subsection subsection_name (end)

\subsection{Taking CSCW seriously: Supporting Articulation Work}

\citet{Schmidt92} begründen mit dieser Arbeit eine Entwicklungsrichtung der \gls{CSCW}, die neben der Unterstützung der eigentlichen produktiven Arbeit auch auf die Unterstützung von „Articulation Work“ fokussiert. Sie beschreiben damit erstmals Anforderungen an die und Möglichkeiten der technische Unterstützung von „Articulation Work“.

\subsubsection{Kontext}

\citet{Schmidt92} widmen sich in ihren Ausführungen der kooperativen Arbeit und der Unterstützung derselben durch Computersysteme. 

\subsubsection{Unterstützung}

\subsubsection{Auswirkungen}

\subsubsection{Bewertung}
Bleibt auf Ebene der Anforderungen, trifft wenig konkrete Aussagen zur Umsetzung.
\\[1em]
\begin{tabular}{| p{3cm} | p{10cm} |}
  \hline
  Domäne & \gls{CSCW} (Workflow-Support, Shared Information Spaces, Cooperative Tools)\\ \hline
  Art von AW & planned explizit, situated explizit und implizit \\ \hline
  Unterstützung & \emph{technisch} \\ \hline
  Auswirkungen & \\ \hline
\end{tabular}


\subsection{Gegenüberstellung und Zusammenfassung} % (fold)
\label{sub:gegenüberstellung_und_zusammenfassung}

% subsection gegenüberstellung_und_zusammenfassung


% subsection modeling_articulation_work_in_software_engineering_processes (end)

\subsection{Weitere Arbeiten zur Thema Articulation Work} % (fold)
\label{sub:weitere_arbeiten_zur_thema_articulation_work}

Die im Folgenden genannten Arbeiten beziehen sich in Einzelaspekten ebenfalls auf „Articulation Work“, treffen jedoch keine Aussage hinsichtlich einer etwaigen Unterstützung derselben. Zumeist wird „Articulation Work“ als erklärendes Rahmenwerk für beobachtete Phänomene verwendet und in der Folge das Hauptaugenmerk auf diese gelegt, ohne nochmals näher auf „Articulation Work“ einzugehen. Aufgenommen wurden auch jene Arbeiten, die sich mit der grundlegenden Konzeption von Articulation Work beschäftigen (und deshalb oben bereits im Detail behandelt wurden), aber keine Aussage zur Unterstützung von „Articulation Work“ treffen. In chronologischer Reihenfolge des Erscheinens trifft dies auf folgende Arbeiten zu:
\begin{description}
	\item[\citet{Strauss85}] prägt in dieser Arbeit den Begriff „Articulation Work“ und beschreibt dieses auf konzeptioneller Ebene ohne eine unmittelbaren Praxis- bzw. Umsetzungsbezug herzustellen.
	\item[\citet{Gasser86}] beschreibt die Integration von Computerunterstützung in alltägliche Arbeitsabläufe und die Anpassungsleistung der arbeitenden Individuen, wenn die aktuelle Arbeitssituation nicht mehr mit dem der Computerunterstützung zugrunde liegenden Modell übereinstimmt. Er identifiziert dabei spezifische Aktivitäten, die im Rahmen der ablaufenden „Articulation Work“ auftreten können.
	\item[\citet{Gerson86}] zeigen die konkrete Manifestation von Articulation Work in einer Fallstudie aus einem Versicherungskonzern und identifizieren daraus die organisationalen Rahmenbedingungen, die zu jenen Problemen führen, die „Articulation Work“ notwendig machen.
	\item[\citet{Bendifallah87}] untersuchen bezugnehmend auf \citet{Gasser86} „Articulation Work“ im Kontext von IT-Support-Arbeit in Unternehmen anhand von zwei Fallstudien und identifizieren dabei zwei unterschiedliche Strategien bei der Durchführung derselben. Im Detail gehen sie jedoch nicht auf die konkret zu setzenden Maßnahmen ein.
	\item[\citet{Fujimura87}] leitet die grundlegende Unterscheidung zwischen „Production Work“ und „Articulation Work“ anhand einer Fallstudie aus dem wissenschaftlich-medizinischen Forschungsbetrieb ab. Sie bleibt dabei auf konzeptueller Ebene und beschreibt die auftretenden Phänomene, geht jedoch nicht auf unterstützende Maßnahmen ein.
	\item[\citet{Strauss88}] detailliert und erweitert seine Konzepte und setzt diese in den Kontext organisationaler Projektarbeit (in dort beschrieben Verständniss im Wesentlichen identisch mit „non-routine“ „collective activity“). Anhand einer Fallstudie aus dem Krankenhaus-Organisations-Bereich zeigt er das Auftreten in der Praxis, beschäftigt sich jedoch nicht mit möglicherweise unterstützenden Interventionen.
	\item[\citet{Schmidt90}] beschreibt ein Framework für die Analyse kooperativer Arbeit und erwähnt dabei „Articulation Work“ als ein zu berücksichtigendes Konzept. Diese Arbeit bildet die Grundlage für die im Hinblick auf die Unterstützung von „Articulation Work“ relevantere Arbeit von \citet{Schmidt92}.
\end{description}

% subsection weitere_arbeiten_zur_thema_articulation_work (end)

% section unterstützung_von_articulation_work (end)

\section{Fazit} % (fold)
\label{sec:fazit}

\textbf{hier muss eine zusammenfassende Tabelle der in der Literatur verfügbaren Information rein}

Die Zielsetzung von „Articulation Work“ formulieren die Proponenten des Ansatzes - allen voran Strauss - klar aus. Offen bleiben jedoch bei allen Autoren direkten Aussagen zum eigentlichen Gegenstand von „Articulation Work“ – also Allem was von den beteiligten Individuen zu artikulieren ist – und den notwendigen Leistungen der Individuen im Prozess der Artikulation. Aussagen zu diesen Aspekten sind aber für die Entwicklung von Ansätzen zur Unterstützung von expliziter „Articulation Work“ notwendig. 

Strauss ist sich dieser Auslassung bewusst\footnote{\emph{„[\ldots] many social scientist pay almost no attention to interior activity: ignoring it, taking it for granted, but leaving it unexamined, or giving it the kind of abstract but not very detailed analysis [\ldots]“}\citep[][S. 131]{Strauss93}}, und beschäftigt sich in späteren Arbeiten \citep{Strauss93} auch mit jenen kognitiven Vorgängen, die von ihm als „thought processes“ oder „mental activities“ bezeichnet werden und die untrennbar mit jeder Art von Tätigkeit und Interaktion verbunden sind\footnote{\emph{„These [thought processes] accompany visible action, as well as precede and follow in conditional and consequential modes“}\citep[][S. 146]{Strauss93}} und diese beeinflussen\footnote{\emph{„Even well-grooved, routine action and interaction may be accompanied by thought [\ldots] directly relevant to the work at hand. As I vacuum the house, barely noticing my movements, still I give myself commands [\ldots]“}\citep[][S. 132]{Strauss93}}. 

Im Kontext der Abstimmung von Tätigkeiten kommt den „thought processes“ der Individuen große Bedeutung zu, da sie den sichtbaren individuellen Handlungen zugrunde liegen bzw. diese beeinflussen. „Articulation Work“ wirkt sich also auf die „thought processes“ der beteiligten Individuen aus. „Thought processes“ umfassen \emph{„images, imaginations, projections of scenes, [...] flashes of insight, rehearsals of action, construction and reconstruction of scenarios, the spurting up of metaphors or comparisons, the reworking and reevaluating of past scenes and one's actions within them, and so on and on“} \citep[][S. 130]{Strauss93} - also im Wesentlichen alle kognitiven Vorgänge, die unmittelbar oder mittelbar im Zusammenhang mit den sichtbaren Arbeitsaspekten, insbesondere den Tätigkeiten zur Zielerreichung und der wahrgenommenen Arbeitsumgebung, stehen. Strauss interessiert sich allerdings ausschließlich für die dynamischen Aspekte der Interaktion zwischen Individuen, nicht aber für die Ausgangspunkte und Ergebnisse der zugrunde liegenden „thought processes“.\footnote{\emph{„I use the gerund 'ing' after 'symbol' [bei der Beschreibung von 'symbolizing', Anm.] to signify that my principal interest is, again, in interaction rather than its products, for symbols are precipitates of interaction“}\citep[][S. 149]{Strauss93}}  Wie bereits oben erwähnt sind aber die Repräsentationen, auf den „thought processes“ beruhen und operieren, für die Unterstützung von „Articulation Work“ von Interesse. Die kognitions-wissenschaftlichen Ansätze zu Schemata (\citep{Rumelhart78} \citep[vgl. nach ][]{Hanke06}) und mentalen Modellen (\citep[vgl. ][]{Seel91}) sind ein Erklärungsansatz für diese Lücke.

% section fazit (end)
% chapter articulation_work (end)


\chapter{Mentale Modelle} % (fold)
\label{cha:mentale_modelle}

In diesem Kapitel wird das Konzept der mentalen Modelle eingeführt, das in dieser Arbeit als Erklärungsansatz für jene Aspekte von "Articulation Work" verwendet wird, die die nicht sichtbaren, kognitiven Beiträge eines beteiligten Individuums betreffen. Nach einer Einführung in die Begriffswelt der mentalen Modelle wird die Argumentation aus dem letzten Kapitel nochmals aufgegriffen und die mögliche Rolle mentaler Modelle für "Articulation Work" erörtert. In der Folge werden Methoden eingeführt mit denen mentale Modelle externalisiert und kommuniziert werden können. Basierend auf diesen Beschreibungen wird im letzten Teil des Kapitels untersucht, welche Herausforderungen sich bei der Anwendung dieser Methoden im Kontext von "Articulation Work" ergeben können.

\section{Articulation Work und mentale Modelle} % (fold)
\label{sec:articulation_work_und_mentale_modelle}

Wie bereits im vorgehenden Kapitel beschrieben, wird in vorhandenen Arbeiten zu Articulation Work deren Auftreten, Kontext und Wirkung beschrieben, nicht aber ihre Durchführung und der eigentliche Gegenstand der Abstimmung. Die Vermeidung von konkreten Aussagen zur Durchführung liegt in der Vielfalt möglicher Ausprägungen begründet. Schon \citet{Strauss88} unterscheidet grob zwischen expliziter und impliziter Articulation Work, begründet die Unterscheidung aus dem Kontext des Auftretens (siehe Abschnitt \ref{sec:arten_von_articulation_work}), lässt aber offen, wie sich implizite und explizite Articulation Work unterscheiden bzw. was explizite Articulation Work im Gegensatz zur ständig im Arbeitsverlauf auftretenden impliziten Articulation Work ausmacht. Der eigentliche Gegenstand der Abstimmung, die im Rahmen der Articulation Work erfolgen soll, wird ebenfalls nicht konkret festgelegt. Strauss spricht von \emph{„putting together tasks, task sequences, task clusters - even aligning larger units such as lines of work and subprojects - in the service of work flow“} \citep[][S. 2]{Strauss88}, und konkretisiert \emph{„the specific questions about tasks of course include: what, where, when, how, for how long, how complex, how weIl defined are their boundaries, how attainable are they under current working conditions, how precisely are they defined in their operational details, and what is the expected level of performance. (Which of those are the most salient dimensions depends on the organizational work context under study, and we cannot emphasize too much that \textbf{it is the researcher who must discover these saliences}.)“} \citep[][S. 6]{Strauss85}. Strauss lässt also offen, was es exakt ist, dass abgestimmt werden muss bzw. verlagert diese Frage in den konkreten Einzelfall. 

Strauss spricht diese Auslassung in einer späteren Arbeit explizit an \citep[][S. 131]{Strauss93} und beschäftigt sich in dieser auch mit jenen kognitiven Vorgängen, die von ihm als „thought processes“ oder „mental activities“ bezeichnet werden und die untrennbar mit jeder Art von Tätigkeit und Interaktion verbunden sind \citep[][S. 146]{Strauss93} und diese beeinflussen \citep[][S. 132]{Strauss93}.  

Im Kontext der Abstimmung von Tätigkeiten kommt den „thought processes“ der Individuen große Bedeutung zu, da sie den sichtbaren individuellen Handlungen zugrunde liegen bzw. diese beeinflussen. „Articulation Work“ wirkt sich also auf die „thought processes“ der beteiligten Individuen aus. „Thought processes“ umfassen \emph{„images, imaginations, projections of scenes, [...] flashes of insight, rehearsals of action, construction and reconstruction of scenarios,  the spurting up of metaphors or comparisons, the reworking and reevaluating of past scenes and one's actions within them, and so on and on“} \citep[][S. 130]{Strauss93} - also im Wesentlichen alle kognitiven Vorgänge, die unmittelbar oder mittelbar im Zusammenhang mit den sichtbaren Arbeitsaspekten, insbesondere den Tätigkeiten zur Zielerreichung und der wahrgenommenen Arbeitsumgebung, stehen. Strauss interessiert sich allerdings ausschließlich für die dynamischen Aspekte der Interaktion zwischen Individuen, nicht aber für die Ausgangspunkte und Ergebnisse der zugrunde liegenden „thought processes“ \citep[][S. 149]{Strauss93}

% section articulation_work_und_mentale_modelle (end)

\section*{Mentale Modelle -- Begriffsbestimmung}

Nach der Identifikation der begrenzten Erklärungsmächtigkeit des Konzepts "Articulation Work" wird nun die Theorie der "mentalen Modelle" herangezogen, um die bei Articulation Work offen bleibenden Aspekte hinsichtlich des Gegenstandes der Abstimmung und deren Untersützung näher zu betrachten. Dazu wird im ersten Schritt der Begriff der "mentalen Modelle" in den historischen Kontext gestellt und dessen Bedeutung dargelegt.

Das Konzept der "mentalen Modelle" wird grundsätzlich verwendet, um zu erklären \emph{"wie Menschen die Welt verstehen -- genauer: wie sie ihr Wissen benutzen, um sich bestimmte Phänomene der Welt subjektiv plausibel zu machen"} \citep[][S. VII]{Seel91}. Mentale Modelle sind dabei Erklärungsmodelle der Welt, die Menschen auf Basis von Alltagserfahrungen, bisherigem Wissen und darauf basierenden Schlussfolgerungen bilden. Ein gebildetes mentales Modell wird dann als Basis verwendet, um die Welt zu verstehen und ggf. Vorhersagen über deren Verhalten zu bilden. \citep[][S. VII]{Seel91}

Im Wesentlichen wurde das Forschungsfeld der mentalen Modelle durch zwei Arbeiten maßgeblich beeinflusst. \citet{Johnson-Laird81} und \citet{de-Kleer81} führen den Begriff als eigenständigen Forschungsgegenstand ein und legen damit die Grundlage für einen Großteil der nachfolgenden Arbeiten in dem Gebiet. Im Kontext dieser Arbeit werden dabei zwei dieser nachfolgenden Arbeiten näher betrachtet. Zum einen stellt \citet{Norman83} den Begriff erstmals im den Kontext der Mensch-Maschine-Interaktion dar. Zum anderen versucht \citet{Seel91} die unterschiedlichen Richtungen der Forschung im Bereich der mentalen Modelle zusammenzuführen und daraus die Bedeutung von Mentalen Modellen für Lernvorgänge (unter die -- im breiten Verständnis von Seel -- auch die hier relevanten Abstimmungsvorgänge fallen) und Möglichkeiten zu deren Unterstützung abzuleiten.

\subsection{Mentale Modelle nach Johnson-Laird}

\section{Mentale Modelle im Gesamtzusammenhang}

\subsection{Realität}

\subsection{Theorien}

\subsection{Experten- vs. Alltags-Modelle}

\subsection{Schemata}


\section{Bildung mentaler Modelle} % (fold)
\label{sec:bildung_mentaler_modelle}

Nach \citep{Seel91} umfasst die Bildung mentaler Modelle zwei Komponenten: Eine \emph{deklarative Komponente}, in der bereichs- bzw. domänen-spezifisches Wissen in der Form von hier nicht näher spezifizierten, strukurierten Wissensbasen abgelegt wird und eine \emph{operative Komponente}, in der auf Grundlage dieser Wissensbasen Schlüsse gezogen und neues Wissen abgeleitet wird, die über das ursprüngliche domänenspezifische Wissen hinausgeht. 

Das in den Wissensbasen repräsentierte Wissen kann auf Alltagserfahrung begründet sein oder durch Vermittlung oder Instruktion begründet werden. Im ersteren Fall ist das Wissen dann als konkret und handlungsbezogen angesehen werden, im zweiten Fall ist das Wissen eher auf abstrakter, formaler Ebene anzusiedeln. Analog dazu kann auch in der operativen Komponente die Schlussfolgerung induktiv auf Basis eines "intuitionsbegründeten" Regelsystems gezogen werden oder durch Deduktion mittels einem formal begründbaren Regelsystem gebildet werden. 

Die Modifikation und Erweiterung der eigenen Wissensbasen und die (Weiter-)Entwicklung der kognitiven Fähigkeiten, die für die Ableitung von Schlussfolgerungen notwendig sind, bezeichnet \citet{Seel91} als "Lernen". Lernen ist \emph{"mit der Verarbeitung individueller Erfahrungen mit sowie vermittelter Information über die Welt, ihre Struktur und Evidenz verbuneen und kann als ein Prozess permanenter konzeptueller Veränderungen verstanden werden."} \citep[][S. 23]{Seel91}. Lernen setzt damit die Fähigkeit und Bereitschaft voraus, \emph{"vermittelte Weltauffassungen zu verstehen, zu akzeptieren und sodann den eigenen gedanklichen Konstruktionen zugrunde zu legen"} \citep[][S. 23]{Seel91}.

% section bildung_mentaler_modelle (end)

\section{Externalisierung mentaler Modelle} % (fold)
\label{sec:externalisierung_mentaler_modelle}

% section externalisierung_mentaler_modelle (end)

% chapter mentale_modelle (end)

% \chapter{Mentale Modelle}
% \label{cha:mentale_modelle}
% 
% In diesem Kapitel wird das Konzept der mentalen Modelle eingeführt, das in dieser Arbeit als Erklärungsansatz für jene Aspekte von "Articulation Work" verwendet wird, die die nicht sichtbaren, kognitiven Beiträge eines beteiligten Individuums betreffen. Nach einer Einführung in die Begriffswelt der mentalen Modelle wird die Argumentation aus dem letzten Kapitel nochmals aufgegriffen und die mögliche Rolle mentaler Modelle für "Articulation Work" erörtert. In der Folge werden Methoden eingeführt mit denen mentale Modelle externalisiert und kommuniziert werden können. Basierend auf diesen Beschreibungen wird im letzten Teil des Kapitels untersucht, welche Herausforderungen sich bei der Anwendung dieser Methoden im Kontext von "Articulation Work" ergeben können.
% 
% \section{Begriffsbestimmung}e
% \label{sec:mentalemodelle_begriffsbestimmung}
% 
% Der Begriff der Mentalen Modelle wurde von \citet{Johnson-Laird81} geprägt. Ein mentales Modell ist nach 
% 
% \subsection{Mentale Modelle nach Johnson-Laird} % (fold)
% \label{sub:mentale_modelle_nach_johnson_laird}
% 
% % subsection mentale_modelle_nach_johnson_laird (end)
% 
% \subsection{Mentale Modelle nach Norman} % (fold)
% \label{sub:mentale_modelle_nach_norman}
% 
% \citet{Norman83a} formuliert ein Verständnis von mentalen Modellen aus Interaktionssicht. Sein Kontext ist die Untersuchung von Mensch-Maschine-Interaktion und den dort auftretenden Interaktionsabläufen. Mentale Modelle sind in diesem Verständnis individuelle Konstrukte, die von Menschen bei der Interaktion mit der Umwelt, mit anderen Menschen oder mit Technologie gebildet werden, um das Verhalten der Gegenseite erklären und vorhersagen zu können\footnote{\emph{„In interaction with the environment, with others, an with the artifacts of technology, people form internal, mental models of themselves and of the things with which they are interacting. These models provide predictive and explanatory power for understanding the interaction“} \citep{Norman83a}}. Um den Begriff abzugrenzen, führt \citeauthor{Norman83a} ein aus vier Elementen bestehendes Begriffssystem ein, das den Diskussionsbereich abgrenzt und definiert:
% \begin{description}
% 	\item[target system] Das Zielsystem ist jenes System, das von einer Person benutzt wird oder dessen Benutzung von dieser Person erlernt wird.
% 	\item[conceptual model of target system] Ein konzeptionelles Modell ist ein Modell, dass das Zielsystem vollständig, konsistent und exakt beschreibt. Konzeptionelle Modelle werden von Entwicklern, Designern, Wissenschaftern oder Lehrern (im Allgemeinen: Experten in der Domäne des Zielsystems) definiert.
% 	\item[mental model of target system] Mentale Modelle werden von Personen bei der Interaktion mit dem Zielsystem entwickelt, um dessen Verhalten zu erklären. Diese Modelle müssen nicht vollständig und exakt sein, müssen aber für die jeweilige Person funktional sein, d.h. für deren Zwecke ausreichendes Erklärungspotential besitzen. Mentale Modelle haben evolutionären Charakter und entwickeln sich während der Interaktion mit dem System weiter. Die Inhalte eines mentalen Modells werden durch das Vorwissen und die Erfahrung der jeweiligen Person beeinflusst.
% 	\item[scientist's conceptualization of mental model] Die Konzeptualisierung eines mentalen Modells ist der Versuch ein mentales Modell mit wissenschaftlichen Mitteln zu erheben und abzubilden. Sie soll die Inhalte des mentalen Modells möglichst vollständig und genau abbilden. Die Konzeptualisierung ist also ein Modell eines Modells.
% \end{description}
% 
% Im Weiteren nennt \citeauthor{Norman83a} sechs generelle Eigenschaften von mentalen Modellen, die er aus eine Vielzahl von Beobachtungen in unterschiedlichen Kontexten ableitet:
% \begin{enumerate}
% 	\item Mentale Modelle sind unvollständig
% 	\item Mentale Modelle können von ihren Trägern nur sehr einschränkt wiedergegeben werden.
% 	\item Mentale Modell sind instabil und werden vor allem in Bereich ungenau, die Teile des Zielsystems abbilden die lange nicht benötigt wurden.
% 	\item Mentale Modelle sind nicht klar voneinander abgrenzbar -- ähnliche Gegenstände oder Situationen werden oft hinsichtlich der angewandten Interaktionsmuster verwechselt.
% 	\item Mentale Modelle sind unwissenschaftlich -- auch mentale Modelle, die inhaltlich (technisch) überflüssiges Verhalten verursachen, werden beibehalten, wenn der Aufwand der physischen Ausführung gering ist.
% 	\item Mentale Modelle sind simpel -- auch wenn eine effizientere Interaktion möglich wäre, wenn mehr Aufwand in die Planung investiert würde bzw. ein komplexeres mentales Modell zum Einsatz käme, präferieren Benutzer einfache Modelle, deren Anwendung höheren „physischen Aufwand“ mit sich bringen.
% \end{enumerate}
% 
% % subsection mentale_modelle_nach_norman (end)
% 
% \subsection{Mentale Modelle nach Senge} % (fold)
% \label{sub:mentale_modelle_nach_senge}
% 
% % subsection mentale_modelle_nach_senge (end)
% 
% \section{Veränderung mentaler Modelle}
% \label{sub:veränderung_mentaler_modelle}
% Assimilation vs. Akkommodation
% 
% \section{Mentale Modelle und Articulation Work}
% \label{sec:mentale_modelle_und_articulation_work}
% 
% Argumentation mit Wissensspirale (Nonaka \& Takeuchi)


% part grundlagen (end)


\part{Unterstützung} % (fold)
\label{prt:umsetzung}

\section*{Einleitung} % (fold)
\label{sec:umsetzung_einleitung}
\thispagestyle{empty}

Basierend auf den diese Arbeit motivierenden Grundlagen, die in den letzten Kapiteln beschrieben wurden, wird in diesem Teil auf die konkreten Maßnahmen zur Unterstützung von Articulation Work eingegangen. Ziel dieses Teils ist es, sowohl das hier vorgeschlagene Vorgehen bei der Unterstützung expliziter Articulation Work als auch die Unterstützung, die ein Werkzeug dabei leisten kann, umfassend dazustellen.

Auf Grundlage der Methodik, die im Kontext von Concept Mapping und Strukturlegetechniken vorgeschlagen wird und unter Berücksichtigung der Anforderungen, die aus dem inhärent kollaborativen Anwendungsszenario abgeleitet werden können, muss das Vorgehen zur Durchführung von expliziter Articulation Work festgelegt werden. 

In Rahmen der Festlegung des Vorgehens werden auch jene Aspekte identifiziert, in denen Unterstützung durch technische Werkzeuge sinnvoll und notwendig ist. Die Anforderungen, die sich aus diesen Aspekten ableiten lassen, bilden die Grundlage für die Konzeption und Umsetzung eines Werkzeugs, das diese Unterstützung bietet. Die technischen Details der Implementierung dieses Werkzeugs und die zugrunde liegenden konzeptionellen und technologischen Grundlagen bilden den Kern dieser Arbeit.

Der Aufbau dieses Teils folgt dem eben umrissenen inhaltlichen Vorgehen. In Kapitel \ref{cha:methodik} wird die Methodik zur Unterstützung expliziter Artikulation Work beschrieben. Aus diesem werden im darauf folgenden Kapitel \ref{cha:anforderungen} jene Bereiche identifiziert, in denen eine technologische Unterstützung notwendig ist und die Anforderungen an ein Werkzeug abgeleitet, das diese Unterstützung bietet. Die vier folgenden Kapitel beschäftigen sich mit der Umsetzung des Werkzeugs. Kapitel \ref{cha:implementierung_Überblick} beschäftigt sich dabei mit den konzeptionellen Grundlagen des Forschungsgebiets "Tangible Interfaces", das die Basis für die technische Umsetzung bildet. Kapitel \ref{cha:input_&_interpretation} beschäftigt sich mit jenen Technologien und Softwarekomponenten, die für die Informationseingabe in das technische System verwendet werden. Dabei wird auch auf die konkrete Interaktion der Benutzer mit dem System eingegangen. Kapitel \ref{cha:visualisierung} beschreibt die Ausgabeseite des technischen Systems und behandelt die Umsetzung des Informationsflusses vom System zu den Benutzern. Letztendlich wird in Kapitel \ref{cha:persistierung} beschrieben, welche Maßnahmen zu Sicherung der Ergebnisse der expliziten Articulation Work getroffen werden müssen und welche Möglichkeiten der technischen Umsetzung bestehen bzw. gewählt wurden.

% section umsetzung_einleitung (end)

\chapter{Methodik} % (fold)
\label{cha:methodik}

% chapter methodik (end)
\chapter{Anforderungen an ein Werkzeug} % (fold)
\label{cha:anforderungen}

\paragraph{Physische Abbildung beliebiger diagrammatischer Modelle} % (fold)
\label{par:physische_abbildung_legen_beliebiger_diagrammatischer_modelle}

Ein Werkzeug zur Unterstützung von Strukturlegetechniken muss das grundlegende Konzept der Methodik vollständig unterstützten. Es muss möglich sein, Konzepte auf einer Modellierungs-Oberfläche zu platzieren und zueinander in Beziehung zu setzen. Der gesamte Modellstatus muss visuell auf der Oberfläche erkennbar sein.

% paragraph physische_abbildung_legen_beliebiger_diagrammatischer_modelle (end)

\paragraph{Kollaborative und unmittelbare Manipulierbarkeit des Modells} % (fold)
\label{par:kollaborative_und_unmittelbare_manipulierbarkeit_des_modells}

Zur Unterstützung von expliziter „Articulation Work“ muss das Werkzeug kollaborative Strukturlege-Prozesse erlauben. Es muss möglich sein, das gelegte Modell simultan zu erweitern oder zu verändern.

% paragraph kollaborative_und_unmittelbare_manipulierbarkeit_des_modells (end)

\paragraph{Nicht vorgegebene Semantik der Modellierungselemente} % (fold)
\label{par:nicht_vorgegebene_semantik_der_modellierungselemente}

Wie oben bereits argumentiert, sind zur Unterstützung von expliziter „Articulation Work“ vor allem Varianten von Strukturlegetechniken geeignet die keine Vorgaben hinsichtlich der zu verwendenden Konzepte und Verknüpfungen machen. Das Werkzeug muss dementsprechend die Offenheit bieten, beliebige Klassen von Konzepten und Verknüpfungen zu definieren (z.B. Klasse „organisationale Rolle“) und von diesen beliebige Instanzen zu bilden und zu benennen (z.B. Instanz „Geschäftsführer“). Gleichzeitig muss sichergestellt werden, dass die festgelegte Semantik im Modell mit abgebildet wird und nicht verloren geht.

% paragraph nicht_vorgegebene_semantik_der_modellierungselemente (end)

\paragraph{Unterstützung der iterativen Aushandlung des Modells} % (fold)
\label{par:unterstützung_der_iterativen_aushandlung_des_modells}

Im Sinne der Unterstützung der Dialog-Konsens-Methodik sind ist der Austausch über das Modell durch das Werkzeug zu unterstützen. Vor allem muss es möglich sein, Anmerkungen über Konsens oder Dissens über einzelnen Modellteile oder das gesamte Modell explizit mit in die Repräsentation aufzunehmen. 

% paragraph unterstützung_der_iterativen_aushandlung_des_modells (end)

\paragraph{Persistente Ablage des Modells und Möglichkeit zur Rekonstruktion} % (fold)
\label{par:persistente_ablage_des_modells_möglichkeit_zur_rekonstruktion}

Die persistente Ablage eines Modells (z.B. als digitale Repräsentation) und Werkzeugunterstützung zur Rekonstruktion eines abgelegten Modells erlaubt die Wiederaufnahme eines unterbrochenen Strukturlegeprozesses bzw. die Reflexion und Anpassung bereits erstellter Modelle zu einem späteren Zeitpunkt.

% paragraph persistente_ablage_des_modells_möglichkeit_zur_rekonstruktion (end)

\paragraph{Ermöglichung experimenteller Veränderungen am Modell} % (fold)
\label{par:ermöglichung_experimenteller_veränderungen_am_modell}

Es muss möglich sein, das Modell experimentell zu verändern und ggf. zu einem früheren stabilen Modellzustand zurückzukehren. Dies erlaubt eine konsequenzlose Erkundung von Lösungsräumen und unterstützt damit den Dialog-Konsens-Prozess. Das Werkzeug muss also stabile Modellzustände erfassen und deren Rekonstruktion unterstützen.
% paragraph ermöglichung_experimenteller_veränderungen_am_modell (end)

\paragraph{Verknüpfung mit digitalen Ressourcen} % (fold)
\label{par:verknüpfung_mit_digitalen_ressourcen}

Die Einbindung von digitalen Ressourcen (Dateien, Hyperlinks,...) ermöglicht die Einbindung des Modells in den organisationalen Kontext und erleichtert so einerseits die Verständnisbildung und ermöglicht andererseits die Verwendung der Repräsentation als unmittelbare Handlungsanleitung mit Verknüpfungen zu den betroffenen Arbeitsgegenständen.

% paragraph verknüpfung_mit_digitalen_ressourcen (end)

\paragraph{Bearbeitung von beliebig komplexen Modellen} % (fold)
\label{par:bearbeitung_von_beliebig_komplexen_modellen}

Komplexe Modelle enthalten oft eine große Anzahl von Konzepten und viele Verknüpfungen. Das Werkzeug muss das Modell in einer Form darstellen, die dessen Erfassung und Manipulation ermöglicht, ohne die Repräsentierenden kognitiv zu sehr zu belasten.
% paragraph bearbeitung_von_beliebig_komplexen_modellen (end)






\footnote{By recording and replaying the authoring process, navigable history can re-situate an author after a gap in the authoring process. Similarly, in a collaborative authoring process, an author can play through the events since his/her last authoring session to quickly determine the activity of the other authors. Finally, in many situations, information becomes harder to interpret as its context changes over time. By returning to the state of the information space at the time of authoring, disambiguation of the information may become possible. For the reader who is not also the writer of the hypertext there are additional uses of navigable history. A reader replaying the author’s writing process can gain insight into the motivation of the author and have a greater understanding of the author’s writing style. Such an understanding is important in collaborative work and in other contexts, like education and literary analysis. \citep{Shipman00}}

% chapter anforderungen (end)

\chapter{Grundlagen der Implementierung} % (fold)
\label{cha:implementierung_Überblick}

Wie im Kapitel "Design" gefordert, wurde zur Umsetzung des Werkzeugs ein "Tangible Tabletop Interface" verwendet. Tabletop Interface zeichnen sich im Generellen dadurch aus, dass im Gegensatz zu handelsüblichen Rechnern nicht nur die Software sondern auch die Hardware applikationsspezifisch ist und nicht generisch eingesetzt werden kann. Die Hardware bildet dabei einen Teil oder die gesamte Benutzungsschnittstelle ab. Im speziellen Fall eines "Tangible Tabletop Interfaces" basiert der Benutzerinteraktion auf der Verwendung physischer Bausteine ("Tokens"), die auf der physischen Oberfläche des Interfaces manipuliert werden. Dieses Paradigma wird ergänzt von Tabletop Interfaces, die die Benutzerinteraktion ausschließlich auf Gesten bzw. Berührungen der Oberfläche abbilden (horizontal verbaute "Touch-" bzw. "Multi-Touch-Displays").

REFs!!! Die Entwicklung von Tabletop Interfaces begann Mitte der 1990er-Jahren mit den Arbeiten von Ishii \& Ullmer. Auch die erste Anwendung, die sich mit Modellierungs-Ansätzen mit Hilfe von Tabletop Interfaces konzentriert, stammt aus dieser Zeit. Mit dem fortschreiten der technologischen Entwicklung ist heute ein Status erreicht, in dem mit Hilfe generischer Identifikations-Frameworks schnell und ohne großen Aufwand Applikationen mit "tangiblen" Inputkanälen erstellt werden können. Zur Zeit noch im Prototypenstatus befinden sich Ansätze, die sich mit generischen Möglichkeiten des tangiblen Informationsoutputs beschäftigt. Der Rückkanal vom Rechner zum Benutzer wird heute zumeist mit der Projektion von Inhalten auf die Arbeitsoberfläche umgesetzt.

In den folgenden Abschnitten wird die historische Entwicklung von Tabletop Interfaces sowie der aktuelle Stand der Entwicklung im Anwendungsbereich dieser Arbeit betrachtet. Es werden dabei die grundlegenden Konzepte und Eigenschaften der jeweiligen Arbeiten betrachtet und das Potential hinsichtlich der Umsetzung von in Kapitel XY identifizierten Anforderungen an das hier entwickelte Werkzeug betrachtet. 

\section{Entwicklung von Tangible Interfaces} % (fold)
\label{sec:tangible_interfaces}

Der Begriff der Tangible bzw. Graspable Interfaces – also der "berührbaren" oder "begreifbaren" Benutzungsschnittstellen — stammt aus der Mitte der neunziger Jahre des zwanzigsten Jahrhunderts. \citet{Fitzmaurice95} werden im Allgemeinen als die ersten betrachtet, die den Begriff des "Graspable User Interfaces" prägen und damit die Manipulierbarkeit digitaler Information durch physische Mittel beschreiben. \citet{Fitzmaurice96} präzisiert später den Begriff durch die Abgrenzung zwischen (herkömmlichen, maus-, tastatur- und bildschirmbasierenden) zeitlich gemultiplexten Schnittstellen, bei denen der Informationsaustausch zwischen Benutzer und System über einen Kanal zeitlich hintereinander erfolgt und den (neuartigen, berührbaren) räumlich gemultiplexten Schnittstellen, bei denen mehrere Kanäle gleichzeitig zur Interaktion zwischen Benutzer und System verwendet werden können. 

Der Begriff des "Tangible User Interfaces" wurde kurz danach bzw. parallel dazu von \citet{Ishii97} eingeführt. \citeauthor{Ishii97} verfolgen dabei bei der Definition den umgekehrten Weg und sprechen von einer "Augmentation der realen Welt durch eine Kopplung von digitaler Information and physische Objekte"\footnote{\emph{“augment the real physical world by coupling digital information to everyday physical objects and environments”}\citep{Ishii97}}. 

\subsection{Ubiquitous Computing}

\subsection{Augmented Reality} 

% subsection tangibles_historischer_hintergrund (end)

\section{Konzeptualisierung und Einteilung von Tangible Interfaces} % (fold)
\label{sec:konzeptualisierungen_von_tangible_interfaces}

Die Entwicklung des Forschungsgebiets der "Tangible Interfaces" wurde von mehreren konzeptuellen Arbeiten maßgeblich beeinflusst. Die dort vorschlagenen Erklärungsmodelle definieren das Gebiet und grenzen es gegenüber anderen Forschungsbereichen ab. Sie dienen außerdem als Grundlage für Erklärung und Konzeption konkreter Tangible Interfaces. Im Folgenden wird die historische Entwicklung dieser konzeptuellen Modelle beschrieben und auf deren Spezifika eingegangen.

Zur struktrierten Betrachtung von Tangible Interfaces ist es außerdem notwendig, jene Dimensionen zu identifizieren, an denen sich einzelne Tangible Interfaces einordnen und unterscheiden lassen. Die Ausprägungen dieser Dimensionen liefern kombiniert ein Begriffssystem, dass bei der Aufbereitung von unterschiedlichen Ansätzen im Bereich der Tangible Interface sowie deren Vergleich helfen kann. Die hier vorgestellten Ansätze tragen unterschiedlich detailliert und aus unterschiedlichen Gesichtspunkten zu dieser Thematik bei. Die einzelnen Ansätze werden hier dargestellt und in Kapitel XY auf das in dieser Arbeit entwickelte System angewandt um so das System-Design aus konzeptueller Sicht zu reflektieren und potentielle Verbesserungs- und Erweiterungsmöglichkeiten zu identifizieren.

\subsection{Graspable User Interfaces}

\citeauthor{Fitzmaurice96} legt in jener Arbeit, in der es den Begriff des "Graspable User Inferfaces" prägt \citep{Fitzmaurice96}, auch Eigenschaften fest, anhand deren sich die "Graspability" einer Benutzungsschnittstelle zeigt und beurteilen lässt. Diese Beurteilung erfolgt auf einer generischen Skala mit Ausprägungen von "niedrig" bis "hoch", wobei "hohe" Werte in mehreren Eigenschaften auf eher hohe "Graspability" hinweist.

\subsubsection{Space Mulitplexing}
\subsubsection{Concurrency}
\subsubsection{Physical Form}
\subsubsection{Spartially aware}
\subsubsection{Spatial recofigurability}

\subsection{Tangible Bits}

\citep{Ishii97}

\subsection{Containers, Tokens und Tools}

\citep{Holmquist99} legen ihre Arbeit als konzeptuelle Betrachtung von interaktiven Systemen an, in denen physische Objekte verwendet werden, um auf digitale Information zuzugreifen bzw. diese zu manipulieren. Das Einteilungsschema, das die Autoren vorschlagen, basiert auf der Art und Weise, in der Information an diese physischen Objekte gebunden ist. 

\subsubsection{Containers}

\subsubsection{Tokens}

\subsubsection{Tools}

\subsection{Das MCRpd Interaktions-Modell}
\citep{Ullmer00}

\subsection{Degree of Coherence}
\citep{Koleva03}

\subsection{Tokens und Constraints nach Shaer et al.}
\citep{Shaer04}

\subsection{Einteilung nach Klemmer, Li, Lin und Landay}
\citep{Klemmer04}

\subsection{Taxonomie nach Fishkin}
\citep{Fishkin04}

\subsection{Tokens und Constraints nach Ullmer et al.}
\citep{Ullmer05}

\subsection{Tangible Bits: Beyond Pixels}
\citep{Ishii08}

% section konzeptualisierungen_von_tangible_interfaces (end)

\section{Tangible Interfaces in kooperativer Verwendung} % (fold)
\label{sub:tangible_interfaces_in_kooperativer_verwendung}
\citep{Hornecker04}
% subsection tangible_interfaces_in_kooperativer_verwendung (end)

% section tangible_interfaces (end)

\section{Tabletop Interfaces} % (fold)
\label{sec:tabletop_interfaces}

Grundlagen

\subsection{Historische Entwicklung} % (fold)
\label{sub:historische_entwicklung_von_tabletop_interfaces}

\subsubsection{Sensetable} % (fold)
\label{subs:sensetable}
Der Sensetable \citep{Patten01}
% subsubsection sensetable (end)

\subsubsection{BUILD-IT} % (fold)
\label{par:build_it}
\citep{Fjeld01}
% subsubsection build_it (end)
% subsection historische_entwicklung_von_tabletop_interfaces (end)
% section tabletop_interface (end)

\section{Tangible Interfaces zur Modellbildung} % (fold)
\label{sub:tangible_interfaces_zur_modellbildung}

% subsection tangible_interfaces_zur_modellbildung (end)

\subsection{Aktuelle verwandte Ansätze} % (fold)
\label{sub:aktuelle_verwandte_ansätze}

% subsection aktuelle_verwandte_ansätze (end)
\begin{itemize}
	\item Historische Entwicklung von Tabletop Interfaces
	\begin{itemize}
		\item Sensetable
		\item Morten Fjeld
		\item ReacTable
		\item Eva Hornecker
	\end{itemize}
	\item Historische Entwicklung von Tangible Interfaces zur Modellbildung
	\begin{itemize}
		\item Sensetable Modeling Application
		\item Designer's Outpost (Klemmer)
	\end{itemize}
	\item Aktuelle verwandte Ansätze
	\begin{itemize}
		\item Antle (TEI Mail-Pointer)
		\item Sun (TEI Demo)
	\end{itemize}
\end{itemize}


% section grundlegende_&_verwandte_arbeiten (end)

% chapter implementierung_Überblick (end)
\chapter{Input \& Interpretation} % (fold)
\label{cha:input_&_interpretation}

\section{Möglichkeiten zur Erfassung von Benutzerinteraktion} % (fold)
\label{sec:möglichkeiten_zur_erfassung_von_benutzerinteraktion}

\subsection{Potentielle technologische Ansätze} % (fold)
\label{sub:potentielle_technologische_ansätze}
\begin{itemize}
	\item optisch
	\item kapazitiv
	\item elektromagnetisch (RFID)
	\item akustisch (Ultraschall)
\end{itemize}

% subsection potentielle_technologische_ansätze (end)
\subsection{Verfügbare Frameworks} % (fold)
\label{sub:verfügbare_frameworks}
\begin{itemize}
	\item ReacTIVision
	\item ARToolkit
	\item Papiermache (Klemmer)
	\item Visual Codes (ETH)
	\item Frameworks aus der TU LVA
\end{itemize}

% subsection verfügbare_frameworks (end)

\subsection{Technologieentscheidung} % (fold)
\label{sub:technologieentscheidung}
-> ReacTIVision
% subsection technologieentscheidung (end)
% section möglichkeiten_zur_erfassung_von_benutzerinteraktion (end)

\section{Konzeption und Umsetzung der Hardwarekomponenten} % (fold)
\label{sec:konzeption_und_umsetzung_der_hardwarekomponenten}

\subsection{Überblick} % (fold)
\label{sub:Überblick}
Grafik aus dem TEI-Paper
Zerlegbarkeit

% subsection Überblick (end)

\subsection{Tokens \& Input-Werkzeuge} % (fold)
\label{sub:tokens_&_input_werkzeuge}

% subsection tokens_&_input_werkzeuge (end)

\subsection{Input auf der Tischoberfläche} % (fold)
\label{sub:input_auf_der_tischoberfläche}

Semitransparente Oberfläche, Projektion von unten (Umlenkspiegel), Kamera von untern - Überleitung zur 
Illumination via Interferenz zwischen Beamer und Kamera

% subsection input_auf_der_tischoberfläche (end)

\subsection{Illumination und Umgebungslichtabhängigkeit} % (fold)
\label{sub:illumination_und_umgebungslichtabhängigkeit}

% subsection illumination_und_umgebungslichtabhängigkeit (end)
% section konzeption_und_umsetzung_der_hardwarekomponenten (end)

\section{Erfassung der Benutzerinteraktion durch Software} % (fold)
\label{sec:erfassung_der_benutzerinteraktion_durch_software}

% section erfassung_der_benutzerinteraktion_durch_software (end)

\section{Interpretation der Rohdaten und Stabilisierung der Erkennungsleistung} % (fold)
\label{sec:interpretation_der_rohdaten_und_stabilisierung_der_erkennungsleistung}

% section interpretation_der_rohdaten_und_stabilisierung_der_erkennungsleistung (end)
% chapter input_&_interpretation (end)
\chapter{Visualisierung und Modellierungsunterstützung} % (fold)
\label{cha:visualisierung_und_modellierungsunterstützung}

\section{Technologische Grundlage der Visualisierung} % (fold)
\label{sec:technologische_grundlage_der_visualisierung}

\subsection{JHotDraw – Überblick} % (fold)
\label{sub:jhotdraw_Überblick}

% subsection jhotdraw_Überblick (end)
\subsection{Einsatz von JHotDraw} % (fold)
\label{sub:einsatz_von_jhotdraw}

% subsection einsatz_von_jhotdraw (end)
\subsection{Projektion von Information auf die Tischoberfläche} % (fold)
\label{sub:projektion_von_information_auf_die_tischoberfläche}

% subsection projektion_von_information_auf_die_tischoberfläche (end)
% section technologische_grundlage_der_visualisierung (end)

\section{Umsetzung der Anforderungen zur Modellierungsunterstützung} % (fold)
\label{sec:umsetzung_der_anforderungen_zur_modellierungsunterstützung}

\subsection{Benennung von Blöcken und Verbindungen} % (fold)
\label{sub:benennung_von_blöcken_und_verbindungen}
inkl. Löschen

% subsection benennung_von_blöcken_und_verbindungen (end)

\subsection{Abstraktion} % (fold)
\label{sub:abstraktion}
Container

% subsection abstraktion (end)

\subsection{Modellierungshistorie} % (fold)
\label{sub:modellierungshistorie}
automatisches Tracking vs. Snapshots

% subsection modellierungshistorie (end)

\subsection{Wiederherstellungsunterstützung} % (fold)
\label{sub:wiederherstellungsunterstützung}

% subsection wiederherstellungsunterstützung (end)
% section umsetzung_der_anforderungen_zur_modellierungsunterstützung (end)
% chapter visualisierung_und_modellierungsunterstützung (end)
\chapter{Persistierung} % (fold)
\label{cha:persistierung}

In den vorangegangen drei Kapiteln wurde die Umsetzung des eigentlichen Werkzeugs beschrieben. Neben der Unterstützung des Modellierungsvorgangs ist aber auch die persistente Speicherung der erstellten Modelle zum Zwecke der Weiterverarbeitung ein hier zu beleuchtender Aspekt. Auf die Persistierung wirken vor allem zwei der in Kapitel XY identifzierten Anforderungen ein. Zum ersten ist die Nachvollziehbarkeit des Modellierungsvorganges sicherzustellen -- dies gilt nicht nur während des Vorgangs selbst, sondern auch danach. Dementsprechend ist sämtliche Information zu persistieren, die zur Wiederherstellung nicht nur des Modells selbst sondern auch der gesamten Modellierungshistorie notwendig ist. Zum zweiten hat die Forderung nach semantischer Offenheit bei der Modellierung auch unmittelbare Auswirkungen auf die Persistierung. Neben dem Modell selbst muss aufgrund dieser Anforderung auch die Bedeutung der verwendeten Modellierungselemente miterfasst und persistiert werden, so dass diese bei der Weiterverarbeitung der Modelle verwendet werden kann.

In diesem Kapitel werden nun aufgrund der eben genannten Forderungen technologische Ansätze identifiziert, beschrieben und schließlich hinsichtlich ihrer Eignung für den konkreten Einsatz beurteilt. Der ausgewählte Ansatz wird im darauf folgenden Abschnitt konzeptuell beschrieben. Die Abbildung der Modelle und der ebenfalls zu persistierenden zusätzlichen Information in ein geeignetes Datenmodell ist Gegenstand des darauf folgenden Abschnitts. Schließlich wird die konkrete technische Umsetzung der Persistierung dargelegt und die dazu notwendigen Software-Module im Detail beschrieben.
 
\section{Möglichkeiten der Persistenzsicherung} % (fold)
\label{sec:möglichkeiten_der_persistenzsicherung}

\begin{itemize}
	\item Serialisierung von Java-Objekten
	\item Relationale Datenbanken
	\item XML Topic Maps
\end{itemize}

% section möglichkeiten_der_persistenzsicherung (end)

\section{Topic Maps} % (fold)
\label{sec:topic_maps}

Topic Maps \citep{TMDM08} sind wie bereits in Abschnitt XY beschrieben ein Mittel zur Abbildung von semantischen Netzen. In Topic Maps können beliebige Daten strukutriert aufbereitet und zueinander in Beziehung gesetzt werden. Die Art der zu repräsentierenden Daten ist dabei irrelvant, eine Topic Map trifft keine Aussage über ein den repräsentierten Daten zugrundeliegendes Begriffsystem (sie ist „ontology-agnostic“ \citep{Vatant04}).

Historisch stammen Topic Maps aus dem Bereich der technischen Repräsentation von Thesauri und Indizes \citep{Pepper00} \citep{Rath03}. Aus diesen Bereichen motivieren sich auch die Bausteine einer Topic Map, wenngleich der Verwendung durch diesen Ursprung nicht eingeschränkt wird. Die grundlegenden Elemente einer Topic Map sind „Topics“, „Associations“ und „Occurrences“ (siehe Abbildung \ref{fig:img_Persistenz_TMBasic}). 

\begin{figure}[htbp]
	\centering
		\includegraphics[width=10cm]{img/Persistenz/TMBasic.png}
	\caption{Grundlegende Elemente einer Topic Map}
	\label{fig:img_Persistenz_TMBasic}
\end{figure}

„Topics“ sind stellen Begriffe dar und bilden die Knoten des semantischen Netzes. Ein Topic kann beliebige Information darstellen, repräsentiert aber immer genau ein Phänomen der realen Welt (d.h. zu einem Topic muss es eine Entsprechung außerhalb der Topic Maps geben, die beobachtbar oder beschreibbar ist und auf die die modellierende Person Bezug nehmen will \footnote{„A subject can be anything whatsoever, regardless of whether it exists or has any other specific characteristics, about which anything whatsoever may be asserted by any means whatsoever. In particular, it is anything about which the creator of a topic map chooses to discourse.“ \citep[][S.8]{TMDM08}}). Eine Topic Map ist damit im Sinne von \citet{Stachowiak73} ein diagrammatisches Modell, das einen bestimmten, für den Modellersteller relevanten Ausschnitt der Realität abbildet.

"Associations" bilden die Beziehungen zwischen Topics ab und stellen damit die Kanten des semantischen Netzes dar. Eine Association verknüpft Topics semantisch miteinander und kann frei mit Bedeutung belegt werden. Die Art der Beziehungen ist also nicht festgelegt und wird wie die Bedeutung der Topics frei gewählt werden. Topics und Associations decken historisch den Bereich der Darstellung von Thesauri ab, in denen Begriffe definiert und zueinanden in Beziehung gesetzt werden. 

Der zweite historische Ursprung von Topic Maps, die Indizes, werden durch das Konstrukt der "Occurences" abgedeckt. Occurences ("Auftreten") sind Referenzen aus der Topic Map in die reale Welt. Sie setzen die Topics einer Topic Map in Bezug zu beliebiger referenzierbarer Information (z.B. Dokumente). Im Kontext der eben genannten Indizes, kann eine Topic Map als der mit Querverweisen versehene Index eines Buches verstanden werden, in dem durch die Angabe von Seitenzahlen auf den Text des Buches verwiesen wird. Diese Verweise durch Angabe der Seitenzahlen sind in diesem Zusammenhang die Occurrences.

Die Ansammlung von durch Associations verknüpften und mit Occurrences versehenen Topics bilden eine Topic Map. Darüber hinaus kann in Topic Maps jedoch noch weiterführende Information repräsentiert werden (siehe Abbildung \ref{fig:img_Persistenz_TMFull}), die Gegenstand der folgenden Abschnitte sein werden.

\begin{figure}[htbp]
	\centering
		\includegraphics[width=10cm]{img/Persistenz/TMFull.png}
	\caption{Umfassende Darstellung der Elemente einer Topic Map}
	\label{fig:img_Persistenz_TMFull}
\end{figure}

\subsection{Topics, Subjects, Topic Names und Variants} % (fold)
\label{sub:topics_subjects_topic_names_und_variants}

Dieser

\begin{figure}[htbp]
	\centering
		\includegraphics[width=10cm]{img/Persistenz/TopicNaming.png}
	\caption{Benennung von Topics}
	\label{fig:img_Persistenz_TopicNaming}
\end{figure}


% subsection topics_subjects_topic_names_und_variants (end)

\subsection{Associations und Roles} % (fold)
\label{sub:associations_und_roles}

% subsection associations_und_roles (end)

\subsection{Occurrences und Datatypes} % (fold)
\label{sub:occurrences_und_datatypes}

% subsection occurrences_und_datatypes (end)

\subsection{Metamodellierung in Topic Maps} % (fold)
\label{sub:metamodellierung_in_topic_maps}

Topic Types, Association Types und Occurrence Types

\begin{figure}[htbp]
	\centering
		\includegraphics[width=10cm]{img/Persistenz/MetaModelExample.png}
	\caption{Beziehungen in der Metamodellbildung in Topic Maps}
	\label{fig:img_Persistenz_MetaModelExample}
\end{figure}

% subsection metamodellierung_in_topic_maps (end)

\subsection{Scopes} % (fold)
\label{sub:scopes}

% subsection scopes (end)

\subsection{Weiterführende Konzepte} % (fold)
\label{sub:tm_weiterführend}

\subsubsection{Reification} % (fold)
\label{ssub:reification}

% subsubsection reification (end)

\subsubsection{Merging} % (fold)
\label{ssub:merging}

% subsubsection merging (end)
% subsection tm_weiterführend (end)

\subsection{Einschränkungen} % (fold)
\label{sub:einschränkungen}

Regeln und verbindliche Strukturvorgaben
Datenhaltung und Abfrage
% subsection einschränkungen (end)
% section topic_maps (end)

\section{Abbildung von Modellen auf Topic Maps} % (fold)
\label{sec:abbildung_von_modellen_auf_topic_maps}
-> DA Matthias
% section abbildung_von_modellen_auf_topic_maps (end)

\section{Technische Umsetzung der Persistierung von Modellen} % (fold)
\label{sec:technische_umsetzung_der_persistierung_von_modellen}
Topic Map Engine Persistence Layer
% section technische_umsetzung_der_persistierung_von_modellen (end)

\section{Zusammenfassung} % (fold)
\label{sec:persistierung_zusammenfassung}

% section persisitierung_zusammenfassung (end)
% chapter persistierung (end)


% part umsetzung (end)



\part{Evaluierung des Instruments} % (fold)
\label{prt:evaluierung}

\section*{Einleitung} % (fold)
\label{sec:evaluierung_einleitung}
\thispagestyle{empty}

\markboth{Einleitung}{Evaluierung des Instruments}

Nach der nun erfolgten Beschreibung der Umsetzung des Werkzeugs wird in diesem Teil die Überprüfung der Verwendbarkeit des entwickelten Instruments und seiner Effekte behandelt. Ziel dieser Arbeit ist es, die effektive Durchführung explizite „Articulation Work“ zu unterstützen. Ziel dieses Teils ist, diese Anforderung hinsichtlich ihrer Erfüllung oder Nicht-Erfüllung zu überprüfen und damit die Beantwortung zweite in Kapitel \ref{cha:einführung} formulierte Forschungsfrage zu vervollständigen.

Wie in Kapitel \ref{cha:mentale_modelle} argumentiert, führt ein möglicher Weg zur Unterstützung expliziter „Articulation Work“ über die (kollaborative) Externalisierung mentaler Modelle. Die aus dieser Externalisierung resultierenden Strukturen sind ihrerseits diagrammatische Modelle. Die Qualität dieser Modelle ist vielschichtig bewertbar, im Kontext des hier verfolgten Verwendungszwecks sind aber einige Bewertungsdimensionen identifizierbar, die bei der Unterstützung von „Articulation Work“ relevant sind. Diese Dimensionen werden ebenfalls hinsichtlich ihrer Ausprägung im hier vorgestellten Werkzeug zu bewerten sein. 

Letztendlich wird die Externalisierung mentaler Modelle technisch durch ein Tabletop Interface unterstützt. Auch die technische Umsetzung bzw. deren Verständlichkeit und Verwendbarkeit hat Auswirkungen auf den Erfolg der Externalisierung und damit der durchgeführten „Articulation Work“. Das Werkzeug selbst und seine Nutzung muss also ebenfalls untersucht und im Kontext der Anforderungen in dieser Arbeit bewertet werden. 

Entsprechend dieser Ausführungen wurde die hier beschriebenen Untersuchung durchgeführt. Sie gliedert sich in drei Teile, die sich mit dem Werkzeug selbst, den erstellten Modellen und den Auswirkungen durchgeführten expliziten „Articulation Work“ auseinandersetzen. Die Struktur dieses Teiles spiegelt diese Aufteilung wieder. In Kapitel \ref{cha:konzeptuelle_evaluierung} wird das Werkzeug aus Sicht seiner Eigenschaften als Tabletop Interface theoretisch-konzeptuell betrachtet und aus den in diesen Bereich verfügbaren Analyse- und Beschreibungsframeworks mögliches Verbesserungspotential identifiziert. Kapitel \ref{cha:eval_ueberblick} beschreibt die Grundlagen der empirischen Untersuchung, in der das hier entwickelte Werkzeug hinsichtlich seiner Verwendbarkeit und Wirkung untersucht wurde. In Kapitel \ref{cha:eval_werkzeug} werden Design und Umsetzung jener Tests beschrieben, in denen die Benutzbarkeit und Verständlichkeit des Werkzeugs überprüft wurden. Die Überprüfung der Effekte des Werkzeugs beginnt im darauf folgenden Kapitel \ref{cha:eval_modell}, in dem die erstellten Modelle und deren Entstehungsprozess Gegenstand der Betrachtung sind. Letztendlich wird in Kapitel \ref{cha:eval_aw} auf die Wirkung des Werkzeugs auf „Articulation Work“ und damit letztendlich auf die im jeweiligen Anwendungsfall zu erzielenden Effekte eingegangen.

% section evaluierung_einleitung (end)

\automark[section]{chapter} 

\chapter{Konzeptuelle Evaluierung} % (fold)
\label{cha:konzeptuelle_evaluierung}

\section{Betrachtung im Lichte des Tangible Bits Ansatzes} % (fold)
\label{sec:betrachtung_tangible_bits}

Grundlage der Betrachtungen in diesem Abschnitt ist das Konzept der Tangible Bits \citep{Ishii97}, der in Abschnitt \ref{sub:tangible_bits} beschrieben wird.

\subsection{Abbildung} % (fold)

Das hier vorgestellte Werkzeug kann hinsichtlich seiner Funktion als eine Instanz des Konzepts „Interactive Surface“ betrachtet werden. Die „Surface“ ist hierbei eine Tischoberfläche, auf der interagiert wird. Die im Rahmen der Beschreibung des „metaDESK“ \citep{Ullmer97} als Beispiel für eine „Interactive Surface“ eingeführten \gls{TUI}-Elemente finden zum Teil auch im hier vorgestellten Werkzeug Anwendung.

Die Modellierungstokens und einbettbaren Tokens des Werkzeugs sind \emph{Phicons}, also passive Träger von digitaler Information. Die Werkzeugtokens zur Manipulation des Modells entsprechen \emph{Phandles}, also Elemente, die dazu verwendet werden, digitale Information zu verändern bzw. festzulegen. Jene Werkzeugtokens, die der Steuerung der Systemfunktionen dienen, sind hingegen als \emph{Instruments} zu klassifizieren. \emph{Lenses} und \emph{Trays} kommen im Werkzeug nicht zum Einsatz.

Hinsichtlich der Metaphorik unterscheiden \citet{Ullmer97} zwischen unterschiedlichen Abstraktionsebenen von Phicons (\emph{generic} -- \emph{symbolic} -- \emph{model}), wobei im vorliegenden System ob der offenen Semantik die Modellierungstokens ausschließlich \emph{generic Phicons} sind bzw. sein können. Die Werkzeugtokens sind zumeist als \emph{symbolic Phicons}, im Falle des Löschtokens -- dem Radiergummi -- eher als \emph{model Phyicon} zu klassifizieren.

\subsection{Bewertung} % (fold)

Für die Bewertung des Werkzeugs ist vor allem dessen Gegenüberstellung zu den vorgeschlagenen Elementen einer „Interactive Surface“ von Interesse. Hier zeigt sich, das die unterschiedlichen Arten von Tokens, die im Werkzeug eingeführt wurden, feingranular auf die unterschiedlichen Element-Arten von \citep{Ishii97} abbildbar sind. Insbesondere die explizite Unterscheidung zwischen \emph{Phandles} und \emph{Instruments} ist eine Alleinstellungsmerkmal der hier vorgeschlagenen Systematik.

Eine mögliche Lücke, die Erweiterungspotential für das Werkzeug anzeigen könnte, ist die Abwesenheit von TUI-Elementen, die als \emph{Lenses} oder \emph{Trays} zu klassifizieren sind. Insbesondere \emph{Trays} erscheinen für die explizite Interaktion mit einzelnen Tokens -- etwa der Benennung oder der Einbettung von Zusatzinformation -- als geeignet. Die dazu notwendigen Interaktionsabläufe würden expliziter auf den Vorgang der Zuordnung von Information eingehen und sich stärken von anderen Interaktionen unterscheiden, die anderen Zwecken, z.B. der Herstellung von Verbindungen zwischen Modellierungstokens, dienen.

\section{Einordnung in das Ordnungssystem von Holmquist et al.}

Grundlage der Einordnung in diesem Abschnitt ist der Ansatz von \citep{Holmquist99}, der in Abschnitt \ref{sub:containers_tokens_tools} beschrieben wurde.

\subsection{Abbildung}

Die von \citeauthor{Holmquist99} verwendete Terminologie ist im Wesentlichen direkt auf jene abbildbar, die in dieser Arbeit verwendet wurde. Die Modellierungstokens und einbettbaren Tokens entsprechen im Wesentlichen \emph{Tokens}. Dies ist dadurch begründbar, dass die Art eines Modellierungstokens in einem Modell immer im gleichen Zusammenhang mit der Art der Information steht, die durch dieses repräsentiert wird. Eine Eigenschaft, die eher \emph{Containern} zuzuordnen ist, ist jedoch die dynamische Festlegbarkeit der Bedeutung einer Art von Modellierungstokens - die physischen Elemente ansich sind vor Beginn der Modellbildung generisch (also \emph{Container}), werden aber im Zuge der Modellierung mit Bedeutung belegt (die dann für alle Instanzen dieser Art von Modellierungstokens gilt) und sind dann eher als \emph{Tokens} zu klassifizieren. 

Die Werkzeugtokens des hier vorgestellten Systems entsprechen in ihrer Konzeption den \emph{Tools}. Sie manipulieren digitale Information, lösen Aktionen aus oder versetzten das System in einen anderen Zustand und entsprechen damit exakt der Definition von \emph{Tools}, die von den Autoren gegeben wird.

\emph{Information Faucets} sind im Kontext des hier vorgestellten Systems einerseits die Tischoberfläche, über die Information zu Modellierungstokens abgerufen werden kann, andererseits ist die Registrierungskamera ein klassisches Faucet im Sinne der Definition, da sie dem Abruf oder der Assoziation von Information an ein Token dient, sobald dieses in den Erfassungsbereich der Kamera gerät.

\subsection{Bewertung}

Die konzeptuellen Elemente des hier vorgestellten Systems sind also auf das Ordnungsstem von \citet{Holmquist99} abbildbar. Die Problematik der nicht eindeutigen Zuordnung von Modellierungstokens zur Kategorie \emph{Tokens} oder \emph{Constraints} ist einerseits auf eine der grundlegenden Design-Paradigmen des hier entwickelten Werkzeugs -- der Flexibiltät der Abbildung -- zurückzuführen, weist aber andererseits auch auf mögliches Verbesserungspotential hin.

Durch die Flexibilisierung nicht nur der Bindung zwischen physischen Elementen und digitaler Repräsentation sondern auch der Verwendung von unterschiedlichen physischen Elementen selbst könnten Modellierungstokens eher \emph{Token}-artiger werden. Indem Modellierende eigenen physische Elemente (auf ihrem Arbeitskontext) einbringen können, könnte die Erfassbarkeit der Bedeutung der physischen Repräsentation unter Umständen verbessert werden können.

\section{Einordnung in die Taxonomie von Fishkin}

Grundlage der Einordnung in diesem Abschnitt ist der Ansatzes von \citep{Fishkin04}, der in Abschnitt \ref{sub:taxonomie_fishkin} beschrieben wurde.

\subsection{Abbildung}
Das in dieser Arbeit entwickelte Werkzeug überspannt aufgrund seiner komplexen Struktur in beiden von \citeauthor{Fishkin04} vorgeschlagenen Dimensionen zur Klassifikation von Tangible Interfaces mehrere Ausprägungen. Um eine umfassende und ins Detail gehende Einordnung vornehmen zu können, werden im Folgenden Einzelaspekte des Systems betrachtet und eingeordnet. Während die Dimension "Embodiment" bereits in Kapitel \ref{cha:visualisierung} betrachtet wurde, um eine strukturierte Zuordnung der Ausgabekanäle vornehmen zu können, werden hier die einzelnen Funktionalitäten des Systems (siehe Abschnitt \ref{sec:benutzerinteraktion_mit_dem_werkzeug}) jeweils beiden Dimensionen zugeordnet (siehe Tabelle \ref{tab:einordnungFishkin})

\begin{table}[htbp]
	\centering
	\begin{tabular}{| p{6cm} || p{3cm} | p{3cm} |} \hline
		 & Embodiment & Metaphor \\ \hline \hline
		Platzieren und Benennen von Modellelementen & distant, nearby (Tastatur), full (Haftnotiz) & verb (Tastatur), verb + noun (Haftnotiz) \\ \hline
		Erstellen von Verbindern & nearby & verb bis noun+verb (Werkzeugtokens), verb (räumliche Nähe)\\ \hline
		Löschen von Verbindern & environmental bis nearby & noun \\ \hline
		Einbetten von Information & full & noun + verb \\ \hline
		Abrufen von Information & distant & verb \\ \hline
		Erstellen von Snapshots & environmental bis nearby & none \\ \hline
		Navigation in der Modell-Historie & distant & verb \\ \hline
		Wiederherstellen eines Modell-Zustandes & nearby & noun + verb\\ \hline
	\end{tabular}
	\caption{Einordnung des Systems in die Taxonomie nach Fishkin}
	\label{tab:einordnungFishkin}
\end{table}

Beim \emph{Platzieren und Benennen von Modellelementen} ist die Benennung auf zwei Arten möglich, die unterschiedlich in die Taxonomie einzuordnen sind. Bei Benennung mittels Auswahl und Tastatur ist durch die Projektion der Benennung die Embodiment-Ausprägung "nearby" zu wählen. Der Vorgang der Auswahl und Benennung kann als analog zur realen Welt gesehen werden, die eingesetzten Werkzeuge sind aber generischer Natur -- Metaphor ist also als "verb" zu klassifizieren. Bei der Benennung mittels Haftnotitz ist durch die unmittelbar auf den Tokens angebrachten Benennungen Embodiment "full", Der Vorgang des Beschriftens wird analog zur realen Welt durchgeführt, auch die Informationsträger (Haftnotizen) entsprechen jenen der realen Welt, Metaphor ist also "verb + noun", wobei  der notwendige Vorgang der expliziten Erfassung einer Beschriftung durch das System eine Klassifikation "full" verhindert und sogar die Einstufung "noun + verb" etwas abschwächt (keine Analogie des Vorgangs zur realen Welt).

Zur \emph{Herstellung von Verbindern} existieren ebenfalls zwei Möglichkeiten. In beiden Fällen ist durch die Projektion der Verbindung die Ausprägung in Embodiment "nearby", sie unterscheiden sich jedoch hinsichtlich "Metaphor". Bei der Verwendung von Werkzeugtokens ist der Vorgang der Auswahl der Endpunkte analog zur realen Welt zu sehen und somit als "verb" einzustufen. Die Verwendung von spezifischen Werkzeugtokens zur Herstellung gerichteter Verbinder zeigt sogar Züge von "noun + verb", da die durch das Token dargestellte Pfeilspitze eine Analogie zur realen Welt bildet.

Das \emph{Löschen von Verbindern} wird durch das Lösch-Token vorgenommen. Dieses ist durch einen Radiergummi symbolisiert, der jedoch nicht als solche eingesetzt wird sondern das System nur in einen Löschmodus versetzt. Die Klassifikation in Metaphor ist demnach "noun". Die Visualisierung des Löschzustandes erfolgt unspezifisch durch die Umfärbung der gesamten Tischoberfläche, womit ein Embodiment von "nearby" oder "environmental" (aufgrund der Unspezifität) gerechtfertigt wäre.

\emph{Einbetten von Information} erfolgt durch die Verwendung der Modellierungstokens als Container und Hineinlegen von kleineren Tokens. Embodiment ist in diesem Fall "full", die die Einbettung physisch nachvollzogen wird. Metaphor ist durch die Analogie des "Hineinlegens" von Information in "Container" in die Ausprägung "noun + verb" einzuordnen.

Das \emph{Abrufen von Information} wird über den sekundären Ausgabekanal abgewickelt und ist daher in Embodiment als "distant" einzuordnen. Der Vorgang des Herausnehmens von Information aus einem Container existiert analog zur realen Welt, das bei diesem Vorgang im Zentrum stehende Objekt, das einbettbare Token, ist jedoch generisch und weist nicht auf die Art der eigebetteten Information hin. Eine Klassifikation von "verb" in Metaphor erscheint daher gerechtfertigt.

Beim \emph{Erstellen von Snapshots} wird die gesamte Tischoberfläche als Feedbackkanal genutzt. Insofern ist Embodiment wie im Falle des Löschens von Verbindern im Bereich "environmental" bis "nearby" anzusiedeln. Das Snapshot-Token selbst ist ein generisches Objekt, das keine Analogie zur realen Welt aufweist. Metaphor ist daher "none".

Die \emph{Navigation in der Modell-Hierarchie} erfolgt mit dem runden Navigations-Token. Zur Ausgabe der gespeicherten Modell-Zustände wird der sekundäre Ausgabekanal
verwendet. Embodiment ist deshalb "distant". Metaphor beschränkt sich auf "verb", da der Drehvorgang zur Navigation analog zum Einstellen einer Uhr erfolgt, das Token selbst aber bis auf seine runde Form generisch ist.

Das \emph{Wiederherstellen eines Modellzustandes} erfolgt durch spezifische Anweisungen auf der Modellierungsoberfläche. Embodiment ist also als "nearby" einzustufen. Der Vorgang der Wiederherstellung erfolgt durch Verschieben der Modellierungstokens, was im Wesentlichen analog zur realen Welt abläuft. Da unmittelbar die Objekte manipuliert werden, kann Metaphor als "noun + verb" eingestuft werden.

\subsection{Bewertung}

Die Taxonomie nach \citeauthor{Fishkin04} ermöglicht eine strukturierte Erfassung einzelner Aspekte eines Tangible User Interfaces. Eine aussagekräftige Gesamteinordnung ist nur bei einfachen \glspl{TUI} möglich, komplexe, mit vielen Interaktionsmöglichkeiten ausgestattete Systeme tendieren dazu, ein sehr breites Spektrum der Taxonomie abzudecken. Für die detaillierte Betrachtung eines komplexen Gesamtsystems erscheint die Taxonomie dennoch geeignet, da einerseits aus den einzelnen Teileinordnungen für den jeweiligen Anwendungsfall ggf. Verbesserungspotentiale abgeleitet werden können und andererseits (nach der Betrachtung des hier entwickelten Systems) scheint, als ob ein die Taxonomie breit abdeckendes Gesamtsystem potentiell Inkonsistenzen im Interaktionsdesign aufweist bzw. unterschiedliche Interaktionsparadigmen vermischt wurden. Vor allem "Ausreißer" aus einem vorwiegend einheitlichen Gesamtbild scheinen einer näheren Betrachtung hinsichtlich eines möglichen Redesigns wert.

Konkret können diese Vermutungen im vorliegenden System vor allem an der Konzeption des Lösch-Tokens und des Snapshot-Tokens festgemacht werden. Der Großteil der Interaktionen mit dem System beinhaltet in der Dimension Metaphor den "verb"-Aspekt (zu etwa gleichen Teilen ausschließlich und in der Kombination mit "noun"). Die Funktionalitäten, die die beiden erwähnten Tokens einbeziehen, laufen diesem Trend entgegen und zeigen in Metaphor die Ausprägung "noun" bzw. "none". Tatsächlich zeigt sich in der Praxis, das die die Anwendbarkeit dieser Tokens von Benutzern missverstanden bzw. nicht verstanden wird. Ein Redesign dieser Tokens mit expliziterer bzw. eher aktivitätsorientierter Metaphor erscheint deshalb untersuchenswert.

Zusammenfassend scheint die Taxonomie vor allem im Zusammenhang mit der Sicherung von konsistenter Interaktion an der Benutzungsschnittstelle sinnvoll anwendbar zu sein. Der Mehrwert des Ansatzes zeigt sich hier nicht so sehr in den absoluten Ausprägungen auf den beiden Dimensionen sondern vielmehr in den relativen Unterschieden, die zwischen den einzelnen Teilen des Tangible User Interfaces auftreten.

\section{Zusammenfassung}

% chapter konzeptuelle_evaluierung (end)

% final draft
% todo: Interkapitel-Referenzen, Kontextgrafik


\chapter{Überblick über die empirische Untersuchung}
\label{cha:eval_ueberblick}

In diesem Kapitel wird ein Überblick über die in dieser Arbeit durchgeführte empirische Untersuchung gegeben. Dabei wird auf die einzelnen zu untersuchenden Aspekte, deren theoretische Grundlagen und die Durchführung der Untersuchung gegeben. Abbildung \ref{fig:img_Kontextgrafiken_k11} stellt dieses Kapitel und dessen Aufbau im Kontext der anderen inhaltlich vor- und nachgelagerten Kapitel dar.


\begin{figure}[htbp]
	\centering
		\includegraphics[scale=0.6]{img/Kontextgrafiken/k11.png}
	\caption{Kapitel „Überblick über die empirische Untersuchung“ im Gesamtzusammenhang}
	\label{fig:img_Kontextgrafiken_k11}
\end{figure}

Die im Rahmen der empirischen Evaluierung zu untersuchenden Aspekte sind Gegenstand des ersten Abschnitts. Neben einer wiederholenden grundlegenden Betrachtung werden hier die jeweiligen Untersuchungsfragen festgelegt. Eine nähere Betrachtung der einzelnen Aspekte, die Festlegung der Methodik und deren Operationalisierung im Rahmen des konkreten Untersuchungsdesigns erfolgt im Rahmen der übrigen Kapitel in diesem Teil der Arbeit.

Im zweiten Abschnitt wird ein Überblick über das globale Untersuchungsdesign gegeben. Auf Basis der zu evaluierenden Aspekte werden die konkret durchgeführten Teile der Evaluation (im Folgenden: "Evaluierungsblöcke") beschrieben und den Aspekten zugeordnet. Diese Evaluierungsblöcke werden überblicksweise hinsichtlich der intendierten Ziele, der Aufgabenstellung und der jeweiligen Anzahl der Teilnehmer beschrieben. Die Beschreibung bildet die Grundlage für die Beschreibung der Evaluierung der zu prüfenden Aspekte in den folgenden Kapiteln.

\section{Zu untersuchende Aspekte} % (fold)
\label{sec:untersuchungsaspekte}

Ziel dieser Arbeit ist die Unterstützung von expliziter Articulation Work. Eine Möglichkeit, explizite Articulation Work zu unterstützen, ist die Externalisierung und Abstimmung der mentalen Modelle über den betreffenden Arbeitsvorgang, die den Handlungen der beteiligten Personen zugrunde liegen (siehe Abschnitt \ref{sec:articulation_work_und_mentale_modelle}). Die Externalisierung mentaler Modelle ist mittels unterschiedlicher Methoden möglich, wobei sich Ansätze, die auf der Abbildung mentaler Modelle in diagrammatischen Strukturen basieren, als gut geeignet erwiesen haben (siehe Abschnitt \ref{sec:externalisierung_mentaler_modelle}). Zwei derartige Methoden sind Concept Mapping und Strukturlegetechniken, die beide Vor- und Nachteil hinsichtlich des Einsatzes in kollaborativen Szenarien zeigen (siehe \ref{cha:methodik}). In dieser Arbeit wird deshalb versucht, die Vorteile der beiden Ansätze methodisch zu vereinigen und zur Vermeidung der Nachteile durch ein Tabletop Interface zu unterstützen (siehe Kapitel \ref{cha:methodik} sowie \ref{cha:anforderungen}).

Anhand dieser Argumentationskette zeigt sich, dass zwischen der Zielformulierung und dem konkreten Werkzeug zur Zielerreichung einige argumentative Schritte liegen, die vorerst lediglich (aus der Literatur begründete) Annahmen darstellen. Im Zuge der Evaluation der Ergebnisse dieser Arbeit müssen nun diese Schritte einzeln betrachtet werden und hinsichtlich der jeweiligen Zielerreichung überprüft werden. Untersuchungsgegenstand ist dabei jeweils das erstellte Werkzeug, die betrachteten Aspekte unterscheiden sich je nach Argumentationsschritt. Die Untersuchungsfragen, die die Argumentationsschritte abdecken sind:
\begin{itemize}
 \item Sind das Werkzeug und dessen Komponenten verständlich und wie intendiert einsetzbar? (Aspekt: Werkzeug)
 \item Erlauben Werkzeug und Methode die Abbildung semantisch offener diagrammatischer Modelle? (Aspekt: Modell)
 \item Unterstützen Werkzeug und Methode Articulation Work? (Aspekt: Articulation Work)
\end{itemize}

Diese Fragen decken die Aspekte der oben beschriebene Argumentationskette ab, die Detaillierung der Fragestellungen ist in den folgenden Abschnitten beschrieben. Die Beschreibung der zu prüfenden Hypothesen sowie die Operationalisierung der Untersuchungsfragen erfolgt in den Kapitel \ref{cha:eval_werkzeug} bis \ref{cha:eval_aw}.

\subsection{Evaluierung des Werkzeugs}
\label{sub:eval_werkzeug}

Die Evaluierung des Werkzeugs an sich beschäftigt sich mit der Beantwortung der ersten Untersuchungsfrage. Diese zielt auf die Verständlichkeit des Werkzeugs im weiteren Sinn ab. Unter Verständlichkeit im weiteren Sinn ist hier zu verstehen, dass einerseits geprüft werden muss, ob die Bedeutung und grundlegende Verwendung der Komponenten des Werkzeugs von Benutzern erfasst und verstanden werden und ob andererseits die Interaktionsabläufe, die zur Auslösung bzw. Abwicklung einer Funktion des Werkzeugs führen, für Benutzer verständlich und nachvollziehbar sind.

Neben der quantitativen Bewertung anhand dieser Metriken ist bei der Untersuchung dieses Aspektes vor allem auch das qualitative Feedback der Benutzer notwendig, um Ansatzpunkte zur Verbesserung der Verwendbarkeit des Werkzeugs zu erhalten. Diese Anregungen können im Sinne eines iterativen Designprozesses umgesetzt und deren Auswirkungen erneut einer Evaluierung unterzogen werden. Neben der Erhebung dieser zusätzlich funktionalen Anforderungen für einen iterativen Designprozess sind in diesem Zusammenhang auch Hinweise hinsichtlich nicht-funktionaler Aspekte des Systems zu berücksichtigen, die der Verwendbarkeit negativ beeinflussen bzw. auch unkritisch sein können.

Die Verwendbarkeit des Werkzeugs kann nicht entkoppelt von der Anwendungsdomäne betrachtet werden, muss also im Kontext der Aufgabe, für die es eingesetzt wird, gesehen werden. Das Werkzeug ist zwar grundsätzlich für die Repräsentation beliebiger diagrammatischer Modelle ausgelegt, eignet sich aufgrund der unterschiedlichen Anforderungen jedoch nicht gleich gut für alle möglichen Anwendungsfälle (so sind z.B. ausschließlich Verbindungen mit zwei Endpunkten erstellbar, Verbindungen mit mehr Endpunkten werden nicht unterstützt). Die Prüfung der Verwendbarkeit des Werkzeugs kann hier fokussiert auf die in dieser Arbeit verfolgten Anwendungsfälle durchgeführt werden, die im Bereich der konzeptionellen Netze (im Wesentlichen Varianten von Concept Maps) und im Bereich der Abbildung von Arbeitsvorgängen (im Wesentlichen kausale Zusammenhänge mit Kontextinformation) zu finden sind. Die Unterstützung anderer Anwendungsfälle ist möglich und unter Umständen erstrebenswert, stellt jedoch kein Beurteilungskriterium dar.

\subsection{Evaluierung der Modellrepräsentationen}
\label{sub:eval_modell}

Der zweite zu evaluierende Aspekt sind die mit dem Werkzeug erstellten Modelle, die als Mittel zur Durchführung expliziter Articulation Work dienen. Eine wesentliche Eigenschaft, die Modelle dabei aufweisen müssen, ist die Adäquatheit der Modellierungssprache hinsichtlich der durch die Benutzer zu repräsentierenden Information. Diese Eigenschaft wird in der vorliegenden Arbeit durch die in Kapitel \ref{cha:mentale_modelle} beschriebene Anforderung der semantischen Offenheit abgedeckt, der jedoch vor allem hinsichtlich der intersubjektiven Verständlichkeit der Modelle und deren Eindeutigkeit nicht nur Vorteile bringt. Grundlegende ist in dieser Phase zu evaluieren, ob die erstellten Modelle den im Rahmen des Einsatzes zur Unterstützung von Articulation Work intendierten Zweck erfüllen. Dabei sind sowohl das Modell als auch das (hier von den Benutzern festgelegte) Metamodell zu betrachten. Anhaltspunkte zur Identifikation der zu evaluierenden Objekte sowie zum Vorgehen bieten hier der Ansatz der „Interactive Process Models“ \citep{Jorgensen04} und die „Grundsätze der ordnungsgemäßen Modellierung“ \citep{Becker00} sowie von diesen Arbeiten abgeleitete Ansätze.

Die eben beschriebenen Ansatzpunkte erlauben eine Evaluierung der erstellten Modelle hinsichtlich der Abbildbarkeit der Kernaspekte von „Articulation Work“ im engeren Sinne (Strauss' „salient dimensions“: \emph{„who, where, when, what and how“} \citep{Fjuk97}), decken also im Wesentlichen eine an organisationalen Abläufen orientierten Sicht auf Modelle ab. Im Sinne der Offenheit der Abbildung müssen aber auch Modelle berücksichtigt werden, die nicht diese „salient dimensions“ zur Grundlage haben, also „Concept Maps“ \citep{Novak06} im allgemeinen Sinn sind und damit die Abbildung mentaler Modelle nicht nur über unmittelbare Arbeitsaspekte sondern über beliebige Sachverhalte erlauben \citep{Ifenthaler06}. Dabei sind Metriken notwendig, die die erstellten Modelle selbst betrachten und deren Eigenschaften und Verwendung beim Concept Mapping bzw. im Rahmen von Strukturlegetechniken berücksichtigen.

Wie bereits im letzten Abschnitt angeführt, ist auch bei diesem Aspekt der Evaluierung der in dieser Arbeit verfolgte Anwendungszweck des Werkzeugs (bzw. hier: der Modelle) zu berücksichtigen. Dies ist insofern ein einschränkender Faktor, als dass hier Modelle lediglich im Kontext der Externalisierung mentaler Modelle und zur Unterstützung von Articulation Work berücksichtigt werden. Das Werkzeug selbst erlaubt auch die Erstellung von Modellen zu anderen Anwendungszwecken, die jedoch hier nicht weiter berücksichtigt werden.  

\subsection{Evaluierung der Articulation Work}
\label{sub:eval_articulation_work}

Letztendlich muss auch die durchgeführte Articulation Work selbst beurteilt werden. In der Literatur zum Thema „Articulation Work“ werden zumeist lediglich das Phänomen „Articulation Work“ und dessen konkrete Ausprägungen beschrieben (siehe Kapitel \ref{cha:articulation_work}), Ansätze zur Bewertung des Erfolgs von „Articulation Work“ sind jedoch selten zu finden. Aus der Verschränkung zwischen „Articulation Work“ und „Production Work“, also jenem Anteil der Arbeit, der unmittelbar der Zielerreichung dient, die von mehreren Autoren, unter anderem \citet{Fujimura87} und \citet{Strauss93}, erwähnt wird, lassen sich jedoch Ansatzpunkte ableiten.

„Articulation Work“ tritt immer dann auf, wenn eine Zielerreichung in der Production Work aufgrund von Unklarheiten oder Problemen zwischen den beteiligen Individuen nicht möglich ist. Ein erfolgreicher Abschuss der „Production Work“ bei am Beginn oder während der Arbeit bestehenden Unklarheiten weißt also unter Umständen auf erfolgreich durchgeführte Articulation Work hin. „Articulation Work“ manifestiert sich im Arbeitsprozess auf unterschiedliche Arten, so dass bei der Evaluierung hinsichtlich der Auswirkungen des Werkzeugs diese von den übrigen Einflussfaktoren (also auf anderen Wegen durchgeführte „Articulation Work“) getrennt werden muss. Dazu ist eine Betrachtung des gesamten Arbeitsablaufs unter Berücksichtigung von Production und Articulation Work notwendig. Metriken, die bei der Bewertung des Erfolgs von „Articulation Work“ zu berücksichtigen sind, sind also einerseits im Ergebnis des Arbeitsprozesses, andererseits auch im Arbeitsprozess selbst zu finden.

Ein zweiter Ansatzpunkt zur Bewertung des Erfolgs von „Articulation Work“ liegt in den Aussagen von \citet{Strauss93} hinsichtlich der wahrgenommenen "Problematik" einer Arbeitssituation, die „Articulation Work“ notwendig macht. Diese Wahrnehmung ist individueller Natur, d.h. „Articulation Work“ ist dann notwendig, wenn zumindest einer am Arbeitsablauf beteiligten Person Aspekte der Arbeit unklar sind oder problematisch erscheinen. Im Gegenzug ist keine „Articulation Work“ notwendig bzw. diese abgeschlossen, wenn alle beteiligten Personen die Situation als unproblematisch empfinden bzw. mit den im Rahmen der (expliziten) „Articulation Work“ erzielten Ergebnissen zufrieden sind. Hier liegt der Ansatzpunkt für eine Evaluierung des Erfolgs der durchgeführten „Articulation Work“, der diese auf Basis der individuellen Wahrnehmungen der beteiligten Personen beurteilt.
% section untersuchungsaspekte (end)

\section{Globales Untersuchungsdesign}
\label{sec:globales_untersuchungsdesign}

Die oben beschriebenen Aspekte müssen nun im Rahmen einer empirischen Untersuchung geprüft werden. Während das detaillierte Untersuchungsdesigns in den folgenden Kapiteln, die sich jeweils einem der drei zu evaluierenden Aspekte widmen, beschrieben wird, wird an dieser Stelle ein Überblick über das globale Untersuchungsdesign und die im Rahmen der Evaluierung durchgeführten Anwendungen des Werkzeugs gegeben.

Im ursprünglichen globalen Untersuchungsdesign war vorgesehen, jedem der zu untersuchenden Aspekte einen Block an Anwendungen des Werkzeugs mit einer auf den jeweiligen Aspekt abgestimmten Aufgabenstellung zuzuordnen. Nach Durchführung der ersten beiden Blöcke wurde offensichtlich, dass sich aus der Anwendung des Werkzeugs heraus zusätzliche Hypothesen ableiten ließen, die -- um sie in der Evaluierung berücksichtigen zu können -- in einem späteren Block geprüft werden mussten. Außerdem wurde offensichtlich, dass vor allem zur Evaluierung des Werkzeugs in allen Blöcken Verbesserungspotential identifiziert werden konnte bzw. Anregungen der Anwender rückgemeldet wurden, die zum Teil im Rahmen des iterativen Entwicklungsprozesses in das Werkzeug einflossen und deren Wirkung in einem späteren Block erneut geprüft werden musste. 

Letztendlich wurden die Blöcke für die Evaluierung mehrerer bzw. aller Aspekte herangezogen, sofern die jeweilige Aufgabenstellung geeignet war. Bei der nun folgenden Beschreibung der Anwendungs-Blöcke wird deshalb jeweils angegeben und begründet, inwieweit diese in die Evaluierung welcher Aspekte einfließen. Ein Überblick über das globale Untersuchungsdesign mit einer überblicksweisen Zuordnung zwischen den zu evaluierenden Aspekten und den Anwendungsblöcken wird in Abschnitt \ref{sec:eval_ueberblick_zusammenfassung} gegeben.

\subsection{Block 1: Technische Evaluierung}
\label{sub:eval_1}

Die Intention von Block 1 war die grundlegende Verständlichkeit und Verwendbarkeit des Werkzeugs zu prüfen. Fokus dieses Blocks an Anwendungen des Werkzeugs war also die Untersuchung der Eigenschaften des Werkzeugs selbst. Zusätzlich wurde hier explorativ die Wirkung des Werkzeugs auf die Modellierungstätigkeit und Kooperation der Anwender untersucht.

\subsubsection{Kontext} % (fold)
\label{ssub:1_kontext}

Die Untersuchung wurde im Rahmen einer Diplomarbeit durchgeführt (\cite{Bohninger10}), wobei die Untersuchungen in keinen einheitlichen realen Arbeitskontext eingebettet waren. Allerdings war die Aufgabenstellung so formuliert, dass die erstellten Modelle aus den Arbeitskontexten der jeweiligen Teilnehmer stammten.

% subsubsection kontext (end)

\subsubsection{Aufgabenstellung und Ablauf} % (fold)
\label{ssub:1_aufgabenstellung}

Den modellierenden Teilnehmern wurde mitgeteilt, dass sie einen Aspekt aus ihrem täglichen Arbeits- oder Privatleben abbilden sollten, der regelmäßig auftritt oder bereits mehrmals für Probleme sorgte. Die bewusste Offenheit der Aufgabenstellung sollte dabei bewirken, dass sich die Teilnehmer nicht zu sehr auf den abzubildenden Sachverhalt, sondern eher auf den Abbildungsprozess selbst fokussierten. Die Modellbildung erfolgte jeweils individuell.

Nur die Hälfte der Teilnehmer erstellte tatsächlich Modelle. Die zweite Hälfte wurde zur Überprüfung der Verständlichkeit der Modelle sowie der Verwendbarkeit des Werkzeugs zur kooperativen Modellierung herangezogen. Dazu wurde nach Abschluss einer Modellbildung jeweils ein nicht modellierender Teilnehmer an die Modellierungsoberfläche gebeten und aufgefordert, die Abbildung zu interpretieren. Die Beurteilung der Adäquatheit dieser Interpretation erfolgte durch den ursprünglich modellierenden Teilnehmer.

In einer dritten Phase wurden beide Teilnehmer aufgefordert, dass Modell gemeinsam zu reflektieren und gegebenenfalls zu verändern, um es den Ergebnissen der Reflexion anzupassen. In dieser Phase war das vorrangige Ziel, die Verwendung des Werkzeugs bei der Veränderung von Modellen und dessen kollaborativer Anwendung zu testen. 

Entsprechend dieser Beschreibung ist die Phase 1 dieses Blocks dem Anwendungsszenario „Verfeinerung mentaler Modelle“ (siehe Abschnitt \ref{sub:verfeinerung_individueller_mentaler_modelle}) zuzuordnen. Die Phasen 2 und 3 sind dem Anwendungsszenario „Wissenstransfer“ (siehe Abschnitt \ref{sub:wissenstransfer}) zuzuordnen.

% subsubsection aufgabenstellung (end)

\subsubsection{Anwendungen und Teilnehmer} % (fold)
\label{ssub:1_teilnehmer}

Insgesamt wurden neun Anwendungen des Werkzeug wie oben beschrieben durchgeführt. Zusätzlich wurde das Untersuchungsdesign im Rahmen von drei Anwendungen getestet (Pretest), woraus hinsichtlich der technischen Eigenschaften des Werkzeugs ebenfalls bereits Erkenntnisse gewonnen werden konnten. Insgesamt nahmen also 24 Personen an diesem Block von Anwendungen teil, 6 davon in der Pretest-Phase.

Die Teilnehmer (exkl. Pretest) stammten aus unterschiedlichen beruflichen Hintergründen und unterschieden sich auch in Art der höchsten abgeschlossenen Ausbildung (7 Universität/FH, 7 Matura, 4 Lehrabschluss). Die Altersspanne lag zwischen 19 und 43 Jahren, 13 Teilnehmer waren weiblich, 11 männlich.

Die Modellierungsphasen (exkl. Pretest) dauerten im Schnitt 8 Minuten ($SD=2:13$), die kürzeste Modellbildung dauerte 5 Minuten, die längste 12 Minuten. Die Interpretations- und Reflexionsphasen (nicht separat aufschlüsselbar, da zum Großteil ineinander übergehend) dauerten im Schnitt 5 Minuten ($SD=1:45$). 

% subsubsection teilnehmer (end)

\subsubsection{Verwendung der Ergebnisse} % (fold)
\label{ssub:1_verwendung_der_ergebnisse}

Die Ergebnisse dieses Blocks flossen in die Evaluierung des Werkzeugs und in die Hypothesenbildung hinsichtlich der erstellten Modelle ein. Für die Evaluierung der Modelle konnten erste Erkenntnisse hinsichtlich der Verständlichkeit der mit offener Semantik gewonnen werden. Keine Ergebnisse brachte dieser Block für die Evaluierung der durchgeführten „Articulation Work“.

% subsubsection verwendung_der_ergebnisse (end)

\subsection{Block 2: Aushandlung von Zusammenarbeit 1}
\label{sub:eval_2}

In Block 2 lag der Fokus der Evaluation erstmals auf der Unterstützung von Articulation Work. In diesem Rahmnen wurden auch die Verwendbarkeit des Werkzeugs im praktischen Anwendungskontext und die Eigenschaften der erstellten Modelle sowie deren Rolle im Prozess der expliziten Articulation Work untersucht.

\subsubsection{Kontext} % (fold)
\label{ssub:2_kontext}

Block 2 wurde im Rahmen eines Seminars aus Wirtschaftsinformatik mit Studierenden dieser Studienrichtung durchgeführt. Die im Seminar zu erstellenden wissenschaftlichen Arbeiten wurden von den Studierenden in Gruppen zu 2-3 Personen ausgearbeitet. Die Gruppen wurden so gebildet, dass sich die Teilnehmer nicht persönlich kannten oder zumindest nicht bereits in anderen Kontexten zusammengearbeitet hatten. Ziel dieser Maßnahme war die Vermeidung der Verfälschung der Untersuchungsergebnissse durch bereits eingespielte Gruppen (Erfahrungen in Seminaren der Vorjahre zeigen tendentiell schlechtere Ergebnisse bei der Zusammenarbeit von einander nicht persönlich bekannten bzw. nicht eingespielten Teilnehmern).

Im Rahmen des Seminars wurden sechs Forschungsgebiete ausgewählt, die in Zusammenhang mit der Erstellung und Verwendung sozio-technischer Systeme stehen (konkret: Organisationales Lernen, eLearning, \gls{CSCW}, Mentale Modelle, Articulation Work und semantische Contentanreicherung). Den Gruppen wurden jeweils zufällig zwei dieser Themen zugewiesen, die Aufgabe für die wissenschaftliche Arbeit war das Finden und Beschreiben einer möglichen Verknüpfung oder eines möglichen Zusammenhanges zwischen diesen Themen. Dieser Zusammenhang sollte im Zentrum der Seminararbeit stehen und aus beiden Grundlagen-Themen argumentiert sein. Ziel dieser Maßnahme war es, die Seminararbeit so offen wie möglich zu gestalten und einen Themenfindungs- bzw. -konkretisierungsprozess in den Ablauf zu integrieren. Außerdem wurde so ein Szenario geschaffen, in dem sich eine strikte Arbeitsteilung der Gruppenteilnehmer ohne weitere Zusammenarbeit während der Ausarbeitung der Inhalte („Production Work“) potentiell auf das Ergebnis auswirkt und sich konkret der fehlenden oder schwachen Verknüpfung der Grundlagen-Themen zeigt.

% subsubsection kontext (end)

\subsubsection{Aufgabenstellung und Ablauf} % (fold)
\label{ssub:2_aufgabenstellung}

Das Werkzeug wurde im Rahmen des Seminars für jede Gruppe zweimal eingesetzt. Die erste Anwendung fand zu Beginn des Seminars nach der Themenzuteilung statt. Die Aufgabe war die Aushandlung der Modalitäten der Zusammenarbeit mit der Zielsetzung, das an der resultierenden wissenschaftlichen Arbeit die Ko-Autorenschaft nicht mehr zu erkennen sein sollte (etwa durch plötzlich wechselnde Schreibstile oder Brüche in der Argumentationskette). Den Teilnehmern wurde das Werkzeug und dessen Funktionen vorgestellt und ohne weitere Vorgaben zur Verfügung gestellt (insbesondere wurden weder Vorgaben hinsichtlich der Topologie des zu erstellenden Modells oder der Bedeutung der Modellierungselemente gemacht).

In der zweiten Anwendung wurde der Zusammenarbeitsprozess reflektiert und gegebenenfalls eine Adaption vereinbart. Die zweite Anwendung fand in der Mitte des Semesters nach Abschluss der Literaturrecherche und der Grobkonzeption, aber vor der Erstellung der eigentlichen wissenschaftlichen Arbeit statt. Konkrete Zielsetzung für die Teilnehmer war hier, auf Basis der bisherigen Erfahrungen die weitere Zusammenarbeit zu vereinbaren. Das Werkzeug wurde ohne neuerliche Vorstellung und ohne Vorgaben hinsichtlich der Verwendung zur Verfügung gestellt.

Entsprechend dieser Beschreibung ist die erste Anwendung des Werkzeugs in diesem Block dem Anwendungsszenario „Aushandlung mentaler Modelle“ (siehe Abschnitt \ref{sub:aushandlung_individueller_mentaler_modelle}) zuzuordnen. Die zweite Anwendung ist in das Anwendungsszenario „Abstimmung mentaler Modelle“ (siehe Abschnitt \ref{sub:abstimmung_individueller_mentaler_modelle}) einzuordnen.
% subsubsection aufgabenstellung (end)

\subsubsection{Anwendungen und Teilnehmer} % (fold)
\label{ssub:2_teilnehmer}

Insgesamt nahmen an diesem Block 19 Personen in 9 Gruppen zu 2 bzw. einmalig 3 Personen teil. Jede der Gruppen setzte das Werkzeug zweimal ein, wodurch insgesamt 18 Anwendungen die Grundlage für die Auswertung der Ergebnisse bilden.

Die Teilnehmer waren allesamt Studierende der Wirtschaftsinformatik im zweiten Studienabschnitt. 18 Personen waren männlich, eine weiblich. Vier Personen hatten insofern Erfahrung mit wissenschaftlichen Arbeiten bzw. den konkreten Anforderungen in der betreffenden Lehrveranstaltungen, als dass sie bereits zuvor eine Lehrveranstaltung gleichen Typs besucht hatten.

In der ersten Runde dauerten die Anwendungen durchschnittlich 20 Minuten 50 Sekunden ($SD=4:18$), in der zweiten Runde lediglich 9 Minuten 49 Sekunden ($SD=5:20$).

% subsubsection teilnehmer (end)

\subsubsection{Verwendung der Ergebnisse} % (fold)
\label{ssub:2_verwendung_der_ergebnisse}

Die in diesem Block erhobenen Daten fließen in die Auswertung alle drei zu evaluierenden Aspekte ein. Zur Auswertung hinsichtlich des Erfolgs von Articulation Work liegen neben den Aufnahmen der Modellierungsvorgänge und den erstellten Modellen selbst auch Prozessreflexionen der Teilnehmer über den Erstellungsprozess der Seminararbeiten sowie die Seminararbeit an sich vor. Die Auswirkungen von „Articulation Work“ können also am Ergebnis (im Vergleich zu Ergebnissen auf Lehrveranstaltungen mit identischem Konzept) und am subjektiv wahrgenommenen Verlauf des Erstellungsprozesses der Arbeit bewertet werden.

Hinsichtlich der Auswertung des Modell-Aspektes wird durch diesen Block die Betrachtung von Modellen ermöglicht, die im Kontext der Arbeitsabstimmung erstellt wurden, also im Wesentlichen der Definition von Vorgehen und Schnittstellen dienen. Untersucht werden hier Aufbau und Inhalt der Modelle, wobei besonderes Augenmerk auf der Prozess und Ergebnis der Bedeutungszuweisung zu den Modellelementen liegt.

Im Rahmen der Werkzeug-Evaluation bringt dieser Block die ersten Hinweise auf die Anforderungen an das Werkzeug bei der Verwendung desselben im Rahmen einer realen Aufgabenstellung. Außerdem wurde in diesem Block erstmals ein durchgängig kollaboratives Szenario eingesetzt, bei dem immer mindestens zwei Personen gleichzeitig das Werkzeug verwenden.

% subsubsection verwendung_der_ergebnisse (end)

\subsection{Block 3: Concept Mapping 1}
\label{sub:eval_3}

Der Fokus von Block 3 lag auf der Erstellung von semantisch vernetzten Strukturen im Allgemeinen, wobei das Konzept der Concept Maps als ein etabliertes Werkzeug zur Externalisierung mentaler Modelle eingesetzt wurde. Inhaltlich fokussierte dieser Block nicht auf die Unterstützung von Articulation Work im engeren Sinne, wohl aber auf die Externalisierung und Abstimmung mentaler Modelle, was wie in Kapitel \ref{cha:mentale_modelle} beschrieben ein Mittel zur Unterstützung expliziter Articulation Work ist. Im Zentrum der Aufmerksamkeit steht in diesem Block also die Evaluierung der erstellten Modelle und der Nutzen des Werkzeugs zur Aushandlung einer einheitlichen auf einen gegebenen Sachverhalt.

\subsubsection{Kontext} % (fold)
\label{ssub:3_kontext}

Der dritte Block wurde im Rahmen einer Lehrveranstaltung zur Schulung von Methoden der Prozess- und Kommunikationsmodellierung durchgeführt. Diese Lehrveranstaltung ist Teil der im Curriculum definierten Basiskompetenz Wirtschaftsinformatik und wird von Studierenden im zweiten bis dritten Studiensemester besucht.

Im Rahmen der Lehrveranstaltung wurden drei unterschiedliche Prozessmodellierungssprachen (SeeMe \citep{Herrmann04a}, Subjekt-orientierte Modellierung mittels JPass \citep{Fleischmann07} und \gls{EPK}s aus dem ARIS-Konzept \citep{Scheer00}) eingeführt und praktisch an einem durchgängigen Beispiel angewandt. Diese Sprachen unterscheiden sich sowohl im Anwendungsgebiet, in den abgebildeten Aspekten des realen Prozesses sowie in der Darstellungsform des Modells. Ziel der letzten Teilaufgabe, die unter Einsatz des hier vorgestellten Werkzeugs durchgeführt wurde, war bei den Studierenden ein Verständnis für die Unterschiede und Gemeinsamkeiten zwischen diesen Sprachen zu erzeugen und sie in die Lage zu versetzen, für einen gegebenen Anwendungsfall eine adäquate Sprache auszuwählen.  

% subsubsection kontext (end)

\subsubsection{Aufgabenstellung und Ablauf} % (fold)
\label{ssub:3_aufgabenstellung}

Die Aufgabe zur Erstellung der Concept Map umfasste zwei Teile, wobei im zweiten Teil das Tabletop Interface eingesetzt wurde. Die Aufgabenstellung lautete in beiden Teilen, eine Concept Map zu erstellen, die die wesentlich erscheinenden Eigenschaften der vorgestellten Sprachen sowie deren Gemeinsamkeiten und Unterschiede darstellt. In der ersten Phase war diese Aufgabe von den Studierenden individuell zu lösen, wobei die Concept Map auf Papier oder mit Hilfe des Werkzeugs CMapTools\footnote{http://cmap.ihmc.us} \citep{Canas04} am Rechner erstellt werden konnte. 

In der zweiten Phase wurden Gruppen zu je drei Teilnehmern gebildet, die nun ihre individuellen Sichten konsolidieren und jeweils eine gemeinsame Concept Map zur gleichen Aufgabenstellung unter Einsatz des hier vorgestellten Werkzeugs erstellen sollten. Die Gruppen wurden zufällig zusammengesetzt, den Teilnehmern war während der individuellen Phase die Zuteilung nicht bekannt, so dass eine Abstimmung vor Anwendung des Werkzeugs weitgehend ausgeschlossen werden kann.

Der zweite Teil der Aufgabenstellung in diesem Block, in dem das Werkzeug zur Anwendung gebracht wurde, entspricht damit einer Ausprägung des Anwendungsszenarios „Abstimmung mentaler Modelle“ (siehe Abschnitt \ref{sub:abstimmung_individueller_mentaler_modelle}).

% subsubsection aufgabenstellung (end)

\subsubsection{Anwendungen und Teilnehmer} % (fold)
\label{ssub:3_teilnehmer}

An den Anwendungen, die in diesem Block durchgeführt wurden, nahmen insgesamt 54 Personen teil, die in 18 Gruppen einmalig mit dem Werkzeug arbeiteten. Alle Teilnehmer waren Studierende der Wirtschaftsinformatik im ersten Studienabschnitt (1-4 Semester), 8 waren weiblich, 46 männlich. Keinem der Teilnehmer war der Ansatz des Concept Mapping vor Beginn der betreffenden Aufgabe bekannt, Erfahrungen mit Prozessmodellierungssprachen (also dem Gegenstand der Concept Map) sammelten alle Teilnehmer erstmals im Rahmen der Lehrveranstaltung, in der dieser Evaluierungs-Block durchgeführt wurde.

Den Teilnehmern wurde das Werkzeug vor Beginn der Anwendung demonstriert und in sämtlichen Anwendungsaspekten erklärt. Die Anwendungen selbst dauerten durchschnittlich 32 Minuten 32 Sekunden ($SD=10:07$), wobei die kürzeste Anwendung 14 Minuten, die längste 45 Minuten dauerte.
% subsubsection teilnehmer (end)

\subsubsection{Verwendung der Ergebnisse} % (fold)
\label{ssub:3_verwendung_der_ergebnisse}

Die Daten, die aus diesem Block gewonnen werden konnten, gehen in die Evaluierung des Modell-Aspekts ein. Hier können einerseits wiederum die erstellten Modelle hinsichtlich Struktur, Inhalt und semantischen Zuweisungen untersucht werden. Der Modellierungsgegenstand ist in diesem Fall jedoch anders gelagert als im vorhergehenden Fall, anstelle eines Arbeitsabstimmung ist hier ein Vergleich von Konzepten durchzuführen. Andererseits können hier die Abstimmungsprozesse der individuellen mentalen Modelle insofern betrachtet werden, als dass für jede Gruppe neben dem kooperativ erstellten Ergebnis auch noch die individuellen Concept Maps vorliegen und ausgewertet werden können.

Wie bereits in den zuvor beschriebenen Blöcken können auch hier wieder Erkenntnisse hinsichtlich der Verwendung des Werkzeugs gewonnen werden. Aufgrund der der Aufgabe innewohnenden Relevanz der Verbindungen zwischen Konzepten ist vor allem deren Verwendung bzw. der Vorgang deren Erstellung zu betrachten.

Der Aspekt Articulation Work bleibt in diesem Block insofern außen vor, als dass kein Arbeitskontext vorliegt, keine aufzulösende Problematik vorliegt und keine Zusammenarbeit auszuhandeln ist. Insofern wird dieser Aspekt in diesem Block nicht explizit behandelt. Aufgrund der Durchführung sämtlicher Schritte, die zur Unterstützung expliziter Articulation Work notwendig sind (Externalisierung, Abstimmung) können aber die einzelnen Anwendungen zur Hypothesenbildung für den Evaluierungs-Aspekt Articulation Work herangezogen werden.

% subsubsection verwendung_der_ergebnisse (end)

\subsection{Block 4: Aushandlung von Zusammenarbeit 2}
\label{sub:eval_4}

Block 4 deckt die erste Anwendung des Werkzeugs im realen Unternehmenskontext ab. Im Rahmen einer Diplomarbeit \citep{Wahlmuller10} wurde das Werkzeug zur Offenlegung unmittelbar relevanter bzw. urgenter Fragestellungen eingesetzt, die im Rahmen eines Workshops zu den Abläufen in und zur Struktur der IT-Abteilung einer Unternehmensgruppe aus dem Bildungsbereich auftraten. Fokus dieses Blocks war die Untersuchung der Einsetzbarkeit des Werkzeugs im praktischen Kontext und dessen tatsächlicher Unterstützungsleistung für Articulation Work. Dazu wurde neben der Begleitung der eigentlichen Modellierungssession in zeitlichem Abstand auch eine Erhebung der wahrgenommenen Wirkungen auf die Arbeitspraxis durchgeführt.

\subsubsection{Kontext} % (fold)
\label{ssub:4_kontext}

Das Werkzeug wird im Kontext einer österreichweit tätigen Unternehmensgruppe im Aus- und Weiterbildungsbereich eingesetzt. Konkret kam das Werkzeug bei einem Workshop zum Einsatz, der von der Abteilung für technisches Produkt- und Service-Management in der konzernweiten IT-Abteilung abgehalten wurde. Die Abteilung hat rund 30 Mitarbeiter, die sich in insgesamt 5 Unterabteilungen gliedern. Zusätzlich ist ein Mitarbeiter abteilungsweit für die Qualitätssicherung der Arbeitsabläufe verantwortlich. Dieser leitete die Workshops, bei denen das Werkzeug zum Einsatz kam und führte in Abstimmung mit den jeweils betroffenen Kollegen die Themenauswahl durch.

% subsubsection kontext (end)

\subsubsection{Aufgabenstellung und Ablauf} % (fold)
\label{ssub:4_aufgabenstellung}

In unterschiedlichen Konstellationen mit Gruppengrößen von 2 bis 6 Personen wurden an zwei Workshop-Terminen Themen aus dem täglichen Arbeitskontext behandelt. Dabei wurden zum Einen Unterabteilungs-interne oder -übergreifende Arbeitsabläufe abgebildet und ausgehandelt, die als potentiell problematisch oder neu einzurichten wahrgenommen wurden. Zum Anderen wurde die wahrgenommene Struktur einer Unterabteilung selbst und deren Außenbeziehungen abgebildet, reflektiert und zwischen den Mitgliedern derselben abgestimmt.

Bei Aufgaben der ersten Kategorie begann die Bearbeitung jeweils mit der kooperativen Repräsentation des Ist-Standes und damit einem Abgleich der individuellen Sichten auf den aktuellen Arbeitsablauf. In weiterer Folge wurde anhand des Modells mögliches Optimierungspotential diskutiert und das Modell ggf. dementsprechend adaptiert.

Bei der Darstellung der Struktur einer Unterabteilung wurden im ersten Schritt die relevanten organisationalen Einheiten und Rollen gesammelt und auf der Oberfläche platziert. In weiterer Folge war die Aufgabe die Zusammenhänge innerhalb der Unterabteilung und deren Beziehungen nach außen durch räumliche Anordnung der definierten Einheiten sowie deren Kommunikationskanäle explizit durch Assoziationen darzustellen. Ziel war eine Repräsentation des Ist-Zustands der Abteilung, die soweit abgestimmt wurde, dass alle Teilnehmer ihre individuelle Sicht auf das Modell abbilden konnten.

Die unterschiedlichen Anwendungen in diesem Block sind entsprechend der obigen Beschreibung als Ausprägungen des Anwendungsszenarios „Abstimmung mentaler Modelle“ (siehe Abschnitt \ref{sub:abstimmung_individueller_mentaler_modelle}) zu betrachten.
% subsubsection aufgabenstellung (end)

\subsubsection{Anwendungen und Teilnehmer} % (fold)
\label{ssub:4_teilnehmer}

Am ersten Workshop-Tag nahmen insgesamt 6 Teilnehmer an 5 Modellierungsdurchgängen in Gruppen von 2 bis 5 Personen teil. Beim zweiten Workshop nahmen insgesamt 8 Teilnehmer an ebenfalls 5 Modellierungsdurchgängen teil. Insgesamt beschäftigten sich 8 Aufgaben mit konkreten Arbeitsabläufen, 2 Aufgaben widmeten sich der Struktur von Unterabteilungen. Die Gruppengröße variierte zwischen 3 und 6 Personen. 10 Teilnehmer nahmen an mehr als einem Modellierungsdurchgang teil, eine Person war an beiden Workshop-Tagen beteiligt.

Durch die Einbindung aller Unterabteilungen kamen Teilnehmer mit unterschiedlichem fachlichen Hintergrund zu Einsatz. Etwa die Hälfte der Teilnehmer war der Gruppe der Techniker oder Softwareentwickler zuzuordnen. Die andere Hälfte setzte sich aus Mitarbeiter im Support, Verkauf, Einkauf sowie der internen Verrechnung zusammen. Eine Teilnehmerin war weiblich, alle anderen Teilnehmer waren männlich.

Allen Teilnehmern wurde einmalig das Werkzeug und dessen Bedienung vorgestellt. Die Modellierungsdurchgänge dauerten zwischen 25 Minuten und etwa 1,5 Stunden. Sämtliche Teilnehmer wurden nach ihrer letzten Teilnahme an einem Durchgang mittels einem Fragebogen sowohl nach der Nützlichkeit des Werkzeugs als auch nach dem wahrgenommenen Nutzen des inhaltlichen Ergebnisses befragt. Um die mittelfristigen Auswirkungen der durchgeführten Modellierungsdurchgänge beurteilen zu können, wurde acht Wochen nach dem zweiten Workshop erneut eine Befragung durchgeführt, in der die wahrgenommene Auswirkungen thematisiert wurden. 

% subsubsection teilnehmer (end)

\subsubsection{Verwendung der Ergebnisse} % (fold)
\label{ssub:4_verwendung_der_ergebnisse}

Die Daten, die das Ergebnis dieses Blocks bilden, werden zur Evaluierung des Aspekts „Articulation Work“ eingesetzt. Betrachtet werden dabei die wahrgenommenen und beobachtbaren Veränderungen am Arbeitsprozess, der unter Einsatz des Werkzeugs reflektiert wurde.

Neben diesem Aspekt werden auch die erstellten Modelle, das in diesem Fall wieder aus der Domäne der Arbeitsabstimmung stammen, betrachtet und hinsichtlich ihrer Struktur und Semantik ausgewertet. 

Der Werkzeug-Aspekt wird in diesem Teil der Untersuchung nicht gesondert betrachtet, Verbesserungs- und Erweiterungspotential wird nur bei Erwähnung oder offensichtlichen Bedienungsfehlern bzw. Verständnisschwierigkeiten explizit identifiziert.

% subsubsection verwendung_der_ergebnisse (end)

\subsection{Block 5: Concept Mapping 2}
\label{sub:eval_5}

In Block 5 wird im Wesentlich der Evaluierungs-Blocks 3 (siehe Abschnitt \ref{sub:eval_3}) inhaltlich erneut durchgeführt (die Modellierungsaufgabe ist identisch). Im Gegensatz zu Block 3, wo die grundlegende Eignung des Werkzeugs zum Concept Mapping im Mittelpunkt stand, wird in Block 5 eine vergleichende Studie durchgeführt, die die Eignung des Tabletop Interfaces zum kollaborativen Concept Mapping mit jener der rechner-basierten CMapTools \citep{Canas04} vergleicht.

\subsubsection{Kontext} % (fold)
\label{ssub:5_kontext}

Die Anwendungssituation ist in diesem Block identisch mit dem in Abschnitt \ref{ssub:3_kontext} beschriebenen Kontext (Lehrveranstaltung im Curriculum Wirtschaftsinformatik zur Schulung von Ansätzen in der Prozess- und Kommunikationsmodellierung).

Der Ablauf der Lehrveranstaltung unterschied sich nur insofern von jenem in Block 3, als dass für jede Modellierungssprache separat eine Reflexion in Gruppen zu zwei Studierenden durchgeführt wurde. In diesen Reflexionen wurden die eigenen Anwendungen der jeweiligen Sprache mit einer Musterlösung gegenübergestellt und hinsichtlich ihrer Korrektheit und dem Vorgehen bei der Modellierung betrachtet.

% subsubsection kontext (end)

\subsubsection{Aufgabenstellung und Ablauf} % (fold)
\label{ssub:5_aufgabenstellung}

Die Aufgabenstellung ist identisch mit jener in Block 3. Ziel ist es, drei in der Lehrveranstaltung vorgestellte Prozessmodellierungssprachen hinsichtlich ihrer als wesentlich empfundenen Eigenschaften und deren Gemeinsamkeiten und Unterschiede zu betrachten und in einer Concept Map abzubilden. Das Vorgehen unterscheiden sich jedoch wegen der unterschiedlichen Zielsetzung der Untersuchung von jenem in Block 3.

Nach Abschluss der letzten Reflexionsphase (also nach drei Modellierungsphasen und drei Reflexionsphasen) wurde eine Gruppeneinteilung für die kollaborative Erstellung der Concept Map vorgenommen. Die Gruppen wurden aus jeweils zwei zufällig ausgewählten Studierenden gebildet. In der Untersuchung erhielt die Hälfte der Gruppen den Aufgabe, die Aufgabenstellung unter Verwendung des Tabletop Interfaces durchzuführen, die andere Hälfte verwendete das rechner-basierte Werkzeug CMapTools \citep{Canas04}, um die Concept Map zu erstellen. Die Gruppen wurden zufällig einem Werkzeug zugeordnet und führten die Aufgabenstellung in beiden Fällen kollaborativ in einer kontrollierten Umgebung durch. Im Gegensatz zu Block 3 entfiel hier die explizit geforderte individuelle Vorbereitungsphase, um eine stärkere inhaltliche Auseinandersetzung mit den Inhalten während der Modellierung zu fördern. 

Wie bereits in Block 3 ist diese Aufgabenstellung eine Ausprägung des Anwendungsszenarios „Abstimmung mentaler Modelle“ (siehe Abschnitt \ref{sub:abstimmung_individueller_mentaler_modelle}).

% subsubsection aufgabenstellung (end)

\subsubsection{Anwendungen und Teilnehmer} % (fold)
\label{ssub:5_teilnehmer}

An der Untersuchung nahmen 49 Studierende in 23 Gruppen teil, wobei 11 Gruppen die Aufgabenstellung unter Verwendung des hier vorgestellen Werkzeugs und 12 Gruppen unter Verwendung der CMapTools durchführten. Die Teilnehmer waren allesamt Studierende der Wirtschaftsinformatik in der ersten Phase des Bakkelauratsstudiums (erstes bis drittes Semester), 40 Teilnehmer waren männlich, 9 weiblich. Keiner der Teilnehmer hatte Vorkenntnisse in der Prozessmodellierung oder im Concept Mapping.

Den Teilnehmern wurde das Werkzeug vor Beginn der Anwendung demonstriert und in sämtlichen Anwendungsaspekten erklärt. Die Anwendungen selbst dauerten im Fall der Durchführung mittels CMapTools durchschnittlich 41 Minuten 10 Sekunden ($SD=8:34$), wobei die kürzeste Anwendung 30 Minuten 16 Sekunden, die längste 54 Minuten dauerte. Im Fall der Durchführung am hier vorgestellten System betrug die Modellierungsdauer im Schnitt 34 Minuten 18 Sekunden ($SD=9:11$), wobei die kürzeste Anwendung 21 Minuten, die längster 54 Minuten dauerte.

% subsubsection teilnehmer (end)

\subsubsection{Verwendung der Ergebnisse} % (fold)
\label{ssub:5_verwendung_der_ergebnisse}

Die in diesem Block erhobenen Daten fließen in vorrangig in den Modell-Aspekt der Evaluierung ein. Hier wird eine vergleichende Studie durchgeführt, die das Ziel hat, die Eignung der beiden verwendeten Ansätze für die Externalisierung von mentalen Modellen gegenüberzustellen. Grundlage dieser Beurteilung ist das erstellte Modell, außerdem wird der auch Modellierungsprozess in der Auswertung berücksichtigt.

Hinsichtlich des Werkzeug-Aspekts wird in diesem Block neben der Identifikation von Verbesserungspotential und Verständnisschwierigkeiten auch die Zufriedenheit mit dem Werkzeug bzw. dessen Akzeptanz bei den Benutzern explizit erhoben. 

Der Aspekt „Articulation Work“ wird hier wie schon in Block 3 und aus den dort angeführten Gründen (siehe Abschnitt \ref{ssub:3_verwendung_der_ergebnisse}) nicht weiter berücksichtigt.

% subsubsection verwendung_der_ergebnisse (end)

\section{Eingesetzte Werkzeuge und Verfahren} % (fold)
\label{sec:eingesetzte_werkzeuge_und_verfahren}

Für die Erfassung und Auswertung der erhobenen Daten kamen unterschiedliche Werkzeuge zum Einsatz. Grundsätzlich wurden sämtliche Anwendungen des Tabletop Interface auf Video erfasst, um eine nachträgliche quantitative und qualitative Auswertung zu ermöglichen. In den Evaluierungsblöcken 1, 4 und 5 kamen aufgrund der Fragestellung außerdem Fragebögen zur Erhebung des Vorwissens bzw. der Erfahrungen bei der Benutzung des Werkzeugs zum Einsatz. Die Auswahl bzw. das Design dieser Fragebögen ist abhängig von den jeweils zu testenden Hypothesen und wird dementsprechend im Rahmen der Beschreibung des Untersuchungsdesigns in den folgenden Kapiteln beschrieben. In allen Anwendungen wurden zudem die Modellierungsergebnisse graphisch festgehalten.

Sämtliche erfassten Daten wurden vor der Auswertung digital aufbereitet. Videos wurden als Mediendateien im MPEG4-Format abgelegt, Fragebögen wurden gescannt und im PDF-Format abgelegt, die graphischen Repräsentationen der Modellierungsergebnisse liegen als Bilddateien im PNG- bzw. JPEG-Format vor.

Die Rohdaten wurden einer quantitativen sowie qualitativen Auswertung unterzogen. Die dazu eingesetzten technischen Werkzeuge und methodischen Ansätze werden in den folgenden Abschnitten näher betrachtet.

\subsection{Werkzeuge} % (fold)
\label{sub:werkzeuge}

Zur Erfassung der quantitativen Daten wurde Microsoft Excel 2007\footnote{http://office.microsoft.com/excel} verwendet. Die deskriptiven statistischen Parameter wurden ebenfalls mit Microsoft Excel sowie mit dem Statistik-Paket R\footnote{http://www.r-project.org/} in der Version 2.9.0 berechnet. Die Parameter der schließenden Statistik wurden ebenfalls mit R berechnet. Die im Bereich der schließenden Statistik eingesetzten Methoden sind Gegenstand der folgenden Abschnitte. Zur Visualisierung der deskriptiven Parameter wurde neben R auch die Software OmniGraphSketcher\footnote{http://www.omnigroup.com/applications/omnigraphsketcher} eingesetzt.

Im Bereich der qualitativen Auswertung war vor allem die Benutzung des Tabletop Interface und die Interaktion der Modellierenden untereinander von Interesse. Zur Auswertung kam dabei einerseits offene Fragen in den eingesetzten Fragebögen und andererseits die von \citet{Hornecker04} vorgeschlagene Variante der Interaktionsanalyse nach \citet{Jordan95} zum Einsatz (siehe Abschnitt \ref{sub:interaktionsanalyse}). Die im Zuge der Durchführung zu erstellenden Transkripte der Interkationsabläufe wurden ohne spezifische Werkzeugunterstützung in einem Texteditor erstellt.
% subsection werkzeuge (end)

\subsection{Signifikanztests} % (fold)
\label{sub:signifikanztests}

Signifikanztests werden verwendet, um zu ermitteln, ob die Unterschiede zwischen zwei Stichproben tatsächlich signifikant sind, d.h. ob sich mit einer gegebenen Irrtumswahrscheinlichkeit (etwa $p<0.05$) auch die beiden den Stichproben zugrundeliegenden Grundgesamtheiten unterscheiden \citep[][S. 496]{Bortz03}. Signifikanztests sind deshalb ein zentraler Bestandteil der quantitativen Hypothesenprüfung.

Je nach Eigenschaften der zugrundeliegenden Grundgesamtheiten und Umfang der Stichprobe müssen unterschiedliche Verfahren zur Signifikanzprüfung eingesetzt werden. In den folgenden Unterabschnitten werden die hier verwendeten Tests kurz beschrieben und die Voraussetzungen für deren Einsatz angeführt. Eine umfassende Beschreibung der Methoden würde den Umfang dieser Arbeit sprengen. Als Grundlage für die Auswahl dienten in dieser Arbeit das einführende Werk von \citet{Bortz03} sowie die Website „Using R for statistical analyses“\footnote{http://gardenersown.co.uk/Education/Lectures/R}. Für eine umfassendere Beschreibung der Methoden sei hier auf diese Quellen verwiesen.

\subsubsection{t-Test} % (fold)
\label{ssub:t_test}

Der t-Test nach Student prüft in der Grundvariante anhand einer Stichprobe, ob der erwartete Mittelwert der entsprechenden Grundgesamtheit gleich, kleiner oder größer einem gegebenen Wert ist. In der -- hier eingesetzten -- Variante für zwei Stichproben prüft der Test, ob der erwartete Mittelwert der der ersten Stichprobe zugrundeliegenden Grundgesamtheit gleich, kleiner oder größer ist als jener der Grundgesamtheit zur zweiten Stichprobe. Für mehr als zwei Stichproben kann der t-Test nicht eingesetzt werden, alternativ kann der Kruskal-Wallis-Test (siehe unten) zur Anwendung gebracht werden.

Der t-Test geht von einer intervallskalierten, normalverteilten Grundgesamtheit aus. Bei Grundgesamtheiten, deren Verteilung unbekannt ist, kann bei ausreichender Stichprobengröße (häufig: $n>30$) auf Grund des zentralen Grenzwertsatzes von einer Normalverteilung ausgegangen werden und der t-Test wiederum eingesetzt werden. Bei kleineren Stichproben kann der t-Test nur dann verwendet werden, wenn eine Normalverteilung der Grundgesamtheit zu erwarten ist. Dies kann auch für kleine Stichproben mit dem Sharpiro-Wilk-Test (siehe unten) überprüft werden. Eine weitere Bedingung für den Einsatz des t-Tests ist, dass die Varianz der Grundgesamtheiten der beiden Stichproben identisch ist. Dies kann mit dem F-Test (siehe unten) festgestellt werden.

% subsubsection t_test (end)

\subsubsection{Wilcoxon-Test} % (fold)
\label{ssub:wilcoxon_text}
Der Wilcoxon-Rangsummentest (oder alternativ: Mann-Whitney-U-Test) ist ein Verfahren zur Überprüfung ob zwei Verteilungen signifikant übereinstimmen. Die Verteilungen müssen im Gegensatz zum t-Test nicht normalverteilt sein, sollten aber eine ähnliche Form aufweisen. Der Wilcoxon-Test ist auch für kleine Stichproben geeignet.

Aufgrund der Stichprobengrößen in den vorliegenden Untersuchungen ist der Wilcoxon-Test dem t-Test hier im Allgemeinen vorzuziehen. 

% % subsubsection wilcoxon_text (end)

\subsubsection{Sharpiro-Wilk-Test} % (fold)
\label{ssub:sharpiro_wilk_test}

Der Sharpiro-Wilk-Test \citep{Shapiro65} testet eine Verteilung auf „Nicht-Normaliät“ (d.h. die Nullhypothese ist, dass die Verteilung nicht normalverteilt ist). Mit einer Irrtumswahrscheinlichkeit von $p<0.05$ kann daher bei Ablehnung der Nullhypothese davon ausgegangen werden, dass die geprüfte Verteilung nicht normalverteilt ist. Dieser Test eignet sich auch für kleine Stichproben (ab $n>3$).

Er wird hier eingesetzt, um zu prüfen, ob der t-Test eingesetzt werden kann oder nicht (da dieser eine Normalverteilung der Parameter voraussetzt).

% subsubsection sharpiro_wilk_test (end)

\subsubsection{F-Test} % (fold)
\label{ssub:f_test}

Der F-Test (oder: Varianzquotienten-Test) überprüft, ob die Varianzen zweier normalverteilter Grundgesamtheiten signifikant übereinstimmen. Er wird hier eingesetzt, um die entsprechende Voraussetzung für den Einsatz des t-Tests zu überprüfen. Muss die Nullhypothese verworfen werden, so muss anstelle des t-Tests der Welch-Test eingesetzt werden, auf den hier nicht näher eingeangen wird.

% subsubsection f_test (end)

\subsubsection{Kruskal-Wallis-Test} % (fold)
\label{ssub:kruskal_wallis_test}

Der Kruskal-Wallis-Test ist wie der Wilcoxon-Test ein Verfahren, mit dem die Übereinstimmung von Verteilungen auf Signifikanz überprüft werden kann. Wie dieser setzt er keine Normalverteilung voraus und eignet sich auch für kleine Stichproben. 

Der Kruskal-Wallis-Test kann jedoch Gegensatz zu den anderen Verfahren auch für die Überprüfung von mehr als zwei Verteilungen gleichzeitig eingesetzt werden. Die Nullhypothese ist, dass sich die Verteilungen nicht unterscheiden. Werden detailliertere Hypothesen benötigt, so muss eine paarweise Überprüfung der Verteilungen mit einem der oben beschriebenen Verfahren vorgenommen werden.

Der Kruskal-Wallis-Test kann ab drei Verteilungen mit einer Stichprobengröße von jeweils mindestens 6 sinnvoll eingesetzt werden. In dieser Arbeit kommt er nur selten zum Einsatz, da großteils lediglich die Signifikanz der Übereinstimmung zweier Verteilungen überprüft werden muss.

% subsubsection kruskal_wallis_test (end)

% subsection signifikanztests (end)

\subsection{Interaktionsanalyse} % (fold)
\label{sub:interaktionsanalyse}

Die Interaktionsanalyse nach \citet{Jordan95} dient der qualitativen Auswertung von Interaktionsabläufen zwischen unterschiedlichen Individuen und den dazu eingesetzten Hilfsmitteln. Grundsätzlich wird die Interaktion aufgezeichnet und transkripiert. Das Transkipt enthält dabei nicht nur die verbale Interaktion sondern auch eine exakte Beschreibung der non-verbalen Aktivitäten der Beteiligten. Insbesondere wurde hier auf die Erfassung der Verwendung der verfügbaren Hilfsmittel geachtet. Die Interaktionsanalyse wurde von \citet{Hornecker04} zur Beschreibung der Wirkung von Tangible Interfaces auf Kooperation zwischen Individuen eingesetzt. Die in diesem Kontext vorgeschlagenen vereinfachte Variation der ursprünglichen Methode (v.a. der Verzicht auf interdisziplinär zusammengesetzte Analysegruppen) wurde in dieser Arbeit übernommen. 

Der Fokus der Analyse lag auf der Nutzung und Wirkung des eingesetzten Werkzeugs, weshalb zur Transkription jene Szenen ausgewählt wurden, in denen diese sichtbar wird. Transkripiert wurde jene Szenen, aus denen im Sinne der festgelegten Auswertungsebenen Schlüsse auf die Verwendung des Werkzeugs, die Wirkung bei der Modellbildung oder den Einfluss auf die Interaktion zwischen den Beteiligten gezogen werden können. Dies umfasst Szenen, in denen
\begin{itemize}
  \item das Werkzeug nicht in der in der Beschreibung des Interaktionsdesigns (siehe Abschnitt \ref{sec:benutzerinteraktion_mit_dem_werkzeug}) festgelegten Art verwendet wurde,
  \item die Funktionalität oder Bedienung des Werkzeugs missverstanden wurde,
  \item das Modell als Referenz verwendet wurde um Sachverhalte individuell zu reflektieren,
  \item das Modell von einem Teilnehmer als Referenz verwendet wurde um anderen Teilnehmern Sachverhalte zu erklären,
  \item das Modell als Mittel zur Fokussierung der inhaltlichen Diskussion zwischen den Teilnehmern verwendet wurde.
\end{itemize}

Bei der Transkription kam das von \citet{Hornecker04} vorgeschlagene Codierungsschema zum Einsatz, das im Folgenden wiedergegeben ist: \emph{„Zeilenweise Transkription nach einfachem, sequentiellem Schema. Zeitlicher Ablauf wird durch die Nummerierung vorne wiedergegeben. Fehlt eine Zeilennummer, so findet das Beschriebene mehr oder minder parallel mit dem Geschehen der darüberstehenden Zeile statt. Zusätzlich alle zehn Sekunden Zeitstempel in eigener Zeile. Sprünge in Zeilennummern zeigen Auslassungen in der Transkription an. Nur angedeutet werden Betonung und Zeitverhalten des Sprechens. Die Zeitdauer von Gestik und manueller Handlung ergibt sich aus den Beschreibungen.“} 

In der Darstellung werden dieses Codierungsschema wiefolgt dargestellt (ebenfalls angelehnt an \citet{Hornecker04}):
\begin{transkript}
	\emph{Zusammenhang, Auslöser der Situation}\\
	Schritt. \textbf{Teilnehmer:} Aussage\\
	\textbf{Teilnehmer:} simultan getätigte Aussage \emph{bzw. Handlung}\\
	Zeitstempel\\	
	Schritt. \textbf{Teilnehmer:} Aussage \emph{(gleichzeitig mit der Aussage ausgeführte Tätigkeit, eingefügt an jener Stelle, an der die Aktivität beginnt)} Aussage (Fortsetzung)\\
	\emph{Interaktion zwischen Teilnehmern oder der Teilnehmer mit dem System}\\
	Schritt. \textbf{Teilnehmer:} Aussage \textbf{Teil der Interaktion bzw. Aussage, der für die Prüfung der jeweiligen Hypothese relevant ist} Aussage (Fortsetzung)\\
\end{transkript}


Beispielhaft dargestellt kann ein Transkript wiefolgt aussehen:
\begin{transkript}
	\emph{Teilnehmer versuchen mit dem Radiergummi und nur einem anderen Marker einen Verbinder zu entfernen.}\\
	1. \textbf{B:} Können wir die nicht so auch einfach löschen?\\
	2. \textbf{C:} Ja mit dem Radiergummi.\\
	3. \textbf{B:} Muss ich den jetzt zuerst so \emph{(Hält den Radiergummi zur Kamera)} hinhalten?\\
	4. \textbf{A:} Nein, ich glaube, \textbf{den musst du einfach da \emph{(zeigt auf den Verbinder)} drauf legen.}\\
	10s\\
	5. \emph{B legt den Radiergummi auf den vom System automatisch erstellten Verbinder.}\\
	20s\\
	6. \textbf{A:} Und jetzt muss man \emph{(legt ein Markierungtoken auf den Verbinder)} Nein.\\
	\emph{Der Verbinder lässt sich auf diese Art nicht löschen und die Teilnehmer entscheiden sich den Fehler mittels der Wiederherstellungsfunktion zu beseitigen.}
\end{transkript}

% subsection interaktionsanalyse (end)
% section eingesetzte_werkzeuge_und_verfahren (end)

% section eingesetzte_verfahren (end)
\section{Zusammenfassung}
\label{sec:eval_ueberblick_zusammenfassung}

In diesem Kapitel wurde das globale Untersuchungsdesign zur Evaluierung der hier vorgestellten Arbeit beschrieben. In den ersten Abschnitten wurden die zu evaluierenden Aspekte identifiziert und beschrieben. Im Rahmen dieser Beschreibungen wurden auch mögliche Ansatzpunkte für die konkrete Untersuchung angeführt, die die Basis für die detaillierte Konzeption der Evaluierung dieser Aspekte in den Kapiteln \ref{cha:eval_werkzeug} bis \ref{cha:eval_aw} bildet. 

Im folgenden Abschnitt wurden die einzelnen im Rahmen der Evaluierung durchgeführten Untersuchungen angeführt. Diese Untersuchungen fokussieren jeweils auf einen der zu evaluierenden Aspekte. Ihnen liegt jeweils ein konkretes Szenario zu Grunde, das in einer Reihe von Anwendungen des Werkzeugs durch verschiedene Benutzer in Modelle umgesetzt wird. Je nach Fokus der Untersuchung werden vor- und nachgelagerte bzw. parallel ablaufende Aktivitäten in die Untersuchung mit einbezogen.

Die ursprüngliche Zuordnung zwischen den zu evaluierenden Aspekten und den einzelnen Evaluierungs-Blöcken ist in Tabelle \ref{tab:evaluierungsMatrixOriginal} nochmals überblicksweise angeführt. Die Zuordnung hatte jeweils Einfluss auf das Szenario, in dem das Werkzeug angewandt wurde sowie auf das Untersuchungsdesign.

\begin{table}[htbp]
	\centering
	\caption{Ursprüngliches globales Untersuchungsdesign}
	\begin{tabular}{| p{3cm} || p{2cm} | p{2cm} | p{2cm} |} \hline
		 & Werkzeug & Modell & Articulation Work \\ \hline \hline
		 Block 1 & x &  &   \\ \hline
		 Block 2 &  &  & x  \\ \hline
		 Block 3 &  & x &   \\ \hline
		 Block 4 &  &  & x  \\ \hline
		 Block 5 &  & x &   \\ \hline
	\end{tabular}
	\label{tab:evaluierungsMatrixOriginal}
\end{table}

Im Zuge der Durchführung der Evaluierung erwies sich die strikte Zuordnung eines Blocks zu genau einem zu evaluierenden Aspekt als nicht durchführbar. Tatsächlich liefern Untersuchungen zu einem (im Sinne der Zielhierarchie) übergeordneten Aspekte (von „unten“ nach „oben“: Werkzeug -- Modell -- Articulation Work) immer auch Erkenntnisse zu den untergeordneten zu evaluierenden Aspekten. Die Zuordnung der Evaluierungs-Blöcke zu den Aspekten verändert sich also wie in Tabelle \ref{tab:evaluierungsMatrix} angegeben. Diese Zuordnung liegt auch den oben angeführten Beschreibungen der Blöcke zugrunde, in denen jeweils die Beiträge eines Blocks zu den zu evaluierenden Aspekten angegeben wurden.

\begin{table}[htbp]
	\centering
	\caption{Einfluss der Untersuchungen auf die zu evaluierenden Aspekte}
	\begin{tabular}{| p{3cm} || p{2cm} | p{2cm} | p{2cm} |} \hline
		 & Werkzeug & Modell & Articulation Work \\ \hline \hline
		 Block 1 & \textbf{x} &  &   \\ \hline
		 Block 2 & x & x & \textbf{x}  \\ \hline
		 Block 3 & x & \textbf{x} &   \\ \hline
		 Block 4 & x & x & \textbf{x}  \\ \hline
		 Block 5 & x & \textbf{x} &   \\ \hline
	\end{tabular}
	\label{tab:evaluierungsMatrix}
\end{table}

In den folgenden Kapiteln wird nun die Evaluierung der einzelnen Aspekte über die Evaluierungs-Blöcke hinweg im Detail beschrieben. Dabei werden die Hypothesenbildung bzw. die Entwicklung der Hypothesen über die Zeit, die möglichen Ansätze zur Evaluierung der jeweiligen Hypothesen sowie das Untersuchungsdesign, das die Prüfung der Hypothesen ermöglicht, beschrieben. Die Kapitel schließen jeweils mit einer Zusammenfassung der Ergebnisse der Hypothesenprüfung und einer Bewertung dieser Ergebnisse im Kontext der globalen Zielsetzung, also der Unterstützung von expliziter Articulation Work.

% section globales_untersuchungsdesign (end)
% chapter eval_ueberblick (end)

\chapter{Evaluierung der Verwendbarkeit des Werkzeugs} % (fold)
\label{cha:eval_werkzeug}

Im ersten empirischen Teil der Evaluierung wurde die grundlegende Verständlichkeit und Verwendbarkeit des Werkzeug geprüft. Ziel war es hier, konzeptionelle und technische Eigenschaften bzw. Verhaltensweisen des Werkzeugs zu identifizieren, die den Modellierungsprozess behindern oder unterbrechen. Darunter fällt grundsätzlich jede Eigenschaft und jede Verhaltensweise, die die Benutzer zwingt, sich mit dem technischen System an sich zu beschäftigen und von der Erfüllung der eigentlichen Aufgabe ablenkt bzw. diese unterbricht. Abbildung \ref{fig:img_Kontextgrafiken_k12} stellt dieses Kapitel und dessen Aufbau im Kontext der anderen inhaltlich vor- und nachgelagerten Kapitel dar.

\begin{figure}[htbp]
	\centering
		\includegraphics[scale=0.6]{img/Kontextgrafiken/k12.png}
	\caption{Kapitel „Evaluierung der Verwendbarkeit des Werkzeugs“ im Gesamtzusammenhang}
	\label{fig:img_Kontextgrafiken_k12}
\end{figure}

Die Untersuchung wurde daneben auch genutzt, um explorativ die inhaltliche Verwendung des Systems zu untersuchen (d.h. wie es für seinen eigentlichen Verwendungszweck, die Modellierung, eingesetzt wurde) und Hypothesen abzuleiten, die in weiteren Schritten getestet werden konnten.

\section{Hypothesen} % (fold)
\label{sec:hypothesen}

In diesem Abschnitt werden die Hypothesen angeführt und begründet, die in diesem Teil der empirischen Untersuchung geprüft werden. Die hier angegebenen Hypothesen gehen auf die Eigenschaften des Werkzeugs in der Verwendung durch die Benutzer ein. Bei der Hypothesenbildung wird auf den Verwendungszweck des Werkzeugs, die Unterstützung der Bildung diagrammatischer Modelle, Rücksicht genommen -- die Modelle selbst sind jedoch nicht Gegenstand der Betrachtung, sondern werden erst im nächsten Kapitel behandelt. Nicht berücksichtigt wird außerdem die Verwendung zur Unterstützung von Articulation Work -- die Implikationen des Werkzeugs auf diese sind Gegenstand von Kapitel \ref{cha:eval_aw}.

\subsection{Konzeptionell begründete Hypothesen} % (fold)
\label{sub:konzeptionell_begründete_hypothesen}

Die folgenden Hypothesen wurden aus der Aufgabenstellung (siehe Kapitel \ref{cha:einführung}) sowie den Anforderungen an das Werkzeug (siehe Kapitel \ref{cha:anforderungen}) abgeleitet. Neben der Formulierung der Hypothese ist jeweils die Begründung aus der Konzeption des Werkzeugs angeführt.

Der grundlegende Anspruch des Werkzeugs ist es, explizite Articulation Work zu unterstützen. Wie in Teil \ref{prt:grundlagen} dieser Arbeit beschrieben, wird dies hier über die Externalisierung und Aushandlung von mentalen Modellen realisiert. Ein gängiges Mittel, um mentale Modelle zu repräsentieren, sind diagrammatische Modelle, worunter die Ergebnisse der vorgeschlagenen Methoden zur Externalisierung -- Concept Mapping und Strukturlegetechniken -- fallen. Das Werkzeug muss also die Repräsentation diagrammatischer Modelle unterstützen. Die Prüfung dieser Hypothese ermöglicht die Beurteilung der Erfüllung der Anforderung \ref{anf:physische_abbildung_legen_beliebiger_diagrammatischer_modelle} (siehe Seite \pageref{anf:physische_abbildung_legen_beliebiger_diagrammatischer_modelle}). 

\begin{hyp}
	\label{hyp:diagmodelle}
	Das Werkzeug ermöglicht die Repräsentation diagrammatische Modelle.
\end{hyp}

„Articulation Work“ ist immer in einen kooperativen Arbeitszusammenhang eingebettet. Die Kollaboration findet dabei nicht nur im produktiven Teil der Arbeit statt, sondern hat immer auch Auswirkungen auf die „Articulation Work“. Jede Unterstützung von „Articulation Work“ muss damit auch in kooperativen Szenarien einsetzbar sein. Dies gilt auch für das hier vorgestellte Werkzeug, das die kooperative Bearbeitung einer Aufgaben (hier: der Externalisierung und Abstimmung mentaler Modelle) ermöglichen muss. Die Prüfung dieser Hypothese ermöglicht die Beurteilung der Erfüllung der Anforderung \ref{anf:kollaborative_und_unmittelbare_manipulierbarkeit_des_modells} (siehe Seite \pageref{anf:kollaborative_und_unmittelbare_manipulierbarkeit_des_modells}).

\begin{hyp}
	\label{hyp:kollaborativ}
	Das Werkzeug ermöglicht kooperatives Arbeiten an einer Aufgabe.
\end{hyp}

Die Aspekte von Arbeit, die im Rahmen von „Articulation Work“ abzustimmen sind, sind unterschiedlicher Natur. Naheliegend ist eine Abstimmung der Abläufe und Schnittstellen zwischen Personen, aber auch nicht-prozedurale Information wie das Verständnis der Struktur und Elemente eines Arbeitszusammenhangs kann Gegenstand von Articulation Work sein. Gleiches gilt für die im Rahmen der Articulation Work abzustimmenden mentalen Modelle -- diese bilden die Basis für Handlungsentscheidungen, umfassen aber im Allgemeinen (in Abgrenzung zu Schemata) nicht nur handlungsleitende Information sondern auch Kontextinformation, die die Bewertung der wahrgenommenen Situation ermöglicht. Demensprechend muss ein Werkzeug zu Unterstützung von expliziter Articulation Work und damit der Externalisierung von mentalen Modellen die Verwendung in unterschiedlichen Kontexten, d.h. für unterschiedliche zu externalisierenden Informationsstrukturen, die in mentalen Modellen abgebildet sind, ermöglichen. Die Prüfung dieser Hypothese ermöglicht die Beurteilung der Erfüllung der Anforderung \ref{anf:nicht_vorgegebene_semantik_der_modellierungselemente} (siehe Seite \pageref{anf:nicht_vorgegebene_semantik_der_modellierungselemente}).

\begin{hyp}
	\label{hyp:kontexte}
	Das Werkzeug ist gleichwertig für Modellierungsaufgaben in unterschiedlichen Kontexten einsetzbar.
\end{hyp}

Die ersten drei hier formulierten Hypothesen sind unmittelbar aus der globalen Zielsetzung abgeleitet und bilden die grundlegenden Anforderungen an das Werkzeug bei der Unterstützung von Articulation Work ab. Die nun folgenden Hypothesen sind konzeptionell nicht mehr direkt auf die globale Zielsetzung ausgerichtet sondern stellen auf Funktionalität des Werkzeugs ab, die den Modellbildungsprozess unterstützen soll. 

Auf Basis der Möglichkeit zur Navigation durch die Entstehungsgeschichte des Modells besteht auch die Möglichkeit, vergangene Modellzustände wiederherzustellen. Das Werkzeug unterstützt dabei die Benutzer durch die Ausgabe von schrittweisen Anweisungen, die den aktuellen Modellzustand in den wiederherzustellenden Zustand überführen. Allgemein bietet diese Funktionalität die Möglichkeit, erkannte Fehler im Modell zu korrigieren, ohne dabei bereits repräsentierte Information zu verlieren. Im kollaborativen Einsatz ermöglicht diese Funktionalität, alternative, individuelle Sichten auf den abzustimmenden Sachverhalt zu repräsentieren und dabei die Möglichkeit bieten, einen für alle Beteiligten akzeptablen Ausgangspunkt wiederherzustellen. Die Prüfung dieser Hypothese ermöglicht die Beurteilung der Erfüllung der Anforderung \ref{anf:ermöglichung_experimenteller_veränderungen_am_modell} (siehe Seite \pageref{anf:ermöglichung_experimenteller_veränderungen_am_modell}).

\begin{hyp}
	\label{hyp:wiederherstellung}
	Die Möglichkeit der Wiederherstellung vergangener Modellzustände fördert die Bereitschaft alternative Repräsentationen auszuprobieren.
\end{hyp}

Die letzten beiden Hypothesen dieses Abschnitts sind ausschließlich auf die Verwendung des Werkzeugs an sich ausgerichtet und stehen nicht im Kontext von Articulation Work oder der Unterstützung der Externalisierung mentaler Modelle. Hypothese \ref{hyp:behinderung} steht für den in der Zielsetzung formulierten Anspruch, dass das Werkzeug in den Hintergrund treten muss und die Beschäftigung mit der eigentlichen Aufgabe nicht behindern darf. Dabei wird hier nicht auf den konkreten Anwendungsfall -- die Erstellung von Modellen -- eingegangen sondern lediglich die allgemeine Funktionsfähigkeit und Bedienbarkeit des Werkzeugs betrachtet. Ersteres ist Gegenstand der Evaluierung der erstellten Modelle, die in Kapitel \ref{cha:eval_modell} beschrieben werden. Die Prüfung dieser Hypothese ermöglicht die Beurteilung der Erfüllung der Anforderung \ref{anf:physische_abbildung_legen_beliebiger_diagrammatischer_modelle} (siehe Seite \pageref{anf:physische_abbildung_legen_beliebiger_diagrammatischer_modelle}).

\begin{hyp}
	\label{hyp:behinderung}
	Das Werkzeug behindert die Modellbildung nicht.
\end{hyp}

Hypothese \ref{hyp:gewöhnung} geht davon aus, dass bei wiederholten Verwendung des Werkzeugs Lern- und Gewöhnungseffekte auftreten, die die Verwendung erleichtern, beschleunigen und zu weniger Fehlbedienung führen. Dies ist ein Effekt, der bei jedem Werkzeug zu erwarten ist, dessen zugrunde liegenden Konzepte den Benutzern bewusst sind. Von dieser Voraussetzung kann durch die inhaltliche Einführung der Benutzer in die das Werkzeug prägenden und motivierenden Ideen ausgegangen werden. Damit wäre zu erwarten, dass das Werkzeug bei wiederholtem Einsatz in den späteren Anwendungen effizienter (im Sinne von schneller und Fehlbedienungen vermeidend) verwendet wird. Die Prüfung dieser Hypothese ermöglicht die Beurteilung der Erfüllung der Anforderung \ref{anf:physische_abbildung_legen_beliebiger_diagrammatischer_modelle} (siehe Seite \pageref{anf:physische_abbildung_legen_beliebiger_diagrammatischer_modelle}).

\begin{hyp}
	\label{hyp:gewöhnung}
	Wiederholte Verwendung des Werkzeugs führt zu schnellerer Modellbildung und weniger Fehlbedienungen.
\end{hyp}

Hinsichtlich der in Kapitel \ref{cha:anforderungen} formulierten Anforderungen können die hier formulierten Hypothesen zusammenfassend wie in Tabelle \ref{hyp:eval_tui} dargestellt eingeordnet werden:

\begin{table}[htbp]
	\centering
	\caption{Hypothesen zur Werkzeugbenutzung und deren Bezug zu den Anforderungen an das Werkzeug}
\begin{tabular}{|c|c|}
  \hline
   Hypothese & Anforderung \\ \hline
   \ref{hyp:diagmodelle} & \ref{anf:physische_abbildung_legen_beliebiger_diagrammatischer_modelle} \\
   \ref{hyp:kollaborativ} & \ref{anf:kollaborative_und_unmittelbare_manipulierbarkeit_des_modells} \\
   \ref{hyp:kontexte} & \ref{anf:nicht_vorgegebene_semantik_der_modellierungselemente} \\
   \ref{hyp:wiederherstellung} & \ref{anf:ermöglichung_experimenteller_veränderungen_am_modell} \\
   \ref{hyp:behinderung} & \ref{anf:physische_abbildung_legen_beliebiger_diagrammatischer_modelle} \\
   \ref{hyp:gewöhnung} & \ref{anf:physische_abbildung_legen_beliebiger_diagrammatischer_modelle} \\ \hline
\end{tabular} 
	\label{hyp:eval_tui}
\end{table}

% subsection konzeptionell_begründete_hypothesen (end)

\subsection{Explorativ gebildete Hypothesen} % (fold)
\label{sub:explorativ_gebildete_hypothesen}

Neben den aus der Aufgabenstellung abgeleiteten Hypothesen wurden einige Hypothesen auch während der Durchführung der einzelnen Evaluierungs-Blöcke gebildet. Diese Hypothesen sind spezifischer auf einzelne Aspekte des Werkzeugs abgestellt und decken beobachtete Auffälligkeiten und Missverständnisse in der Verwendung des Werkzeugs ab. 

Die erste in diesem Zusammenhang beobachtete Auffälligkeit betrifft die Herstellung von Verbindern zwischen einzelnen Modellelementen. Wie in Abschnitt \ref{sub:verbinden_von_modellelementen} beschrieben, existieren zwei Möglichkeiten, diese Funktion auszuführen. Einerseits können die beiden Modellelemente, die verbunden werden sollen, mit Markierungs-Tokens ausgewählt werden, worauf hin eine Verbindung hergestellt werden. Andererseits können Verbinder auch durch das Zusammenführen der zu verbindenden Blöcke (bis sich deren Breitseiten berühren) hergestellt werden. In der ersten Implementierung des Werkzeugs, die im Evaluierungs-Block 1 und im ersten Teil des zweiten Blocks verwendet wurde, war lediglich die erste Variante verfügbar. Die Möglichkeit zur Herstellung von Verbindern wurde in den in diesen Blöcken durchgeführten Anwendungen kaum eingesetzt. Dies führte einerseits zur Bildung der Hypothese \ref{hyp:keine_verbinder} (siehe Abschnitt \ref{sub:m_explorativ_gebildete_hypothesen}), andererseits wurde bei ersten Auswertungen der Beobachtungen der im Verhältnis zum übrigen Modellierungs-Prozess hohe Zeit-Aufwand bei der Herstellung von Verbindern offensichtlich. Dieser Aufwand ist den Maßnahmen zur Stabilisierung der Erkennungsleistung des Werkzeugs geschuldet und kann mit den eingesetzten Interaktionsablauf nicht reduziert werden. Aufgrund einer Anregung eines Untersuchungsteilnehmers wurde deshalb die oben beschriebene zusätzliche Möglichkeit zur Herstellung von Verbindungen implementiert. Zu untersuchen ist nun, ob diese Maßnahme die Nutzung von Verbindern bei der Modellbildung tatsächlich erhöht.

\begin{hyp}
	\label{hyp:verbinder}
	Die Einführung der alternativen Möglichkeit zur Verbindungsherstellung erhöht die Nutzung von Verbindern bei der Modellerstellung.
\end{hyp}

Die zweite hier aufgestellte Hypothese betrifft eine Auffälligkeit bei der Verwendung des Löschtokens. Das Löschtoken wird verwendet, um das Werkzeug in einen Modus zu versetzen, in dem Verbinder gelöscht werden können. Schon die konzeptionelle Einordnung des Werkzeugs in Kapitel \ref{cha:konzeptionelle_evaluierung} zeigte Potential für Missverständnisse in der Verwendung dieses Tokens (siehe z.B. die Abschnitte \ref{sec:spezifikation_des_tac_schemas_nach_shaer_et_al_} und \ref{sec:einordnung_in_die_taxonomie_von_fishkin}). Zusammengefasst liegt die aus der Theorie ableitbare Problematik darin, dass durch die äußere Form des Tokens -- einem Radiergummi -- eine Metapher für dessen Verwendung („ausradieren“ von Elementen) suggeriert wird, die in dieser Form im Werkzeug nicht umgesetzt ist, da das Token lediglich als Schalter fungiert. Erste Beobachtungen deuteten darauf hin, dass die Verwendung des Löschtoken tatsächlich unverständlich oder missverständlich zu sein scheint. Die zugehörige Hypothese ist positiv formuliert, zu erwarten wäre demnach, dass sie verworfen werden muss.

\begin{hyp}
	\label{hyp:radierer}
	Das Löschtoken ermöglicht intuitives Löschen von Modellelementen.
\end{hyp}

% subsection explorativ_gebildete_hypothesen (end)
% section hypothesen (end)

\section{Untersuchungsdesign und Durchführung} % (fold)
\label{sec:untersuchungsdesign}

In diesem Abschnitt wird auf Basis der oben formulierten Hypothesen das Untersuchungsdesign abgeleitet und die Durchführung der Untersuchung beschrieben. Der erste Teil des Abschnitts beschreibt die Operationalisierung der Hypothesen und damit die Festlegung wie diese konkret geprüft werden können. Im zweiten Teil des Abschnitts wird die Durchführung der Prüfung beschrieben. Hier erfolgt neben der Zuordnung der einzelnen Evaluierungsblöcke (siehe Abschnitt \ref{sec:globales_untersuchungsdesign}) auch die Darstellung rein beschreibender Parameter der Werkzeugverwendung, die nicht unmittelbar in die Prüfung der Hypothesen eingehen. 

\subsection{Operationalisierung} % (fold)
\label{sub:operationalisierung}

In diesem Abschnitt wird für jede Hypothese identifiziert, in welcher Form sie geprüft werden kann. Dies umfasst die Festlegung der Messpunkte sowie der jeweiligen Mess- und Auswertungsmethode (letzte bezugnehmend auf den in Abschnitt \ref{sec:eingesetzte_werkzeuge_und_verfahren} beschriebenen Verfahren). Zudem werden jene Evaluationsblöcke festgelegt, die für die jeweilige Untersuchung herangezogen wurden.

Für jede Hypothese wird also spezifiziert, anhand welcher Aspekte diese geprüft werden kann (= abhängige Variablen). Zudem wird festgelegt welche Ausgangssituation bei der Anwendung gewählt werden muss, um die Prüfung durchführen zu können (= unabhängige Variable) und welche Faktoren die Beurteilung ggf. ungewollt beeinflussen können (= Störvariablen).

\subsubsection{Repräsentation diagrammatischer Modelle} % (fold)
\label{ssub:repräsentation}

Gegenstand dieses Abschnitts ist die Prüfung der Hypothese \ref{hyp:diagmodelle}. Diese bezieht sich auf die Eignung des Werkzeugs für die Repräsentation diagrammatischer Modelle.

Voraussetzung für die Prüfung der Hypothese ist der Einsatz von Modellierungsaufgaben, die so formuliert sind, dass es grundsätzlich möglich ist, sie durch die Beschreibung in einem diagrammatischen Modell zu erfüllen. Keinen Einfluss auf die Untersuchung haben die eingesetzte Methodik sowie eventuell vorhandene Modellierungsvorkenntnisse, da die grundsätzlich Möglichkeit der Erstellung diagrammatischer Modelle unabhängig von der Art der Verwendung und von der Kompetenz der Benutzer ist. 

Geprüft wird die Hypothese hier an der Repräsentation, die mit Hilfe des Werkzeugs erstellt wurde. Ein diagrammatisches Modell zeichnet nach \citep{Larkin87} aus, dass in ihm Konzepte und deren Zusammenhänge visuell-graphisch dargestellt werden können (in Abgrenzung zu textuellen Beschreibungen). Zur Bewertung der Hypothese werden deshalb die erstellten Repräsentationen herangezogen und überprüft, ob sie den Anforderungen an ein diagrammatisches Modell -- das Vorhandensein von Konzepten und Beziehungen zwischen diesen -- erfüllen.

% subsubsection repräsentation (end)

\subsubsection{Kooperatives Arbeiten} % (fold)
\label{ssub:kollaboratives_arbeiten}

Gegenstand dieses Abschnitts ist die Prüfung der Hypothese \ref{hyp:kollaborativ}. Dabei wird überprüft, ob das Werkzeug kooperatives Arbeiten an einer Modellierungsaufgabe erlaubt.

Dazu muss eine Modellierungsaufgabe gewählt werden, in der die kooperatives Erstellung des Modells vorgesehen ist. Etwaige Modellierungsvorkenntnisse haben keinen Einfluss auf die Beurteilung der hier betrachteten Hypothese.

Zur Beurteilung eignen sich in diesem Fall die Zeitverteilung der Beteiligung der einzelnen Benutzer am Modellierungsvorgang, das Verhalten der Benutzer bei simultaner Manipulation eines Modells auf der Modellierungsoberfläche sowie der subjektive Eindruck der Benutzer über deren Kooperation untereinander. Der erstgenannte Aspekt kann quantitativ gemessen werden, wobei eine tendenziell zeitlich gleichverteilte Einbindung der Beteiligten in die Modellbildung für die Annahme der Hypothese spricht. Zusätzlich kann mittels dem zweiten und dritten Aspekt qualitativ beurteilt werden, ob und wie eine kooperative Manipulation des Modells durch mehrere Benutzer gleichzeitig möglich ist.

% subsubsection kollaboratives_arbeiten (end)

\subsubsection{Einsetzbarkeit in unterschiedlichen Kontexten} % (fold)
\label{ssub:einsetzbarkeit_in_unterschiedlichen_kontexten}

Gegenstand dieses Abschnitts ist die Operationalisierung der Hypothese \ref{hyp:kontexte}. Diese Hypothese zielt dabei auf die Eignung des Werkzeugs zur Modellbildung in unterschiedlichen Kontexten, d.h. für unterschiedliche Modellierungsaufgaben. 

Zur Beurteilung dieser Hypothese muss die Modellierungsaufgabe entsprechend den unterschiedlichen Einsatzkontexten variiert werden. Etwaige Modellierungsvorkenntnisse können die individuelle Beurteilung insofern beeinflussen, als das sie Werkzeugs für eine bestimmte Aufgabe als besser oder schlechter geeignet erscheinen lassen.

Zur Prüfung der Hypothese bieten sich sind in diesem Fall die Wahrnehmung der Eignung durch die Benutzer, die qualitativ beurteilt wird, und die Korrelation der Größe der erstellten Modelle mit der benötigten Modellierungsdauer an. Korrelliert die Modellgröße positiv mit der Modellierungsdauer, so ist der Zeitanteil, der zu Beschäftigung mit dem Werkzeug selbst (und nicht mit der Modellierungsaufgabe) tendenziell stabil. Daraus kann abgeleitet werden, dass das Werkzeug die verglichenen Modellierungsaufgaben gleich gut (oder schlecht) unterstützt.

% subsubsection einsetzbarkeit_in_unterschiedlichen_kontexten (end)

\subsubsection{Wiederherstellung vergangener Modellzustände} % (fold)
\label{ssub:wiederherstellung_vergangener_modellzustände}

Gegenstand dieses Abschnitts ist die Operationalisierung der Hypothese \ref{hyp:wiederherstellung}. Gegenstand der Überprüfung ist die Verwendung der Wiederherstellungsfunktionalität zum Zwecke der versuchsweisen Veränderung des Modells.

Zur Prüfung dieser Hypothese muss die Modellierungsaufgabe so gestaltet, dass sinnvoll unterschiedliche Repräsentationen gebildet werden können. Modellierungsvorkenntnisse haben keine Auswirkungen auf diese Untersuchung.

Zur Beurteilung dieser Hypothese wird ist die \emph{Anzahl der Verwendungen der Wiederherstellungsfunktionalität zur Korrektur inhaltlich verworfener Repräsentationen} herangezogen. Werte über 0 deuten hier auf eine Annahme der Hypothese hin. Zusätzlich können qualitative Aussagen zur Nutzung dieser Funktionalität und deren \emph{wahrgenommenen Nutzen} zur Beurteilung verwendet werden. 

% subsubsection wiederherstellung_vergangener_modellzustände (end)

\subsubsection{Nicht-Behinderung} % (fold)
\label{ssub:nicht_behinderung}

Gegenstand dieses Abschnitts ist die Operationalisierung der Hypothese \ref{hyp:behinderung}. Dabei wird überprüft, ob bei der Verwendung des Werkzeugs dieses in den Aufmerksamkeitsfokus der Benutzer tritt oder sich diese auf die eigentliche Modellierungsaufgabe konzentrieren können. 

Die Modellierungsaufgabe hat keinen Einfluss auf die Überprüfung dieser Hypothese, lediglich etwaig vorhandene \emph{Modellierungsvorkenntnisse} können als \textbf{Störvariable} wirken, da sie Einfluss auf die erwartete Funktionalität des Werkzeugs haben kann.

Zur Beurteilung, ob bzw. inwieweit das Werkzeug die Modellbildung behindert, werden sowohl quantitativ als auch qualitative beurteilbare Metriken herangezogen. Die Anzahl von \emph{Fehlfunktionen des Werkzeugs} bzw. das \emph{Auftreten von Systemabstürzen} kann als Indikator für eine behindernde Wirkung des Werkzeugs herangezogen werden. Das Auftreten von Missverständnissen und daraus resultierende Fehlbedienungen können ebenfalls eine Behinderung des Modellierungsvorgangs interpretiert werden. Zudem werden Aussagen der Benutzer hinsichtlich hinderlicher Faktoren bei der Werkzeugbenutzung als Maß für die wahrgenommene Behinderung durch das Werkzeug herangezogen. Der Einfluss von Modellierungsvorkenntnissen kann in diesem Fall nicht mit statistischen Maßnahmen kompensiert werden. Etwaige Vorkenntnisse werden dementsprechend bei der Auswertung angeführt und müssen bei der Diskussion der Hypothese berücksichtigt werden.

% subsubsection nicht_behinderung (end)

\subsubsection{Gewöhnung an das Werkzeug} % (fold)
\label{ssub:gewöhnung_an_das_werkzeug}

Gegenstand dieses Abschnitts ist die Operationalisierung der Hypothese \ref{hyp:gewöhnung}. Dabei wird überprüft, ob wiederholte Benutzung des Werkzeugs Auswirkung auf die Qualität der Interaktion hat. Eine Erhöhung der Qualität äußert sich in schnellerer Modellbildung und weniger Fehlbedienung.

Bei der Prüfung der Hypothese muss eine etwaige veränderte Funktionalität des Werkzeugs zwischen den verglichenen Evaluierungsblöcken berücksichtigt werden, die die Interaktion einerseits erleichtern kann, andererseits aber auch zu Fehlbedienung aufgrund von unbekannten Interaktionsmustern führen kann. Auch unterschiedliche Modellierungsaufgaben, die ein Individuum in den aufeinander folgenden Anwendungen bearbeitet, können die Beurteilung erschweren, weil potentiell andere (noch unbekannte) Funktionen des Werkzeugs zum Einsatz kommen können.

Zur Beurteilung der Qualität der Interaktion sind einerseits die Anzahl der Fehlbedienungen des Werkzeugs pro Zeiteinheit und andererseits die Arbeitsdauer am Werkzeug\footnote{Die Arbeitsdauer am Werkzeug ist im Gegensatz zur gesamten Modellierungsdauer um jenen Zeitanteil reduziert, in dem die Teilnehmer interagieren, ohne am Werkzeug zu arbeiten.} in Abhängigkeit der Modellgröße heranzuziehen. Die Normierung der Arbeitsdauer ist notwendig, um vergleichbare Werte für unterschiedliche Werkzeug-Anwendungen zu erhalten. Sinken beide Werte zwischen zwei Evaluierungsblöcken, die auf der gleichen Stichprobe aufbauen, signifikant, so kann die Hypothese bestätigt werden. Um eine Vergleichbarkeit zwischen den Anwendungen herzustellen, ist es sinnvoll, in beiden Blöcken eine identische Modellierungsaufgabe zu stellen und die Funktionalität des Werkzeugs nicht zu verändern. Identische Modellierungsaufgaben können durch die wiederholte inhaltliche Beschäftigung mit der Aufgabe zu schnellerer Arbeit bzw. zu kompakteren Modellen führen. Dies kann wiederum durch die Berücksichtigung der reinen Arbeitszeit am Werkzeug sowie der Normierung derselben in Abhängigkeit der Modellgröße kompensiert werden.

% subsubsection gewöhnung_an_das_werkzeug (end)

\subsubsection{Herstellung von Verbindern} % (fold)
\label{ssub:herstellung_von_verbindern}

Gegenstand dieses Abschnitts ist die Operationalisierung der Hypothese \ref{hyp:verbinder}. Mit Hilfe dieser Hyothese soll überprüft werden, ob die Einführung der alternativen Möglichkeit zur Herstellung von Verbindern deren Verwendung signifikant gesteigert hat.

Bei der Messung muss der Einfluss der Modellierungsaufgabe (da sie die Anzahl der benötigten Verbinder beeinflussen kann) und eventuell vorhandene Modellierungsvorkenntnisse (da diese Einfluss auf die Struktur des Modells haben können) berücksichtigt werden. Um den Einfluss dieser Aspekte zu reduzieren, wird die Beurteilung in zwei Evaluierungsblöcken vorgenommen, in denen die gleiche Stichprobe mit der gleichen Aufgabenstellung das Werkzeug mit der gleichen Methodik anwandte. Lediglich die Funktionalität des Werkzeugs wurde zwischen den beiden Anwendungen um den alternativen Weg zur Herstellung von Verbindern erweitert.  

Zur Beurteilung des Ausmaßes der Verwendung von Verbindern kann die \emph{Connectedness} des Modells herangezogen werden. Die Connectedness ist das Verhältnis zwischen der Anzahl der im Modell verwendeten Verbinder und der Anzahl der verwendeten Knoten (Modellierungselemente). Hier ist zu prüfen, ob die Connectedness in jenem Evaluierungs-Block, in dem der alternative Weg zur Herstellung von Verbindungen verfügbar war, signifikant höher ist als in jenem Block, in dem sie nicht verfügbar war.

% subsubsection herstellung_von_verbindern (end)

\subsubsection{Verwendung des Löschtokens} % (fold)
\label{ssub:löschtoken}

Gegenstand dieses Abschnitts ist die Operationalisierung der Hypothese \ref{hyp:radierer}. Dabei wird überprüft, ob das Löschtoken intuitiv korrekt verwendet wird oder ob es zu Fehlinterpretationen kommt.

Die Verwendbarkeit des Löschtokens ist unabhängig von der Modellierungsaufgabe, der angewandten Methodik und auch von eventuell vorhandenen Modellierungsvorkenntnissen. Hinsichtlich des Anwendungskontext des Werkzeugs sind also keine Voraussetzungen zu beachten.

Zur Beurteilung der intuitiven Verwendbarkeit werden quantitative und qualitative Merkmale der Werkzeugverwendung herangezogen. Quantitativ beurteilbar ist der Anteil der Fehlbedienungen des Löschtokens in Bezug auf alle Anwendungen des Werkzeugs, in denen es grundsätzlich verwendet wurde. Qualitativ wird die Art des Missverständnisses, das zu den jeweiligen Fehlbedienungen führt, beurteilt.

Zur Messung der quantitativen Variablen wird für jede Anwendung die Anzahl der Fehlbedienungen erhoben, die durch das Löschtoken verursacht wurden. Dieser Wert wird in Bezug zur Gesamtanzahl der Fehlbedienungen gesetzt, so dass der Anteil der durch das Löschtoken verursachten Fehlbedienungen berechnet werden kann. Bei "gleich guter" intuitiver Bedienbarkeit aller Werkzeuge wäre eine Gleichverteilung der Fehler zu erwarten. Ist der Anteil der durch das Löschtoken verursachten Fehlbedienungen höher als der Anteil, der bei Gleichverteilung zu erwarten wäre, so deutet dies auf eine Ablehnung der Hypothese hin.

Qualitativ werden Modellierungssituationen betrachtet, in denen das Löschtoken zum Einsatz kommt. Auf Basis von Transkripten der Interaktion zwischen den Benutzern und dem Werkzeug, bei denen es zu Fehlbedierungen kam, werden die aufgetretenen Missverständnisse explizit identifziert.

% subsubsection löschtoken (end)

% subsection operationalisierung (end)

\subsection{Durchführung} % (fold)
\label{sub:durchführung}

In diesem Abschnitt werden die für diesen Evaluierungs-Teil relevanten deskriptiven Parameter der berücksichtigten Anwendungs-Blöcke angeführt.
Als Grundlage der Überprüfung der Hypothesen werden hier die Evaluierungs-Blöcke 1 bis 5 verwendet. Dabei wurden für die quantitativ zu prüfenden Variablen die Blöcke 2 und 3 herangezogen, da in diesen die größten Stichproben zur Verfügung standen. In die qualitative Auswertung der Ergebnisse wurden hingegen alle Blöcke (1-5) mit einbezogen.

\subsubsection{Stichprobe} % (fold)

Für die Untersuchung der Hypothesen in diesem Kapitel wurden die Evaluierungsblöcke 1 bis 5 herangezogen. Die Stichprobe setzt sich wie in Tabelle \ref{tab:stichprobe_tui} beschrieben zusammen.

\begin{table}[htbp]
	\centering
	\caption{Stichproben der Evaluierung zur Werkzeugverwendung}

		\begin{tabular}{| l || c | c |}
		\hline
			Evaluierungsblock & $n_{Anwendungen}$ & $n_{Teilnehmende}$ \\ \hline
			technische Evaluierung		  &  9 & 18 \\
			Aushandlung 1 (1. Durchgang)  &  9 & 19 \\
			Aushandlung 1 (2. Durchgang)  &  9 & 18 \\
			Concept Mapping 1			  & 18 & 54 \\
			Aushandlung 2				  & 10 & 13 \\
			Concept Mapping 2 (Tisch)     & 11 & 24 \\  \hline
			Gesamt						  & 66 & 146 \\ \hline
	\end{tabular}
	\label{tab:stichprobe_tui}
\end{table}

\subsubsection{Dauer der Werkzeugverwendung} % (fold)

Die Dauer der Werkzeug-Verwendung wurde den Blöcken 2 („Aushandlung“) und 3 („Concept Mapping“) erhoben. Die Bearbeitungszeit ist wie in Tabelle \ref{tab:dauer_werkzeugverwendung} dargestellt verteilt (siehe auch Abbildung \ref{fig:img_Evaluierung_usageTimeOverview}\footnote{In allen Boxplots gilt folgende Notation: 
\begin{itemize}
	\item breite horizontale Linie: Bereich zwischen 25\%- und 75\%-Quantil
	\item breite vertikale Linie: Median
	\item linke schmale Linie: Bereich zwischen 2,5\%- und 25\%-Quantil
	\item rechte schmale Linie: Bereich zwischen 75\%- und 97,5\%-Quantil
	\item Kreuze: Ausreißer (außerhalb 2,5\%- und 97,5\%-Quantil)
\end{itemize}
}):

\begin{table}[htbp]
	\centering
	\caption{Dauer der Werkzeugverwendung}
\begin{tabular}{| p{1cm} || p{3cm} | p{3cm} | p{3cm} |}
  \hline
   & Aushandlung (1. Durchgang) & Aushandlung (2. Durchgang) & Concept Mapping \\ \hline
   $t_{min}$ & 11m 54s & 2m 5s & 14m 1s \\ 
   $\overline{t}$ & 20m 53s & 9m 49s & 32m 32s \\ 
   $s(t)$ & 4m 18s & 5m 20s & 10m 7s \\
   $t_{max}$ & 27m 30s & 19m 29s & 45m 0s \\ \hline
\end{tabular} 
	\label{tab:dauer_werkzeugverwendung}
\end{table}

\begin{figure}[htbp]
	\centering
		\includegraphics[width=15cm]{img/Evaluierung/usageTimeOverview.png}
	\caption{Dauer der Werkzeugverwendung -- Überblick}
	\label{fig:img_Evaluierung_usageTimeOverview}
\end{figure}

Die erhobene Dauer der Werkzeug-Verwendung teilt sich ein einen Anteil, an dem tatsächlich mit dem Werkzeug interagiert wird und einen Anteil, der anderen Tätigkeiten (wie inhaltlicher Diskussion, Bedeutungsaushandlung, \ldots) gewidmet ist. Diese beiden Anteile sind in den einzelnen Blöcken wie folgt verteilt (siehe auch die Abbildungen \ref{fig:img_Evaluierung_usageTimeConceptMapping} und \ref{fig:img_Evaluierung_usageTimeNegotiation}):

\begin{figure}[htbp]
	\centering
		\includegraphics[width=15cm]{img/Evaluierung/usageTimeConceptMapping.png}
	\caption{Dauer der Werkzeugverwendung -- Concept Mapping}
	\label{fig:img_Evaluierung_usageTimeConceptMapping}
\end{figure}

\begin{figure}[htbp]
	\centering
		\includegraphics[width=15cm]{img/Evaluierung/usageTimeNegotiation.png}
	\caption{Dauer der Werkzeugverwendung -- Aushandlung}
	\label{fig:img_Evaluierung_usageTimeNegotiation}
\end{figure}

% subsection durchführung (end)
% section untersuchungsdesign (end)

\section{Ergebnisse} % (fold)
\label{sec:ergebnisse}

\subsection{Repräsentation diagrammatischer Modelle} % (fold)
\label{sub:repräsentation_diagrammatischer_modelle}

Gegenstand der hier beschriebenen Untersuchung ist Hypothese \ref{hyp:diagmodelle} („Das Werkzeug ermöglicht die Repräsentation diagrammatische Modelle.“). Als Grundlage dieser Untersuchung dienen die Ergebnisse aller Evaluierungsblöcke, da die Aufgaben in allen Fällen auf die Erstellung einer Repräsentation in Form eines diagrammatischen Modells gefordert war.

Ausgewertet wird hier, ob die Ergebnisse der Modellierung jeweils als diagrammatisches Modell zu klassifizieren sind. Ein diagrammatisches Modell zeichnet nach \citep{Larkin87} aus, dass in ihm Konzepte und deren Zusammenhänge visuell-graphisch dargestellt werden. Eine Darstellung von Beziehungen kann durch die explizite Darstellung von Verbindungen zwischen Konzepten oder durch andere graphische Mittel wie Gruppierung von Konzepten in räumlicher Nähe erfolgen. Um eine eindeutige Auswertbarkeit gewährleisten zu können, wird hier auf die explizite Darstellung von Verbindungen eingeschränkt. 

\subsubsection{Auswertung} % (fold)

In allen vorliegenden Modellen wurden Konzepte als Grundelemente des diagrammatischen Modells verwendet. Das Kriterium zur Klassifizierung als diagrammatisches Modell ist im Folgenden also das Vorhandensein von Verbindungen. Bei der Auswertung ergab sich die in Tabelle \ref{tab:modelle_mit_verbindern} dargestellte Verteilung.

\begin{table}[htbp]
	\centering
	\caption{Anzahl der Modelle mit Verbindern}

\begin{tabular}{| p{3cm} || p{3cm} | p{3cm} |}
  \hline
   Block & Modelle gesamt & Modelle mit Verbindern \\ \hline
   1 & 9 & 0 \\ 
   2 & 18 & 9 \\ 
   3 & 18 & 17 \\ 
   4 & 10 & 10 \\ 
   5 & 11 & 11 \\ \hline
   Gesamt & 66 & 47 \\ \hline
\end{tabular}
	\label{tab:modelle_mit_verbindern}
\end{table}

Insgesamt sind in 66 Modellen, die als Ergebnis vorliegen, 47 Modelle zu identifizieren, in denen explizit Verbindungen zur Darstellung von Beziehungen zwischen Konzepten verwendet werden ($71,2\%$). Eine implizite Darstellung von Beziehungen ist jedoch in allen vorliegenden Modellen zu erkennen. Nicht explizit durch Verbindungen abgebildete Beziehungen werden in allen Fällen durch die räumliche Konfiguration der Konzepte zueinander dargestellt.

\subsubsection{Diskussion} % (fold)

Legt man das Kriterium des Vorhandenseins von Verbindungen zwischen Konzepten an, so sind $71,2\%$ der betrachteten Modelle als diagrammatische Modelle zu klassifizieren. Dies erscheint vordergründig eine geringe Zahl zu sein, die gegen die allgemeine Gültigkeit der Hypothese sprechen würde. Allerdings sind in allen Modelle implizite Verbindungen zwischen Konzepten eindeutig zu identifizieren. Außerdem ist zu erkennen, dass der Anteil an diagramatischen Modellen über die Evaluierungsblöcke (und damit die Weiterentwicklung des Werkzeugs über die Zeit) hinweg stetig ansteigt, bis er in den letzten beiden Blöcken jeweils $100\%$ erreicht. Dies ist durch technische Fehlfunktionen zu erklären, die es in ersten Evaluierungsblöcken schwer bzw. teilweise unmöglich machten, explizite Verbindungen intentional zu erstellen. Unter Anbetracht dieser Erkenntnisse erscheint die Annahme der Hypothese \ref{hyp:diagmodelle} als gerechtfertigt.

Die Abbildung von Verbindungen durch räumliche Konfiguration ist Gegenstand der Prüfung von Hypothese \ref{hyp:keine_verbinder} in Kapitel \ref{cha:eval_modell} und wird dort einer näheren Betrachtung unterzogen.

\subsubsection{Ergebnis} % (fold)

\textbf{Hypothese \ref{hyp:diagmodelle} kann auf Basis der Untersuchung bestätigt werden.} Die Abbildung von Konzepten und Beziehungen zwischen diesen wurde in allen vorliegenden Modellen erfolgreich umgesetzt, wenngleich die Modellierung von expliziten Verbindungen in den ersten beiden Evaluierungsblöcken aufgrund von technischen Unzulänglichkeiten nicht durchgeführt wurde.

% subsection repräsentation_diagrammatischer_modelle (end)

\subsection{Kooperatives Arbeiten} % (fold)
\label{sub:kollaboratives_arbeiten}

Gegenstand der hier beschriebenen Untersuchung ist Hypothese \ref{hyp:kollaborativ} („Das Werkzeug ermöglicht kooperatives Arbeiten an einer Aufgabe.“). Zur Untersuchung der quantitativ beurteilbaren Aspekte wurden die Werkzeuganwendungen aus den Evaluierungsblöcken 2 ($n=9$) und 3 ($n=18$) herangezogen, wobei in Block 2 in Gruppen zu zwei Personen modelliert wurde (in einem Fall drei Personen), in Block 3 in Gruppen zu drei Personen (in drei Fällen nur zwei Personen). Zusätzlich wurden zur qualitative Beurteilung Daten aus Block 4 verwendet.

In Evaluierungsblock 4 wurde hinsichtlich der subjektiven Wahrnehmung der Kooperation eine Befragung der Teilnehmer mittels eines Fragebogens durchgeführt (diese umfasste auch weitere Aspekte, die in späteren Abschnitten besprochen werden). Die Fragestellungen zur Kooperation wurde in 4 geschlossenen Items codiert, die auf einer 7-teiligen Likert-Skala zu beantworten waren. Zusätzlich wurden offene Fragen hinsichtlich der Nützlichkeit der Werkzeugs eingesetzt, die an dieser Stelle ebenfalls hinsichtlich Aussagen zur Kooperation zwischen den Teilnehmern ausgewertet werden.

\subsubsection{Auswertung} % (fold)

Grundlage des ersten Teils der Auswertung ist die Verteilung der Modellierungsdauer zwischen den Teilnehmern. Um die unterschiedliche Gesamt-Modellierungsdauer in den einzelnen Anwendungen zu kompensieren, wurden die Berechnungen auf Basis der prozentuellen Zeitanteile der einzelnen Teilnehmer durchgeführt. Die einzelnen Datensätze wurden so sortiert, dass die anteilsmäßige Modellierungsdauer von Teilnehmer A bis Teilnehmer C (bzw. B) abnimmt. In den einzelnen Evaluierungsblöcken ergeben sich die in Abbildung \ref{fig:img_Evaluierung_timeDist} dargestellten Verteilungen.

\begin{figure}[htbp]
	\centering
		\includegraphics[height=2.5in]{img/Evaluierung/timeDistSE1.png}
		\includegraphics[height=2.5in]{img/Evaluierung/timeDistSE2.png}
		\includegraphics[height=2.5in]{img/Evaluierung/timeDistUE.png}
	\caption{Zeitverteilung zwischen den Teilnehmern}
	\label{fig:img_Evaluierung_timeDist}
\end{figure}

\todo Zu prüfen ist hier, ob die Zeit-Anteile der einzelnen Teilnehmer signifikant unterschiedlich sind. Dazu wird für jeden Block die Signifikanz zwischen den Verteilung der einzelnen Teilnehmerklassen berechnet (eine Teilnehmerklasse setzt sich aus all jenen Teilnehmern zusammen, die am längsten, am zweitlängsten bzw. am drittlängsten aktiv waren).  Aufgrund der geringen Stichprobengröße kommt zur Prüfung der Signifikanz der t-Test nicht in Frage, es wird der \emph{Wilcoxon-Test} herangezogen. Der t-Test setzt außerdem Normalverteilung der Prüfgrößen voraus, was zumindest bei einer der Verteilungen nicht der Fall ist (Sharpiro-Wilk-Test für $conn_{B22}$: $p=6.29e^{-5}$, damit ist von Nicht-Normalverteilung auszugehen).

Im zweiten Teil der Auswertung wurde in einem Fragebogen in 4 Items aggregiert die Frage nach kooperativen Aspekten bei der Modellbildung gestellt. Diese wurden im Schnitt als sehr hoch oder hoch beurteilt ($M = 1.79$, $SD = 0.56$, $t4(13) = -14.28$, $p<.001$). Dieses Ergebnis steht in Übereinstimmung mit den qualitativen Aussagen der Benutzer, von denen 10 explizit auf die kooperationsfördende Wirkung des Werkzeugs hinwiesen. Auch in Auswertungen der  Videoaufnahmen der betreffenden Modellierungsdurchgänge ist zu erkennen, dass zwischen $40$ und $70\%$ der gesamten Modellierungsdauer der Interaktion zwischen den Teilnehmern zuzurechnen ist.

\subsubsection{Diskussion} % (fold)

\subsubsection{Ergebnis} % (fold)


% subsection kollaboratives_arbeiten (end)

\subsection{Einsetzbarkeit in unterschiedlichen Kontexten} % (fold)
\label{sub:einsetzbarkeit_in_unterschiedlichen_kontexten}

Gegenstand der hier beschriebenen Untersuchung Hypothese \ref{hyp:kontexte} („Das Werkzeug ist gleichwertig für Modellierungsaufgaben in unterschiedlichen Kontexten einsetzbar.“). Als Grundlage dieser Untersuchung dienen die Ergebnisse der Evaluierungsblöcke 2 und 4.

\subsubsection{Auswertung} 

\subsubsection{Diskussion} 

\subsubsection{Ergebnis} 

% subsection einsetzbarkeit_in_unterschiedlichen_kontexten (end)

\subsection{Wiederherstellung vergangener Modellzustände} % (fold)
\label{sub:wiederherstellung_vergangener_modellzustände}

Gegenstand der hier beschriebenen Untersuchung ist Hypothese \ref{hyp:wiederherstellung} („Die Möglichkeit der Wiederherstellung vergangener Modellzustände fördert die Bereitschaft alternative Repräsentationen auszuprobieren.“). Als Grundlage dieser Untersuchung dienen die Ergebnisse der Evaluierungsblöcke 2 bis 5, da die Funktion zur Wiederherstellung vergangener Modellzustände erst in diesen Blöcken funktionsfähig zur Verfügung stand.

\subsubsection{Auswertung} 

Für alle Anwendungen des Werkzeugs in den Evaluierungsblöcken 2 bis 5 wurde hier untersucht, wie oft die Möglichkeit zur Wiederherstellung vergangener Modellzustände eingesetzt wurde, um alternative Modellierungswege auszuprobieren. Nicht berücksichtigt wurden Einsätze derselben Funktion, die zur Korrektur von Modellierungsfehlern durch Fehlerkennungen des Systems verwenden wurden (verstärkt in den Evaluierungsblöcken 2 und 3 aufgetreten, in 4 und 5 durch Stabilisierung der Erkennungsleistung nicht mehr relevant). Die Verteilung des Einsatzes der Funkion ist in absoluten Zahlen in Tabelle \ref{tab:anzahl_wiederherstellung} für jeden Evaluierungsblock angeführt

\begin{table}[htbp]
	\centering
	\caption{Anzahl des Einsatzes der Wiederherstellungsfunkion}
\begin{tabular}{| c || c | c | c | c || c | c | c | c |}
  \hline
   EB    & 0 E. & 1 E. & 2 E. & 3+ E. \\ \hline
   2     & 18 & 0 & 0 & 0 \\ 
   3     & 14 & 4 & 0 & 0 \\ 
   4     & 10 & 0 & 0 & 0 \\ 
   5     & 10 & 1 & 0 & 0 \\ \hline
   Ges.  & 52 & 5 & 0 & 0 \\ \hline
\end{tabular} \\
\footnotesize EB \ldots Evaluierungsblock, x E.\ldots x Einsätze der Wiederherstellungsfunktion
	\label{tab:anzahl_wiederherstellung}
\end{table}

Die Wiederherstellungsfunktion wurde also insgesamt in $8.77\%$ der Fälle ($n=57$) eingesetzt und kam maximal einmal je Anwendung zum Einsatz.  Aus den Videoanalysen ist außerdem erkennbar, dass die Wiederherstellungsfunktion -- falls ihre Verwendung überhaupt in Betracht gezogen wird -- in den meisten Fällen lediglich zur Fehlerkorrektur eingesetzt wird. (in 52 Anwendungen wurde die Wiederherstellungsfunkion in 37 Fällen -- $71.2\%$ -- mindestens einmal zur Korrektur von Erkennungsfehlern und 5 mal zur Korrektur von inhaltlich verworfenen Modellierungswegen verwendet).

Bei der in den Blöcken 1, 4 und 5 durchgeführten Befragung der Teilnehmer hinsichtlich der Erfahrungen mit dem Werkzeug wurde unter anderem nach als besonders nützlich empfundenen Funktionen bzw. Eigenschaften des Werkzeugs gefragt. Die Wiederherstellungsfunktion wurde in diesem Zusammenhang von keinem Teilnehmer ($n=55$) erwähnt. 

\subsubsection{Diskussion} 

Die Ergebnisse der Auswertung der Untersuchung zu dieser Hypothese zeigt ein geringes Ausmaß der Verwendung der Wiederherstellungsfunktion zum Zwecke der Erstellung von Modellalternativen. Die Funktion wurde in $71.2\%$ der Anwendungen verwendet, was für ein hohes Bewusstsein über deren Existenz spricht. Lediglich in $8.77\%$ der Anwendungen wurde die Funktion zur Verfolgung alternativer Modellierungswege eingesetzt, in $61.5\%$ der Anwendungen wurde sie lediglich zur Fehlerkorrektur verwendet. Auch in der qualitativen Erhebung der als nützlich wahrgenommenen Werkzeugfunktionalitäten wurde die Wiederherstellungsfunktion in keinem Fall genannt. Auf Basis dieser Ergebnisse kann die Hypothese nicht bestätigt werden. 

\subsubsection{Ergebnis} 

\textbf{Hypothese \ref{hyp:wiederherstellung} kann auf Basis der Untersuchung nicht bestätigt werden.} Die Wiederherstellungsfunktion wird nur in unter $10\%$ der untersuchten Anwendungen  zur Verfolgung alternativer Modellierungswege genutzt. Die Funktion wird außerdem von den Anwendern bei der Frage nach den als nützlich wahrgenommene Funktionen nicht genannt.

% subsection wiederherstellung_vergangener_modellzustände (end)

\subsection{Nicht-Behinderung} % (fold)
\label{sub:nicht_behinderung}

Gegenstand der hier beschriebenen Untersuchung ist Hypothese \ref{hyp:behinderung} („Das Werkzeug behindert die Modellbildung nicht.“). Als Grundlage dieser Untersuchung dienen die Ergebnisse der Evaluierungsblöcke 2 bis 5, da sich das Werkzeug erst in diesen Blöcken hinsichtlich der Funktionalität in vollständigem Zustand befand. Zu berücksichtigen ist bei der Auswertung, dass im Laufe der Evaluierungsblöcken 4 und 5 eine Überarbeitung der Implementierung vorgenommen wurde, mittels der das Auftreten von Fehlerkennungen verringert werden konnte und deren Korrektur weniger aufwändig wurde. Befragungen der Modellierenden hinsichtlich einer etwaigen Behinderung durch das Werkzeug wurden in den Blöcken 1, 4 und 5 durchgeführt, wobei lediglich die Anmerkungen aus den letzen beiden Blöcken für den aktuellen Entwicklungsstand des Werkzeugs relevant sind.

\subsubsection{Auswertung} 

In Tabelle \ref{tab:fehlfunktionen} wird gegliedert nach Evaluierungsblocken dargestellt, wie oft es in einer einzelnen Anwendung zu Fehlfunktionen in der Erkennung kam, die den Modellierungsfluss unterbrachen. Als Fehlerkennungen wurde das Verschwinden von Blöcken oder Fehlzuordnungen von Benennungen sowie die unbeabsichtigte oder von System eigenständig vorgenommene Erstellung oder Entfernung von Verbindern bzw. Richtungspfeilen eingeordnet. Zusätzlich wurden Systemabstürze als massive Unterbrechung, die zum Gesamtverlust des bis zum Zeitpunkt des Absturzes erstellten Modells führten, separat ausgewertet.

\begin{table}[htbp]
	\centering
	\caption{Fehlfunktionen und Abstürze des Werkzeugs}
\begin{tabular}{| c || c || c | c | c | c || c |}
  \hline
   EB    & Anw. & 0 Ff. & 1-3 Ff. & 4-6 Ff. & 7+ Ff. & Systemabstürze \\ \hline
   2     & 18 & 0 &  8 &  5 &  5 &  4 \\ 
   3     & 18 & 1 & 10 &  4 &  3 &  5 \\ 
   4     & 10 & 0 &  2 &  2 &  5 &  1 \\ 
   5     & 11 & 0 &  3 &  3 &  4 &  5 \\ \hline
   Ges.  & 57 & 1 & 23 & 14 & 17 & 15 \\ \hline
\end{tabular} \\
\footnotesize EB \ldots Evaluierungsblock, Anw. \ldots Anzahl der Anwendungen, x Ff.\ldots x Fehlfunktionen
	\label{tab:fehlfunktionen}
\end{table}

In der Gesamtheit der betrachteten Anwendungen ($n=57$) ergibt sich folgende Verteilung der Anzahl der Fehlerkennungen je Anwendung, die auch in Abbildung \ref{fig:img_Evaluierung_fehlerkennungen} graphisch dargestellt ist. In $1.75\%$ der Fälle ($n_{0}=1$) trat keine Fehlerkennung während der Anwendung auf. In $40.35\%$ der Fälle ($n_{1-3}=23$) traten zwischen 1 und 3 Fehlerkennungen auf. 4-6 Fehlerkennungen konnten in $24.56\%$ der Fälle ($n_{4-6}=14$) festgestellt werden. 7 oder mehr Fehlerkennungen traten in $29.82\%$ der Fälle ($n_{7+}=17$) auf. In $26.32\%$ der Fälle ($n_{Absturz}=15$) kam es zu Systemabstürzen, wobei diese in 10 Fällen nach Ende des eigentlichen Modellierungsvorgangs auftraten.

\begin{figure}[htbp]
	\centering
		\includegraphics[width=10cm]{img/Evaluierung/fehlerkennungen.png}
	\caption{Verteilung der Anzahl der Fehlerkennungen je Anwendung -- Übersicht}
	\label{fig:img_Evaluierung_fehlerkennungen}
\end{figure}

\todo qualitative Daten

\subsubsection{Diskussion} 

\todo Die Daten der quantitativen Auswertung zeigen, dass es in annähernd allen betrachteten Fällen zu zumindest einer Fehlerkennung kam. Es ist davon auszugehen, dass jede Fehlerkennung den Modellierungsfluss unterbricht, da das dann inkorrekte Modelle korrigiert werden muss. Insofern 

\todo Bestätigung und Relativierung durch die Überarbeitung der Interaktion durch die qualitativen Ergebnisse.

Der hohe Anteil von Systemabstürzen ist insofern zu relativieren, als dass diese in zwei Drittel der Fälle nach Abschluss der eigentlichen Modellierungstätigkeit auftraten und somit die Modellerstellung selbst nicht mehr unterbrachen. Abstürze traten durchgängig vor allem in langen Modellierungsdurchgängen etwa ab Minute 40 auf, da ab diesem Zeitpunkt der Speicherbedarf der Historie tendenziell an die Grenzen des verfügbaren Arbeitsspeichers stößt. Alternativ kam es an Tagen mit starker Modellierungstätigkeit ab etwa 5 Stunden durchgängiger Betriebsdauer zu Überhitzungen des Rechners, auf dem die Software ausgeführt wurde, was zum Gesamtabsturz des Betriebssystems führte. Lediglich in 5 Fällen war der Absturz auf fehlerhaftes Programmverhalten (abgesehen von der Speicherproblematik) zurückzuführen. Diese Fälle traten in den Evaluierungsblöcken 2 und 3 auf. Trotzdem sind auch Systemabstürze in der Endphase der Anwendung nach der Modellierung durch den auftretenden Datenverlust nicht akzeptabel und sprechen somit gegen die Annahme der Hypothese.

Insgesamt kann die hier geprüfte Hypothese aus den angeführten Gründen nicht bestätigt werden.

\subsubsection{Ergebnis} 

\textbf{Hypothese \ref{hyp:behinderung} kann auf Basis der Untersuchung nicht bestätigt werden.}


% subsection nicht_behinderung (end)

\subsection{Gewöhnung an das Werkzeug} % (fold)
\label{sub:gewöhnung_an_das_werkzeug}

Gegenstand der hier beschriebenen Untersuchung ist Hypothese \ref{hyp:gewöhnung} („Wiederholte Verwendung des Werkzeugs führt zu schnellerer Modellbildung und weniger Fehlbedienungen.“). Als Grundlage dieser Untersuchung dienen die Ergebnisse des Evaluierungsblocks 2, da in diesem für jede Teilnehmerzusammenstellung jeweils zwei Anwendungen des Werkzeugs durchgeführt wurden.

\subsubsection{Auswertung} 

Zur Auswertung der Modellierungsgeschwindigkeit (hinsichlich des Hypothesenteils „schnellere Modellbildung“) wurde die reine Modellierungszeit jeder Anwendung (ohne Diskussionszeit) mit der jeweiligen Modellgröße normiert. In Tabelle 	\ref{tab:normierte_zeiten} sind die Anwendungszeiten und Modellgrößen sowie die daraus errechneten normierten Werte für beide Anwendungen der Gruppen in Evaluierungsblock 2 angegeben. 

\begin{table}[htbp]
	\centering
	\caption{Modellierungszeiten in Abhängigkeit der Modellgröße in Evaluierungsblock 2}
\begin{tabular}{| c || c | c | c || c | c | c |}
  \hline
   Gruppe    & $t_{1}$ & $n_{1}$ & $t'_{1}$ & $t_{2}$ & $n_{2}$ & $t'_{2}$ \\ \hline
   1     & 620 & 20 & 31.0 & 300 &  6 & 50.0 \\ 
   2     & 450 & 13 & 34.6 & 420 &  7 & 60.0 \\ 
   3     & 240 & 12 & 20.0 & 285 &  6 & 47.5 \\ 
   4     & 215 & 10 & 21.5 & 420 & 13 & 32.3 \\ 
   5     & 577 & 12 & 48.1 & 270 &  7 & 38.6 \\ 
   6     & 339 & 10 & 33.9 & 330 &  8 & 41.3 \\ 
   7     & 348 &  8 & 43.5 & 110 &  5 & 22.0 \\ 
   8     & 855 & 16 & 53.4 & 510 & 12 & 42.5 \\ 
   9     & 735 &  8 & 91.9 & 195 & 12 & 16.3 \\ \hline
\end{tabular} \\
\footnotesize $t_{x}$ \ldots Modellierungsdauer in Sekunden, $n_{x}$ \ldots Anzahl der Elemente, $t'_{1}$ \ldots normierte Modellierungdauer in Sekunden
	\label{tab:normierte_zeiten}
\end{table}

Zusammenfassend ist zwischen der ersten Anwendung (normierte Modellierungsdauer: $M=42.0, SD=21.8, n=9$) und der zweiten Anwendung (normierte Modellierungsdauer: $M=38.9, SD=13.7, n=9$) keine signifikante Verringerung der normierten Modellierungsdauer zu erkennen (einseitiger Wilcoxon-Test für gepaarte Stichproben: $V=21, p=0.590$\footnote{Aufgrund der beiden kleinen Stichproben und der Nicht-Normalverteilung der beiden Stichproben (Shapiro-Wilk-Test 1. Anwendungsdurchgang: $W=0.853, p=0.081$, 2. Anwendungsdurchgang: $W=0.972, p=0.910$) kann der t-Test ($t=0.286, df=8, p=0.391$) trotz der gleichen Varianz der Stichproben (F-Test: $F=2.53, p=0.211$) nicht angewandt werden.}).

Die Anzahl der Fehlbedienungen ist die Anwendungen in beiden Modellierungsdurchgängen in Evalierungsblock 2 in Tabelle \ref{tab:fehlbedienungen} angegeben. Als Fehlbedienungen wurden all jene Interaktionen mit dem Werkzeug eingestuft, in denen die Bedienung nicht dem intendierten Interaktionsdesign folgte. Fehlfunktionen des Werkzeugs wurden nicht berücksichtigt.

\begin{table}[htbp]
	\centering
	\caption{Anzahl der Fehlbedienungen in Evaluierungsblock 2}
\begin{tabular}{| c || c | c |}
  \hline
   Gruppe    & $FB_{1}$ & $FB_{2}$ \\ \hline
   1     & 1 & 0 \\ 
   2     & 4 & 1 \\ 
   3     & 2 & 1 \\ 
   4     & 0 & 0 \\ 
   5     & 0 & 0 \\ 
   6     & 6 & 1 \\ 
   7     & 3 & 1 \\ 
   8     & 6 & 1 \\ 
   9     & 4 & 2 \\ \hline
\end{tabular} \\
\footnotesize $FB_{x}$ \ldots Anzahl der Fehlbedienungen
	\label{tab:fehlbedienungen}
\end{table}

Zusammenfassend konnte hier gezeigt werden, dass die Anzahl der Fehlbedienungen zwischen Anwendung 1 ($M=2.89, SD=2.32, n=9$) und Anwendung 2 ($M=0.78, SD=0.67, n=9$) signifikant geringer geworden ist (einseitiger Wilcoxon-Test für gepaarte Stichproben: $V=28, p=0.0109$\footnote{Aufgrund der beiden kleinen Stichproben und der Nicht-Normalverteilung der ersten Stichprobe (Shapiro-Wilk-Test 1. Anwendungsdurchgang: $W=0.9144, p=0.348$, 2. Anwendungsdurchgang: $W=0.813, p=0.0284$)   sowie der unterschiedlichen Varianz der Stichproben (F-Test: $F=12.06, p=0.00199$) kann der t-Test ($t=3.333, df=8, p=0.00517$) nicht angewandt werden.}).

\subsubsection{Diskussion} 

Eine signifikante Beschleunigung der Modellierungsgeschwindigkeit konnte in obiger Untersuchung nicht festgestellt werden. Die mit der Modellgröße normierte Modellierungszeit verringerte sich zwischen den beiden Anwendungen im Schnitt nur geringfügig. Dieses Ergebnis kann somit nicht als Indikator für die Bestätigung der Hypothese gesehen werden. In den anderen Evaluierungsblöcken (3, 4 und 5) liegt die durchschnittliche normierte Modellierungsdauer in ähnlichen Bereichen wie in den beiden Durchgängen von Evaluierungsblock 2. Bei Anwendung des Werkzeugs durch den Entwickler selbst ist die normierte Modellierungsdauer hingegen auf ungefähr den halben Wert reduziert. Benutzer ohne tiefgehende und mehrfach wiederholte Anwendungserfahrungen scheinen also keinen signifikant messbaren Beschleunigungseffekt bei der Bedienung des Werkzeugs zu erfahren.

Hingegen ist die Anzahl der Fehlbedienungen in den jeweils zweiten Anwendungen des Werkzeugs im Vergleich zur jeweils ersten Anwendung signifikant gesunken. Dies spricht für die Bestätigung der hier geprüften Hypothese. Betrachtet man die Fehlbedienungen detaillierter, so ist ein Großteil der aufgetretenen Fälle sowohl in der ersten als auch in der zweiten Anwendung auf Verständnisschwierigkeiten bei der Bedienung des Löschtokens (zum Zeitpunkt der Evaluierung noch mit dem zustandsbehafteten Interaktionsdesign implementiert, siehe Abschnitt \ref{sub:verwendung_des_löschtokens}) und der Verwendung der Wiederherstellungsfunktion zurückzuführen. Das Interaktionsdesign beider Aspekte wäre also zu hinterfragen (bzw. wurde im Falle des Löschtokens hinterfragt).

Insgesamt kann die hier untersuchte Hypothese nur zum Teil bestätigt werden, da der vermutete Beschleunigungseffekt nicht nachzuweisen war.

\subsubsection{Ergebnis} 

\textbf{Hypothese \ref{hyp:gewöhnung} kann auf Basis der vorliegenden Daten teilweise bestätigt werden.} Während kein signifikanter Beschleunigungseffekt bei wiederholter Verwendung des Werkzeugs gemessen werden konnte, war eine signifikante Verringerung der Anzahl der Fehlbedienungen des Werkzeugs bei wiederholtem Einsatz feststellbar.

% subsection gewöhnung_an_das_werkzeug (end)

\subsection{Herstellung von Verbindern} % (fold)
\label{sub:herstellung_von_verbindern}

Gegenstand der hier beschriebenen Untersuchung ist Hypothese \ref{hyp:verbinder} („Die Einführung der alternativen Möglichkeit zur Verbindungsherstellung erhöht die Nutzung von Verbindern bei der Modellerstellung.“). Zur Untersuchung herangezogen wurden die Werkzeuganwendungen aus Evaluierungsblock 2 ($n=9$). Dieser wurde gewählt, da in diesem Block alle Teilnehmer das Werkzeug zweimal mit der gleichen Aufgabenstellung anwandten, wobei in der ersten Anwendungsrunde lediglich die ursprüngliche Funktionalität zur Herstellung von Verbindern verfügbar war, in der zweiten Runde aber bereits der alternative Funktionalität implementiert war. Zur weiteren Überprüfung der Ergebnisse werden außerdem die Ergebnisse aus Block 3 ($n=17$) herangezogen, bei dessen Durchführung ebenfalls bereits die alternative Funktionalität verfügbar war.

\subsubsection{Auswertung} % (fold)

Grundlage der Auswertung ist das Modellmerkmal „Connectedness“, worunter hier das Verhältnis zwischen der Anzahl der in einem Modell verwendeten Verbindern und den verwendeten Modellelementen verstanden wird. In den einzelnen Evaluierungsblöcken verteilt sich die Connectedness wie in den Abbildungen \ref{fig:img_Evaluierung_connectednessAushandlung1}, \ref{fig:img_Evaluierung_connectednessAushandlung2} und \ref{fig:img_Evaluierung_connectednessConceptMapping} dargestellt.

\begin{figure}[htbp]
	\centering
		\includegraphics[height=2in]{img/Evaluierung/connectednessAushandlung1.png}
	\caption{Connectedness in Evaluierungsblock 2 - Durchgang 1}
	\label{fig:img_Evaluierung_connectednessAushandlung1}
\end{figure}

\begin{figure}[htbp]
	\centering
		\includegraphics[height=2in]{img/Evaluierung/connectednessAushandlung2.png}
	\caption{Connectedness in Evaluierungsblock 2 - Durchgang 2}
	\label{fig:img_Evaluierung_connectednessAushandlung2}
\end{figure}

\begin{figure}[htbp]
	\centering
		\includegraphics[height=2in]{img/Evaluierung/connectednessConceptMapping.png}
	\caption{Connctedness in Evaluierungsblock 3}
	\label{fig:img_Evaluierung_connectednessConceptMapping}
\end{figure}

Zu prüfen ist, ob die Connectedness in jenem Evaluierungs-Blöcken bzw. -Durchgängen, in denen die alternative Funktionalität zur Verbindungs-Herstellung verfügbar war, signifikant höher ist, als in jenen, in denen dies nicht der Fall war. Berechnet wird die Signifikanz zwischen den Ergebnissen der beiden Durchgänge von Block 2 ($conn_{B21}$ und $conn_{B22}$) sowie zwischen den Ergebnissen ersten Durchgang von Block 2 und den Ergebnissen von Block 3 ($conn_{B3}$). Im zweiten Fall ist zu beachten, dass die Aufgabenstellung nicht identisch war und somit eine potentielle Störvariable wirksam wird. Aufgrund der geringen Stichprobengröße kommt zur Prüfung der Signifikanz der t-Test nicht in Frage, es wird der \emph{Wilcoxon-Test} herangezogen. Der t-Test setzt außerdem Normalverteilung der Prüfgrößen voraus, was zumindest bei einer der Verteilungen nicht der Fall ist (Sharpiro-Wilk-Test für $conn_{B22}$: $p=6.29e^{-5}$, damit ist von Nicht-Normalverteilung auszugehen).

Die Null-Hypothese des Wilcoxon-Tests ist, das die beiden Verteilungen identisch verteilt sind. Entsprechend dem erwarteten Ergebnis (dass die mit der alternativen Funktionalität durchgeführten Anwendungen höhere Connectedness aufweisen) wurde die Alternativ-Hypothese so festgelegt, dass sie angenommen wird, wenn die Verteilung des zweiten Blocks gegenüber dem ersten Block nach rechts verschoben (also wertemäßig höher) ist.

Der Wilcoxon-Test für ungepaarte Stichproben ergibt für $conn_{B21}$ und $conn_{B22}$ und der eben beschriebenen Alternativ-Hypothese $p=0.9854$ -- die Alternativ-Hypothese ist damit anzunehmen, die zweite Verteilung (jene mit Einsatz der alternativen Funktionalität der Verbindungsherstellung) weist eine signifikant höhere Connectedness auf als die erste Verteilung (ohne diese Funktionalität). 

Für $conn_{B21}$ und $conn_{B3}$ ergibt der Wilcoxon-Test für ungepaarte Stichproben mit der gleichen Alternativ-Hypothese $p=0.98$ -- auch hier ist die Alternativ-Hypothese anzunehmen.

Für $conn_{B22}$ und $conn_{B3}$ ergibt der Wilcoxon-Test für ungepaarte Stichproben mit der gleichen Alternativ-Hypothese $p=0.7586$ -- auch hier ist die Alternativ-Hypothese anzunehmen.

\subsubsection{Diskussion} % (fold)

Aufgrund der Ergebnisse der berechneten Signifikanztests ist die Hypothese anzunehmen. Mit der Einführung der alternativen Möglichkeit zur Herstellung von Verbindungen war in den einzelnen Anwendungen des Werkzeugs eine Zunahme der Verwendung von Verbindern zu beobachten. Während die Benutzer bei der ursprünglichen Funktion zur Herstellung von Verbindungen zum Großteil auf diese verzichteten (auch bereits in Evaluierungsblock 1), wurden Verbinder unabhängig von der Aufgabenstellung mit der Einführung der alternativen Funktionalität verstärkt eingesetzt.

Die Connectedness eignet sich als Parameter zur vergleichenden Beurteilung des Ausmaßes der Verwendung von Verbindern, da durch die Einbeziehung der Größe des Modells (repräsentiert durch die Anzahl der verwendeten Modellelemente) in die Berechnung den Wert für unterschiedliche Modelle vergleichbar macht. 

Einfluss auf die Höhe der Connectedness hat aber die Aufgabenstellung, die zur Bildung des Modells führt. Unterschiedliche Modellierungsaufgaben führen zu unterschiedlichen Modell-Topologien, die sich wiederum in der Anzahl der verwendeten Verbinder auswirkt. Dies zeigt sich am Ergebnis des Wilcoxon-Tests für $conn_{B22}$ und $conn_{B3}$ -- in beiden Fällen stand die alternative Möglichkeit zur Verbindungsherstellung zur Verfügung $conn_{B3}$ ist trotzdem signifikant höher als $conn_{B22}$. Die kann darin begründet liegen, dass die Concept-Mapping-Aufgabe aus $conn_{B3}$ eher zu stärker verbundenen Modellen führt als eher zur ablauforientierten Modellen führende Arbeitsabstimmungs-Aufgabe aus $conn_{B22}$. Während bei Concept Mapping beliebige Konzepte in Beziehung stehen können, stehen Elemente bei ablauf-orientierten Modellen vor allem mit ihren kausalen Vorgängern und Nachfolgern in Beziehung, was die Anzahl der Verbinder einschränkt.

Aufgrund der großen Rolle der Aufgabenstellung ist bei der Überprüfung der Hypothese wichtig, diese Störvariable möglichst auszuschalten. Zur Beurteilung wird deswegen ausschließlich der Wilcoxon-Test zwischen $conn_{B21}$ und $conn_{B22}$ herangezogen, da in diese beiden Verteilungen mit der gleichen Aufgabenstellung und identischer Stichprobe (jedoch in zeitlichem Abstand von ca. einem Monat) zustande gekommen sind (da die Messungen unabhängig voneinander entstanden, wird ein Wilcoxon-Test für ungepaarte Variablen verwendet). Das Resultat des Wilcoxon-Tests spricht stark für die Annahme der Alternativhypothese des Tests und damit für die Annahme von Hypothese \ref{hyp:verbinder}. Zu berücksichtigen ist hier jedoch die geringe Stichprobengröße, die die Aussagekraft des Ergebnisses wieder in Frage stellt.

\subsubsection{Ergebnis} % (fold)

Die Auswertung zeigt eine signifikant höhere Verwendung von Verbindern bei Verfügbarkeit der alternativen Funktionalität zur Verbindungs-Herstellung. Auch die Natur der Aufgabenstellung scheint hohen Einfluss auf die Verwendung von Verbindern zu haben (siehe dazu auch die Diskussion von Hypothese \ref{hyp:keine_verbinder} in Abschnitt \ref{sub:abbildung_von_zusammenhängen_ohne_verbinder}). \textbf{Hypothese \ref{hyp:verbinder} kann auf Basis der vorliegenden Daten bestätigt werden.}

% subsection herstellung_von_verbindern (end)

\subsection{Verwendung des Löschtokens} % (fold)
\label{sub:verwendung_des_löschtokens}

In diesem Abschnitt werden die Ergebnisse der Überprüfung der Hypothese \ref{hyp:radierer} („Das Löschtoken ermöglicht intuitives Löschen von Modellelementen.“) vorgestellt.

\subsubsection{Auswertung} % (fold)

\begin{transkript}
	\emph{Die Teilnehmer möchten einen Block umbenennen.}\\
	\textbf{A:} Wie haben wir jetzt gesagt \emph{(markiert den roten Baustein)} keine Modellierungsvorgabe \emph{(gibt Bezeichnung ein)}\\
	\emph{System übernimmt die neue Beschriftung für den Baustein nicht.}\\
	\textbf{A:} Wo wurde das hingeschrieben? \emph{(Pause)} Radiergummi? Glaubst du kann man das wegradieren?\\
	\textbf{B:} Probiere es aus.\\
	\textbf{\emph{A legt Radiergummi zum Block mit der Absicht die Beschriftung zu löschen}}\\
	\textbf{B:} Nein! Du löscht alles. Hör auf! \\
	\textbf{A:} Ok wie war das zuerst? Lassen wir das mal weg. \emph{(legt Baustein zur Seite)}\\
	\emph{A legt den Block zur Seite.} 
\end{transkript}

Ein ähnliches Missverständnis zeigt sich auch in folgender Situation:

\begin{transkript}
	\emph{TLN A und B stellen jeweils ihren Marker zu den Blöcken, die verbunden werden sollen. Dabei wird eine gerichtete Verbindung erstellt.}\\
	\textbf{C:} Jetzt haben wir aber einen Pfeil gebastelt.\\
	\textbf{B:} Ja stimmt. Interessant.\\
	\textbf{A:} Wie war das mit dem Radiergummi. \emph{(nimmt Radiergummi und legt ihn auf die Verbindung)}\\
	\textbf{B:} Nein\\
	\textbf{C:} Nein, mit dem Glas! Du löscht alles!\\
	\textbf{A:} Nein nur die Verbindung. \textbf{\emph{(Macht Radierbewegungen auf der Verbindung)}}\\
	\textbf{C:} Ich glaube dass wir das Glas nehmen müssen.\\
	\emph{A schiebt die Blöcke zwischen denen die Verbindung gelöscht werden soll zusammen.}\\
	\textbf{A:} Da es funktioniert. \emph{(schiebt die Blöcke weiter auseinander und bemerkt dass die Verbindung nicht gelöscht wurde)} Nein.\\
	\textbf{B:} Ich glaube der Radiergummi vernichtet alles.\\
	\textbf{A:} Nein der Radiergummi vernichtet nur Verbindungen. Nur welche? \emph{(schiebt beide Blöcke wieder zusammen – nimmt Radiergummi weg und schiebt Blöcke in die Ausgangsposition)}
\end{transkript}

\begin{transkript}
	\emph{Es wird eine falsche Beschriftung eingefügt. Die Teilnehmer wollen diese löschen, verwenden den Radiergummi allerdings falsch.}\\
	\textbf{B:} Aber irgendwie steht jetzt Ereignisse nicht bei dem Ding \emph{(zeigt auf gelben Block)} sondern dort \emph{(zeigt auf beschriftete Verbindung)}.\\
	\emph{A verrückt den gelben Block ein wenig.}\\
	\textbf{B:} Normal ist das nicht oder?\\
	\textbf{C:} Nein.\\
	\emph{A nimmt den Radiergummi.}\\
	\textbf{A:} Ich glaube das. \emph{(setzt den Radiergummi auf die Arbeitsfläche)}\\
	\textbf{C:} Aber nicht alles!\\
	\emph{A nimmt Radiergummi wieder weg. System erstellt eine Verbindung zwischen zwei roten Blöcken. Teilnehmer lachen. \textbf{A legt Radiergummi auf die erstellte Verbindung, und nimmt ihn wieder weg.} A nimmt die beiden verbundenen Blöcke und verschiebt sie.}\\
	\textbf{A:} Vielleicht so. \emph{(führt die Blöcke zusammen)}
\end{transkript}

\begin{transkript}
	\emph{In der Szene erstellt das System einen ungewollten Verbinder, die Teilnehmer versuchen auf verschiedene Arten den Verbinder zu löschen.}\\
	\textbf{B:} Und wie kann ich die Verbindungen löschen?\\
	\textbf{B:} Warte einmal, da gibt es irgendwo das mit dem Radiergummi.\\
	\textbf{A:} murmelt zustimmend \\
	\emph{\textbf{B nimmt den Radiergummi und platziert ihn direkt auf dem Verbinder}}\\
	\emph{Das System färbt den Tisch rot}\\
	\textbf{A:} Nein, warte. Da löscht du Alles!\\
	\emph{\textbf{B verschiebt den Radiergummi auf dem Tisch, hebt ihn an und platziert ihn direkt auf einem Block.}}\\
	\emph{Sobald der Radiergummi von der Oberfläche auf den Block gelegt wurde, entfernt das System die rote Färbung.}\\
	\textbf{A:} Ich glaube da löscht du Alles.\\
	\emph{B legt den Radiergummi an mehreren Stellen trotz der Warnung von TN A auf die Oberfläche}\\
	\textbf{B:} Nein, es will eh nicht.\\
\end{transkript}

\begin{transkript}
	\emph{C versucht die Benennung eines Verbinders mittels Radiergummi zu entfernen.}\\
	\textbf{B:} Aber irgendwie steht jetzt Ereignisse nicht bei dem Ding \emph{(deutet auf einen Block)} sondern dort \emph{(deutet auf einen Verbinder)}. Das wollen wir nicht oder?\\
	\textbf{A:} Nein.\\
	\textbf{C:} Ich glaube das. \emph{\textbf{(nimmt den Radiergummi und legt ihn auf den Verbinder den die Teilnehmer entfernen wollen.)}}\\
	\textbf{A:} Aber nicht alles.\\
	\emph{C entfernt den Radiergummi wieder von der Modellierungsoberfläche. In diesem Moment erstellt das System automatisch einen neuen Verbinder. C versucht den neuen Verbinder mittels Radiergummi zu entfernen.}\\
	\textbf{A:} Oh Gott.\\
	\textbf{C:} Vielleicht so \emph{(schiebt die beiden betroffenen Blöcke zusammen)}, nein.\\
	\textbf{B:} Nein.\\
	\textbf{A:} Oh Gott oh Gott oh Gott.\\
	\textbf{B:} Gehen wir einen Prozessschritt zurück.\\
	\textbf{C:} Genau.\\
\end{transkript}

\begin{transkript}
	\emph{Teilnehmer versuchen mit dem Radiergummi und nur einem anderen Marker einen Verbinder zu entfernen.}\\
	\textbf{B:} Können wir die nicht so auch einfach löschen?\\
	\textbf{C:} Ja mit dem Radiergummi.\\
	\textbf{B:} Muss ich den jetzt zuerst so \emph{(Hält den Radiergummi zur Kamera)} hinhalten?\\
	\textbf{A:} Nein, ich glaube, \textbf{den musst du einfach da \emph{(zeigt auf den Verbinder)} drauf legen.}\\
	\emph{B legt den Radiergummi auf den vom System automatisch erstellten Verbinder.}\\
	\textbf{A:} Und jetzt muss man \emph{(legt ein Markierungtoken auf den Verbinder)} Nein.\\
	\emph{Der Verbinder lässt sich auf diese Art nicht löschen und die Teilnehmer entscheiden sich den Fehler mittels der Wiederherstellungsfunktion zu beseitigen.}
\end{transkript}

\subsubsection{Diskussion} % (fold)

\subsubsection{Ergebnis} % (fold)

% subsection verwendung_des_löschtokens (end)
% section ergebnisse (end)

% chapter eval_tui (end) 
\chapter{Evaluierung der erstellten Modelle} % (fold)
\label{cha:eval_modell}

% chapter eval_modell (end)
\chapter{Evaluierung der durchgeführten Articulation Work} % (fold)
\label{cha:eval_aw}

\section{Hypothesen} % (fold)
\label{sec:a_hypothesen}

\subsection{Konzeptuell begründete Hypothesen} % (fold)
\label{sub:a_konzeptuell_begründete_hypothesen}

\begin{hyp}
	Das Werkzeug verbessert den Prozess der Abstimmung zwischen Personen.
\end{hyp}

\begin{hyp}
	Die Anwendung des Werkzeugs verbessert die Ergebnisse kollaborativer Arbeit.
\end{hyp}

% subsection konzeptuell_begründete_hypothesen (end)

\subsection{Explorativ gebildete Hypothesen} % (fold)
\label{sub:a_explorativ_gebildete_hypothesen}

% subsection explorativ_gebildete_hypothesen (end)

% section hypothesen (end)

\section{Untersuchungsdesign und Durchführung} % (fold)
\label{sec:a_untersuchungsdesign}

% section untersuchungsdesign (end)

\section{Ergebnisse} % (fold)
\label{sec:a_ergebnisse}

% section ergebnisse (end)

% chapter eval_aw (end)

%\input{Untersuchungsdesign}
%\chapter{Untersuchungsergebnisse} % (fold)
\label{cha:untersuchungsergebnisse}

\section{Erhobene Daten} % (fold)
\label{sec:erhobene_daten}

\subsection{Phase 1} % (fold)
\label{sub:phase_1}

In Phase 1 wurden 9 Modellierungsdurchgänge mit insgesamt 18 Personen durchgeführt. An dem vorangegangenen Pretest nahmen 12 Personen teil.
% subsection phase_1 (end)

\subsection{Phase 2} % (fold)
\label{sub:phase_2}

In Phase 2 wurden Untersuchungen im Rahmen zweier Lehrveranstaltungen durchgeführt. An der ersten Untersuchung nahmen 18 Studierende der Wirtschaftsinformatik teil, die in Gruppen zu 2 Personen insgesamt 17 Modellierungsdurchgänge durchführten. An der zweiten Untersuchung nahmen 54 Studierende in Gruppen zu 3 Personen an insgesamt 18 Modellierungsdurchgängen teil.
% subsection phase_2 (end)

\subsection{Phase 3} % (fold)
\label{sub:phase_3}

% subsection phase_3 (end)
% section erhobene_daten (end)

\section{Auswertung \& Interpretation} % (fold)
\label{sec:auswertung_&_interpretation}

% section auswertung_&_interpretation (end)
% chapter untersuchungsergebnisse (end)


% part evaluierung (end)


\part*{}

\chapter{Schlussbetrachtungen} % (fold)
\label{cha:schlussbetrachtungen}

\begin{figure}[htbp]
	\centering
		\includegraphics[width=10cm]{img/Schlussbetrachtungen/ArbeitInteraktionMentaleModelleTabletop.png}
	\caption{Gesamtzusammenhang der in dieser Arbeit verwendeten Konzepte}
	\label{fig:img_Schlussbetrachtungen_ArbeitInteraktionMentaleModelleTabletop}
\end{figure}

Hypothesen zeigen: Geeignet für Articulation Work, eher nicht geeignet für detaillierte Modellbildungen, technisch keine Probleme mehr.

Schlussfolgerung: Werkzeug unterstützt den Prozess der kommunikativen Abstimmung von mentalen Modellen, nicht aber die vollständige Externalisierung derselben.

\section{Anwendungsszenarien} % (fold)
\label{sec:anwendungsszenarien}

\subsection{Problembeschreibung und Arbeitsabstimmung}

Einzel- oder Gruppensessions. 

Aufgabenstellung: meist aus Arbeitsabläufen der beteiligten Personen

Merkmal: Tisch ist Mittel zum Zweck, gelegtes Modell fungiert als Diskussionsgrundlage. Modell ist statisch, wird einmal gelegt und nicht mehr verändert. Eher kompakte Modelle, die den Kontext eines Problems beschreiben. Die eigentliche Problematik ist selten explizit im Modell dargestellt.

Anwendungsbeispiele: Block 1, Block 2, Block 4 (Session 1-3).

Vorteile:

\subsection{Concept Mapping}

Einzel- oder Gruppensessions. 

Aufgabenstellung: zur Erhebung bzw. Überprüfung von domänenspezifischen (Struktur-)Wissen

Merkmal: Tisch 

Anwendungsbeispiele: Block 3, Block 5

\subsection{Strukturaufstellung und Manipulation}

Anwendungsbeispiele: Block 4 (Session 4).

AUfgabenstellung: Erhebung und Reflexion der Strukturen in denen Arbeitsabläufe situiert sind (Abteilungen, Personen, Kommunikationskanäle).

Merkmal: Modelle sind nicht statisch - werden nach der Erstellung zwar meist nicht erweitert aber in ihrer Struktur verändert (räumliche Relation der Knoten zueinander). 

% chapter schlussbetrachtungen (end)

\appendix
\pagenumbering{Roman}

\part*{Anhang}
\addcontentsline{toc}{part}{Anhänge}

% final draft

\chapter{Literatur zum Themengebiet Articulation Work} % (fold)
\label{cha:literatur_zum_themengebiet_articulation_work}

Dieser Anhang stellt die Literatur zu „Articulation Work“ umfassend dar. Er dient als Ergänzung zu den in Kapitel \ref{cha:articulation_work} eingeführten Konzepten und Inhalten. Insbesondere wurden die hier beschriebenen Arbeiten hinsichtlich ihrer Relevanz für die Begriffsbestimmung zu „Articulation Work“ und deren Unterstützung beurteilt. Die als relevant identifizierten Arbeiten wurden in den Abschnitten \ref{sec:aw_begriffsbestimmung}, \ref{sec:arten_von_articulation_work} und \ref{sec:unterstützung_von_articulation_work} umfassender dargestellt.

\section{Literaturquellen} % (fold)
\label{sec:literaturquellen}

In der Literatursuche wurden Datenbanken aus den Bereichen Informatik, Psychologie, Soziologie, den Wirtschaftswissenschaften sowie der Organisationslehre durchsucht. Nach der initialen Suche wurde jeweils auch die in den gefundenen Arbeiten referenzierte Sekundärliteratur aufgearbeitet. Des weiteren wurden mit Hilfe von rückwärts verlinkenden Datenbanken (wo vorhanden) Publikationen erfasst, die auf die bislang gefundenen Arbeiten referenzieren. Die so identifizierten Publikationen wurden ebenfalls hinsichtlich ihrer Relevanz überprüft.

Die in der Suche berücksichtigten Datenbanken bzw. Meta-Suchmaschinen sind:
\begin{description}
	\item[Domänenspezifische Datenbanken]\ 
		\begin{itemize}
			\item INSPEC\footnote{via http://ovidsp.ovid.com} (Naturwissenschaften)
			\item Business Source Premier\footnote{via http://search.ebscohost.com/} (Wirtschaftswissenschaften)
			\item PsycINFO\footnote{via http://ovidsp.ovid.com} (Psychologie)
			\item PSYNDEXplus\footnote{via http://ovidsp.ovid.com} (Psychologie)
			\item SocINDEX\footnote{via http://search.ebscohost.com/} (Soziologie)
			\item ERIC\footnote{http://www.eric.ed.gov/} (Pädagogik)
			\item ACM Guide\footnote{http://portal.acm.org/guide.cfm} (Informatik)
		\end{itemize}
	\item[Verlags-Datenbanken]\ 
		\begin{itemize}
			\item ACM Digital Library\footnote{http://portal.acm.org/dl.cfm} (Informatik)
			\item IEEE XPlore\footnote{http://ieeexplore.ieee.org} (Informatik)
			\item SpringerLink\footnote{http://www.springerlink.de} (fächerübergreifend)
			\item ScienceDirect\footnote{http://www.sciencedirect.com} (fächerübergreifend)
			\item Emerald\footnote{http://www.emeraldinsight.com} (Wirtschaftswissenschaften)
			\item Wiley Interscience\footnote{http://www3.interscience.wiley.com} (fächerübergreifend)
		\end{itemize}
	\item[Meta-Suchmaschinen]\ 
	 	\begin{itemize}
	 		\item Google Scholar\footnote{http://scholar.google.com/} (fächerübergreifend)
	 		\item CiteSeerX\footnote{http://citeseerx.ist.psu.edu/} (Naturwissenschaften und Informatik)
	 	\end{itemize}
\end{description}

% section literaturquellen (end)

\section{Relevante Literatur} % (fold)
\label{sec:relevante_literatur}

Die im Folgenden genannten Arbeiten beziehen sich in unterschiedlicher Weise auf das Themengebiet „Articulation Work“. Es konnten vier Kategorien von Arbeiten identifiziert werden, die sich hinsichtlich ihres inhaltlichen Fokus unterscheiden:
\begin{enumerate}[(I)]
	\item Arbeiten, die sich mit der grundlegenden Konzeption von „Articulation Work“ beschäftigen und keine Aussage zu deren Unterstützung machen.
	\item Arbeiten, in denen „Articulation Work“ als erklärendes Rahmenwerk für beobachtete Phänomene verwendet wird, und in der Folge das Hauptaugenmerk auf diese Phänomene gelegt wird, ohne nochmals näher auf „Articulation Work“ einzugehen.
	\item Arbeiten, die auf die Unterstützung von „Articulation Work“ eingehen.
	\item Arbeiten, in denen „Articulation Work“ lediglich erwähnt wird, allerdings nicht näher darauf Bezug genommen wird.
\end{enumerate}

In chronologischer Reihenfolge des Erscheinens sind die folgenden Arbeiten einer oder mehreren der genannten Kategorien zuzuordnen (Kategorie jeweils in Klammer angeführt):

\begin{description}
	\item[\citet{Strauss85}] (I) prägt in dieser Arbeit den Begriff „Articulation Work“ und beschreibt dieses auf konzeptueller Ebene ohne eine unmittelbaren Praxis- bzw. Umsetzungsbezug herzustellen.
	\item[\citet{Gasser86}] (I) beschreibt die Integration von Computerunterstützung in alltägliche Arbeitsabläufe und die Anpassungsleistung der arbeitenden Individuen, wenn die aktuelle Arbeitssituation nicht mehr mit dem der Computerunterstützung zugrunde liegenden Modell übereinstimmt. Er identifiziert dabei spezifische Aktivitäten, die im Rahmen der ablaufenden „Articulation Work“ auftreten können.
	\item[\citet{Gerson86}] (I, II) zeigen die konkrete Manifestation von „Articulation Work“ in einer Fallstudie aus einem Versicherungskonzern und identifizieren daraus die organisationalen Rahmenbedingungen, die zu jenen Problemen führen, die „Articulation Work“ notwendig machen.
	\item[\citet{Bendifallah87}] (II) untersuchen bezugnehmend auf \citet{Gasser86} „Articulation Work“ im Kontext von IT-Support-Arbeit in Unternehmen anhand von zwei Fallstudien und identifizieren dabei zwei unterschiedliche Strategien bei der Durchführung derselben. Im Detail gehen sie jedoch nicht auf die konkret zu setzenden Maßnahmen ein.
	\item[\citet{Fujimura87}] (I) leitet die grundlegende Unterscheidung zwischen „Production Work“ und „Articulation Work“ anhand einer Fallstudie aus dem wissen\-schaftlich-medizinischen Forschungsbetrieb ab. Sie bleibt dabei auf konzeptueller Ebene und beschreibt die auftretenden Phänomene, geht jedoch nicht auf unterstützende Maßnahmen ein.
	\item[\citet{Strauss88}] (I) detailliert und erweitert seine Konzepte und setzt diese in den Kontext organisationaler Projektarbeit (im dort beschrieben Verständnis im Wesentlichen identisch mit „non-routine collective activity“). Anhand einer Fallstudie aus dem Krankenhaus-Organisations-Bereich zeigt er das Auftreten von „Articulation Work“ in der Praxis, beschäftigt sich jedoch nicht mit möglicherweise unterstützenden Interventionen.
	\item[\citet{Schmidt90}] (I) beschreibt ein Framework für die Analyse kooperativer Arbeit und erwähnt dabei „Articulation Work“ als ein zu berücksichtigendes Konzept. Diese Arbeit bildet die Grundlage für die im Hinblick auf die Unterstützung von „Articulation Work“ relevantere Arbeit von \citet{Schmidt92}.
	\item[\citet{Mi91}] (III) betrachten „Articulation Work“ im Kontext der Softwareentwicklung und argumentieren für deren explizite Berücksichtigung in Software Engineering Prozessen. Sie schlagen einen formalisierten Prozess zur Durchführung von „Articulation Work“ vor und führen einen Satz von regelbasierten Heuristiken zur konkreten Durchführung ein. Sie sind damit die ersten, die sich explizit mit der Unterstützung von „Articulation Work“ beschäftigen.
	\item[\citet{Schmidt92}] (I, III) begründen mit dieser Arbeit eine Entwicklungsrichtung der \gls{CSCW}, die neben der Unterstützung der eigentlichen produktiven Arbeit auch auf die Unterstützung von „Articulation Work“ fokussiert. Sie beschreiben damit erstmals Anforderungen an die technische Unterstützung von „Articulation Work“ und Möglichkeiten zu deren Umsetzung.
	\item[\citet{Bannon93}] (II) zeigen in diesem Sammelwerk die ersten Ergebnisse des COMIC-Projektes \citep{Rodden95} und erwähnen dabei in einzelnen Beiträgen „Articulation Work“ als ein im Bereich der \gls{CSCW} zu berücksichtigendes Konzept.
	\item[\citet{Corbin93}] (I) beschäftigen sich mit der Festlegung von Interaktionsmodalitäten in kooperativer Arbeit durch „Articulation Work“ und detaillieren dabei das Verständnis von expliziter „Articulation Work“, indem sie mögliche Zeitpunkte des Auftretens sowie Schritte bei deren Durchführung nennen.
	\item[\citet{Strauss93}] (I) fasst im Rahmen einer umfassenderen Arbeit zur Entwicklung einer „Theory of Action“ seine Überlegungen zur Rolle und Ausgestaltung von „Articulation Work“ zusammen und würdigt diese kritisch. Konkrete Maßnahmen zur Unterstützung oder Ermöglichung von „Articulation Work“ sind aber auch hier nicht vorhanden.
	\item[\citet{Bowers94}] (III) ist Editor eines COMIC-Deliverables \citep{Rodden95}, in dem zum ersten Mal auf die in \citep{Schmidt96} ausformulierten Anforderungen zur technischen Unterstützung von „Articulation Work“ eingegangen wird.
	\item[\citet{Lenoir94}] (IV) erwähnt „Articulation Work“ (konkret die Arbeit von \citet{Fujimura87}) als Beispiel der Verknüpfung unterschiedlicher wissenschaftlicher Arbeitskontexte, geht aber dann nicht näher auf „Articulation Work“ ein.
	\item[\citet{Schmidt94}] (III) rephrasiert im Wesentlichen \citep{Schmidt90} mit Fokus auf den Aspekt der kooperativen Arbeit (und nicht der Computerunterstützung derselben). Er detailliert darin die Artikulationsnotwendigkeiten bei kooperativer Arbeit, führt jedoch hinsichtlich der Unterstützung von „Articulation Work“ keine zusätzlichen Anforderungen ein.
	\item[\citet{Schmidt95}] (III) basiert wie \citep{Schmidt94} auf \citep{Schmidt90}, leitet jedoch inhaltlich bereits zu der oben im Detail behandelten Arbeit von \citet{Schmidt96} über.
	\item[\citet{Grinter95}] (II) beschreibt die Verwendung von Konfigurations-Management-Systemen zur Koordination von Softwareentwicklungs-Prozessen. Sie bezieht sich dabei am Rand auf „Articulation Work“ (via \citep{Schmidt92}), führt diesen Aspekt aber nicht näher aus. Diese Arbeit bildet jedoch die Grundlage für die hinsichtlich der Unterstützung von „Articulation Work“ relevantere Arbeit derselben Autorin \citep{Grinter96}.
	\item[\citet{Simone95}] (III) konkretisieren die in \citet{Schmidt96} beschriebene Notation zur Spezifikation von Koordinationsmechanismen in \gls{CSCW}-Systemen und bereiten damit den Weg zur technischen Unterstützung von „coordinating predefined work“, die in \citep{Divitini00} umfassend beschrieben ist.
	\item[\citet{Grinter96}] (III) betrachtet die Rolle von „Articulation Work“ im Kontext der Softwareentwicklung und zeigt anhand zweier qualitativer empirischer Studien die Auswirkungen eines computerbasierten Configuration Management Systems bei der kooperativen Erstellung von Software.
	\item[\citet{Schmidt96}] (I, III) entwickeln in ihrer Arbeit ein generisches Vorgehen zur Konzeption von technischer Unterstützung von „Articulation Work“. Aufbauend auf früheren Arbeiten der Autoren (z.B. \citep{Schmidt90} und \citep{Schmidt92}) formulieren die Autoren eine Notation zur Spezifikation von \gls{CSCW}-Systemen, die auf der Unterstützung von „Articulation Work“ aufbauen.
	\item[\citet{Bannon97}] (II) beschreiben die Verwendung von „Common Information Spaces“ im Kontext von \gls{CSCW} und identifizieren die Artikulations-Bedürfnisse, die im Rahmen der Verwendung derselben auftreten können. Die Autoren gehen nicht näher auf die Umsetzung oder Unterstützung dieser konkreten Ausprägungen von „Articulation Work“ ein.
	\item[\citet{Fjuk97}] (I, III) versuchen, die Konzepte von „Articulation Work“ durch eine Abbildung auf die Konzepte der „Activity Theory“ zu konkretisieren. Die Autoren geben dabei neben der Erweiterung der konzeptuellen Grundlagen auch mögliche Ansatzpunkte für die Unterstützung durch rechnerbasierte Werkzeuge an.
	\item[\citet{Fjuk97a}] (II) verwendet die Ansätze von \citet{Strauss93}, um die Interaktion in computerbasierten kooperativen Lernumgebungen (also in \gls{CSCL}-Systemen) zu betrachten. Sie verwendet dabei „Articulation Work“ als Analysedimension (als jener Teil des Arbeitsablaufs, in dem Interaktion zwischen den Lernenden vorrangig auftritt), gehen jedoch nicht näher auf deren Unterstützung ein.
	\item[\citet{Simone97}] (IV) beschreiben ein System zur Generierung von Awareness in kooperativen Anwendungen und erwähnen dabei am Rande „Articulation Work“, als einen Aspekt, bei dessen Unterstützung das System interessant sein könnte.
	\item[\citet{Simone97a}] (III) berichten über den aktuellen Stand der Entwicklung bei der technischen Unterstützung von Koordinationsmechanismen in CSCW-Systemen. Sämtliche hier enthaltenen Ergebnisse werden umfassender in \citep{Divitini00} dargestellt.
	\item[\citet{Kling98}] (II) beschreiben „Articulation Work“ als einen Aspekt, dessen Unterstützung bei der Gestaltung von „human centered (computer) systems“ zu berücksichtigen ist. Sie gehen jedoch nicht unmittelbar auf die mögliche Form der Unterstützung ein.
	\item[\citet{Carstensen99}] (III) führen Aspekte von \citep{Schmidt96} genauer oder aus einem anderen Betrachtungswinkel aus, fügen aber dem Verständnis von „Articulation Work“ bzw. deren Unterstützung keine neuen Aspekte hinzu.
	\item[\citet{Schmidt99}] (III) betonen basierend auf \citep{Schmidt96} den dynamischen Charakter von „Articulation Work“, die in einem Arbeitsablauf je nach Kontext unterschiedliche Ausprägungen annehmen kann. Sie fordern eine Berücksichtigung dieser Dynamik in technischen Werkzeugen zur Unterstützung von „Articulation Work“, fügen aber den Ausführungen von \citep{Schmidt96} keine fundamental neuen Anforderungen hinzu. Die Autoren leiten mit dieser Arbeit über zu der erstmals in \citet{Simone99} vorgestellten technischen Implementierung des in den vorgegangenen Publikationen konzipierten Werkzeugs.
	\item[\citet{Simone99}] (III) stellen als Umsetzung der in \citep{Schmidt96} aufgestellten Forderungen zur Unterstützung von „Articulation Work“ durch \gls{CSCW}-Systeme den „Reconciler“ vor, ein auf auf Java und \gls{CORBA} basierendes Software-Modul, das den globalen Kontext und Zustand eines (digitalen) Arbeitsobjektes bei dessen Bearbeitung durch ein Individuum offenlegt und dadurch die Entwicklung einer gemeinsamen Sicht auf geteilt benutzte Objekte ermöglicht und die Vermeidung von Konflikten unterstützt. Der „Reconciler“ ist damit ein technisches Werkzeug zur Unterstützung von „situated Articulation Work“, die Arbeit detailliert jedoch lediglich die in \citep{Schmidt96} genannten Unterstützungsaspekte um diese technisch implementierbar zu machen.
	\item[\citet{Suchman99}] (II) beschäftigt sich mit „invisible work“ in denen Arbeitsartefakte an die tatsächlichen Erfordernisse des jeweiligen Arbeitskontext angepasst werden („design-for-use“). Sie argumentiert für die Anerkennung (also Sichtbarmachung) dieser Arbeit durch die Entwicklung expliziter Design-Praktiken, geht aber nicht näher auf deren Ausgestaltung ein.
	\item[\citet{Star99}] (III) verfassen die einzige Arbeit, in der Strauss selbst Stellung zur Unterstützung von „Articulation Work“ im Generellen und der Unterstützung durch Computersysteme im Speziellen Stellung nimmt. Die Autoren  würdigen die Argumente und Forderungen aus \citep{Schmidt96} kritisch und argumentieren gegen „Sichtbarkeit von Arbeit um jeden Preis“. „Articulation Work“ bedingt nicht notwendigerweise die vollständige Offenlegung aller Arbeitsaspekte sondern geht immer nur soweit wie für eine Wiederaufnahme bzw. Aufrechterhaltung der produktiven Arbeit notwendig. Als Konsequenz fordern sie \gls{CSCW}-Systeme, die -- zusätzlich zu den von \citet{Schmidt96} formulierten Anforderungen -- die Kontrolle über die Sichtbarkeit der eigenen Arbeit bei den arbeitenden Individuen belassen (und stärken damit die Anforderung, die bereits von \citet{Schmidt92} aufgestellt wurde, von \citet{Schmidt96} jedoch nicht explizit berücksichtigt wurde).
	\item[\citet{Berg00}] (II) beschreiben „Articulation Work“ im Kontext von „order and disorder“ in kooperativen Arbeitssituationen (konkret im medizinischen Sektor). „Articulation Work“ ist dabei eine Ausprägung von „disordered work“, im dem Sinne, dass sie nicht vorgegebenen Regeln gehorcht bzw. zur Anwendung kommt, wenn spezifizierte, routinierte Arbeit („ordered work“) nicht mehr funktioniert. Die Autoren gehen jedoch nicht auf eine mögliche Unterstützung von „Articulation Work“ oder „ordered work“ ein.
	\item[\citet{Divitini00}] (III) stellen ein System zur Unterstützung von etablierter kooperativer Arbeit in Form eines adaptiven Workflow-Systems vor, dessen Verhalten durch die Durchführung von „Articulation Work“ beeinflusst werden kann bzw. die Durchführung derselben unterstützt.
	\item[\citet{Schmidt00}] (III) führen im Kontext von CSCW die bereits in \citep{Schmidt96} entwickelten Konzepte nochmals weiter und zeigen, dass bei „Articulation Work“ die Grenze zwischen der Herstellung von „mutual awareness“ (als Bezeichnung einer ad-hoc durchgeführten Abstimmung) und der Verwendung „coordinative artifacts and protocols“ (als Ausprägung eine Koordination von etablierten Arbeitsprozessen) fließend ist. Sie fordern als Folge, dass eine technische Unterstützung beide Arten von „Articulation Work“ unterstützen muss, detaillieren oder verändern aber die konkreten Anforderungen aus \citep{Schmidt96} nicht weiter.
	\item[\citet{Simone00}] (II) beschreibt die Rolle von „classification schemes“ für \gls{CSCW}, die der Klassifikation von Domänenkonzepten zugrunde liegen. Anhand mehrerer Fallstudien beschreibt die Autorin die lokale, informelle und emergente Bildung von Klassifikations-Schemata in Gruppen. Sie argumentiert letztlich dafür, dass diese Schema-Bildung Teil von „Articulation Work“ ist und unterstützt werden muss, um eventuell auftretende Inkonsistenzen zwischen den Schemata einzelner Gruppen oder Individuen zu vermeiden. Letztlich beschreibt die Autorin, dass das in \citep{Simone99} vorgestellte System diese Anforderung erfüllen kann. 
	\item[\citet{Christensen01}] (II) beschäftigt sich mit „Articulation Work“ in Arbeitssituationen, in die mobil arbeitende Individuen involviert sind und konzentriert sich auf jene Arbeits-Aspekte, die spezifisch für derartige Situationen zusätzlich zu artikulieren sind. Er identifiziert diese Aspekte im Rahmen einer Studie und beschreibt ausschließlich den Status quo ohne konkrete Unterstützung-Maßnahmen anzuführen. Weiterführende Arbeiten zu diesem Ansatz sind nicht publiziert.
	\item[\citet{Fuchs01}] beschreiben die technische Unterstützung von „Articulation Work“ in (verteilten) Gruppen mittels \gls{CSCW}-Technologie. Die Autoren präsentieren ein konkret umgesetztes System, das eine Reihe von Werkzeugen zur Unterstützung von „Articulation Work“ bietet.
	\item[\citet{Raposo01}] (III) stellen ein konzeptuelles Framework vor, das die Koordination von voneinander abhängigen Aufgaben in Gruppen erlauben soll und damit „Articulation Work“ mit dem Ziel „coordination of predefined work“ unterstützen soll. Dabei schlagen die Autoren eine Struktur vor, die es erlaubt, für eine Abhängigkeit zwischen Aufgaben unterschiedliche Koordinationsstrategien festzulegen, die dann kontextabhängig ausgewählt werden können. Das Framework wird in \citep{Raposo02} weiter konkretisiert und dessen Umsetzung in einem technischen System beschrieben.
	\item[\citet{Simone01}] (II, III) entwickeln die Ansätze hinsichtlich der Unterstützung der Bildung von „classification schemes“ aus \citep{Simone00} weiter und konzentrieren sich dabei auf deren Adaptierung an konkrete Arbeitssituationen. Sie führen dabei aber keine neuen Aspeke hinsichtlich der Unterstützung von „Articulation Work“ ein.
	\item[\citet{Bossen02}] (II) baut auf der Arbeit von \citep{Bannon97} zu „Common Information Spaces“ auf und identifiziert im Rahmen einer Fallstudie im medizinischen Bereich Gestaltungsparameter, in deren Rahmen auch „Articulation Work“ als in unterschiedlichen Ausprägungen zu unterstützendes Phänomen genannt wird, ohne näher auf die Implikationen dieser Forderung einzugehen.
	\item[\citet{Davenport02}] (II, III) beschreibt „Articulation Work“ als eine Form von „alltäglichem Wissensmanagment“, mit Hilfe dessen beteiligte Individuen im Arbeitsprozess lernen und ihre Kompetenzen erweitern („situated learning“). Anhand einer Fallstudie zeigt sie, dass das Konzept der „Communities of Practice“ \citep{Wenger98} und deren Methoden geeignet sind, diese Form von „Articulation Work“ zu unterstützen. Die Autorin deutet die Möglichkeit einer Unterstützung durch rechnerbasierte Werkzeuge an, führt diese Idee aber nur am Rande aus.
	\item[\citet{Herrmann02}] (II, III) beschäftigen sich mit Modellen von soziotechnischen Arbeitsprozessen und zeigen auf, dass zu deren (kooperativen Erstellung) „Articulation Work“ notwendig ist. 
	\item[\citet{Mark02}] (II) stellen eine Kurzfassung des in \citep{Mark02a} ausführlich beschriebenen Tests des „Reconciler“-Systems vor.
	\item[\citet{Mark02a}] (II) beschreiben einen ersten Test des „Reconciler“-Systems und zeigen, dass das Werkzeug tatsächlich bei der Entwicklung eines gemeinsamen Sichtweise über die Arbeitsdomäne betragen kann.
	\item[\citet{Raposo02}] beschäftigen sich aufbauend auf \citep{Raposo01} mit der Konkretisierung des Frameworks zur Unterstützung der Koordination von Aufgaben, die in gegenseitiger Abhängigkeit stehen. Die Autoren bereiten damit das Feld für die technische Umsetzung des Frameworks, die in \citep{Raposo04} beschrieben wird.
	\item[\citet{Sarini02}] (III) beschäftigen sich mit „recursive Articulation Work“, also jener Form, deren Gegenstand selbst wiederum „Articulation Work“ ist. Die Autoren leiten Anforderungen an die Unterstützung dieser Form von „Articulation Work“ ab und zeigen die konkrete Umsetzung als Teil des „Reconciler“-Systems.
	\item[\citet{Sarini02a}] beschreiben in Form einer Kurzfassung die wesentlichen Konzepte und Implementierungsansätze des „Reconciler“-Systems.
	\item[\citet{Schmidt02}] (IV) beschäftigt sich konzeptuell mit der Unterstützung von Awareness in \gls{CSCW}-Systemen und erwähnt dabei am Rande, dass Awareness oft ein wichtiger Aspekt von „Articulation Work“ ist.
	\item[\citet{Simone02}] (IV) beschreibt die im Rahmen des „Reconciler“-Projektes durchgeführte Arbeit im Kontext von Wissensmanagement und „Organizational Memories“\footnote{für einen Überblick zu diesem Themengebiet siehe \citep{Maier08}}. Sie zeigt, in welchen Aspekten Berührungspunkte zwischen Wissensmanagment und \gls{CSCW} bestehen und weist auf mögliche Unterstützungsleistungen hin. Auf „Articulation Work“ wird nur im Zusammenhang mit dem im Wissensmanagement relevanten Abgleich von Ontologien verwiesen, der als „Articulation Work“ gesehen werden kann.
	\item[\citet{Eschenfelder03}] (II) beschreibt eine qualitative Studie über das Management von content-zentrierten Websites und zieht „Articulation Work“ (in Bezugnahme auf das von \citet{Corbin93} beschriebene Verständnis) als das der Analyse zugrundeliegende Framework heran. Die Autorin zeigt im zweiten Teil der Arbeit auf, wie Content Management Systeme den Verwaltungsprozess unterstützen können, geht aber nicht weiter auf „Articulation Work“ ein.
	\item[\citet{Olesen03}] (IV) beschreiben die Veränderung des Arbeitsablaufs der Rezeptausstellung in einem Krankenhaus durch Einführung eines technischen Systems, das die elektronische Verschreibung von Medikamenten erlaubt. Die Autoren verfolgen dabei einen kulturwissenschaftlichen Ansatz und weisen lediglich in der Einleitung auf „Articulation Work“ als eine bei der Umstellung des Arbeitsablaufs notwendige Tätigkeit hin.
	\item[\citet{Sarini03}] (III) fasst die konzeptuellen Grundlagen, die Implementierung und den Test des „Reconciler“-Systems in Form seiner Dissertation zusammen. Er führt dabei jedoch keine nicht bereits in früheren Publikationen veröffentlichten Argumente oder Anforderungen ein.
	\item[\citet{Gerson04}] (III) beschreibt die Verwendung von „Reconciliation Mechanisms“ zur Auflösung von Problemen in der Zusammenarbeit bei räumlich verteilt durchgeführter Arbeit. Diese „Reconciliation Mechanisms“ sind vorrangig organisationale oder soziale Maßnahmen, die die Zusammenarbeit verbessern bzw. wieder möglich machen. Ein expliziter Bezug zu „Articulation Work“ wird nicht hergestellt, ausgehend von der Beschreibung sind „Reconciliation Mechanisms“ aber ein Mittel zur Durchführung von „Articulation Work“. \citeauthor{Gerson04} gibt exemplarisch vier dieser Mechanismen an (z.B. „shared ressource pools“ oder „participant review“), ohne jedoch deren detaillierte Ausgestaltung einzugehen.
	\item[\citet{Jorgensen04}] (III) beschreibt die Verwendung von „interaktiven“ Prozessmodellen in organisationalen Arbeitsprozessen und die Veränderung dieser Prozesse durch Modellierungsvorgänge. Dabei bezeichnet er den Modellierungsvorgang als „Articulation Work“. „Interaktive“ Prozesse sind dabei solche, die wissensintensiv sind, im Vorhinein spezifiziert werden können und deren konkreter Ablauf erst zum Zeitpunkt der Ausführung festgelegt wird (was jenen Arbeitsabläufen entspricht, die als „problematic“ oder „non-routine“ bezeichnet werden). Der Autor entwickelt im Rahmen der Arbeit eine Methodik zur Modellierung derartiger Prozesse und ein technisches Werkzeug, dass die Erstellung und Instanzierung dieser Modelle unterstützt bzw. ermöglicht.
	\item[\citet{Raposo04}] decken in ihrer Arbeit zur (technischen) Unterstützung kooperativer Arbeit explizit alle Zeitpunkte ab, in denen „Articulation Work“ auftreten kann („pre-articulation“, „coordination“, „post-articulation“). Sie schlagen zur Koordination formalisiert festgeschriebene „Commitments“ vor, die in der „pre-articulation“-Phase definiert werden und während der „post-articulation“ evaluiert werden. Damit decken die Autoren auch „recursive Articulation Work“ \citep{Sarini02} ab. In der Arbeit wird im wesentlichen der vorgeschlagene Formalismus und dessen konzeptuelle Anwendung dargestellt.
	\item[\citet{Faergemann05}] (I, II) beschreiben „Articulation Work“ in Arbeitsprozessen, die unterschiedlich große Personenkreise umfassen, die verschieden stark miteinander vertraut sind. Die Autoren leiten auf ihren empirischen Beobachtungen vier unterschiedliche Arten von „Articulation Work“ ab, die sich jeweils in der Größe ihres Durchführungskontexts (d.h. des Teilnehmerkreises) unterscheiden. Sie beschreiben die Charakteristika dieser Arten von „Articulation Work“, gehen aber nicht auf deren Unterstützung ein (wobei sie andeuten, dass eine technische Unterstützung jeweils unterschiedlich ausfallen muss, bezeichnen dies jedoch als eine offene Forschungsfrage).
	\item[\citet{Hasu05}] (II, IV) beschreibt die Einbindung neuer (computer-basierter) Werkzeuge in Arbeitsabläufe durch technische Laien (konkret die Verwendung eines neuen, komplexen medizinischen Gerätes durch Neurologen). Sie klassifiziert die im Zuge dessen anfallenden Aktivitäten als „invisible Articulation Work“. Sie geht im Übrigen darauf ein, wie derartige Prozesse durch ethnogaphische Forschung erfasst werden können, die Unterstützung von „Articulation Work“ selbst wird aber nicht weiter thematisiert.
	\item[\citet{Hampson05}] (II) beschreiben die Arbeit im interaktiven (d.h. hier telemediengestützten) Kundenservice als „Articulation Work“. Die Autoren führen eine Klassifikation von unterschiedlichen Arten von Arbeitsabläufen ein, um ihren Fokus abzugrenzen. In der Folge beschreiben sie die im Rahmen des „interactive customer service“ auftretende Phänomene, deren konkrete Ausprägunen und die Reaktionen der Kundenbetreuer. Sie legen dar, welche Rolle „Articulation Work“ in diesem Kontext spielt, gehen dabei aber nicht darauf ein, wie diese Abläufe unterstützt werden können. 
	\item[\citet{Cabitza06}] (III) entwickeln den „Reconciler“-Ansatz weiter und wenden ihn unter Bezugnahme auf \citet{Faergemann05} auf „globale Articulation Work“ an. Sie entwickeln dabei ein konzeptuelles Framework, das (wie im Falle des „Reconciler“-Ansatzes) auf Artefakten als Artikulations-Objekten beruht und geben eine Methodik an, wie derartige Artefakte entwickelt werden können (im Sinne der „recursive Articulation Work“). Die vollständige Umsetzung sowie eine Evaluierung des Konzepts stand zum Zeitpunkt der Publikation der Arbeit noch aus. 
	\item[\citet{Crabtree06}] (II, III) zeigen die Relevanz von „Articulation Work“ in Situationen, in denen Personen einander Hilfestellungen geben. Die Autoren beschreiben dabei Situationen, in denen die Hilfestellung „remote“ (d.h. aus der Entfernung) erfolgt. Sie zeigen, welche Artikulationsprozesse dabei regelmäßig auftreten und leiten Anforderungen an eine mögliche Unterstützung für derartige Arbeitsprozesse ab. Obwohl grundsätzlich relevant für die Unterstützung von „Articulation Work“, bleibt diese Arbeit hinsichtlich der Anforderungen jedoch relativ abstrakt und unspezifisch. 
	\item[\citet{Kaghan06}] (II, III) (bzw. der ebenfalls vorliegende ausführlichere Preprint \citep{Kaghan04}) zeigen mit kulturwissenschaftlichem Hintergrund, wie organisationale Artefakte (also Ergebnisse bzw. Gegenstände von Arbeit) kooperativ erstellt, verwendet und angepasst werden und in der Folge die mit ihnen verbundene Arbeit widerspiegeln. Die Autoren bedienen sich dabei einer Fallstudie aus dem Bereich des Technologietransfers zwischen Universitäten und Wirtschaft, wo Verträge als Artefakte bzw. die Vertragsverhandlung als relevanter Arbeitsablauf im Detail betrachtet werden. „Articulation Work“ kommt dabei im Rahmen der Vertragsanbahnung („arranging deals“) zum Einsatz. Allgemein hat „Articulation Work“ hier das Ziel, ein erreichtes gemeinsames Verständnis so in einem Artefakt abzubilden, dass dieses von den Beteiligten als Repräsentant der vereinbarten Zusammenarbeit akzeptiert wird. Die Autoren nehmen wie \citet{Davenport02} Bezug auf „Communities of Practice“ als wesentliches Konzept bei der Entwicklung dieser Artefakte.
	\item[\citet{Baker07}] (I, II) beschreiben, wie „Articulation Work“ im Rahmen des Designs von „information infrastructure“ (als Bezeichnung von Systemen, die Verwaltung und strukturierte Manipulation von Information erlauben) zur Anwendung kommt. Die Autoren beziehen sich auf eine von ihnen durchgeführte empirische Studie und fassen aufgrund ihrer Erkenntnisse den Begriff „Articulation Work“ so breit, dass er nicht nur die Abstimmung der eigentlichen Arbeitsabläufe umfasst, sondern etwa auch die Aushandlung eines gemeinsamen Verständnisse über den Aufbau der Arbeitsdomäne. 
	\item[\citet{Cabitza07}] (IV) beschreiben die Verwendung von „dokumentatischen Artefakten“ und deren Computer-Unterstützung im medizinischen Bereich. Die Autoren gehen dabei nur in einem Nebensatz explizit auf „Articulation Work“ ein, die Arbeit dient aber gemeinsam mit \citet{Cabitza06} als Grundlage der weiteren Entwicklungen zur Unterstützung von „Articulation Work“, die in \citep{Cabitza09} beschrieben wird.
	\item[\citet{Convertino08}] (IV) beschreiben, wie in kooperativen Arbeitsprozessen ein gemeinsames Verständnis der Arbeitsdomäne („common ground“) entwickelt werden kann und in weiterer Folge eine einfachere Zusammenarbeit zwischen den beteiligten Individuen ermöglicht. Die Autoren beziehen sich aber nur im Rahmen der in der Arbeit beschriebenen empirischen Studie am Rande auf „Articulation Work“.
	\item[\citet{Larsen08}] (II) beschreiben, wie in kooperativen Arbeitsprozessen Information über die Kompetenzen und Verantwortlichkeiten der beteiligten Individuen ausgetauscht wird. Aus der vorgestellten empirischen Studie leiten die Autoren ab, dass -- trotz fortgeschrittener technischer Möglichkeiten -- die Abstimmung von Kompetenzen und Verantwortlichkeiten in kooperativer Arbeit in synchronen Anwendungsszenarien zu einem besseren Ergebnis führt als in asynchronen Settings.
	\item[\citet{Cabitza08}] (IV) betrachtet die Ausführungen aus \citet{Cabitza07} aus Perspektive des Wissensmanagement und bildet damit ebenso die Grundlage für die in \citep{Cabitza09} vorgestellte technische Lösung vor. „Articulation Work“ als Konzept wird hier nicht explizit angesprochen.
	\item[\citet{Cabitza09}] (III) schlagen „active artifacts“ als Mittel zur Unterstützung von „Articulation Work“ zwischen „Communities“ im Arbeitsablauf (im Sinne von „coordinating predefined work“) vor. „Active artifacts“ können dabei nicht nur Information tragen, sondern auch auf ihren aktuellen Kontext reagieren und selbständig aktiv Information vermitteln. Dabei führen die Autoren auch ein Konzept an, wie das Verhalten derartiger „active artifacts“ spezifiziert werden können. Hinsichtlich der Unterstützung von „Articulation Work“ ist die Arbeit als technische Detaillierung und Verfeinerung der in \citet{Cabitza06} bereits vorgestellten Konzepte zu sehen.
	\item[\citet{Cabitza09a}] (III) stellen die in \citet{Cabitza06} vorgeschlagene und in \citep{Cabitza09} eingesetzte Sprache zur Spezifikation von Koordinations-Artefakten in „global Articulation Work“ im Detail vor. Diese basiert im Wesentlichen auf der Formulierung von \gls{ECA}-Regeln, die im operativen Betrieb die Grundlage der Koordinations-Unterstützung bilden.
	\item[\citet{Cabitza09b}] (II) stellen eine empirische Studie zur Motivation des in \citep{Cabitza09} vorgestellten Systems vor und zeigen dessen unterstützende Wirkung bei der Durchführung von „Articulation Work“.
\end{description}

Betrachtet man diese Arbeiten in ihrer Gesamtheit, so zeigt sich die historische Entwicklung der Forschung zum Thema „Articulation Work“ oder unter Verwendung derselben. Vor allem wird ein starker Bezug zur Konzeption von \gls{CSCW}-Systemen sichtbar, in deren Kontext ein Großteil der verfügbaren Arbeiten verfasst wurden. Zudem sind auch Gruppen von Publikationen zu erkennen, die im gleichen Kontext publiziert wurden und sich nur in Einzelaspekten unterscheiden. Abbildung \ref{fig:img_ArticulationWork_ArticulationWorkLiteratur} auf Seite \pageref{fig:img_ArticulationWork_ArticulationWorkLiteratur} zeigt diese Zusammenhänge.

\begin{figure}[htbp]
	\centering
		\includegraphics[width=\textwidth]{img/ArticulationWork/ArticulationWorkLiteratur.png}
	\caption{Literatur zu Articulation Work im Kontext}
	\label{fig:img_ArticulationWork_ArticulationWorkLiteratur}
\end{figure}

Beginnend mit den Arbeiten von Strauss in der linken oberen Ecke ist vertikal die zeitliche Dimension der Publikation von Arbeiten zu Artikulation Work aufgetragen. Die Seitenbreite wird zur thematischen Gruppierung der Publikationen verwendet. Die Pfeile zwischen Publikationen bzw. Publikationsgruppen stellen einen inhaltlichen Bezug dar. Die Publikationen am Endpunkt des Pfeils nehmen dabei Bezug auf jene, die sich am Ausgangspunkt des Pfeils befinden.

Am linken Rand der Darstellung sind die Arbeiten zu finden, die im soziologischen Kontext verfasst wurden. Die meisten der dort angesiedelten Publikationen sind Grundlagenarbeiten, die den Begriff „Articulation Work“ und dessen konzeptuellen Kontext erörtern oder anhand von Fallstudien das Auftreten von „Articulation Work“ zeigen.

Im Zentrum der Darstellung steht die größte Gruppe von Arbeiten, die im Kontext von \gls{CSCW} verfasst wurde. Die Arbeiten, die sich auf \gls{CSCW} beziehen, haben dabei zum Teil die Ableitung für Anforderungen an eine technische Unterstützung von „Articulation Work“ zum Ziel, der Rest der Arbeiten beschäftigt sich eher mit der technischen Umsetzung der Unterstützung. Jene Publikationen, die eher ersterer Gruppe zuzuordnen sind, sind eher links angeordnet, die technisch orientierten Publikationen befinden sich eher rechts. Die Entfernung zur Mittelachse hat dabei keine Aussagekraft, sondern ist nur einer übersichtlichen Anordnung geschuldet. 

Innerhalb der \gls{CSCW}-Gruppe gibt es zwei bedeutende Sub-Gruppen, die untereinander in Beziehung stehen. Einerseits ist die Gruppe von Publikationen zu nennen, die im Rahmen des COMIC-Projektes 1992-1995 entstanden sind \citep{Rodden95}. In diesem Projekt wurde die Grundlage der Berücksichtigung von „Articulation Work“ als Thema von \gls{CSCW} gelegt. Bereits im Rahmen des COMIC-Projektes beginnend, publiziert die Gruppe um \citeauthor{Simone00} Arbeiten zur konkreten technischen Umsetzung der Unterstützung durch computerbasierte Werkzeuge. Die Implementierungen, auf die dabei immer wieder Bezug genommen wird, werden als „Ariadne“ (für den Koordinierungsaspekt von „Articulation Work“) „Reconciler“ (für den Awareness-Aspekt von „Articulation Work“) bezeichnet, was auch als Namensgeber dieser Gruppe von Arbeiten herangezogen wurde.

Weiter rechts am oberen Rand der Abbildung befinden sich Arbeiten, die „Articulation Work“ im Kontext der Software-Entwicklung betrachten. Dies sind die ersten Arbeiten, die eine konkrete Anwendung der Konzepte um „Articulation Work“ außerhalb der Soziologie bzw. der Community um Strauss zeigen. Die Unterstützung von „Articulation Work“ ist hier nur teilweise Gegenstand der Betrachtung, wo sie aber angesprochen wird, ist sie entsprechend der Anwendungsdomäne eher technisch orientiert.

Ganz rechts sind jene Arbeiten zu finden, die auf „Articulation Work“ Bezug nehmen, jedoch nicht einer der bisher beschriebenen Gruppen zuzuordnen sind. Hier finden sich Publikationen, die vor philosophischem, organsationswissenschaftlichem Hintergrund oder mit Bezug zum Wissensmanagement verfasst wurden. Herauszugreifen ist hier die Arbeit von \citet{Jorgensen04}, der die Rolle von Modellen (konkret konzeptuellen Modellen von Arbeit) bei der Durchführung von „Articulation Work“ betrachtet und damit erstmals einen konkreten Unterstützungsbezug zwischen „Artikulation Work“ und der Domäne der Organisationswissenschaften herstellt (was wiederum für die Betrachtung von „Articulation Work“ im organisationalen Kontext von Interesse ist).

Insgesamt ist in der Abbildung ein starker Schwerpunkt auf die technische Unterstützung von Arbeit, konkret „Articulation Work“, zu erkennen. Dieser Schwerpunkt wurde sowohl konzeptuell als auch technisch ab Beginn der 90er-Jahre des 20. Jahrhunderts bis etwa 2005 ausführlich bearbeitet. In den letzten Jahren treten verstärkt Fallstudien auf, die einen Bezug zu „Articulation Work“ herstellen, jedoch nur bedingt auf deren Unterstützung eingehen.

% section relevante_literatur (end)

% chapter literatur_zum_themengebiet_articulation_work (end)
\chapter{Daten der empirischen Untersuchung} % (fold)
\label{cha:daten_der_empirischen_untersuchung}

Die Darstellung der erhobenen Rohdaten der empirischen Untersuchung und deren detaillierte Auswertung würden an dieser Stelle den Rahmen der Arbeit sprengen. Durch die große Anzahl an Videoaufnahmen, die insgesamt etwa XY \gls{GB} an Platzbedarf einnehmen, ist auch die Beilage eines Datenträgers nicht möglich. 

Die Rohdaten, die durchgeführten Auswertungen und Transkripte sowie die Skripte der statistischen Tests mit der Software R können via eMail unter

\begin{center} stefan@oppl.info \end{center}

angefordert werden. 

\section{Verfügbare Rohdaten} % (fold)
\label{sec:verfügbare_rohdaten}

Zu den Untersuchungen stehen im Einzelnen folgende Rohdaten zur Verfügung:
\begin{itemize}
	\item Evaluierungsblock 1 (siehe auch \citep{Bohninger10})
		\begin{itemize}
			\item Videoaufnahmen der Modellbildung (Detailansicht der Modellierungsoberfläche)
			\item Fotos der finalen Versionen der erstellten Modelle
		\end{itemize}
	\item Evaluierungsblock 2
		\begin{itemize}
			\item Videoaufnahmen der Modellbildung aus jeweils 2 Perspektiven (Gesamtübersicht inkl. Personen sowie Detailansicht der Modellierungsoberfläche)
			\item Fotos bzw. Graphische Abbildungen der finalen Versionen der erstellten Modelle
		\end{itemize}
	\item Evaluierungsblock 3
		\begin{itemize}
			\item Videoaufnahmen der Modellbildung aus jeweils 2 Perspektiven (Gesamtübersicht inkl. Personen sowie Detailansicht der Modellierungsoberfläche)
			\item Graphische Abbildungen der finalen Versionen der erstellten Modelle
		\end{itemize}
	\item Evaluierungsblock 4 (siehe auch \citep{Wahlmuller10})
		\begin{itemize}
			\item Videoaufnahmen der Modellbildung können aus Gründen der Schutzes unternehmensinterner Information auf diesem Wege nicht weitergegeben werden. Etwaige Anfragen sind an Patrick Wahlmüller (Kontaktdaten in \citep{Wahlmuller10}) zu richten.
			\item Graphische Abbildungen der finalen Versionen der erstellten Modelle
		\end{itemize}
	\item Evaluierungsblock 5 (siehe auch \citep{Bindreiter10})
		\begin{itemize}
			\item Videoaufnahmen der Modellbildung mittels CMapTools und am Modellierungstisch aus jeweils 2 Perspektiven (Gesamtübersicht inkl. Personen sowie Detailansicht der Modellierungsoberfläche)
			\item Graphische Abbildungen der finalen Versionen der erstellten Modelle
			\item Als XML exportierte Repräsentationen der mit CMapTools erstellten Modelle (inkl. Modellierungshistorie)
			\item Fragebogen 
		\end{itemize}
\end{itemize}

% section verfügbare_rohdaten (end)

\section{Durchgeführte Auswertungen} % (fold)
\label{sec:durchgeführte_auswertungen}

Zu den Untersuchungen wurden folgende Auswertungen durchgeführt und archiviert. Die einzelnen Auswertungsmethoden sind auf den folgenden Seiten näher beschrieben.

\begin{itemize}
	\item Evaluierungsblock 1 (siehe auch \citep{Bohninger10})
		\begin{itemize}
			\item Überblicksauswertung aller Anwendungen und Modelle
			\item  
		\end{itemize}
	\item Evaluierungsblock 2
		\begin{itemize}
			\item Überblicksauswertung aller Anwendungen und Modelle
			\item Interaktionsanalyse aller Anwendungen
		\end{itemize}
	\item Evaluierungsblock 3
		\begin{itemize}
			\item Überblicksauswertung aller Anwendungen und Modelle
			\item Interaktionsanalyse aller Anwendungen
		\end{itemize}
	\item Evaluierungsblock 4 (siehe auch \citep{Wahlmuller10})
		\begin{itemize}
			\item Überblicksauswertung aller Anwendungen und Modelle
			\item Interaktionsanalyse aller Anwendungen
		\end{itemize}
	\item Evaluierungsblock 5 (siehe auch \citep{Bindreiter10})
		\begin{itemize}
			\item Überblicksauswertung aller Anwendungen und Modelle
			\item Interaktionsanalyse aller Anwendungen
		\end{itemize}
\end{itemize}

\subsection{Überblicksauswertung}

Die Überblicksauswertung fasst die wesentlichen Eigenschaften des in einer Anwendung erstellten Modells sowie die während der Erstellung aufgetretenen Ereignisse zusammen. Die Daten wurden in Form einer Openoffice-Tabelle aufbereitet und stehen als \gls{ODF}-Dateien zur Verfügung. Sie dienen als Grundlage aller weiteren deskriptiven und schließenden statistischen Auswertungen.

Die konkrete Ausgestaltung der Tabelle variiert je nach Evaluierungsblock (abhängig von der durchgeführten Modellierungsaufgabe und der im Werkzeug implementierten Funktionalität) leicht, in Abbildung \ref{fig:img_AnhangEmpirie_raster} ist der Raster aus Evaluierungsblock 5 dargestellt, in dem die höchste Anzahl von Merkmalen erhoben wurden.

\begin{figure}[htbp]
	\centering
		\includegraphics[width=0.9\textwidth]{img/AnhangEmpirie/raster.pdf}
	\caption{Raster der Überblicksauswertung}
	\label{fig:img_AnhangEmpirie_raster}
\end{figure}

Die Befüllung der Raster erfolgte auf Basis der angefertigten Video-Aufnahmen der Werkzeuganwendungen. In den Blöcken 3 und 5 wurden die Raster redundant unabhängig voneinander von jeweils zwei Personen befüllt. Sofern Abweichungen bei der Auswertung der quantitativen Parameter festgestellt wurden, wurde das jeweilige Merkmal durch eine dritte Person geprüft und ggf. entsprechend korrigiert. In den Blöcken 1, 2 und 4 standen nicht ausreichend personelle Ressourcen für eine redundante Auswertung zur Verfügung.

\subsection{Interaktionsanalyse}

Die Interaktionsanalyse wurde wie in Abschnitt \ref{sub:interaktionsanalyse} beschrieben durchgeführt und dokumentiert. Die Dokumentation erfolgte in Textdokumenten, die als \gls{ODF}-Dateien zur Verfügung stehen.

In den Evaluierungsblöcken 3 und 5 wurde die Interaktionsanalyse für jede Werkzeuganwendung von zwei Personen redundant durchgeführt. In den Blöcken 2 und 4 standen die personellen Ressourcen für eine redundante Auswertung nicht zur Verfügung. In den Fällen, in denen redundant ausgewertet wurde, wurden Transkripte, die nur von einer der auswertenden Personen erfasst wurden, von einer dritten Person geprüft und bestätigt bzw. verworfen.

% section durchgeführte_auswertungen (end)

\section{Verwendete Fragebögen} % (fold)
\label{sec:frageboegen}

Bei der Durchführung der Evaluierungsblöcke 1, 4 und 5 wurden zusätzlich zu den direkt aus der Modellbildung erhobenen Daten auch Benutzerbefragungen mittels Fragebögen durchgefürht. Im Folgenden sind für jeden Evaluierungsblock die verwendeten Fragebögen angeführt. Zusätzlich werden die Fragebögen hinsichlich ihrer Relevanz für die untersuchten Hypothesen (siehe Kapitel \ref{cha:eval_werkzeug} bis \ref{cha:eval_aw}) eingeordnet. Die Auswertungen der ausgefüllten Fragebögen sind in digitaler Form detailliert (siehe Abschnitt \ref{sec:verfügbare_rohdaten}) bzw. aggregiert (siehe Abschnitt \ref{sec:durchgeführte_auswertungen}) in digitaler Form verfügbar.


\fboxrule0.4mm
\fboxsep0.1mm

\subsection{Fragebögen aus Evaluierungsblock 1}
\label{sub:fb_eval1}

Die Abbildungen \ref{fig:img_AnhangEmpirie_fb5-01} bis \ref{fig:img_AnhangEmpirie_fb5-05} zeigen den in Evaluierungsblock 1 verwendeten Fragebogen. Der Aufbau des Fragebogens wurde von \cite{Bohninger10} detailliert beschrieben und begründet.

\subsection{Fragebögen aus Evaluierungsblock 4}
\label{sub:fb_eval4}

Die Abbildungen \ref{fig:img_AnhangEmpirie_fb5-01} bis \ref{fig:img_AnhangEmpirie_fb5-05} zeigen den in Evaluierungsblock 4 verwendeten Fragebogen. Der Aufbau des Fragebogens wurde von \citet{Wahlmuller10} detailliert beschrieben und begründet.


\subsection{Frageböqgen aus Evaluierungsblock 5}
\label{sub:fb_eval5}

Die Abbildungen \ref{fig:img_AnhangEmpirie_fb5-01} bis \ref{fig:img_AnhangEmpirie_fb5-05} zeigen den in Evaluierungsblock 5 verwendeten Fragebogen. Der Aufbau des Fragebogens wurde von \citet{Bindreiter10} detailliert beschrieben und begründet.

\begin{figure}[htbp]
	\centering
	\fbox{%
		\includegraphics[width=0.9\textwidth]{img/AnhangEmpirie/fb5-01.jpeg}%
	}
	\caption{Fragebogen für Evaluierungsblock 5 - Seite 1}
	\label{fig:img_AnhangEmpirie_fb5-01}
\end{figure}

\begin{figure}[htbp]
	\centering
	\fbox{%
		\includegraphics[width=0.9\textwidth]{img/AnhangEmpirie/fb5-02.jpeg}%
	}
	\caption{Fragebogen für Evaluierungsblock 5 - Seite 2}
	\label{fig:img_AnhangEmpirie_fb5-02}
\end{figure}

\begin{figure}[htbp]
	\centering
	\fbox{%
		\includegraphics[width=0.9\textwidth]{img/AnhangEmpirie/fb5-03.jpeg}%
	}
	\caption{Fragebogen für Evaluierungsblock 5 - Seite 3}
	\label{fig:img_AnhangEmpirie_fb5-03}
\end{figure}

\begin{figure}[htbp]
	\centering
	\fbox{%
		\includegraphics[width=0.9\textwidth]{img/AnhangEmpirie/fb5-04.jpeg}%
	}
	\caption{Fragebogen für Evaluierungsblock 5 - Seite 4}
	\label{fig:img_AnhangEmpirie_fb5-04}
\end{figure}

\begin{figure}[htbp]
	\centering
	\fbox{%
		\includegraphics[width=0.9\textwidth]{img/AnhangEmpirie/fb5-05.jpeg}%
	}
	\caption{Fragebogen für Evaluierungsblock 5 - Seite 5}
	\label{fig:img_AnhangEmpirie_fb5-05}
\end{figure}


% section frageboegen (end)

% chapter daten_der_empirischen_untersuchung (end)

\part*{Verzeichnisse}
\addcontentsline{toc}{part}{Verzeichnisse}

\addcontentsline{toc}{chapter}{Abbildungsverzeichnis}
\listoffigures

\addcontentsline{toc}{chapter}{Tabellenverzeichnis}
\listoftables

%\addcontentsline{toc}{chapter}{Stichwortverzeichnis}
%\printindex

\addcontentsline{toc}{chapter}{Abkürzungsverzeichnis}
\printglossary[type=\acronymtype,style=long,title=Abkürzungsverzeichnis]

\addcontentsline{toc}{chapter}{Bildquellen}
\chapter*{Abbildungsquellen}

Dieser Anhang enthält Quellenangaben für alle in dieser Arbeit verwendeten Abbildungen, sofern sie nicht vom Autor erstellt wurden. Sofern nicht anders angegeben sind alle Abbildungen und Fotos Werke des Autors und dürfen nicht ohne ausdrückliche Zustimmung verwendet werden.

\begin{description}
 \item[Abbildung \ref{fig:img_ImplementierungUeberblick_MCRpd}] Bild des MVC- und MCRpd-Modells entnommen aus \citep{Ullmer00}
 \end{description}

\begin{description}
 \item[Abbildung \ref{fig:img_ImplementierungInput_artoolkit}] Bild des AR Toolkit Markers entnommen von www.hitl.washington.edu/artoolkit/ (Website der Entwickler)
 \end{description}

\begin{description}
 \item[Abbildung \ref{fig:img_ImplementierungInput_visualcodes}] Bild des Visual Code Markers entnommen aus \citep{Rohs04}
 \end{description}

\begin{description}
 \item[Abbildung \ref{fig:img_Persistenz_SubjectVsOccurrence}] Foto der Tasse lizenzfrei unter http://www.oldskoolman.de/bilder/\\freigestellte-bilder/essen-trinken/kaffee-tasse-freigestellt/, übrige Abbildung eigene Darstellung
 \end{description}

Grafiken aus ImplementierungInput fehlen noch

\addcontentsline{toc}{chapter}{Publikationen im Kontext dieser Arbeit}
\chapter*{Publikationen im Kontext dieser Arbeit}

\begin{description}
	\item[\citet{Oppl05a}] description
	\item[\citet{Oppl06}] description
	\item[\citet{Oppl06a}] description 
	\item[\citet{Oppl07b}] description
	\item[\citet{Oppl07}] description
	\item[\citet{Oppl07a}] description
	\item[\citet{Furtmuller07a}] description
	\item[\citet{Oppl08}] description
	\item[\citet{Oppl08a}] description
	\item[\citet{Oppl09}] description
	\item[\citet{Oppl09a}] description
	\item[\citet{Oppl09b}] description
	\item[\citet{Oppl09c}] description
	\item[\citet{Oppl09d}] description
\end{description}


%\bibliography{/home/oppl/Dokumente/Literatur/Archiv}
\bibliography{/Users/oppl/Documents/Dokumente/Literatur/Archiv}

\newpage
\cleardoublepage

\vspace*{\fill}

\begin{center}
Der eine fragt: Was kommt danach? \\
Der andre fragt nur: Ist es recht? \\
Und also unterscheidet sich \\
der Freie von dem Knecht.
\end{center}

\begin{flushright}
Theodor Storm
\end{flushright}

\vspace*{\fill}



\end{document}