\chapter{Evaluierung der erstellten Modelle} % (fold)
\label{cha:eval_modell}

Im zweiten Teil der Evaluierung wird nicht der Umgang mit dem Werkzeug betrachtet (siehe dazu Kapitel \ref{cha:eval_werkzeug}), sondern auf das unmittelbare Resultat der Werkzeugverwendung, also das erstellte Modell eingegangen. Ebenfalls nicht Gegenstand der Untersuchung ist in diesem Abschnitt die Wirkung der Modellbildung auf die operative Arbeit („Production Work“), die in Kapitel \ref{cha:eval_aw} betrachtet wird.

Ausgehend von den erstellten Modellen und deren Entstehungsprozess wird also in diesem Kapitel untersucht, wie das Werkzeug auf die in dieser Arbeit gewählte Form der Unterstützung expliziter „Articulation Work“, nämliche der kooperativen Externalisierung und Abstimmung mentaler Modelle, wirkt. Dementsprechend sind die im folgenden Abschnitt beschrieben Hypothesen aus den Ausführungen der Kapitel über mentale Modelle (Kapitel \ref{cha:mentale_modelle}) und der Methodik zur Externalisierung derselben (Kapitel \ref{cha:methodik}) abgeleitet. 

Zusätzlich wird eine explorativ gebildete Hypothese untersucht, die sich auf inhaltlicher Ebene mit dem Phänomen der Abbildung von Zusammenhängen durch räumliche Konfiguration der Konzepte in einem Modell beschäftigt, das in der Untersuchung der Hypothese \ref{hyp:diagmodelle} bereits hinsichtlich der Werkzeugverwendung beschrieben wurde.

\section{Hypothesen} % (fold)
\label{sec:m_hypothesen}

In diesem Abschnitt werden die Hypothesen abgeleitet, die in diesem Kapitel geprüft werden. Wie in Kapitel \ref{cha:eval_werkzeug} ist zwischen konzeptuell aus der Aufgabenstellung bzw. der entwickelten Methodik abgeleiteten Hypothesen über die Wirkung des Werkzeugs und explorativ während der Evaluierung selbst gebildeten Hypothesen über Eigenschaften des Werkzeugs bzw. dessen Verwendung im Kontext der Modellbildung zu unterscheiden.

\subsection{Konzeptuell begründete Hypothesen} % (fold)
\label{sub:m_konzeptionell_begründete_hypothesen}

Die folgenden Hypothesen sind aus der Aufgabenstellung bzw. den Ausführungen zur Modellierungs-Methodik abgeleitet. Auf die entsprechenden Ausführungen in den Kapiteln \ref{cha:mentale_modelle} bzw. \ref{cha:methodik} wird jeweils bei der Begründung der Hypothesen verwiesen.

Ein wesentlicher Aspekt bei der Externalisierung mentaler Modelle ist die Offenheit der Repräsentationssprache. Diese ist aus Abschnitt \ref{sub:concept_mapping} begründbar und in Anforderung \ref{anf:nicht_vorgegebene_semantik_der_modellierungselemente} abgebildet. Unter „Offenheit“ ist in diesem Zusammenhang die Eigenschaft der Repräsentationssprache gemeint, keine vordefinierte Semantik der Modellelemente vorzugeben sondern diese von den Modellierenden festlegen zu lassen. Dies umfasst im vorliegenden Fall sowohl die Bedeutung der unterschiedlichen Konzepttypen als auch die Bedeutung der Verbindungen zwischen Konzepten. Das Werkzeug darf also in diesem Zusammenhang die Benutzer nicht bei der Wahl der Repräsentationskonzepte und damit bei der Externalisierung selbst einschränken.

\begin{hyp}
	Das Werkzeug schränkt Benutzer semantisch nicht bei der Externalisierung ihrer mentalen Modelle ein.
\end{hyp}

Das Argument der Nicht-Beschränkung der Benutzer bei der Externalisierung hat neben der eben beschriebenen Sprach-Dimension auch eine konkrete Modell-Dimension. Während sich obige Hypothese auf semantische Einschränkungen der Externalisieungsmöglichkeiten bezieht, ist sind auch konkrete, strukturelle Einschränkungen bei der Verwendung des Werkzeugs zur Externalisierung eines bestimmten mentalen Modells zu berücksichtigten. Die Externalisierung muss -- um nicht beschränkend zu wirken -- beliebig komplexe Modelle ermöglichen. „Komplex“ bedeutet hier, dass das Modell beliebig viele Elemente enthalten können muss und diese beliebig untereinander in Beziehung gesetzt werden können.
	
\begin{hyp}
	Das Werkzeug ermöglicht die Repräsentation beliebig komplexer Modelle.
\end{hyp}

Im Rahmen der Verwendung der Externalisierung von mentalen Modellen zum Zwecke der Durchführung von „Articulation Work“ ist es -- wie in Kapitel \ref{cha:methodik} ausgeführt und in Anforderung \ref{anf:kollaborative_und_unmittelbare_manipulierbarkeit_des_modells} abgebildet -- notwendig, eine kooperative Nutzung des unterstützenden Werkzeugs zu ermöglichen. Ein wesentlicher Schritt zur erfolgreichen Durchführung von „Articulation Work“ ist neben der eigentlichen Externalisierung (die in den ersten beiden Hypothesen dieses Kapitels abgebildet wurde) die Abstimmung der indviduellen mentalen Modelle der Beteiligten. „Abstimmung“ bedeutet hier einen Abgleich der indviduellen Verständnisse jener Arbeitsaspekte, die im Sinne von Kapitel \ref{cha:articulation_work} „problematisch“ sind bzw. enge Kooperation der Beteiligten in der „Production Work“ bedingen.

\begin{hyp}
	Das Werkzeug ermöglicht die Abstimmung individueller Modelle.
\end{hyp}

Die Externalisierung mentaler Modelle mit Hilfe von computer-gestützten Werkzeugen ist keine originäre Idee dieser Arbeit. Computerunterstützung existiert vor allem im Bereich des „Concept Mapping“ (siehe Abschnitt \ref{sub:concept_mapping}), das methodisch maßgeblich in das vorgeschlagene Vorgehen des hier vorgestellten Ansatzes einfließt (siehe Kapitel \ref{cha:methodik}). In den existierenden Werkzeugen werden jedoch die kooperative Erstellung und kommunikative Validierung der externalisierten Modelle nicht explizit berücksichtigt. Beide Aspekte sind jedoch -- wie bei der Beschreibung der Strukturlegetechniken REF ausgeführt -- wichtig für den Abgleich mentaler Modelle und damit für die erfolgreiche Durchführung von „Articulation Work“. Die Ermöglichung und Stärkung der Kooperation der Beteiligten untereinander ist also ein wesentlicher Teilaspekt der Anforderung \ref{anf:kollaborative_und_unmittelbare_manipulierbarkeit_des_modells} an das Werkzeug („Kooperative und unmittelbare Manipulierbarkeit des Modells“).

\begin{hyp}
	Die Verwendung des Werkzeugs führt zu stärkerer Kooperation bei der Modellerstellung als die Verwendung von bildschirm-basierten Werkzeugen.
\end{hyp}

% subsection konzeptionell_begründete_hypothesen (end)

\subsection{Explorativ gebildete Hypothesen} % (fold)
\label{sub:m_explorativ_gebildete_hypothesen}

Im Verlauf der beiden Evaluationen war die Herstellung von Verbindungen zwischen Modellelementen aus technischen Gründen schwierig zu benutzen und sehr anfällig für Fehlfunktionen. Dies führte dazu, dass Verbinder nahezu nicht verwendet wurden (siehe dazu die Auswertungen zu Hypothese \ref{hyp:diagmodelle} in Abschnitt \ref{sub:repräsentation_diagrammatischer_modelle}). In dieser Situation wurden Beziehungen zwischen Modellelementen von den Benutzern durch die räumliche Anordnung der Elemente ausgedrückt. Diese implizite Darstellung von relationaler Information erfolgte in allen Fällen spontan und ohne Anleitung oder Instruktion. Dies führte zu der Vermutung, dass die Verwendung von Verbindern zur Abbildung von Beziehungen bzw. Zusammenhängen zwischen Elementen nicht notwendig ist. Um diese Vermutung zu prüfen, wurde sie formal als Hypothese \ref{hyp:keineverbinder} in die Untersuchung aufgenommen.

\begin{hyp}
	\label{hyp:keineverbinder}
	Zur Abbildung von Zusammenhängen ist die Verwendung von Verbindern nicht notwendig.
\end{hyp}

% subsection explorativ_gebildete_hypothesen (end)

% section hypothesen (end)

\section{Untersuchungsdesign und Durchführung} % (fold)
\label{sec:m_untersuchungsdesign}

In diesem Abschnitt wird auf Basis der oben formulierten Hypothesen das Untersuchungsdesign abgeleitet und die Durchführung der Untersuchung beschrieben. Der erste Teil des Abschnitts beschreibt die Operationalisierung der Hypothesen und damit die Festlegung wie diese konkret geprüft werden können. Im zweiten Teil des Abschnitts wird die Durchführung der Prüfung beschrieben. Hier erfolgt neben der Zuordnung der einzelnen Modellierungsblöcke (siehe Abschnitt \ref{sec:globales_untersuchungsdesign}) auch die Darstellung rein beschreibender Modell-Parameter, die nicht unmittelbar in die Prüfung der Hypothesen eingehen. 

\subsection{Operationalisierung} % (fold)
\label{sub:m_operationalisierung}

In diesem Abschnitt wird für jede Hypothese identifiziert, in welcher Form sie geprüft werden kann. Dies umfasst die Festlegung der Messpunkte sowie der jeweiligen Mess- und Auswertungsmethode (letzte bezugnehmend auf den in Abschnitt \ref{sec:eingesetzte_werkzeuge_und_verfahren} beschriebenen Verfahren). Zudem werden jene Evaluationsblöcke festgelegt, die für die jeweilige Untersuchung herangezogen wurden.

Für jede Hypothese wird also spezifiziert, anhand welcher Aspekte diese geprüft werden kann (= abhängige Variablen). Zudem wird festgelegt welche Ausgangssituation bei der Anwendung gewählt werden muss, um die Prüfung durchführen zu können (= unabhängige Variable) und welche Faktoren die Beurteilung ggf. ungewollt beeinflussen können (= Störvariablen).

\subsubsection{Keine semantische Einschränkung der Externalisierung} % (fold)
\label{ssub:keine_semantische_einschränkung_der_externalisierung}



% subsubsection keine_semantische_einschränkung_der_externalisierung (end)

\subsubsection{Repräsentation beliebig komplexer Modelle} % (fold)
\label{ssub:repräsentation_beliebig_komplexer_modelle}

% subsubsection repräsentation_beliebig_komplexer_modelle (end)

\subsubsection{Abstimmung individueller Modelle} % (fold)
\label{ssub:abstimmung_individueller_modelle}

% subsubsection abstimmung_individueller_modelle (end)

\subsubsection{Wirkung auf die Kooperation bei der Modellerstellung} % (fold)
\label{ssub:wirkung_auf_die_kooperation_bei_der_modellerstellung}

% subsubsection wirkung_auf_die_kooperation_bei_der_modellerstellung (end)

\subsubsection{Abbildung von Zusammenhängen ohne Verbinder} % (fold)
\label{ssub:abbildung_von_zusammenhängen_ohne_verbinder}

% subsubsection abbildung_von_zusammenhängen_ohne_verbinder (end)
% subsection m_operationalisierung (end)

\subsection{Durchführung} % (fold)
\label{sub:m_durchführung}

% subsection m_durchführung (end)
% section untersuchungsdesign (end)

\section{Ergebnisse} % (fold)
\label{sec:m_ergebnisse}

\subsection{Keine semantische Einschränkung der Externalisierung} % (fold)
\label{sub:keine_semantische_einschränkung_der_externalisierung}

\subsubsection{Auswertung} % (fold)

\subsubsection{Diskussion} % (fold)

\subsubsection{Ergebnis} % (fold)

% subsection keine_semantische_einschränkung_der_externalisierung (end)

\subsection{Repräsentation beliebig komplexer Modelle} % (fold)
\label{sub:repräsentation_beliebig_komplexer_modelle}

\subsubsection{Auswertung} % (fold)

\subsubsection{Diskussion} % (fold)

\subsubsection{Ergebnis} % (fold)

% subsection repräsentation_beliebig_komplexer_modelle (end)

\subsection{Abstimmung individueller Modelle} % (fold)
\label{sub:abstimmung_individueller_modelle}

\subsubsection{Auswertung} % (fold)

\subsubsection{Diskussion} % (fold)

\subsubsection{Ergebnis} % (fold)

% subsection abstimmung_individueller_modelle (end)

\subsection{Wirkung auf die Kooperation bei der Modellerstellung} % (fold)
\label{sub:wirkung_auf_die_kooperation_bei_der_modellerstellung}

\subsubsection{Auswertung} % (fold)

\subsubsection{Diskussion} % (fold)

\subsubsection{Ergebnis} % (fold)

% subsection wirkung_auf_die_kooperation_bei_der_modellerstellung (end)

\subsection{Abbildung von Zusammenhängen ohne Verbinder} % (fold)
\label{sub:abbildung_von_zusammenhängen_ohne_verbinder}

\subsubsection{Auswertung} % (fold)

\subsubsection{Diskussion} % (fold)

\subsubsection{Ergebnis} % (fold)

% subsection abbildung_von_zusammenhängen_ohne_verbinder (end)
% section m_ergebnisse (end)

% chapter eval_modell (end)