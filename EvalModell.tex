\chapter{Evaluierung der erstellten Modelle} % (fold)
\label{cha:eval_modell}

\section{Hypothesen} % (fold)
\label{sec:m_hypothesen}

\subsection{Konzeptuell begründete Hypothesen} % (fold)
\label{sub:m_konzeptionell_begründete_hypothesen}

\begin{hyp}
	Das Werkzeug schränkt Benutzer nicht bei der Externalisierung ihrer mentalen Modelle ein.
\end{hyp}
Offenheit, Anzahl der Element-Arten
	
\begin{hyp}
	Das Werkzeug ermöglicht die Repräsentation beliebig komplexer Modelle.
\end{hyp}
Größe der Oberfläche, Einbettungen

\begin{hyp}
	Das Werkzeug ermöglicht die Abstimmung individuelle Modelle.
\end{hyp}
kollaborative Modellbildung

\begin{hyp}
	Die Verwendung des Werkzeugs zur kollaborativen Modellierung führt zu besseren Modellen als bei der Verwendung von bildschirm-basierten Werkzeugen.
\end{hyp}

% subsection konzeptionell_begründete_hypothesen (end)

\subsection{Explorativ gebildete Hypothesen} % (fold)
\label{sub:m_explorativ_gebildete_hypothesen}

\begin{hyp}
	\label{hyp:keineverbinder}
	Zur Abbildung von Zusammenhängen ist die Verwendung von Verbindern nicht notwendig.
\end{hyp}
Connectedness

% subsection explorativ_gebildete_hypothesen (end)

% section hypothesen (end)

\section{Untersuchungsdesign und Durchführung} % (fold)
\label{sec:m_untersuchungsdesign}

% section untersuchungsdesign (end)

\section{Ergebnisse} % (fold)
\label{sec:m_ergebnisse}

% section ergebnisse (end)

\subsection{Connectedness} % (fold)
\label{sub:connectedness}

% subsection connectedness (end)

% chapter eval_modell (end)