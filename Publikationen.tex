\chapter*{Publikationen im Kontext dieser Arbeit}

\begin{description}
	\item[\citet{Oppl05a}] „Towards Human-Centered Design of Diagrammatic Representation Schemes“: Konferenz-Beitrag (8 Seiten ACM), in der erstmals der Ansatz der semantischen Offenheit in der Prozessmodellierung auf einer Fallstudie heraus argumentiert wird und als Mittel der Abbildung individueller Erklärungsmuster oder mentaler Modelle in Prozessmodellen bezeichnet wird.
	\item[\citet{Oppl06}] „Towards Intuitive Work Modeling with a Tangible Collaboration Interface Approach“: Workshop-Paper (6 Seiten IEEE), das die technologischen Fragen bei der Konzeption eines tangiblen Modellierungswerkzeugs beschreibt und die bei der technischen Umsetzung zu lösenden Probleme aufzeigt.
	\item[\citet{Oppl06a}] „Towards Tangible Work Modeling“: Konferenz-Beitrag (6 Seiten Oldenburg-Verlag), in der die erste Umsetzung des tangiblen Modellierungswerkzeugs beschrieben wird. Die Argumentation erfolg aus den Arbeiten zur digital augementierem Montessori-Material heraus, welche auf das Anwendungsgebiet der Aufgabenmodellierung abgebildet werden. Der erste Prototyp des Systems und eine Möglichkeit zur digitalen Repräsentation der individuell eingebrachten Information werden aus technischer Sicht beschrieben.
	\item[\citet{Oppl07b}] „Spielen Sie noch? -- Bausteine im Unternehmenskontext“: Workshop-Paper (4 Seiten Oldenburg-Verlag), dass erstmalig den Gegenstands der Modellbildung von Geschäftsprozessen hin zu Arbeitswissen verallgemeinert und den Ansatz aus dem Forschungsgebiet des Organisationalen Lernen heraus argumentiert. 
	\item[\citet{Oppl07}] „Flexibility of Content for Organisational Learning - A Topic Map Approach“: Master-Arbeit (270 Seiten A4), in der die Repräsentation der Modelle mittels Topic Maps eingeführt wird und deren Notwendigkeit zur erfolgreichen Durchführung von organisationalen Lernprozessen argumentiert wird. Konzeption und Beschreibung der auch in dieser Arbeit zum Einsatz gebrachten Topic-Map-Engine.
	\item[\citet{Oppl07a}] „Human Intervention in cross-organizational Process Development“: Konferenz-Beitrag (10 Seiten, A4), der aus einem durchgeführten Projekt heraus für eine humanzentrierte Sichtweise bei der organisationsübergreifenden Festlegung von Arbeitsprozessen argumentiert und darin einen Beitrag zur Fundierung dieser Arbeit liefert.
	\item[\citet{Furtmuller07a}] „A Tuple-Space based Middleware for Collaborative Tangible User Interfaces“: Workshop-Paper (6 Seiten IEEE), in dem eine Middleware zur flexiblen und dynamischen Kopplung der Informationserfassungs-, -interpretations- und -ausgabemöglicheiten am Tabletop Interface beschrieben wird und deren Einsatz im konkreten Werkzeug umrissen wird.
	\item[\citet{Oppl08}] „Graspable Work Modeling“: Konferenz-Beitrag (4 Seiten Oldenburg-Verlag), in dem erstmals die aktuelle Implementierung des Werkzeugs und die Umsetzung des zweiten Hardware-Prototypen beschrieben wird. Dabei wird erstmalig auf die Fundierung des Ansatzes durch „Articulation Work“ verwiesen.
	\item[\citet{Oppl08a}] „Begreifbare Modellierung von Arbeit“: Workshop-Beitrag (3 Seiten GI LNI), der ebenfalls auf die aktuelle Implementierung des Werkzeugs eingeht und den Mehrwert der tangiblen Interaktionsmöglichkeiten im konkreten Anwendungsfall argumentiert.
	\item[\citet{Oppl09}] „Tabletop Concept Mapping“: Konferenz-Beitrag (8 Seiten ACM), in dem die Verwendung des Werkzeugs zur Durchführung von Concept Mapping beschrieben wird und in dem erstmal empirische Ergebnisse der Werkzeugverwendung angeführt sind. Fokus dieser Arbeit war die Modellbildung mit dem Werkzeug, nicht dessen Einsatz zur Durchführung von „Articulation Work“.
	\item[\citet{Oppl09b}] „Unterstützung expliziter Articulation Work“: Workshop-Beitrag (4 Seiten A4), in der die Wirkung des Werkzeugs im Kontext von techologiegestütztem Lernen (\gls{TEL}) betrachtet wird. 
	\item[\citet{Oppl09c}] „A Tabletop Interface to support Concept Mapping“: Workshop-Paper (3 Seiten A4), in dem eine Kurzübersicht über das Werkzeug zum Einsatz im universitären Lern-Kontext gegeben wird.
	\item[\citet{Oppl09d}] „Konsistente Verwendung von Metaphern als Erfolgskriterium für komplexe Tangible User Interfaces“: Workshop-Paper (4 Seiten GI LNI), in dem die Einflussfaktoren auf die Verständlichkeit von Werkzeugen zur Interaktion mit Tangible Interfaces an einem konkreten Beispiel konzeptuell erörtert und empirisch belegt werden.
	\item[\citet{Oppl09e}] „Using a Tangible Tabletop Interface to facilitate Articulation Work“: Workshop-Paper (2 Seiten A4), in dem eine Kurzübersicht über das Werkzeug zum Einsatz in der Unterstützung von „Articulation Work“ wiederum in Lernszenarien gegeben wird.
	\item[\citet{Oppl10}] „Unterstützung expliziter Articulation Work durch Externalisierung von Arbeitswissen“: Buchbeitrag (25 Seiten A4), in dem das gesamte Forschungsprojekt konzeptuell wie in dieser Arbeit beschrieben argumentiert wird und die Umsetzung des Werkzeugs sowie die ersten empirischen Ergebnisse ausführlich dargestellt werden.
	\item[\cite{Oppl10a}] „Supporting Self-regulated Learning with Tabletop Concept Mapping“: Buchbeitrag (15 Seiten A4, noch nicht erschienen), in dem das Konzept des hier vorgestellten Werkzeugs mit Ansätzen zur Unterstützung von selbst-gesteuertem Lernen und automatisierten, kontextsensitiven Interventionen in den Lernprozess in Zusammenhang gebracht wird und mögliche Anküpfungspunkte identifiziert werden.
\end{description}
