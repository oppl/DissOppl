\chapter{Anforderungen an ein Werkzeug} % (fold)
\label{cha:anforderungen}

\footnote{By recording and replaying the authoring process, navigable history can re-situate an author after a gap in the authoring process. Similarly, in a collaborative authoring process, an author can play through the events since his/her last authoring session to quickly determine the activity of the other authors. Finally, in many situations, information becomes harder to interpret as its context changes over time. By returning to the state of the information space at the time of authoring, disambiguation of the information may become possible. For the reader who is not also the writer of the hypertext there are additional uses of navigable history. A reader replaying the author’s writing process can gain insight into the motivation of the author and have a greater understanding of the author’s writing style. Such an understanding is important in collaborative work and in other contexts, like education and literary analysis. \citep{Shipman00}}

% chapter anforderungen (end)