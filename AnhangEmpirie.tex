\chapter{Daten der empirischen Untersuchung} % (fold)
\label{cha:daten_der_empirischen_untersuchung}

Die Darstellung der erhobenen Rohdaten der empirischen Untersuchung und deren detaillierte Auswertung würden an dieser Stelle den Rahmen der Arbeit sprengen. Durch die große Anzahl an Videoaufnahmen, die insgesamt etwa 35 \gls{GB} an Speicherplatz einnehmen, ist auch die Beilage eines Datenträgers nicht möglich. 

Die Rohdaten, die durchgeführten Auswertungen und Transkripte sowie die Skripte der statistischen Tests mit der Software R können via eMail unter

\begin{center} stefan@oppl.info \end{center}

angefordert werden. 

\section{Verfügbare Rohdaten} % (fold)
\label{sec:verfügbare_rohdaten}

Zu den Untersuchungen stehen im Einzelnen folgende Rohdaten zur Verfügung:
\begin{itemize}
	\item Evaluierungsblock 1 (siehe auch \citep{Bohninger10})
		\begin{itemize}
			\item Videoaufnahmen der Modellbildung (Detailansicht der Modellierungsoberfläche)
			\item Fotos der finalen Versionen der erstellten Modelle
			\item Fragebögen der Befragungen der modellierenden Teilnehmer
			\item Fragebögen der Befragungen der interpretierenden Teilnehmer
			\item Fragebögen zur Korrektheit der Interpretation 
		\end{itemize}
	\item Evaluierungsblock 2
		\begin{itemize}
			\item Videoaufnahmen der Modellbildung aus jeweils 2 Perspektiven (Gesamtübersicht inkl. Personen sowie Detailansicht der Modellierungsoberfläche)
			\item Fotos bzw. Graphische Abbildungen der finalen Versionen der erstellten Modelle
			\item Tagebücher über den Erstellungsprozess der Seminararbeit (= „Production Work“) aller Teilnehmer
			\item Seminararbeiten (= Ergebnis der „Production Work“) bei Einsatz des hier vorgestellten Werkzeugs und aus Lehrveranstaltungen mit identischer Aufgaben- und Themenstellung ohne Einsatz von unterstützenden Maßnahmen bei der Durchführung von „Articulation Work“
		\end{itemize}
	\item Evaluierungsblock 3
		\begin{itemize}
			\item Videoaufnahmen der Modellbildung aus jeweils 2 Perspektiven (Gesamtübersicht inkl. Personen sowie Detailansicht der Modellierungsoberfläche)
			\item Graphische Abbildungen der finalen Versionen der erstellten Modelle
		\end{itemize}
	\item Evaluierungsblock 4 (siehe auch \citep{Wahlmuller10})
		\begin{itemize}
			\item Videoaufnahmen der Modellbildung können aus Gründen der Schutzes unternehmensinterner Information auf diesem Wege nicht weitergegeben werden. Etwaige Anfragen sind an Patrick Wahlmüller (Kontaktdaten in \citep{Wahlmuller10}) zu richten.
			\item Graphische Abbildungen der finalen Versionen der erstellten Modelle
			\item Fragebögen zur Arbeit mit dem Werkzeug
			\item Fragebögen zu den Auswirkungen der Durchführung von „Articulation Work“
		\end{itemize}
	\item Evaluierungsblock 5 (siehe auch \citep{Bindreiter10})
		\begin{itemize}
			\item Videoaufnahmen der Modellbildung mittels CMapTools und am Modellierungstisch aus jeweils 2 Perspektiven (Gesamtübersicht inkl. Personen sowie Detailansicht der Modellierungsoberfläche)
			\item Graphische Abbildungen der finalen Versionen der erstellten Modelle
			\item Als XML exportierte Repräsentationen der mit CMapTools erstellten Modelle (inkl. Modellierungshistorie)
			\item Fragebögen zur Arbeit mit dem Werkzeug
		\end{itemize}
\end{itemize}

% section verfügbare_rohdaten (end)

\section{Durchgeführte Auswertungen} % (fold)
\label{sec:durchgeführte_auswertungen}

Zu den Untersuchungen wurden folgende Auswertungen durchgeführt und archiviert. Die einzelnen Auswertungsmethoden sind auf den folgenden Seiten näher beschrieben.

\begin{itemize}
	\item Evaluierungsblock 1 (siehe auch \citep{Bohninger10})
		\begin{itemize}
			\item Überblicksauswertung aller Anwendungen und Modelle
			\item Deskriptive Parameter aller quantitativ erhobenen Merkmale
		\end{itemize}
	\item Evaluierungsblock 2
		\begin{itemize}
			\item Überblicksauswertung aller Anwendungen und Modelle
			\item Interaktionsanalyse aller Anwendungen
			\item Deskriptive Parameter aller quantitativ erhobenen Merkmale
			\item Signifikanztests der zur Hypothesenprüfung verwendeten Merkmale
		\end{itemize}
	\item Evaluierungsblock 3
		\begin{itemize}
			\item Überblicksauswertung aller Anwendungen und Modelle
			\item Interaktionsanalyse aller Anwendungen
			\item Deskriptive Parameter aller quantitativ erhobenen Merkmale
			\item Signifikanztests der zur Hypothesenprüfung verwendeten Merkmale
		\end{itemize}
	\item Evaluierungsblock 4 (siehe auch \citep{Wahlmuller10})
		\begin{itemize}
			\item Überblicksauswertung aller Anwendungen und Modelle
			\item Interaktionsanalyse aller Anwendungen
			\item Deskriptive Parameter aller quantitativ erhobenen Merkmale
			\item Deskriptive Parameter der Items der Benutzerbefragung
			\item Signifikanztests der zur Hypothesenprüfung verwendeten Merkmale und Befragungsitems
			\item Codierung der zur Hypothesenprüfung verwendeten offenen Items der Benutzerbefragung
		\end{itemize}
	\item Evaluierungsblock 5 (siehe auch \citep{Bindreiter10})
		\begin{itemize}
			\item Überblicksauswertung aller Anwendungen und Modelle
			\item Interaktionsanalyse aller Anwendungen
			\item Deskriptive Parameter aller quantitativ erhobenen Merkmale
			\item Deskriptive Parameter der Items der Benutzerbefragung
			\item Signifikanztests der zur Hypothesenprüfung verwendeten Merkmale und Befragungsitems
			\item Codierung der zur Hypothesenprüfung verwendeten offenen Items der Benutzerbefragung
		\end{itemize}
\end{itemize}

\subsection{Überblicksauswertung}

Die Überblicksauswertung fasst die wesentlichen Eigenschaften des in einer Anwendung erstellten Modells sowie die während der Erstellung aufgetretenen Ereignisse zusammen. Die Daten wurden in Form einer Openoffice-Tabelle aufbereitet und stehen als \gls{ODF}-Dateien zur Verfügung. Sie dienen als Grundlage aller weiteren deskriptiven und schließenden statistischen Auswertungen.

Die konkrete Ausgestaltung der Tabelle variiert je nach Evaluierungsblock (abhängig von der durchgeführten Modellierungsaufgabe und der im Werkzeug implementierten Funktionalität) leicht, in Abbildung \ref{fig:img_AnhangEmpirie_raster} ist der Raster aus Evaluierungsblock 5 dargestellt, in dem die höchste Anzahl von Merkmalen erhoben wurden.

\begin{figure}[htbp]
	\centering
		\includegraphics[width=0.9\textwidth]{img/AnhangEmpirie/raster.pdf}
	\caption{Raster der Überblicksauswertung}
	\label{fig:img_AnhangEmpirie_raster}
\end{figure}

Die Befüllung der Raster erfolgte auf Basis der angefertigten Video-Aufnahmen der Werkzeuganwendungen. In den Blöcken 3 und 5 wurden die Raster redundant unabhängig voneinander von jeweils zwei Personen befüllt. Sofern Abweichungen bei der Auswertung der quantitativen Parameter festgestellt wurden, wurde das jeweilige Merkmal durch eine dritte Person geprüft und ggf. entsprechend korrigiert. In den Blöcken 1, 2 und 4 standen nicht ausreichend personelle Ressourcen für eine redundante Auswertung zur Verfügung.

\subsection{Interaktionsanalyse}

Die Interaktionsanalyse wurde wie in Abschnitt \ref{sub:interaktionsanalyse} beschrieben durchgeführt und dokumentiert. Die Dokumentation erfolgte in Textdokumenten, die als \gls{ODF}-Dateien zur Verfügung stehen.

In den Evaluierungsblöcken 3 und 5 wurde die Interaktionsanalyse für jede Werkzeuganwendung von zwei Personen redundant durchgeführt. In den Blöcken 2 und 4 standen die personellen Ressourcen für eine redundante Auswertung nicht zur Verfügung. In den Fällen, in denen redundant ausgewertet wurde, wurden Transkripte, die nur von einer der auswertenden Personen erfasst wurden, von einer dritten Person geprüft und bestätigt bzw. verworfen.

\subsection{Deskriptive Parameter} % (fold)
\label{sub:deskriptive_parameter}

Zur deskriptiven Beschreibung der erhobenen metrischen und ordinalen Merkmale in der Modellbildung und bei der Benutzerbefragung wurden im Wesentlichen folgende Parameter herangezogen:

\begin{itemize}
	\item Stichprobengröße
	\item Mittelwert
	\item Standardabweichung
	\item Boxplot, dieser codiert:
		\begin{itemize}
			\item 2.5\%-Quantil
			\item 25\%-Quantil
			\item Median
			\item 75\%-Quantil
			\item 97.5\%-Quantil
			\item Ausreißer (Werte unterhalb des 2.5\% und oberhalb des 97.5\%-Quantils)
		\end{itemize}
\end{itemize}

Diese Parameter wurden für alle fünf Evaluierungsblöcke berechnet. Teilweise wurden einzelne Merkmale zueinander in Beziehung gesetzt, um eine Normierung bzw. Vergleichbarkeit der Werte über Anwendungen bzw. Evaluierungsblöcke hinweg gewährleisten zu können.
% subsection deskriptive_parameter (end)

\subsection{Signifikanztests} % (fold)
\label{sub:signifikanztests_fb2}

Signifikanztests wurden lediglich für jene Werte durchgeführt, die zur Hypothesenprüfung verwendet wurden. Da die Untersuchungen teilweise Bestandteil von für sich genommen umfangreicher angelegter Untersuchungen im Rahmen von Masterarbeiten waren, sind nicht aller erhobenen Merkmale (vor allem in der Benutzerbefragung) für diese Arbeit von Relevanz. Auswahlkriterien für Signifikanztests und Anmerkungen zu deren Durchführung wurden in Abschnitt \ref{sub:signifikanztests} angegeben. Signifikanztests wurden in den Evaluierungsblöcken 2 bis 5 durchgeführt.

% subsection signifikanztests (end)

\subsection{Codierung offener Items} % (fold)
\label{sub:codierung_offener_items}

Bei der Auswertung der für die Hypothesenprüfung relevanten offenen Items in der Benutzerbefragung der Evaluierungsblöcke 4 und 5 wurde jeweils ein Codierungsschema entwickelt, um inhaltlich identische Antworten zusammenzufassen. Dies wurde wiederum nicht für alle in den Fragebögen angeführten offenen Items durchgeführt, die diese teilweise Bestandteil von für sich genommen umfangreicher angelegter Untersuchungen im Rahmen von Masterarbeiten waren. 

% subsection codierung_offener_items (end)
% section durchgeführte_auswertungen (end)

\section{Verwendete Fragebögen} % (fold)
\label{sec:frageboegen}

Bei der Durchführung der Evaluierungsblöcke 1, 4 und 5 wurden zusätzlich zu den direkt aus der Modellbildung erhobenen Daten auch Benutzerbefragungen mittels Fragebögen durchgeführt. Im Folgenden sind für jeden Evaluierungsblock die verwendeten Fragebögen angeführt. Zusätzlich werden die Fragebögen hinsichlich ihrer Relevanz für die untersuchten Hypothesen (siehe Kapitel \ref{cha:eval_werkzeug} bis \ref{cha:eval_aw}) eingeordnet. Die Auswertungen der ausgefüllten Fragebögen sind in digitaler Form detailliert (siehe Abschnitt \ref{sec:verfügbare_rohdaten}) bzw. aggregiert (siehe Abschnitt \ref{sec:durchgeführte_auswertungen}) in digitaler Form verfügbar.


\fboxrule0.4mm
\fboxsep0.1mm
\clearpage
\subsection{Fragebögen aus Evaluierungsblock 1}
\label{sub:fb_eval1}

Die Abbildungen \ref{fig:img_AnhangEmpirie_fb1_1-01} bis \ref{fig:img_AnhangEmpirie_fb1_1-03} zeigen den in Evaluierungsblock 1 verwendeten Fragebogen für die Modellierenden. Die Abbildungen \ref{fig:img_AnhangEmpirie_fb1_2-01} bis \ref{fig:img_AnhangEmpirie_fb1_2-02} zeigen den in Evaluierungsblock 1 verwendeten Fragebogen für die interpretierenden Teilnehmer. Abbildung \ref{fig:img_AnhangEmpirie_fb1_3-01} zeigt den Fragebogen bezüglich der Korrektheit der Interpretation, der durch die Modellierenden ausgefüllt wurde. Abbildung \ref{fig:img_AnhangEmpirie_fb1_4-01} zeigt den Fragebogen zur Erhebung der Modellierungsvorkenntnisse. Dieser Fragebogen wurde von allen Teilnehmern ausgefüllt. Der Aufbau der Fragebögen wurde von \cite{Bohninger10} detailliert beschrieben und begründet.

\begin{figure}[htbp]
	\centering
	\fbox{%
		\includegraphics[width=0.9\textwidth]{img/AnhangEmpirie/fb1_1-01.jpeg}%
	}
	\caption{Erster Fragebogen für Modellierer in Evaluierungsblock 1 - Seite 1}
	\label{fig:img_AnhangEmpirie_fb1_1-01}
\end{figure}

\begin{figure}[htbp]
	\centering
	\fbox{%
		\includegraphics[width=0.9\textwidth]{img/AnhangEmpirie/fb1_1-02.jpeg}%
	}
	\caption{Erster Fragebogen für Modellierer in Evaluierungsblock 1 - Seite 2}
	\label{fig:img_AnhangEmpirie_fb1_1-02}
\end{figure}

\begin{figure}[htbp]
	\centering
	\fbox{%
		\includegraphics[width=0.9\textwidth]{img/AnhangEmpirie/fb1_1-03.jpeg}%
	}
	\caption{Erster Fragebogen für Modellierer in Evaluierungsblock 1 - Seite 3}
	\label{fig:img_AnhangEmpirie_fb1_1-03}
\end{figure}

\begin{figure}[htbp]
	\centering
	\fbox{%
		\includegraphics[width=0.9\textwidth]{img/AnhangEmpirie/fb1_2-01.jpeg}%
	}
	\caption{Fragebogen für Interpretierer in Evaluierungsblock 1 - Seite 1}
	\label{fig:img_AnhangEmpirie_fb1_2-01}
\end{figure}

\begin{figure}[htbp]
	\centering
	\fbox{%
		\includegraphics[width=0.9\textwidth]{img/AnhangEmpirie/fb1_2-02.jpeg}%
	}
	\caption{Fragebogen für Interpretierer in Evaluierungsblock 1 - Seite 2}
	\label{fig:img_AnhangEmpirie_fb1_2-02}
\end{figure}

\begin{figure}[htbp]
	\centering
	\fbox{%
		\includegraphics[width=0.9\textwidth]{img/AnhangEmpirie/fb1_3-01.jpeg}%
	}
	\caption{Fragebogen bezüglich der Korrektheit der Interpretation in Evaluierungsblock 1}
	\label{fig:img_AnhangEmpirie_fb1_3-01}
\end{figure}

\begin{figure}[htbp]
	\centering
	\fbox{%
		\includegraphics[width=0.9\textwidth]{img/AnhangEmpirie/fb1_4-01.jpeg}%
	}
	\caption{Fragebogen bezüglich Modellierungsvorkenntnissen in Evaluierungsblock 1}
	\label{fig:img_AnhangEmpirie_fb1_4-01}
\end{figure}

\clearpage
\subsection{Fragebögen aus Evaluierungsblock 4}
\label{sub:fb_eval4}

Die Abbildungen \ref{fig:img_AnhangEmpirie_fb4_1-01} bis \ref{fig:img_AnhangEmpirie_fb4_1-08} zeigen den ersten in Evaluierungsblock 4 verwendeten Fragebogen. Die Abbildungen \ref{fig:img_AnhangEmpirie_fb4_2-01} bis \ref{fig:img_AnhangEmpirie_fb4_2-08} zeigen den zweiten in Evaluierungsblock 4 verwendeten Fragebogen. Der Aufbau der Fragebögen wurde von \citet{Wahlmuller10} detailliert beschrieben und begründet.

\begin{figure}[htbp]
	\centering
	\fbox{%
		\includegraphics[width=0.9\textwidth]{img/AnhangEmpirie/fb4_2-01.jpeg}%
	}
	\caption{Erster Fragebogen für Evaluierungsblock 4 - Seite 1}
	\label{fig:img_AnhangEmpirie_fb4_1-01}
\end{figure}

\begin{figure}[htbp]
	\centering
	\fbox{%
		\includegraphics[width=0.9\textwidth]{img/AnhangEmpirie/fb4_2-02.jpeg}%
	}
	\caption{Erster Fragebogen für Evaluierungsblock 4 - Seite 2}
	\label{fig:img_AnhangEmpirie_fb4_1-02}
\end{figure}

\begin{figure}[htbp]
	\centering
	\fbox{%
		\includegraphics[width=0.9\textwidth]{img/AnhangEmpirie/fb4_2-03.jpeg}%
	}
	\caption{Erster Fragebogen für Evaluierungsblock 4 - Seite 3}
	\label{fig:img_AnhangEmpirie_fb4_1-03}
\end{figure}

\begin{figure}[htbp]
	\centering
	\fbox{%
		\includegraphics[width=0.9\textwidth]{img/AnhangEmpirie/fb4_2-04.jpeg}%
	}
	\caption{Erster Fragebogen für Evaluierungsblock 4 - Seite 4}
	\label{fig:img_AnhangEmpirie_fb4_1-04}
\end{figure}

\begin{figure}[htbp]
	\centering
	\fbox{%
		\includegraphics[width=0.9\textwidth]{img/AnhangEmpirie/fb4_2-05.jpeg}%
	}
	\caption{Erster Fragebogen für Evaluierungsblock 4 - Seite 5}
	\label{fig:img_AnhangEmpirie_fb4_1-05}
\end{figure}

\begin{figure}[htbp]
	\centering
	\fbox{%
		\includegraphics[width=0.9\textwidth]{img/AnhangEmpirie/fb4_2-06.jpeg}%
	}
	\caption{Erster Fragebogen für Evaluierungsblock 4 - Seite 6}
	\label{fig:img_AnhangEmpirie_fb4_1-06}
\end{figure}

\begin{figure}[htbp]
	\centering
	\fbox{%
		\includegraphics[width=0.9\textwidth]{img/AnhangEmpirie/fb4_2-07.jpeg}%
	}
	\caption{Erster Fragebogen für Evaluierungsblock 4 - Seite 7}
	\label{fig:img_AnhangEmpirie_fb4_1-07}
\end{figure}

\begin{figure}[htbp]
	\centering
	\fbox{%
		\includegraphics[width=0.9\textwidth]{img/AnhangEmpirie/fb4_2-08.jpeg}%
	}
	\caption{Erster Fragebogen für Evaluierungsblock 4 - Seite 8}
	\label{fig:img_AnhangEmpirie_fb4_1-08}
\end{figure}

\begin{figure}[htbp]
	\centering
	\fbox{%
		\includegraphics[width=0.9\textwidth]{img/AnhangEmpirie/fb4_1-01.jpeg}%
	}
	\caption{Zweiter Fragebogen für Evaluierungsblock 4 - Seite 1}
	\label{fig:img_AnhangEmpirie_fb4_2-01}
\end{figure}

\begin{figure}[htbp]
	\centering
	\fbox{%
		\includegraphics[width=0.9\textwidth]{img/AnhangEmpirie/fb4_1-02.jpeg}%
	}
	\caption{Zweiter Fragebogen für Evaluierungsblock 4 - Seite 2}
	\label{fig:img_AnhangEmpirie_fb4_2-02}
\end{figure}

\begin{figure}[htbp]
	\centering
	\fbox{%
		\includegraphics[width=0.9\textwidth]{img/AnhangEmpirie/fb4_1-03.jpeg}%
	}
	\caption{Zweiter Fragebogen für Evaluierungsblock 4 - Seite 3}
	\label{fig:img_AnhangEmpirie_fb4_2-03}
\end{figure}

\begin{figure}[htbp]
	\centering
	\fbox{%
		\includegraphics[width=0.9\textwidth]{img/AnhangEmpirie/fb4_1-04.jpeg}%
	}
	\caption{Zweiter Fragebogen für Evaluierungsblock 4 - Seite 4}
	\label{fig:img_AnhangEmpirie_fb4_2-04}
\end{figure}

\begin{figure}[htbp]
	\centering
	\fbox{%
		\includegraphics[width=0.9\textwidth]{img/AnhangEmpirie/fb4_1-05.jpeg}%
	}
	\caption{Zweiter Fragebogen für Evaluierungsblock 4 - Seite 5}
	\label{fig:img_AnhangEmpirie_fb4_2-05}
\end{figure}

\begin{figure}[htbp]
	\centering
	\fbox{%
		\includegraphics[width=0.9\textwidth]{img/AnhangEmpirie/fb4_1-06.jpeg}%
	}
	\caption{Zweiter Fragebogen für Evaluierungsblock 4 - Seite 6}
	\label{fig:img_AnhangEmpirie_fb4_2-06}
\end{figure}

\begin{figure}[htbp]
	\centering
	\fbox{%
		\includegraphics[width=0.9\textwidth]{img/AnhangEmpirie/fb4_1-07.jpeg}%
	}
	\caption{Zweiter Fragebogen für Evaluierungsblock 4 - Seite 7}
	\label{fig:img_AnhangEmpirie_fb4_2-07}
\end{figure}

\begin{figure}[htbp]
	\centering
	\fbox{%
		\includegraphics[width=0.9\textwidth]{img/AnhangEmpirie/fb4_1-08.jpeg}%
	}
	\caption{Zweiter Fragebogen für Evaluierungsblock 4 - Seite 8}
	\label{fig:img_AnhangEmpirie_fb4_2-08}
\end{figure}

\clearpage
\subsection{Fragebögen aus Evaluierungsblock 5}
\label{sub:fb_eval5}

Die Abbildungen \ref{fig:img_AnhangEmpirie_fb5-01} bis \ref{fig:img_AnhangEmpirie_fb5-07} zeigen den in Evaluierungsblock 5 verwendeten Fragebogen. Der Aufbau des Fragebogens wurde von \citet{Bindreiter10} detailliert beschrieben und begründet.

\begin{figure}[htbp]
	\centering
	\fbox{%
		\includegraphics[width=0.9\textwidth]{img/AnhangEmpirie/fb5-01.jpeg}%
	}
	\caption{Fragebogen für Evaluierungsblock 5 - Seite 1}
	\label{fig:img_AnhangEmpirie_fb5-01}
\end{figure}

\begin{figure}[htbp]
	\centering
	\fbox{%
		\includegraphics[width=0.9\textwidth]{img/AnhangEmpirie/fb5-02.jpeg}%
	}
	\caption{Fragebogen für Evaluierungsblock 5 - Seite 2}
	\label{fig:img_AnhangEmpirie_fb5-02}
\end{figure}

\begin{figure}[htbp]
	\centering
	\fbox{%
		\includegraphics[width=0.9\textwidth]{img/AnhangEmpirie/fb5-03.jpeg}%
	}
	\caption{Fragebogen für Evaluierungsblock 5 - Seite 3}
	\label{fig:img_AnhangEmpirie_fb5-03}
\end{figure}

\begin{figure}[htbp]
	\centering
	\fbox{%
		\includegraphics[width=0.9\textwidth]{img/AnhangEmpirie/fb5-04.jpeg}%
	}
	\caption{Fragebogen für Evaluierungsblock 5 - Seite 4}
	\label{fig:img_AnhangEmpirie_fb5-04}
\end{figure}

\begin{figure}[htbp]
	\centering
	\fbox{%
		\includegraphics[width=0.9\textwidth]{img/AnhangEmpirie/fb5-05.jpeg}%
	}
	\caption{Fragebogen für Evaluierungsblock 5 - Seite 5}
	\label{fig:img_AnhangEmpirie_fb5-05}
\end{figure}

\begin{figure}[htbp]
	\centering
	\fbox{%
		\includegraphics[width=0.9\textwidth]{img/AnhangEmpirie/fb5-06.jpeg}%
	}
	\caption{Fragebogen für Evaluierungsblock 5 - Seite 6}
	\label{fig:img_AnhangEmpirie_fb5-06}
\end{figure}

\begin{figure}[htbp]
	\centering
	\fbox{%
		\includegraphics[width=0.9\textwidth]{img/AnhangEmpirie/fb5-07.jpeg}%
	}
	\caption{Fragebogen für Evaluierungsblock 5 - Seite 7}
	\label{fig:img_AnhangEmpirie_fb5-07}
\end{figure}

% section frageboegen (end)

% chapter daten_der_empirischen_untersuchung (end)