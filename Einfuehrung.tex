\chapter{Einführung} % (fold)
\label{cha:einführung}

\emph{„How people work is one of the best kept secrets in America.“} (Wellman, D. zitiert nach \cite{Suchman95}).

Diese Aussage David Wellmans diente Lucy Suchman in den 90er-Jahren des 20. Jahrhunderts als Motivator für ihre Forschung über die Natur menschlicher Arbeit im Computerzeitalter und deren Unterstützung durch neue Technologien. Ihre Arbeiten und viele andere (etwa \cite{Schmidt92} oder \cite{Sachs95}) argumentieren für eine stärkere Berücksichtigung der Rolle des Menschen im Arbeitsprozess.

Wellman spielt mit diesem Zitat auf die oft auftretende Diskrepanz zwischen der (niedergeschriebenen) Definition eines Arbeitsablaufs und dessen tatsächlicher Umsetzung in der konkreten betrieblichen Umgebung an.

Der Druck in der heutigen Geschäftswelt, bestimmte Qualitätskriterien garantiert erfüllen zu können, hat zu einer beinahe flächendeckenden Verbreitung von Qualitätszertifizierungen geführt. Die bekanntesten Vertreter dieser Zertifizierungen sind wohl die Standards aus der Familie der ISO 9000 Normen \cite{ISO05}. In der ISO 9001-Norm \cite{ISO00}, in der die Anforderungen an Qualitätsmanagement-Systeme definiert sind, ist festgeschrieben, dass eine prozessorientierte Organisation eine der Voraussetzungen für erfolgreiches Qualitätsmanagement ist. Ein wesentliches Merkmal einer prozessorientierten Organisation ist, dass ihre organisationalen Prozesse — also ihre Arbeitsorganisation und -abläufe — bekannt, benannt und definiert sind. 

Die Unterschiede zwischen festgeschriebenen und gelebten Arbeitsabläufen wurde schon 1978 von Argyris und Schön \cite{Argyris78} beschrieben. Mit der Unterscheidung zwischen „\emph{espoused theories}“ (= \emph{die offiziell veröffentlichten Theorien über Arbeit}) und den „\emph{theories-in-use}“ (= \emph{die tatsächlich handlungsleitenden Theorien}) wurden dieser Gegensatz auch explizit benannt. Sachs \cite{Sachs95} beschreibt das gleiche Phänomen und unterscheidet zwischen dem „\emph{explicit organisational view}“ und dem „\emph{tacit organisational view}“ auf Arbeit. Sie beschreibt damit einerseits eine explizit formulierte und statische Sicht auf Arbeit und andererseits eine informelle, im Fluss befindliche und zum Zeitpunkt der Betrachtung nirgendwo niedergeschriebene Sicht auf Arbeit. Letztere kann lediglich aus der Analyse der tatsächlichen Arbeitsabläufe gewonnen werden kann, die den Tätigkeiten zugrunde liegenden Annahmen („theories-in-use“ \cite{Argyris78}) sowie deren vom Arbeitenden konkret wahrgenommene organisationale Rahmenbedingungen bleiben allerdings verborgen — die Frage nach dem „Warum?“, die die Form des konkreten Arbeitsablaufs motiviert, kann nicht unmittelbar beantwortet werden.

Wie Sachs verdeutlichen auch Wellman sowie Argyris und Schön, dass das zweitgenannte Verständnis von Arbeit nicht explizit niedergeschrieben und formal definiert ist — in seinem Wesen also „unbekannt“ ist. Wenn „Arbeit“ oder die ihr zugrunde liegenden Annahmen unbekannt sind, kann eine Veränderung ihrer selbst oder der Umgebung, in der sie durchgeführt wird, zu schwerwiegenden Problemen führen (wie z.B. von \cite{Nonaka95}, \cite{Krogh00} oder \cite{Gerson86} beschrieben). Der Begriff „Veränderung“ deckt dabei nicht nur tiefgreifende organisatorische Änderungen im Unternehmen ab, sondern durchaus auch „marginale“ Änderungen wie z.B. die Einstellung neuer Mitarbeiter oder dem Einsatz eines neuen Werkzeugs \cite{Olesen03}. Daraus kann man schließen, dass potentiell „problematische“ Situationen häufig auftreten können. 

„Problematische“ Situationen sind dabei all jene Situationen, in denen die Zielerreichung erschwert wird, weil die dazu notwendigen Schritte entweder unklar sind oder nicht operationalisiert werden können. Im Kontext der obigen Aussage bedeutet dies, dass durch eine organisationale Veränderung neue Arbeitsschritte notwendig werden bzw. die bisherigen nicht mehr funktionieren oder angemessen sind. Beim Auftreten einer derartigen Veränderung ist daher eine erneute Planung bzw. Abstimmung der zur Zielerreichung notwendigen Arbeitsschritte notwendig. Fujimura \cite{Fujimura87} unterscheidet in diesem Sinne zwischen zwei Formen von Arbeit — der „Produktion“ („\emph{production}“) und der „Artikulation“ („\emph{articulation}“), wobei letztere alle Tätigkeiten umfasst, die die Umsetzung bzw. Aufrechterhaltung der „Produktion“ ermöglichen.

„Artikulation“ ist ein integraler Bestandteil von Arbeit \cite{Strauss85}. Mit der Komplexität der „Produktion“ steigt auch der Aufwand der dazu notwendigen „Artikulation“ an \cite{Strauss88}. Die Komplexität steigt hier mit der Anzahl der benötigten Arbeitsschritte, den dazu benötigten Kompetenzen und der Anzahl der involvierten Personen. Nicht bekannte, falsche oder zurückgehaltene Information über die „Produktion“ erschweren die „Artikulation“ oder machen sie unmöglich \cite{Fujimura87}. Dies hat jedoch nicht nur negative Auswirkungen auf die „Produktion“ sondern verhindert auch eine tiefergehende Beschäftigung mit der aktuellen Arbeitspraxis und eine potentielle Verbesserung derselben \cite{Argyris78}. Erfolgreiche Artikulation ist damit nicht nur eine Voraussetzung für eine funktionierende Produktion sondern auch ein Enabler für organisationale Veränderungen im Sinne eines organisationalen Lernprozesses (z.B. \cite{Kim93} oder \cite{Firestone03a}).

Eine methodische und technische Unterstützung kann Artikulation ermöglichen oder deren erfolgreichen Ablauf erleichtern (\cite{Schmidt92}, \cite{Simone99}, \cite{Jorgensen04}, \cite{Baker07}).

\fbox{\parbox{13cm}{\textbf{In der Arbeit sind die methodischen und technischen Möglichkeiten zur Ermöglichung und Unterstützung von Articulation Work zu ergründen, die gewonnenen Erkenntnissen in einem Werkzeug umzusetzen und dessen Auswirkungen auf die Interaktion zur Verbesserung der Production Work zu bewerten.}}}

\section{Forschungsfragen} % (fold)
\label{sec:forschungsfragen}

\begin{enumerate}
	\item Wie kann Articulation Work ermöglicht und unterstützt werden?
		\begin{enumerate}
			\item Durch welche Aktivitäten zeigt sich Articulation Work im Arbeitsprozess?
			\item Welche Rahmenbedingungen ermöglichen bzw. begünstigen Articulation Work?
			\item Wie können die in 1.1. identifizierten Aktivitäten unterstützt werden?
			\item Wie können die in 1.2. identifizierten Rahmenbedingungen und die in 1.3. identifizierten Anforderungen in einer Methodik umgesetzt werden?
		\end{enumerate}
	\item Was muss ein Werkzeug zur Unterstützung von Articulation Work leisten?
		\begin{enumerate}
			\item Welche Anforderungen an ein Werkzeug ergeben sich aus der in 1.4. entwickelten Methodik?
			\item Wie können diese Anforderungen technologisch umgesetzt werden?
		\end{enumerate}
	\item Inwiefern unterstützt das entwickelte Werkzeug die Durchführung von Articulation Work?
		\begin{enumerate}
			\item Wie kann die Unterstützungsleistung bewertet werden?
			\item Welche Auswirkungen auf die Interaktion hat die Anwendung von Methodik und Werkzeug?
		\end{enumerate}
\end{enumerate}

% section forschungsfragen (end)

\section{Aufbau der Arbeit} % (fold)
\label{sec:aufbau_der_arbeit}

% section aufbau_der_arbeit (end)

% chapter einführung (end)